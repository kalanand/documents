\label{sec:summary}

We have presented the differential distributions in jet mass for inclusive dijet
and V$+$jet events, 
%correcting energies of individual particles within jets that are
defined through the 
anti-$k_{\mathrm{T}}$ algorithm for a size parameter of 0.7 for ungroomed jets, as well as
for jets groomed through filtering, trimming, and pruning. In
addition, similar distributions for V+jet events were
given for pruned Cambridge--Aachen jets with a size parameter of 0.8,
as well as for filtered Cambridge--Aachen jets with a size parameter of 1.2. 
The impact of pileup on jet mass was also investigated.
%, as well as comparisons at the detector level
%between anti-$k_{\mathrm{T}}$-0.5, anti-$k_{\mathrm{T}}$-0.7,
%anti-$k_{\mathrm{T}}$-0.8, CA-0.8 and CA-1.2 jets.  

%{\bf To do: Add detector-level comparisons of the jet algorithms.}

Higher-order QCD matrix-element predictions for partons, coupled to
parton-shower Monte Carlo programs that generate jet mass in dijet and
V+jet events, are found to be in good agreement with data. 
A comparison of data with MC simulation indicates that
both \PYTHIA and \HERWIG reproduce the data reasonably well, 
and that the \HERWIG predictions for more aggressive grooming 
algorithms, i.e., those that remove larger fractions of contributions
to the original ungroomed jet mass, agree somewhat better with
observations. It is also observed that the more aggressive grooming
procedures lead to somewhat better agreement between data and MC simulation. 

In comparing the results from the V+jet analysis with those for the two leading jets in multijet events, 
the predictions provide slightly better agreement with the
V+jet data. This observation suggests that simulation of quark jets is better than of gluon jets. 
Differences between data and simulation are larger at small jet mass values, which also correspond to the region 
more affected by pileup and soft QCD radiation.

These studies represent the first detailed investigations of
techniques for characterizing jet substructure based on data 
collected by the CMS experiment at a center-of-mass energy of 7 \TeV. For the trimming and pruning algorithms, 
these studies mark the first publication on this subject from the LHC, and provide an important benchmark 
for their use in searches for massive particles. Finally, the intrinsic stability of these algorithms to pileup effects 
is likely to contribute to a more rapid and widespread use of these techniques in future high-luminosity 
runs at the LHC.

