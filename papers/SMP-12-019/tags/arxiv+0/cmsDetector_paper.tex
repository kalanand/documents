\label{sec:cms_detector}


The CMS detector~\cite{:2008zzk}
is a general-purpose device with 
many features suited for reconstruction of 
energetic jets, specifically, the finely segmented electromagnetic
and hadronic calorimeters and charged-particle tracking detectors.


CMS uses a right-handed coordinate system, with origin 
defined by the center of the CMS detector, 
the $x$ axis pointing to the center of the LHC ring, 
the $y$ axis pointing up, perpendicular to the plane of the LHC ring, 
and the $z$ axis along the direction of the counterclockwise beam. 
The polar angle $\theta$ is measured relative to the positive $z$ 
axis and the azimuthal angle $\phi$ relative to the $x$ axis in the $x$-$y$ plane.


Charged particles are reconstructed in the inner silicon tracker,
which is immersed in a $3.8$~T axial magnetic field. 
%The tracker consists
%of three concentric layers and two endcap disks 
%in each, which are made of pixel sensors, and ten
%concentric layers and twelve endcap disks of strip sensors.  
The CMS tracking detector consists of an inner silicon pixel detector
composed of three concentric central layers and two sets of disks
arranged forward and backward of the center, 
and up to ten silicon strip central layers and three inner and nine
outer strip disks forward and backward of the center. 
This arrangement provides 
full azimuthal coverage for $|\eta| < 2.5$, where 
$\eta = -\ln\tan(\theta/2)$ is the pseudorapidity.
The pseudorapidity approximates the rapidity $y$
and 
equals $y$ for massless particles.
Since many of the reconstructed jets
are not massless, we use the rapidity $y$ for characterizing
jets in this analysis.


A lead tungstate crystal electromagnetic calorimeter (ECAL) and 
a brass/scintillator hadronic calorimeter (HCAL) surround the tracking
volume and provide photon, electron, and jet reconstruction up to $|\eta|=3$.
The ECAL and HCAL cells are grouped into towers projecting radially 
outward from the center of the detector.  
In the central region ($|\eta|<1.74$), 
the towers have dimensions of $\Delta\eta = \Delta\phi = 0.087$ 
that increase at larger $|\eta|$.  
ECAL and HCAL cell energies above some chosen noise-suppression 
thresholds are combined within each tower to define the tower energy. 
Muons are measured in gas-ionization detectors embedded in the steel 
return yoke outside the solenoid. 
To improve reconstruction of jets, the tracking and calorimeter 
information is combined in a ``particle-flow'' (PF)
algorithm~\cite{particleflow}, which is described in 
Section~\ref{sec:reconstruction}.



For the analysis of dijet events, samples are simulated with 
\PYTHIA.4 (Tune Z2) ~\cite{pythia,Field:2010bc}, 
\PYTHIAEIGHT  (Tune 4c)~\cite{pythia8}, 
and \HERWIG (Tune 23)~\cite{herwig}, and
propagated through the simulation of the CMS
detector based on \GEANT4 \cite{Geant4}.
Underlying event (UE) and pileup (PU) are included in the
simulations, which are also reweighted to have the simulated
PU distribution match the observed PU distribution
in the data. 



For the V+jet analysis, events with a vector boson produced in 
association with jets are simulated using \MADGRAPH 5.1~\cite{madgraph}. 
This matrix element generator is also used to simulate $\ttbar$ events. 
The \MADGRAPH events are subsequently subjected to parton showering, simulated 
with \PYTHIA using the Z2 Tune~\cite{Field:2010bc}. 
To compare hadronization in different generators, we generate V+jet
samples in which parton showering and hadronization are 
simulated with \HERWIG.  
Diboson (WW, WZ, and ZZ) events are also generated with \PYTHIA. 
Single-top-quark samples are produced with \POWHEG~\cite{powheg}, and the 
lepton enriched dijet samples are produced with \PYTHIA using the Z2 Tune. 
CTEQ6L1~\cite{cteq} is the default set of parton distribution
functions used 
in all these samples, except for the single-top-quark MC, which uses CTEQ6M.
%The \PYTHIA parameters for the underlying event are set to the Z2 tune.


 
