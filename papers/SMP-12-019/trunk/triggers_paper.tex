\subsection{Dijet trigger selection}
\label{sec:dataSampleAndEventSelection}

Events are collected using single-jet triggers, which are
based on jets reconstructed only from calorimetric information. 
This procedure yields inferior resolution to jets reconstructed
offline with PF constituents, but provides faster 
reconstruction that meets trigger requirements.
As the instantaneous luminosity is time-dependent, the specific
jet-$\pt$ thresholds change with time.
%\label{sec:trigAssignment}
The triggers used to select dijet events have partial overlap. 
Those with lower-$\pt$ thresholds have high prescale settings to accommodate the higher
data-acquisition rates, and some
events selected with these lower-$\pt$ triggers are also collected at
higher thresholds.

To avoid double counting of phase space, each event is assigned
to a specific trigger. 
To do this, we compute the trigger 
efficiency as a function of reconstructed $\pt^{AVG}$, select 
an interval in trigger efficiency where the efficiency is maximum ($>
95$\%) for 
that range of $\pt^{AVG}$, and assign that trigger to the appropriate $\pt^{AVG}$ interval. 
The assignment is based on the
jet $\pt$ values
reconstructed offline (but not groomed). %Table~\ref{TriggerTurnOns}
%shows the $\pt$ thresholds for each of the dijet triggers, and the corresponding interval 
%used for the reconstructed $\pt^{AVG}$ in the event.
Table~\ref{TriggerTurnOns}
shows the $\pt$ thresholds for each of the jet triggers used in the
analysis, and the corresponding intervals of $\pt$ to which the 
triggered events are assigned. 

\begin{table}[h]
  \centering
  \caption{Trigger $\pt$ thresholds for individual jets, 
    and corresponding $\pt^{AVG}$ intervals used to assign the
    triggered events in the dijet analysis.\label{TriggerTurnOns}}
  \begin{tabular}{ |c|c|}
    \hline 
\rule{0pt}{12pt}
Trigger $\pt$ threshold (\GeVns) & $\pt^{AVG}$ range (\GeVns) \\ 
\hline
%60 & 0-150   \\
%100& 150-220 \\
190& 220--300  \\
240& 300--450  \\
370& $>$450 \\
   \hline 
  \end{tabular}
\end{table}
 
\subsection{V+jet trigger selection}
\label{sec:dataSampleAndEventSelectionVjet}


Several triggers are also used to collect events corresponding to
the topology of V+jet events, where the V decays via electrons or
muons in the final state. 
For the \PW+jet channels, the triggers consist of several single-lepton 
triggers, with lepton identification criteria applied online. 
To assure an acceptable event rate, leptons are required to be isolated from other
tracks and energy depositions in the calorimeters. 
For the \PW$(\mu\nu_\mu)$ channel, the trigger thresholds for the
muon $\pt$ are in the range of 17 to 40\GeV.
The higher thresholds are used at higher instantaneous luminosity. 
The combined trigger efficiency for signal events 
%that pass all trigger and offline requirements
that pass offline requirements
(described in Section~\ref{sec:evsel_paper}) is ${\approx} 92\%$.


For the $\PW(\Pe\Pgne)$ events, the electron $\pt$ threshold ranges 
from 25 to 65 \GeV. 
To enhance the fraction of \PW+jet events in the data, the
single-electron triggers are also required to have minimum thresholds 
on the magnitude of the imbalance 
in transverse energy ($\met$) and on the transverse mass 
($m_\mathrm{T}$) of the (electron + $\met$) system,
where $m_\mathrm{T}^2 = 2E_\mathrm{T}^{\mathrm{e}}\met(1-\cos\phi)$,
and $\phi$ is the angle between the directions of $p_{\mathrm{T}}^{\mathrm{e}}$ and $\met$. 
The combined efficiency for electron \PW+jet events that pass the offline 
criteria is ${\approx} 99\%$.


 The $\Z(\mu\mu)$ channel uses the same single-muon triggers as the
 $\PW(\mu\nu_\mu)$ channel. The $\Z (\Pe\Pe)$ channel uses dielectron triggers
 with lower thresholds for $\pt$ (17 and 8\GeV), and additional isolation
 requirements. These triggers are 99\% efficient for all
 {\Z}$+$jet events that pass the final offline selection criteria.

\subsection{Binning jets as a function of $\pt$}
\label{sec:ptBinAssignment}


The jet $\pt$ bins introduced in Eq.~(\ref{eq:pdf_mjet_i}) are given in
Table~\ref{tab:ptBins} for V+jet and dijet events. The jet $\pt$ is re-evaluated for each grooming
algorithm. 
Because there are large biases due to jet misassignment in the dijet
events, especially at small $\pt$ (when three particle-level jets are
often reconstructed as two jets in the detector, or vice versa), 
the $\pt$ intervals for these events begin at 220 GeV. 
Furthermore, the smaller number of events in the V+jet samples precludes the
study of these events beyond $\pt=$ 450 GeV.


\begin{table}[h]
  \centering
  \caption{Intervals in ungroomed jet $\pt$ for the V+jet and dijet analyses. \label{tab:ptBins}}
  \begin{tabular}{ |ccc|}
    \hline 
\rule{0pt}{12pt}
    Bin & $\pt$ interval (\GeVns) & Analysis\\ 
    \hline
%    1 & 50-125 \GeV \\
    1 & 125--150 & V+jet \\
    2 & 150--220 & V+jet  \\
    3 & 220--300 & V+jet,dijet  \\
    4 & 300--450 & V+jet,dijet  \\
    5 & 450--500 & dijet  \\
    6 & 500--600 & dijet  \\
    7 & 600--800 & dijet  \\
    8 & 800--1000 & dijet  \\
    9& 1000--1500 & dijet  \\
%    11& $>$1500  \\
   \hline 
  \end{tabular}
\end{table}




