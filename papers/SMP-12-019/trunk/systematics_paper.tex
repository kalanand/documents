\label{sec:systematics}

Before comparison of the jet mass distributions with QCD predictions, the data are
corrected to the particle level for detector effects, such as resolution and
acceptance. The simulated particle-level jets are
reconstructed with the same algorithm and with the same parameters as
the PF jets. We use the unfolding procedure described in
Refs.~\cite{unfolding_extra1,unfolding_extra2,unfolding_extra3,unfolding_extra4,agostini} 
to correct the jet mass, through
an iterative technique for finding the
maximum-likelihood solution of the unfolding problem.
The detector response matrix is obtained in MC studies of jets. 
In general, the number of iterations must be tuned to minimize the
impact of
statistical fluctuations on the result. In practice, however, the
procedure is largely insensitive to
the precise settings and binning of events and four iterations usually
suffice. A larger number of iterations were found to
provide the same results except for small fluctuations in the tails of
distributions. A simpler bin-by-bin unfolding is used as a
cross-check, 
and is found to provide similar results, with fluctuations
in the tails of the distributions. The jet transverse momenta are
not unfolded. 

Systematic uncertainties are estimated by modifying the 
response matrix for each source of uncertainty by $\pm 1$ standard
deviation, and comparing the mass
distribution to the nominal results, based on simulated \PYTHIA
events. The difference in the unfolded mass spectrum from such a change is taken as
the uncertainty arising from that source.


The experimental uncertainties that can affect the unfolding
of the jet mass
include the jet energy scale (JES),
jet energy resolution (JER), and jet angular resolution (JAR). 
The uncertainty from JES is estimated by raising and lowering the jet
four-momenta by the measured uncertainty as a function of jet $\pt$
and $\eta$~\cite{citeJEC}, which typically corresponds to 1--2\% for the jets
in this analysis. Two additional $\pt$- and $\eta$-independent
uncertainties are included: a 1\% uncertainty to account for
differences observed between the measured and
predicted $\wboson$ mass for high-$\pt$ jets in a $\ttbar$-enriched sample, and a
3\% uncertainty to account for differences in the groomed and
ungroomed energy responses found in MC simulation~\cite{EXO-11-006}. 

The impact of uncertainties in JER and JAR on $m_J$ are evaluated by
smearing the jet energies, as well as the resolutions in $\eta$ and
$\phi$, each by 10\% in the MC simulation relative to the
particle-level generated jets~\cite{citeJEC}. 
These estimated uncertainties on JER and JAR are found to be
essentially the same for all jet grooming techniques in MC studies. 
Since this analysis uses jets constructed from PF constituents, the
charged particles have excellent energy and angular
resolutions, but their use induces a dependence on tracking
uncertainties, e.g., tracking efficiency. This dependence is accounted for
implicitly in the $\pm$10\% changes in jet energy and angular
resolutions, since such changes would lead to a difference between
expected and observed values of these quantities. The same is true for the
neutral electromagnetic
component of the jet (primarily from $\pi^0 \rightarrow \gamma\gamma$
decays). 

The remaining sources of uncertainty are estimated from MC simulation.
The tracking information is not sensitive to the neutral hadronic
component of jets, and this small contribution is taken
directly from simulation. 
We estimate this remaining uncertainty by comparing the unfolded data using \PYTHIA 
and using \HERWIG, and assign the difference as a systematic uncertainty.
This also accounts for the uncertainty from modeling parton showers. 
The latter effect often comprises the largest uncertainty in the unfolded jet mass
distributions as described below. 
Other theoretical ambiguities that can affect the unfolding of the jet
mass include the variation of the parton distribution functions and
the modeling of initial and final-state radiation (ISR/FSR). The former
was investigated and found to be much smaller than the difference
between the unfolding with \PYTHIA and the unfolding with \HERWIG, and
hence is neglected. The latter is included implicitly in the uncertainty between
\PYTHIA and \HERWIG. 

%The jet energy and angular resolution are estimated by increasing and decreasing the jet energy, 
%$\eta$ and $\phi$ resolution by 10\%.
%The nominal value chosen is 10\%, motivated by the observed difference between data %and simulation 
%in the jet energy resolution~\cite{citeJEC}, so the up and down variations 
%correspond to 20\% and no additional smearing with respect to the Monte Carlo, respectively. 
As described in Section~\ref{sec:reconstruction}, the jets used
in this analysis are reconstructed after removing
the charged hadrons that appear to emanate from subleading primary
vertices.  
This procedure produces a dramatic (${\approx} 60\%$) reduction in the pileup
contribution to jets. 
The residual uncertainty from pileup is obtained through MC simulation, 
estimated by increasing and decreasing the cross section for minimum-bias events by 8\%. 


% The jets used at CMS are created from constituents reconstructed 
% with the particle-flow algorithm. Due to this fact, the systematic
% effects that are considered in this analysis are handled differently
% than previous measurements at ATLAS~\cite{atlasJS}, for instance. 
% In that measurement, explicit uncertainties on the jet mass scale and
% resolution (JMS and JMR) are estimated using the ratio of masses
% of jets reconstructed with only calorimetric information to the
% masses of jets reconstructed with only tracking information. 
% To be explicit, this methodology corrects for the differences
% in jet mass in data and MC 
% due to the charged components of the jets only. The
% remaining differences due to the neutral components are taken 
% directly from the MC. 

% At
% CMS, the tracking and calorimetric measurements are synthesized with the
% particle-flow algorithm, and hence the same method cannot be used to
% estimate the charged components of the JMS and JMR. 
% However, since the charged components are directly measured
% with the tracking information, this is already accounted for
% in the uncertainties on the jet mass due to the energy and
% angular resolutions. In addition, the jet mass scale is explicitly
% checked with the difference in the $\wboson$ mass in data
% and MC as described above, and also included already. 
% The remaining differences due to the neutral components
% are also taken directly from the MC, from the differences
% between the \pythia and \herwig jet masses. 




In the dijet analysis, there can be incorrect assignments of leading
reconstructed jets relative to the generator level, e.g., two
generator-level jets can be matched to three reconstructed jets, or vice versa. 
This effect causes a bias in the unfolding procedure, which becomes greater
at small $\pt$. This bias is corrected through MC studies
of matching of particle-level jets to reconstructed jets,
and the magnitude of the bias correction is also 
added to the overall systematic uncertainty.
Such misassignments are negligible in the V+jet analysis. 

