\subsection{Vector Boson Reconstruction and Selection}
\label{sec:Vselection}

Reconstruction of W and Z bosons begins with the identification
and selection of charged leptons and pfMET described in the previous 
section.  Given the unique signature of a highly boosted vector 
boson recoiling from jets, a minimal selection is sufficient to 
identify highly pure samples of V+jets events. The background is dominated by $t\bar{t}$ event (and in lower extent from single top events) in the $W$+jet topology, while in the $Z(\ell\ell)$+jet analysis the additional constraint on the di-lepton mass kills almost completely these backgrounds.  

Candidate \ZtoLL\ decays are reconstructed by combining 
isolated electrons and muons and requiring the dilepton invariant 
mass to satisfy $80<M_{\ell\ell}<100\GeV$.  
%Figure~\ref{fig:InclV} shows
%the dimuon invariant mass in events selected with this loose criteria,
%and including two central jets with a minimum threshold of $20\GeV$ on
%the transverse momentum.  Selection of \ZtoNN\ decays is accomplished 
%simply by requiring $\mathrm{pfMET}>160\GeV$.

%\begin{figure}[tbp]
%  \begin{center}
%    \includegraphics[width=0.49\textwidth,height=0.25\textheight]{figures/Zmm-Inclusive}
%    \hfill
%    \includegraphics[width=0.49\textwidth,height=0.25\textheight]{figures/Wmn-Inclusive}
%    \caption{Distributions of dimuon mass (left) and W transverse momentum
%    (right) for inclusive \ZmmJ\ and \WmnJ\ events selected with two central jets without
%    boosting.  These plots for purely for illustration of what the data looks like before
%    boosting and tagging.}
%    \label{fig:InclV}
%  \end{center}
%\end{figure}

Candidate \WtoLN\ decays are identifed primarily by the topology
of a single isolated lepton and additional missing energy.  The
transverse momentum \ptW\ and mass \mtW\ of the W candidate are 
computed as:
\begin{eqnarray}
\ptW & = & \sqrt{(\texttt{pfMET}_x + p^{\ell}_x)^2 + (\texttt{pfMET}_y + p^{\ell}_y)^2}\texttt{, and}\\
\mtW & = & \sqrt{(\texttt{pfMET}+{\ptl})^2 - {\ptW}^2}.
\end{eqnarray}

%Figure~\ref{fig:InclV} shows the distributions of \mtW\ in events 
%selected with pfMET$ > 40\GeV$ and pfMETsig$>2$, and 
%the addition of two central jets with transverse momentum above $30\GeV$.

%It is observed that in the boosted regime, where the QCD background is
%much reduced, simply requiring $\ptW>\sim 150\GeV$ is sufficient to select a 
%relatively clean sample of real W decays.  For inclusive W production, 
%the distribution of \mtW\ reflects the characteristic Jacobian peak and 
%is very effective at separating signal from the large background of 
%generic QCD production at small values of the transverse mass.  In 
%contrast, for the high boost used in this analysis, the neutrino begins 
%to overlap with the lepton in azimuth, creating a broad flat region in 
%\mtW\ between $0$--$50\GeV$ that reduces the effectiveness of this 
%variable in rejecting QCD background.  Therefore, no selection is applied 
%on \mtW\ in the reconstruction of W candidates in the signal region. 
%However, \mtW\ remains effective at reducing QCD background in the 
%low-boost \Wbb\ control region (see Sec.~\ref{BkgControl}), and in 
%generally cleaning up the background in all of the control regions for 
%the electron mode.

The jet mass analysis in $V$+ jet event is carried on in a boosted kinematic regime, namely $p_T (V)> 120$ GeV. We further require the leading jet in the event (independently of the clustering or radius) to have $p_T>$ 125 GeV.

Simply requiring a boosted regime is highly effective to suppress the QCD background (in addition to the tight isolation cuts on the leptons, naturally). 
In the \WtoLN\ +jet analysis, further QCD rejection is achieved by requiring PFMET $>$ 30 GeV and $M_T(W)>$ 50 GeV.


