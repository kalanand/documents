\subsection{Event reconstruction}
\label{evrecosection}

\ifnpas
All data are reconstructed using CMSSW 4.2.x.
\fi
\label{sec:preselection}
\label{sec:reconstruction}
As indicated above, events are reconstructed using the particle-flow 
algorithm, which
combines the information from all subdetectors to reconstruct
the particle candidates
in an event. 
The algorithm categorizes particles into muons,
electrons, photons, charged hadrons, and neutral hadrons. 
The resulting PF candidates are passed through each jet clustering 
algorithm of Section~\ref{sec:algos},  
as implemented in FastJet (Version 3.0.1) \cite{fastjet1,fastjet2}.


%An extra correction is applied to the data to account for a residual
%nonlinearity that is not observed in the simulation. No pileup
%corrections are applied. 
%Charged hadrons identified as pileup are removed from the inputs to the jet %clustering algorithms.
%The charged hadrons are classified as belonging to a 
%pileup vertex when they are used to reconstruct a vertex that is not
%the highest $\pt$ primary vertex. The primary vertices are
%reconstructed with a deterministic annealing filter (DAF) technique from the tracks %in
%the event. There are also quality criteria
%placed on the primary vertex, in that it must contain at least four
%degrees of freedom in the spatial fit (roughly corresponding to at least four
%tracks), and satisfy $\chi^2 / ndof < 8$. It must also be within the
%physical region of the pixel detector. 

The reconstructed interaction vertex characterized by the largest value 
of $\sum_i ({\pt}^{\mathrm{trk}}_i)^2$, where ${\pt}^{\mathrm{trk}}_i$ is the transverse momentum of the 
$i^{\mathrm{th}}$ charged track associated with the vertex, is defined
as the leading primary vertex (PV) of the event. 
This vertex is used as the reference vertex for all PF objects in the event. 
A pileup interaction can affect the reconstruction of 
jet momenta and $\met$, as well as lepton isolation and b-tagging efficiency. 
To mitigate these effects, a track-based algorithm is used to remove 
all charged hadrons that are not consistent with originating from the
leading PV. 

Electron reconstruction requires the matching of an energy cluster in the 
ECAL with a track extrapolated from the silicon 
tracker~\cite{CMS-PAS-EGM-10-004}.  
Identification criteria based on the energy distribution of showers in the 
ECAL %the matching of tracks to ECAL clusters, 
and consistency of tracks 
with the primary vertex are imposed on electron candidates. 
Additional requirements remove any electrons produced through
conversions of photons in detector material. 
The analysis considers electrons only in the range of $|\eta|<2.5$, 
excluding the transition region $1.44<|\eta|<1.57$ between the central and endcap ECAL detectors 
because of poorer resolution for electrons in this region.
Muons are reconstructed using two algorithms~\cite{CMS-PAS-MUO-10-004}: 
(i) in which tracks in the silicon tracker are matched to signals in 
the muon chambers, and (ii) in which a global fit is performed to a track 
seeded by signals in the external muon system. 
The muon candidates are required to be reconstructed through both algorithms. 
Additional identification criteria are imposed on muon candidates to reduce 
the fraction of tracks misidentified as muons, and to reduce
contamination from muon 
decays in flight. 
These criteria include the number of hits detected 
in the tracker and in the outer muon system, the quality of the fit to 
a muon track, and its consistency of originating from the leading PV.



Charged leptons from V-boson decays are expected to be isolated from 
other energy depositions in the event. 
For each lepton candidate, a cone with radius 0.3 for muons and 0.4
for electrons is chosen around the direction of the 
track at the event vertex. 
When the scalar sum of the transverse momenta of reconstructed particles
within that cone, excluding the contribution from the 
lepton candidate, 
exceeds ${\approx} 10\%$ of the $\pt$ of the 
lepton candidate, that lepton is ignored.  
The exact isolation requirement depends on the $\eta$, $\pt$, and flavor of
the lepton.  
Muons and electrons are required to have $\pt > 30$~GeV and $>80$~GeV, respectively. 
The large threshold for electrons ensures good trigger efficiency.
To avoid double counting, isolated charged leptons are removed from 
the list of PF objects that are clustered into jets.
%The very high offline threshold on the eletron momentum is motivated by the criteria to avoid turn-off effects in the trigger efficiency for the single electron trigger. 

%Furthermore, charged leptons with an isolation
%of 15\% (muons) or 20\% (electrons) of the lepton transverse momentum 
%are removed from consideration for the jet clustering algorithm, 
%where the isolation is defined as the
% energy from charged particles, neutral particles, 
%and photons (counted separately). 


After removal of isolated leptons and charged hadrons 
%attributable to accompanying 
from pileup vertices, only the
neutral hadron component from pileup remains and is included in the
jet clustering. %This pileup contribution is corrected for as follows. 
This remaining component of pileup to the jet energy is removed by
applying a
correction based on a 
mean $\pt$ per unit area of ($\Delta y \times \Delta \phi$) 
originating
from neutral particles~\cite{jetarea_fastjet,jetarea_fastjet_pu}.
This quantity is computed using
the $k_{\mathrm T}$ 
algorithm, and corrects the jet energy
by the amount of energy expected from pileup in the jet cone. 
This ``active area'' method adds a large number of 
soft ``ghost'' particles to the
clustering sequence to determine the effective area subtended by each
jet. 
This procedure is done for all grooming algorithms just
as for the ungroomed jets. 
The active area of a groomed jet is smaller than that of an ungroomed jet, and the pileup correction is therefore also smaller.. 
Different responses in the endcap and central barrel calorimeters 
necessitate using $\eta$-dependent jet corrections. 
The amount of energy expected from the remnants of the hard collision
(the underlying
event) is estimated from minimum-bias data and MC events, and
is added back into the jet. 



In addition, the pileup-subtracted jet four-momenta in data are
corrected for nonlinearities in $\eta$ and $\pt$ by using a
$\pt$- and $\eta$-dependent correction to account for the difference 
between the response in MC-simulated events and 
data~\cite{citeJEC}.  
The jet corrections are derived for the ungroomed
jet algorithms but are also applied to the groomed algorithms, thereby adding 
additional systematic uncertainty in the energy of groomed jets. 


% These corrections are then
% applied as 
% \begin{equation}
% \pt^{PU-corr} = \pt^{RAW} \times \left( 1 - F(\eta) \times \frac{A_{jet} (\rho - <\rho_{UE}>)}{<A_{jet}> <(\rho_{PU,NPV=1})>} \right)
% \end{equation}
% The function $F(\eta)$ is an $\eta$-dependent correction designed to correct for
% different responses between the barrel and endcap calorimeters.
% It is derived from an assumption of a linear dependence of jet
% momenta as a function of the number of reconstructed primary vertices, 
% and then computed over a range of jet pseudorapidities. 
% The area of the jet $A_{jet}$ is computed as described in Ref~\cite{fastjet_area},
% and depends ont he jet algorithm. The mean $\pt$ per unit area for underlying
% event ($\rho_{UE}$) is added back into the jet, since the cachement area approach
% subtracts this as well (which should, however, be added in the jet momentum measurement
% to be consistent with theoretical predictions). The average jet area $<A_{jet}>$ is
% derived for the algorithm in question in a dijet sample, and $<\rho_{PU,NPV=1}>$ is the
% expected average mean $\pt$ per unit area in events with exactly one primary vertex. 


%An accurate measurement of $\met$ is essential for distinguishing 
%the W signal from background processes. 
%The $\met$ in the event is defined using the PF objects,
%and this analysis requires $\met > 50$~GeV. 
