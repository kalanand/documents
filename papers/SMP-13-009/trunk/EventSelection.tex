%\section{Physics objects reconstructions and event selection}
%\label{sec:eventsel}


% MC samples
\section{Event simulation }
\label{sec:mc}
All Monte Carlo (MC) simulation samples, except for the single-top-quark samples, are generated with the
MADGRAPH 5.1.3.22 \cite{MadGraph} event generator using the CTEQ6L1 parton
distribution functions (PDF). Single-top-quark samples are generated with {\sc POWHEG} (v1.0, r1380)\cite{Alioli:2009je,Re:2010bp} with the CTEQ6M PDF 
set~\cite{Pumplin:2002vw,Nadolsky:2008zw}. 
The matrix element calculation is used, and outgoing partons are matched to parton showers from
{\sc PYTHIA} 6.426 \cite{pythia} tune $Z2^*$ \cite{Collaboration:2012tb} with
a matching threshold of 20~GeV and a dynamic factorization ($\mu_F$) and
renormalization ($\mu_R$) scale given by $\sqrt{m^{2}_{W/Z} + p^{2}_{T,W/Z}}$. 
The next-to-leading-order/leading-order (NLO/LO) cross section correction factors (K-factors) for WW$\gamma$, WZ$\gamma$ and AQGC diagrams 
are derived using the NLO cross sections calculated with {\sc aMC@NLO}~\cite{Frixione:2002ik}. The MSTW2008nlo68cl~\cite{Martin:2009iq} PDF 
set is used to calculate the PDF uncertainty following the prescription of Ref.~\cite{Frederix:2011ss}. 
The K-factor obtained for WW$\gamma$ and WZ$\gamma$ is consistent with a
constant value of 2.1 for photons with $E_{T}>30$~GeV. The K-factor for AQGC diagrams is
found to be close to 1.2.
A \GEANTfour-based simulation~\cite{GEANT4} of the CMS
detector is used in the production of all MC samples. All simulated events are reconstructed and analyzed using the same algorithms as for the LHC 
collision events. Additional corrections (scale factors) are applied to take into account the difference in lepton 
reconstruction and identification efficiencies observed between data and simulated events.
For all simulated samples, the hard-interaction collision is overlaid with a number of simulated
minimum bias collisions. The resulting events are weighted to reproduce the distribution of
the number of inelastic collisions per bunch crossing (pileup) inferred from data.
A summary of the contributing processes is given in Table~\ref{tab:samples}.

\begin{table}[htb]
\centering
  \caption{Cross sections used to normalize the simulated samples. All cross sections assume a photon $E_T$ $>$ 10~GeV, $|\eta^{\gamma}|<2.5$.  
The order of the cross section calculation is also indicated. The normalization for the W$\gamma$+Jets sample is derived from data. The contribution from the 
jet$\rightarrow\gamma$ process is also obtained from data as described in Section~\ref{sec:BackgroundModeling}. }
\scalebox{1.05}{
  \begin{tabular}{|c|c|c|}
  \hline
  Process & Cross section [pb] \\
  \hline
  \hline
  SM WW$\gamma$                  & (NLO)      0.090 $\pm$ 0.021    \\
  SM WZ$\gamma$                  & (NLO)      0.012 $\pm$ 0.003    \\
  \hline
  W$\gamma$ + jets               & (data)     10.9  $\pm$ 0.8    \\
  Z$\gamma$ + jets               & (LO)       0.63   $\pm$ 0.13 \\
  $t\overline{t}\gamma$          & (LO)       0.62   $\pm$ 0.12 \\
  Single t + $\gamma$(inclusive) & (NLO)      0.31   $\pm$ 0.01  \\
  \hline
  \end{tabular}}
  \label{tab:samples}
\end{table}

To simulate the signal events for a given AQGC parameter set, several samples are generated with a range of parameter values, 
with the other aQCG parameters set to zero.

\section{Event reconstruction and selection}
\label{sec:recosel}
% Intro

The dataset used in this analysis corresponds to a total integrated
luminosity of 19.3~$\pm$~0.5 (19.2~$\pm$~0.5) $\fbinv$ ~\cite{lumiPAS} collected with
the CMS detector in the muon (electron) channel in pp collisions at
$\sqrt{s}=8$\TeV in 2012.  The data were recorded with single-lepton
triggers using $p_T$ thresholds of 24~\GeV for muons and 27~\GeV for
electrons. The overall trigger efficiency is about 94\% (90\%) for
muon (electron) data, with a small dependence (a few percent) on $p_T$
and $\eta$. Simulated events are corrected for the trigger efficiency
as a function of lepton $p_T$ and $\eta$.


The events used in this analysis are characterized by the production
of a photon plus a pair of massive gauge bosons (WW or WZ), where a
W boson decays to leptons and the other boson (W or Z) decays to
quarks. The selections that are applied to select leptonic W boson decays include a
requirement on the transverse mass (large than 30~GeV~\cite{WZCMS:2010}), with either 
one muon ($p_T >$ 25~GeV, $|\eta| <$ 2.1) or one electron ($P_T >$ 30 GeV, $|\eta| <$ 2.5, excluding the transition 
region between the ECAL barrel and endcaps $1.44 < |\eta| < 1.57$) in the final state. Events with
additional leptons are vetoed in order to reduce backgrounds. The escaping 
neutrino induces physical missing transverse energy ($\MET$) in the reconstructed 
event. A selection requirement of $\MET >$ 35~GeV 
is therefore applied to reject the multijet backgrounds. The two most energetic jet candidates are required to
satisfy $p_T >$ 30~GeV and $|\eta| <$ 2.4. The photon candidate must satisfy $E_T >$ 30~GeV and $|\eta| <$ 1.44. 


% Jets & MET
Jets and 
$\MET$~\cite{Chatrchyan:2011tn,WZCMS:2010} are formed from particles reconstructed using the PF algorithm.
Jets are formed with the anti-$k_T$ clustering
algorithm~\cite{ANTIKT} with a distance parameter of 0.5. 
Charged particles with tracks not originating from
the primary vertex are not considered for jet
clustering~\cite{fastjet1, fastjet2}. The primary vertex of the event is chosen to be the vertex with the highest
$\sum p_T^2$ of its associated tracks.
Jets are required to satisfy
identification criteria that eliminate candidates originating from noisy
channels in the hadron calorimeter~\cite{Chatrchyan:2009hy}.  Jet
energy corrections \cite{Chatrchyan:2011ds} are applied to
account for the jet energy response as a function of $\eta$ and $p_{T}$,
and to correct for contributions from event pileup. Jets from pileup are
identified and removed using the trajectories of tracks associated with the jets, 
the topology of the jet shape and the constituent multiplicities.

The azimuthal separation between the highest $p_T$ jet and the $\MET$
direction is required to be larger
than 0.4. This criterion reduces the QCD multijet background, for
which the \MET is generated sometimes by a mismeasurement of the
leading jet energy. To
reduce the background from W$\gamma$+jets events, requirements on the
dijet invariant mass window of 70 $< m_{jj} <$ 100~GeV, and on the
separation between the jets of $|\Delta\eta_{jj}| <$ 1.4, are
imposed. In order to reject top-quark backgrounds, the two jets are
also required to fail a b quark jet tagging requirement. The combined
secondary vertex algorithm~\cite{Chatrchyan:2012jua} is used, with a
discriminator based on the displaced vertex expected from b hadron
decays. The algorithm selects b hadrons with about 70\% efficiency,
and has a 1\% misidentification probability.

% Muons
Muon candidates are reconstructed by combining information from the
silicon tracker and from the muon detector by means of a global
fit. The muon candidates are required to pass the standard CMS muon
identification and the track quality criteria~\cite{muReco}. The isolation
variables used in the muon selection are based on the PF
algorithm and are corrected for the contribution from pileup. The muon
candidates have a selection efficiency of approximately 96\%.

% Electrons
Electrons are reconstructed from clusters~\cite{Chatrchyan:2013dga} of ECAL energy
deposits matched to tracks in the silicon tracker within the ECAL
fiducial volume, with the exclusion of the transition region between the barrel and
the endcaps, defined above. The electron candidates are
required to be consistent with a particle originating from the primary
vertex in the event. The isolation variables used in the electron selection are based on the PF
algorithm and are corrected for the contribution from pileup. 
The electron selection efficiency is approximately
80\%. To suppress the $Z\rightarrow e^+e^-$ background in the electron
channel, where one electron is misidentified as a photon, a Z boson mass veto of $|M_Z - m_{e\gamma}| >$
10~GeV is applied.

% Photons
Photon candidates are reconstructed from clusters of cells with
significant energy deposition in the electromagnetic calorimeter;
these are merged into ``superclusters''. The photons are required to be within the ECAL barrel fiducial region ($|\eta| <$ 1.44). 
The observables used in the photon selection are isolation variables based on the particle flow algorithm and are corrected for 
the contribution from pileup, the ratio of hadronic energy in the HCAL that is matched in $(\eta,\phi)$ to the electromagnetic 
energy in the ECAL, the transverse width of the electromagnetic shower and an electron track veto.

% Backgrounds and additional selection
\section {Background modeling}
\label{sec:BackgroundModeling}

The main background contribution arises from W$\gamma$+jets
production. After imposing the requirements described above, a binned
maximum likelihood fit to the dijet invariant mass distribution
\mjj~of the two leading jets is performed.  The signal region
corresponding to the W and Z mass windows 70 $<$ \mjj $<$ 100~GeV 
is excluded from the fit. The shape of the W$\gamma$+jets $\mjj$ distribution is obtained from
simulation and the normalization of this background component is unconstrained in the fit.
The normalization of the contribution from misidentified photons is allowed to float with a
Gaussian distributed penalty term of 14\% (Section~\ref{sec:Syst}). The post-fit ratio $K =
\sigma_{\textrm{fit}}/\sigma_{\textrm{LO}}$ for the W$\gamma$+jets
background is measured to be $1.10\pm 0.07$ ($1.07 \pm 0.09$) in the muon (electron) channel.

The background from misidentified photons arises mainly from the
W+3~jets process, in which one jet passes the photon identification
criteria. The total contribution from misidentified photons is
estimated using a data control sample, where all selection criteria
except for the isolation requirement are applied. The shower shape
distribution is then used to estimate the total rate of misidentified
photons. Details on the method can be found in Ref.~\cite{Vgamma}. The
fraction of the total background from misidentified photons varies
with photon $E_{T}$ from a maximum of 23\% ($p_{T}$ = 30~GeV) to 8\%
($p_{T}>135$~GeV).

The multijet background results from misidentified leptons caused by
jets passing the muon or electron selection requirements. It is
estimated using a two component fit to the \MET distribution in
data. The procedure is described in~\cite{Chatrchyan:2012bd}, and was
repeated for the 8~\TeV data. The multijet contribution is estimated
to be 6.2\% for the electron channel, with a 50\% systematic
uncertainty, and is negligible for the muon channel.

Other background contributions arise from top-quark pair production,
single-top-quark production, and Z$\gamma$+jets. These are taken from
simulation and are fixed to their SM expectations, with the central
values and uncertainties listed in Table~\ref{tab:samples}. The
top-quark pair process contribution comes from the presence of two W
bosons in the decays. The b quark jet tagging efficiency is
approximately 70\%, with 2\% uncertainty in the scale factor. The
Z$\gamma$+jets background can mimic the signal when the Z decays
leptonically and one of the leptons is lost, resulting in
\MET. Together the top-quark pair, single-top-quark, and
Z$\gamma$+jets backgrounds represent about 8\% of the expected SM
background rate.

\section{Systematic uncertainties}
\label{sec:Syst}

The uncertainties contributing to the measured rate of misidentified
photons arise from two sources. First, the statistical uncertainty is estimated
from the W+3~jets simulation to be 5.6\% rising to 37\% with increasing photon
$E_T$. The second arises from a bias in the shower shape in the control sample due to the
inverted isolation requirements. The uncertainty is estimated to be less than 11\%. 

The uncertainty in the measured value of the 
luminosity~\cite{lumiPAS} is 2.6\% and contributes to the signal and
those backgrounds that are taken from the MC prediction. Jet energy
scale uncertainties contribute via selection thresholds on the
jet $p_{T}$ and dijet invariant mass by 4.3\%. A small difference in \MET
resolution \cite{Chatrchyan:2011tn} between data and simulation
affects the signal selection efficiency by less than 1\%. Systematic
uncertainties due to the trigger efficiency in the data (1\%) and
lepton reconstruction and selection efficiency (2\%) are also accounted
for. Photon reconstruction efficiency and energy scale uncertainties contribute to the signal selection efficiency at the 1\% level. The uncertainty
from the b jet tagging procedure is 2\% on the data/simulation
efficiency correction factor~\cite{Chatrchyan:2012jua}. This
has an effect of 11\% on the t$\overline{\mathrm t}\gamma$ background, 5\% on the single-top-quark
background, and a negligible effect on the signal. The theoretical
uncertainty in the $t\overline{t}\gamma$ and
Z$\gamma$+jets production is 20\%. The theoretical uncertainties in the
WW$\gamma$, WZ$\gamma$, and AQGC signal cross sections are evaluated
using {\sc aMC@NLO} samples. We vary the renormalization and factorization
scales each by factors of 1/2 and 2, and require $\mu_R=\mu_F$, as described in
Ref.~\cite{Frederix:2011ss}. We find that the scale-related uncertainties are
23.4\%, and that the uncertainty due to the choice of PDF is 3.6\%.
  
\section{Estimate of the standard model WV$\gamma$ cross section}
\label{sec:xsec}

For the SM WV$\gamma$ search a ``cut-and-count'' approach is adopted. The selected numbers of candidate events 
in the data are 183(139) in the muon(electron) channel. The predicted
number of background plus signal events is $194.2 \pm 11.5$($147.9 \pm 10.7$) in the muon(electron) channel, where the 
uncertainty includes statistical, systematic and luminosity related uncertainties. 
The event yield per process is summarized in Table ~\ref{tab:evt}.

\begin{table}[htb]
\centering
  \caption{Expected number of events per process. The predicted number of events for the W$\gamma$+jets and WV+jet processes, where the jet is reconstructed as 
a photon, are derived from data. The "Total prediction" represents the sum of all the individual contributions.}
\scalebox{0.92}{
  \begin{tabular}{|c|c|c|}
  \hline
  Process  & Muon channel & Electron channel  \\
    & number of events & number of events  \\
  \hline
  \hline
  SM WW$\gamma$                       & 6.6   $\pm$ 1.5  & 5.0   $\pm$ 1.1  \\
  SM WZ$\gamma$                       & 0.6   $\pm$ 0.1  & 0.5   $\pm$ 0.1  \\
  \hline
  W$\gamma$ + jets                    & 136.9 $\pm$ 10.5 & 101.6 $\pm$ 8.5  \\
  WV + jet, jet $\rightarrow \gamma$  & 33.1  $\pm$ 4.8  & 21.3  $\pm$ 3.3  \\
  MC $t\overline{t}\gamma$            & 12.5  $\pm$ 3.0  & 9.1   $\pm$ 2.2  \\
  MC single top quark                 & 2.8   $\pm$ 0.8  & 1.7   $\pm$ 0.6  \\
  MC Z$\gamma$ + jets                 & 1.7   $\pm$ 0.1  & 1.5   $\pm$ 0.1  \\
  Multijets                           &  -  & 7.2   $\pm$ 5.1  \\
  \hline
  \hline
  Total prediction           & 194.2 $\pm$ 11.5  & 147.9 $\pm$ 10.7  
\\
  \hline
  \hline
  Data                             & 183               & 139     \\
  \hline
  \end{tabular}}
  \label{tab:evt}
\end{table}

It is only possible to set a one-sided upper limit on the WW$\gamma$ and WZ$\gamma$ cross section with the amount of data
collected so far. The limit is calculated from the
event yields in Table~\ref{tab:evt} using a profile likelihood
asymptotic approximation method~(\cite{CMS-NOTE-2011-005},
Appendix A.1.3). An observed upper limit of 311~fb is calculated for the inclusive
cross section at 95\% confidence level(CL), which is about 3.4 times as large as
 the standard model prediction of 91.6 $\pm$ 21.7 ~fb (with photon $E_{T} >$~30~GeV and $|\eta|<1.44$), calculated with {\sc aMC@NLO}. The 
expected limit is calculated to be 403~fb (4.4 times the SM) 
with 1-$\sigma$(2-$\sigma$) band of [223,598]([151,839])~fb.

