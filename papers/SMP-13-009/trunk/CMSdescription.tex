\section{The CMS detector }
\label{sec:CMSdet}

The central feature of the Compact Muon Solenoid (CMS) apparatus is a
superconducting solenoid of 6\unit{m} internal diameter and 13\unit{m} length, providing a
magnetic field of 3.8\unit{T}. Within the superconducting solenoid
volume are a silicon pixel and strip tracker, a lead tungstate crystal
electromagnetic calorimeter (ECAL), and a brass/scintillator hadron
calorimeter (HCAL). Muons are reconstructed in gas-ionization detectors
embedded in the steel flux return yoke outside the solenoid. Extensive
forward calorimetry complements the coverage provided by the barrel
and endcap detectors.

The CMS experiment uses a right-handed coordinate system, with the origin at the
nominal interaction point, the $x$ axis pointing to the center of the
LHC, the $y$ axis pointing up (perpendicular to the LHC plane), and
the $z$ axis along the counterclockwise beam direction. The polar angle
$\theta$ is measured from the positive $z$ axis and the azimuthal
angle $\phi$ is measured in radians in the $x$-$y$ plane. 
The pseudorapidity $\eta$ is defined as $\eta = -\ln(\tan(\theta/2))$.

The energy resolution for photons with transverse energy ($\ET$) of $60$\GeV varies between 1.1\% and 2.6\% in the ECAL 
barrel, and from 2.2\% to 5\% in the endcaps ~\cite{Chatrchyan:2013dga}. The HCAL, when combined with the ECAL, measures jets 
with a resolution $\Delta E/E \approx 100\% / \sqrt{E\,[\GeVns]} \oplus 5\%$~\cite{CMS:2011esa}. To improve
reconstruction of jets, the tracking and calorimeter information is
combined using a particle flow (PF) reconstruction technique~\cite{PFT-09-001}. The jet energy resolution 
amounts typically to 15\% at 10\GeV, 8\% at 100\GeV, and 4\% at 1\TeV, to be 
compared to about 40\%, 12\%, and 5\% obtained when the calorimeters alone are used for jet clustering.


A more detailed description of the CMS detector can be found in Ref.~\cite{Chatrchyan:2008zzk}. 
