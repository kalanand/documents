\clearpage{}
\section{Introduction}
\label{sec:intro}

The standard model (SM) of particle physics provides a good description of the existing 
high-energy experimental data~\cite{Beringer:1900zz}. The diboson WW and WZ production cross sections have been
precisely measured at the Large Hadron Collider (LHC) and were found to be in agreement
with the SM expectation~\cite{CMSdiboson,Chatrchyan:2012bd,Chatrchyan:2013yaa,ATLAS:2012mec,Aad:2012twa}. This paper presents a 
search for three gauge boson WW$\gamma$ and WZ$\gamma$ (in what follows denoted as WV$\gamma$) production and is an extension of the 
diboson production measurement presented in Ref. \cite{Chatrchyan:2012bd}, with the additional requirement of an energetic photon 
in the final state. Previous searches of triple vector boson production, when at
least two bosons are massive, were performed at LEP~\cite{wwaLEP:1999,Achard:2001eg,Abdallah:2003xn,Heister:2004yd,Schael:2013ita}.

The structure of gauge boson self-interactions emerges
naturally in the SM from the non-Abelian $SU(2)_{L} \otimes U(1)_{Y}$ gauge
symmetry. Together with the triple WW$\gamma$ and
WWZ gauge boson vertices, the SM also predicts the existence of the quartic
WWWW, WWZZ, WWZ$\gamma$, and WW$\gamma\gamma$ vertices.  The direct
investigation of gauge boson self-interactions provides a crucial test
of the gauge structure of the SM and is expected to play a significant 
role at LHC energies~\cite{Belyaev:1998ih}.  

The study of gauge boson self-interactions may provide evidence of the
existence of new phenomena at a higher energy
scale~\cite{aihara1996,Du:2012vh,Fichet:2013ola,Giudice:2007fh}.
Possible new physics, expressed in a model independent
way by higher-dimensional effective operators~\cite{500GeVNLC,
Belanger:1999, Bosonic:2004PRD, Eboli:2006wa,Yang:2012vv,Ye:2013psa}, leads to anomalous
triple and quartic gauge couplings (AQGC). Both types of vertices are
included in triple gauge boson production processes. The production of three gauge bosons is more sensitive to the quartic 
gauge coupling than the diboson channel, which is more suited to measuring the triple gauge coupling. 
Any deviation of the observed coupling from the SM prediction would indicate new physics.
Such a deviation would typically manifest itself in an enhanced production cross section, as well as
changing the shape of the kinematic distributions of the  WV$\gamma$ system.
A stringent limit on the anomalous WW$\gamma\gamma$ quartic coupling was recently obtained
via the exclusive two-photon production of W$^+$W$^-$ ~\cite{Chatrchyan:2013foa}.

This paper presents a study of WV$\gamma$ production
in the semileptonic final state, which includes W($\to\ell\nu$)W($\to
jj$)$\gamma$ and W($\to\ell\nu$)Z($\to jj$)$\gamma$ processes, with $\ell=e,\mu$.  This
decay mode is chosen due to the higher branching fraction compared to
the fully leptonic mode. One characteristic of this decay mode is that
the two production processes WW$\gamma$ and WZ$\gamma$ cannot be
clearly differentiated, due to the detector dijet mass resolution
($\sigma\sim 10$\%), which is comparable to the mass difference between
the W and Z bosons. Since the WW$\gamma$ and WZ$\gamma$ processes share the same dominant background 
(W($\to\ell\nu$)$\gamma$+jets where the dijet invariant mass falls in the W/Z mass window) they are treated as a single 
combined signal in the analysis.

The next section provides a brief description of the theoretical
framework selected for this analysis. Section~\ref{sec:CMSdet}
describes the CMS detector, and section~\ref{sec:mc} the selection of
event generators and simulators used to create simulation
samples. Section~\ref{sec:recosel} describes how data events are
selected and how physics objects within those events are
reconstructed. Section~\ref{sec:BackgroundModeling} details how various
backgrounds are modeled, including the use of a dijet invariant mass
fit to extract the main background. Section~\ref{sec:Syst} lists the
systematic uncertainties considered in the
analysis. Section~\ref{sec:xsec} describes the ``cut-and-count''
WV$\gamma$ cross section measurement. Section~\ref{sec:limits_pT}
describes the use of the photon $E_T$ distribution to set limits on
anomalous couplings. Section~\ref{sec:summary} summarizes the results
of our investigation.
