\section{Jet Reconstruction}

\label{sec:jets}

Three main types of jets are reconstructed at CMS,  which differently combine
individual contributions from subdetectors to form the inputs
to the jet clustering algorithm: calorimeter jets, 
Jet Plus Track (JPT) jets and  Particle Flow (PF) jets.
Jets in the studies presented here are reconstructed using the anti-kt~\cite{bib:akt}
clustering algorithm with the size parameter $R=0.5$.
To evaluate their performance, in Monte Carlo simulations 
"generator jets" (GenJets)
are reconstructed as well by applying the same jet clustering algorithm to all stable 
($c\tau > 1$ cm) generated particles ("MC truth"). 
These generator jets are then associated to jets
reconstructed from the simulated detector signals, 
by requiring a small distance 
$\Delta R = \sqrt{\Delta\eta^2+\Delta\phi^2}$
between the jet axes 
in $\eta-\phi$ space.

{\bf Calorimeter} (Calo) jets 
are reconstructed using energy deposits in the electromagnetic (ECAL) 
and hadronic (HCAL)
calorimeter cells, combined into calorimeter towers. 
%A calorimeter tower 
%consists
%of one or more HCAL cells and the geometrically corresponding ECAL crystals. In the barrel
%region of the calorimeters ($|\eta| < 1.4$), the unweighted sum of one single HCAL cell and
%5x5 ECAL crystals form a projective calorimeter tower. The association 
%between HCAL cells and ECAL crystals is more complex in the endcap regions of the
%electromagnetic calorimeter (1.4 $< |\eta| < $3.0).  
%In the forward region (3.0 $< |\eta| < $5.0) a different calorimeter technology is
%employed, using the Cerenkov light signals collected by short and long quartz-fiber
%readouts to aid the separation of electromagnetic and hadronic signals.
%In order to reduce the
%contribution from calorimeter readout electronics 
%noise, thresholds are applied on energies of individual cells when building towers
%for reconstruction of jets and missing transverse energy ($E_T^{miss}$). These 
%thresholds are listed in 
%Table~\ref{tab:Threshold_Scheme6}.
%In addition, to reduce contribution from event pile-up (additional proton-proton
%interactions in the same bunch crossing), 
%calorimeter towers with transverse energy of 
%$E_T^{towers}<0.3$~GeV are not used in jet 
%reconstruction.
%\begin{table}[h]
%  \centering
%  \begin{tabular}{ |c| c |}
%    \hline
%    Section  & Threshold (in GeV)	\\ \hline 
%    HB  & 0.7	 \\ \hline 
%    HE   & 0.8	 \\ \hline 
%    HO  & 1.1/3.5    (Ring 0/Ring 1,2)	 \\ \hline 
%    HF (Long)  & 0.5	 \\ \hline 
%    HF (Short) & 0.85	 \\ \hline 
%    EB  & 0.07 (per crystal, double sided)	 \\ \hline 
%    EE & 0.3 (per crystal, double sided)	 \\ \hline 
%    EB Sum & 0.2	 \\ \hline 
%    EE Sum & 0.45	 \\ \hline
%  \end{tabular}
%  \caption{Calorimeter offline cell thresholds used in 
%calorimeter and JPT jet reconstruction.  The "Section" label refers to the 
%calorimeter subsystems~\cite{bib:cms}: Hadronic Barrel
%(HB), Endcap (HE), Outer (HO) and Forward (HF), and ECAL Barrel (EB)
%and Endcap (EE). Independent thresholds are placed in different
%sections (rings) of HO, and in different (long and short fiber) readouts in HF.
%In ECAL, in addition to energy thresholds on readouts from individual 
%crystals, thresholds are applied on the sum of crystal readouts 
%corresponding to the same tower. 
%  \label{tab:Threshold_Scheme6}}
%\end{table}


The {\bf Jet Plus Tracks} (JPT) algorithm~\cite{bib:PAS_jpt} exploits the excellent performance 
of the CMS tracking detectors to improve the \pt response and resolution of calorimeter 
jets. Calorimeter jets are  reconstructed first as described above, then charged 
particle tracks are associated with each jet based on spatial separation in $\eta- \phi$
between the jet axis and the track momentum measured at the interaction vertex. 
%The associated tracks are projected onto the surface of the calorimeter and 
%classified as in-cone tracks if they point to within the jet cone around the 
%jet axis on the calorimeter surface. If the 3.8 T magnetic field of CMS has 
%instead bent the track out of the jet cone,
%it is classified as a out-of-cone track. 
The momenta of charged tracks  are then used to improve the determination of the  energy 
and direction of the associated 
calorimeter jet. 
%For in-cone tracks the expected average energy deposition in 
%the calorimeters is subtracted based on the momentum of the track.
%The direction of the axis 
%of the original calorimeter jet is also corrected by the algorithm.

The {\bf Particle Flow} (PF) algorithm combines the information
from all CMS sub-detectors to identify and reconstruct all particles in the event, 
namely muons, electrons, photons, charged hadrons
and neutral hadrons. The detailed description of the algorithm
and its performance can be found in References 
~\cite{bib:PAS_pflow09,bib:PAS_pflow}.
Charged hadrons, in particular, are 
reconstructed from tracks in the central tracker. Photons 
and neutral hadrons are reconstructed from energy clusters in the electromagnetic and 
hadron calorimeters. Clusters separated from the extrapolated position of tracks in the 
calorimeters constitute a clear signature of these neutral particles. A neutral particle 
overlapping with charged particles in the calorimeters can be detected as a calorimeter energy excess 
with respect to the sum of the associated track momenta.  
PF jets are then reconstructed from the resulting list of particles.
The jet momentum and spatial resolutions are 
improved with respect to calorimeter jets
as the use of the tracking detectors and of the excellent granularity of the ECAL 
allows to resolve and precisely measure charged hadrons and photons inside jets,
which constitute $\sim$90\% of the jet energy.  
