% svn info. These are modified by svn at checkout time.
% The last version of these macros found before the maketitle will be the one on the front page,
% so only the main file is tracked.
% Do not edit by hand!
\RCS$Revision: 79369 $
\RCS$HeadURL: svn+ssh://svn.cern.ch/reps/tdr2/papers/JME-10-011/trunk/JME-10-011.tex $
\RCS$Id: JME-10-011.tex 79369 2011-09-22 15:21:47Z alverson $
%%%%%%%%%%%%% ptdr definitions %%%%%%%%%%%%%%%%%%%%%
%\input{ptdr-definitions}
% removed some \mathrm, not completely
%DEFINITION OF JET TYPES
\newcommand{\calojets}{CaloJets}
\newcommand{\jptjets}{JPTJets}
\newcommand{\pfjets}{PFJets}

\newcommand{\ptave}{\ensuremath{p_{{T}}^{{ave}}}\xspace}
\newcommand{\sigmaA}{\ensuremath{\sigma_A}\xspace}
\newcommand{\ptfirst}{\ensuremath{p_{{T}}^{{Jet}1}}\xspace}
\newcommand{\ptthird}{\ensuremath{p_{{T}}^{{Jet}3}}\xspace}
\newcommand{\ptthirdmax}{\ensuremath{p_{{T,max}}^{{Jet}3}}\xspace}
\newcommand{\ptthirdraw}{\ensuremath{p_{{T,raw}}^{{Jet}3}}\xspace}
\newcommand{\ptthirdcorr}{\ensuremath{p_{{T,cor}}^{{Jet}3}}\xspace}
%\newcommand{\ptthirdpara}{\ensuremath{p^{{rel}}_{Jet3,\parallel}}\xspace}
\newcommand{\ptthirdpara}{\ensuremath{p_{{T,rel}}^{Jet3,\parallel}}\xspace}
\newcommand{\ptrelthree}{\ensuremath{\ptthird/\ptave}\xspace}
%\newcommand{\ptrelthree}{\ensuremath{p_{T}^{Jet3}/p_{T}^{ave}}\xspace}

\newcommand{\ptsecond}{\ensuremath{p_{{T}}^{{Jet}2}}\xspace}
\newcommand{\ptsecondmax}{\ensuremath{p_{{T,max}}^{{Jet}2}}\xspace}


%\newcommand{\kt}{$k_\mathrm{T}$\xspace}
\newcommand{\antikt}{Anti-$k_{T}$\xspace}
\newcommand{\aachen}{Cambridge/Aachen\xspace}
\newcommand{\siscone}{SISCone\xspace}
\newcommand{\itcone}{Iterative Cone\xspace}

\newcommand{\Rsoft}{\ensuremath{R_{\mathrm{soft}}}\xspace}
\newcommand{\ksoft}{\ensuremath{\mathrm{k}_{\mathrm{soft}}}\xspace}
\newcommand{\kpli}{\ensuremath{\mathrm{k}_{\mathrm{pli}}}\xspace}


\newcommand{\nhits}{\ensuremath{n^{90}_{\mathrm{hits}}}\xspace}
\newcommand{\fhpd}{\ensuremath{f_{\mathrm{HPD}}}\xspace}


\newcommand{\DeltaR}{\ensuremath{\Delta R}\xspace}
\newcommand{\DeltaRMax}{\ensuremath{\Delta R_{\mathrm{max}}}\xspace}
\newcommand{\Rch}{\ensuremath{R_{\mathrm{ch}}}\xspace}
\renewcommand{\pt}{\ensuremath{p_{{T}}}\xspace}
\newcommand{\ptmin}{\ensuremath{p^{{min}}_{{T}}}\xspace}
\newcommand{\ptcorr}{\ensuremath{p_{\mathrm{T}}^{\mathrm{corr}}}\xspace}
\newcommand{\pthat}{\ensuremath{\hat{p}_{\mathrm{T}}}\xspace}
\newcommand{\ptgen}{\ensuremath{p_{\mathrm{T}}^{\mathrm{gen}}}\xspace}
\newcommand{\ptref}{\ensuremath{p_{\mathrm{T}}^{\mathrm{REF}}}\xspace}
\newcommand{\etagen}{\ensuremath{\eta^{\mathrm{gen}}}\xspace}
\newcommand{\etamax}{\ensuremath{\eta^{\mathrm{max}}}\xspace}
\newcommand{\phigen}{\ensuremath{\varphi^{\mathrm{gen}}}\xspace}
\newcommand{\dphijj}{\ensuremath{\left|\Delta\varphi(\mathrm{j}_1\mathrm{j}_2)\right|}\xspace}

\def\eslash{\ensuremath{{\hbox{$E$\kern-0.6em\lower-.05ex\hbox{/}\kern0.10em}}}}
\def\vecmet{\mbox{$\vec{\eslash}_T$}\xspace} %missing ET vector
\def\MET{\mbox{$\eslash_T$}\xspace}
\def\met{\mbox{$\eslash_T$}\xspace} %missing ET, no space
\def\mex{\mbox{$\eslash_x$}\xspace} %missing Ex
\def\mey{\mbox{$\eslash_y$}\xspace} %missing Ey

%\newcommand{\MeV}{\,\ensuremath{\mathrm{Me\kern-0.1em V}}\xspace}
%\newcommand{\GeV}{\,\ensuremath{\mathrm{Ge\kern-0.1em V}}\xspace}
%\newcommand{\gev}{\,\ensuremath{\mathrm{Ge\kern-0.1em V}}\xspace}
%\newcommand{\TeV}{\,\ensuremath{\mathrm{Te\kern-0.1em V}}\xspace}

\newcommand{\etabarrel}{\,\ensuremath{0<|\eta|\leq1.4}\xspace}
\newcommand{\etaendcaps}{\,\ensuremath{1.4<|\eta|\leq2.6}\xspace}
\newcommand{\etatransition}{\,\ensuremath{2.6<|\eta|\leq3.2}\xspace}
\newcommand{\etaforward}{\,\ensuremath{3.2<|\eta|\leq4.7}\xspace}

%\newcommand{\cm}{\,\mathrm{cm}\xspace}

\newcommand{\ipb}{\,pb^{-1}\xspace}
\newcommand{\ifb}{\,fb^{-1}\xspace}



\newcommand{\fasym}{\ensuremath{f_{\text{Asym}}}\xspace}
\newcommand{\fasymdata}{\ensuremath{f^{\text{Data}}_{\text{Asym}}}\xspace}
\newcommand{\fasymmc}{\ensuremath{f^{\text{MC}}_{\text{Asym}}}\xspace}
\newcommand{\fresp}{\ensuremath{f_{\text{Resp}}}\xspace}


%%%%%%%%%%%%%%%  Title page %%%%%%%%%%%%%%%%%%%%%%%%
\cmsNoteHeader{JME-10-011} % This is over-written in the CMS environment: useful as preprint no. for export versions
\title{Determination of Jet Energy Calibration and Transverse Momentum Resolution in CMS}
\date{\today}
\abstract{
Measurements of the jet energy calibration and transverse momentum resolution in CMS are presented, performed with a data sample collected in proton-proton collisions at a centre-of-mass energy of 7\TeV, corresponding to an integrated luminosity of $36\pbinv$. The transverse momentum balance in dijet and $\gamma$/Z+jets events is used to measure the jet energy response in the CMS detector, as well as the transverse momentum resolution. The results are presented for three different methods to reconstruct jets: a calorimeter-based approach, the ``Jet-Plus-Track" approach, which improves the measurement of calorimeter jets by exploiting the associated tracks, and the ``Particle Flow" approach, which attempts to reconstruct individually each particle in the event, prior to the jet clustering, based on information from all relevant subdetectors.
}

\hypersetup{%
pdfauthor={Konstantinos Kousouris, Niki Saoulidou},%
pdftitle={Determination of Jet Energy Calibration and Transverse Momentum Resolution in CMS},%
pdfsubject={CMS},%
pdfkeywords={CMS, jets, JetMET, energy scale, energy resolution}}

\maketitle

\clearpage

\section{Introduction}

Jets are the experimental signatures of quarks and gluons produced in high-energy processes such as hard scattering of partons in proton-proton collisions. The detailed understanding of both the jet energy scale and of the transverse momentum resolution is of crucial importance for many physics analyses, and it is an important component of the systematic uncertainty. This paper presents studies for the determination of the energy scale and resolution of jets, performed with the Compact Muon Solenoid (CMS) at the CERN Large Hadron Collider (LHC), on proton-proton collisions at $\sqrt{s}=7\TeV$, using a data sample corresponding to an integrated luminosity of $36\pbinv$.

The paper is organized as follows: Section~\ref{sec:detector} describes briefly the CMS detector, while Section~\ref{sec:jets} describes the jet reconstruction methods considered here. Sections~\ref{sec:data} and~\ref{sec:methods} present the data samples and the experimental techniques used for the various measurements. The jet energy calibration scheme is discussed in Section~\ref{sec:jec} and the jet transverse momentum resolution is presented in Section~\ref{sec:res}. 

\section{The CMS Detector}\label{sec:detector}

A detailed description of the CMS detector can be found elsewhere~\cite{CMS}. A right-handed coordinate system is used with the origin at the nominal interaction point (IP). The x-axis points to the center of the LHC ring, the y-axis is vertical and points upward, and the z-axis is parallel to the counterclockwise beam direction. The azimuthal angle $\phi$ is measured with respect to the x-axis in the xy-plane and the polar angle $\theta$ is defined with respect to the z-axis, while the pseudorapidity is defined as $\eta=-\ln\left[\tan\left(\theta/2\right)\right]$. The central feature of the CMS apparatus is a superconducting solenoid, of 6\,m internal diameter, that produces a magnetic field of 3.8\,T. Within the field volume are the silicon pixel and strip tracker and the barrel and endcap calorimeters ($|\eta| < 3$), composed of a crystal electromagnetic calorimeter (ECAL) and a brass/scintillator hadronic calorimeter (HCAL). Outside the field volume, in the forward region ($3 < |\eta| < 5$), there is an iron/quartz-fibre hadronic calorimeter. The steel return yoke outside the solenoid is instrumented with gaseous detectors used to identify muons. The CMS experiment collects data using a two-level trigger system, the first-level hardware trigger (L1)~\cite{PTDRI} and the high-level software trigger (HLT)~\cite{HLT}. 

\section{Jet Reconstruction}\label{sec:jets}

Jets considered in this paper are reconstructed using the anti-$k_T$ clustering algorithm~\cite{AKT} with a size parameter $R=0.5$ in the $y-\phi$ space, implemented in the \textit{FastJet} package~\cite{FASTJETI,FASTJETII}. In some cases, jets with a size parameter $R=0.7$ are also considered. The clustering is performed by four-momentum summation. The rapidity $y$ and the transverse momentum \pt of a jet with energy $E$ and momentum $\vec{p}=(p_x,p_y,p_z)$ are defined as $y=\frac{1}{2}\ln\left(\frac{E+p_z}{E-p_z}\right)$ and $\pt=\sqrt{p_x^2+p_y^2}$ respectively. The inputs to the clustering algorithm are the four-momentum vectors of detector energy deposits or of particles in the Monte Carlo (MC) simulations. Detector jets belong to three types, depending on the way the individual contributions from subdetectors are combined: Calorimeter jets, Jet-Plus-Track jets and Particle-Flow jets.  

{\bf Calorimeter (CALO) jets} are reconstructed from energy deposits in the calorimeter towers. A calorimeter tower consists of one or more HCAL cells and the geometrically corresponding ECAL crystals. In the barrel region of the calorimeters, the unweighted sum of one single HCAL cell and 5x5 ECAL crystals form a projective calorimeter tower. The association between HCAL cells and ECAL crystals is more complex in the endcap regions. In the forward region, a different calorimeter technology is employed, using the Cerenkov light signals collected by short and long quartz readout fibers to aid the separation of electromagnetic and hadronic signals. A four-momentum is associated to each tower deposit above a certain threshold, assuming zero mass, and taking as a direction the tower position as seen from the interaction point.

{\bf Jet-Plus-Track (JPT) jets} are reconstructed calorimeter jets whose energy response and resolution are improved by incorporating tracking information, according to the Jet-Plus-Track algorithm~\cite{JME-09-002}. Calorimeter jets are first reconstructed as described above, and then charged particle tracks are associated with each jet, based on the spatial separation between the jet axis and the track momentum vector, measured at the interaction vertex, in the $\eta-\phi$ space. The associated tracks are projected onto the front surface of the calorimeter and are classified as \textit{in-cone} tracks if they point to within the jet cone around the jet axis on the calorimeter surface. The tracks that are bent out of the jet cone because of the CMS magnetic field are classified as \textit{out-of-cone} tracks. The momenta of charged tracks are then used to improve the measurement of the energy of the associated calorimeter jet: for \textit{in-cone} tracks, the expected average energy deposition in the calorimeters is subtracted and the momentum of the tracks is added to the jet energy. For \textit{out-of-cone} tracks the momentum is added directly to the jet energy. The Jet-Plus-Track algorithm corrects both the energy and the direction of the axis of the original calorimeter jet.

The {\bf Particle-Flow (PF) jets} are reconstructed by clustering the four-momentum vectors of particle-flow candidates. The particle-flow algorithm~\cite{PFT-09-001,PFT-10-002} combines the information from all relevant CMS sub-detectors to identify and reconstruct all visible particles in the event, namely muons, electrons, photons, charged hadrons, and neutral hadrons. Charged hadrons, electrons and muons are reconstructed from tracks in the tracker. Photons and neutral hadrons are reconstructed from energy clusters separated from the extrapolated positions of tracks in ECAL and HCAL, respectively. A neutral particle overlapping with charged particles in the calorimeters is identified as a calorimeter energy excess with respect to the sum of the associated track momenta. The energy of photons is directly obtained from the ECAL measurement, corrected for zero-suppression effects. The energy of electrons is determined from a combination of the track momentum at the main interaction vertex, the corresponding ECAL cluster energy, and the energy sum of all bremsstrahlung photons associated with the track. The energy of muons is obtained from the corresponding track momentum. The energy of charged hadrons is determined from a combination of the track momentum and the corresponding ECAL and HCAL energy, corrected for zero-suppression effects, and calibrated for the non-linear response of the calorimeters. Finally, the energy of neutral hadrons is obtained from the corresponding calibrated ECAL and HCAL energy. The PF jet momentum and spatial resolutions are greatly improved with respect to calorimeter jets, as the use of the tracking detectors and of the high granularity of ECAL allows resolution and measurement of charged hadrons and photons inside a jet, which together constitute $\sim$85\% of the jet energy. 

The {\bf Monte Carlo particle jets} are reconstructed by clustering the four-momentum vectors of all stable ($c\tau > 1$ cm) particles generated in the simulation. In particular, there are two types of MC particle jets: those where the neutrinos are excluded from the clustering, and those where both the neutrinos and the muons are excluded. The former are used for the study of the PF and JPT jet response in the simulation, while the latter are used for the study of the CALO jet response (because muons are minimum ionizing particles and therefore do not contribute appreciably to the CALO jet reconstruction).

The {\bf Particle-Flow missing transverse energy} ($\vecmet$), which is needed for the absolute jet energy response measurement, is reconstructed from the particle-flow candidates and is defined as $\vecmet=-\displaystyle\sum_{i}{\left(E_i\sin\theta_i\cos\phi_i\hat{\mathbf{x}}+E_i\sin\theta_i\sin\phi_i\hat{\mathbf{y}}\right)}=\mex\hat{\mathbf{x}}+\mey\hat{\mathbf{y}}$, where the sum refers to all candidates and $\hat{\mathbf{x}},\hat{\mathbf{y}}$ are the unit vectors in the direction of the x and y axes.


\section{Event Samples and Selection Criteria}\label{sec:data}

In this Section, the data samples used for the various measurements are defined. In all samples described below, basic common event preselection criteria are applied in order to ensure that the triggered events do come from real proton-proton interactions. First, the presence of at least one well-reconstructed primary vertex (PV) is required, with at least four tracks considered in the vertex fit, and with $|\text{z}(\mathrm{PV})|<24\cm$, where $\text{z}(\mathrm{PV})$ represents the position of the proton-proton collision along the beams. In addition, the radial position of the primary vertex, $\rho(\mathrm{PV})$, has to satisfy the condition $\rho(\mathrm{PV})<2\cm$.

Jet quality criteria (``Jet ID'') have been developed for CALO jets~\cite{JME-09-008} and PF jets~\cite{JME-10-003}, which are found to retain the vast majority ($>99\%$) of genuine jets in the simulation, while rejecting most of the misidentified jets arising from calorimeter and/or readout electronics noise in pure noise non-collision data samples: such as cosmic-ray trigger data or data from triggers on empty bunches during LHC operation. Jets used in the analysis are required to satisfy proper identification criteria.

\subsection{Zero Bias and Minimum Bias Samples}

The zero bias and minimum bias samples are used for the measurement of the energy clustered inside a jet due to noise and additional proton-proton collisions in the same bunch crossing (pile-up, or PU), as described in Section~\ref{sec:offset}. The zero bias sample is collected using a random trigger in the presence of a beam crossing. The minimum bias sample is collected by requiring coincidental hits in the beam scintillating counter~\cite{HLT} on either side of the CMS detector. 

\subsection{Dijet Sample}
\label{sec:jjsample}
The dijet sample is composed of events with at least two reconstructed jets in the final state and is used for the measurement of the relative jet energy scale and of the jet \pt resolution. This sample is collected using dedicated high-level triggers which accept the events based on the value of the average uncorrected \pt (\pt not corrected for the non-uniform response of the calorimeter) of the two CALO jets with the highest \pt (leading jets) in the event. The selected dijet sample covers the average jet $\pt$ range from $15\GeV$ up to around $1\TeV$.

 
\subsection{$\gamma+$jets Sample}

The $\gamma+$jets sample is used for the measurement of the absolute jet energy response and of the jet \pt resolution. This sample is collected with single-photon triggers that accept an event if at least one reconstructed photon has $\pt>15\GeV$. Offline, photons are required to have transverse momentum $\pt^{\gamma}>15\GeV$ and $|\eta|<1.3$. The jets used in the $\gamma+$jets sample  are required to lie in the $|\eta|<1.3$ region. The $\gamma+$jets sample is dominated by dijet background, where a jet mimics the photon. To suppress this background, the following additional photon isolation and shower-shape requirements~\cite{EGM-10-005} are applied:

\begin{itemize}

\item
\textbf{HCAL isolation}: the energy deposited in the HCAL within a cone of radius $R=0.4$ in the $\eta-\phi$ space, around the photon direction, must be smaller than $2.4\GeV$ or less than $5\%$ of the photon energy ($E_{\gamma}$);

\item
\textbf{ECAL isolation}: the energy deposited in the ECAL within a cone of radius $R=0.4$ in the $\eta-\phi$ space, around the photon direction, excluding the energy associated with the photon, must be smaller than $3\GeV$ or less than $5\%$ of the photon energy;

\item
\textbf{Tracker isolation}: the number of tracks in a cone of radius $R=0.35$ in the $\eta-\phi$ space, around the photon direction, must be less than three, and the total transverse momentum of the tracks must be less than $10\%$ of the photon transverse momentum;

\item
\textbf{Shower shape}: the photon cluster major and minor must be in the range of 0.15-0.35, and 0.15-0.3, respectively. Cluster major and minor are defined as second moments of the energy distribution along the direction of the maximum and minimum spread of the ECAL cluster in the $\eta-\phi$ space;

\end{itemize}

The selected $\gamma+$jets sample covers the $\pt^{\gamma}$ range from $15\GeV$ up to around $400\GeV$.

\subsection{$Z(\mu^+\mu^-)$+jets Sample}

The $Z(\mu^+\mu^-)$+jets sample is used for the measurement of the absolute jet energy response. It is collected using single-muon triggers with various \pt thresholds. Offline, the events are required to have at least two opposite-sign reconstructed global muons with $\pt>15\GeV$ and $|\eta^\mu|<2.3$ and at least one jet with $|\eta|<1.3$. A global muon is reconstructed by a combined fit to the muon system hits and tracker hits, seeded by a track found in the muon systems only. The reconstructed muons must satisfy identification and isolation requirements, as described in Ref.~\cite{EWK-10-002}. Furthermore, the invariant mass $M_{\mu\mu}$ of the two muons must satisfy the condition $70<M_{\mu\mu}<110\GeV$. Finally, the reconstructed Z is required to be back-to-back in the transverse plane with respect to the jet with the highest \pt: $|\Delta\phi(Z,jet)|>2.8 rad$. 

\subsection{$Z(e^+e^-)$+jets Sample}

The $Z(e^+e^-)$+jets sample is used for the measurement of the absolute jet energy response. It is collected using single-electron triggers with various \pt thresholds. Offline, the events are required to have at least two opposite-sign reconstructed electrons with $\pt>20\GeV$ in the fiducial region $|\eta|<1.44$ and $1.57<|\eta|<2.5$ and at least one jet with $|\eta|<1.3$. The reconstructed electrons must satisfy identification and isolation requirements, as described in Ref.~\cite{EWK-10-002}. Furthermore, the invariant mass $M_{ee}$ of the electron-positron pair must satisfy the condition $85<M_{ee}<100\GeV$. Finally, the reconstructed Z is required to be back-to-back in the transverse plane with respect to the jet with the highest \pt: $|\Delta\phi(Z,jet)|>2.7 rad$. 







\section{Experimental Techniques}\label{sec:methods}

%------------------------------- Dijet balance -------------------------

\subsection{Dijet \pt-Balancing}

The dijet \pt-balancing method is used for the measurement of the relative jet energy response as a function of $\eta$. It is also used for the measurement of the jet \pt resolution. The technique was introduced at the CERN p$\bar{\text{p}}$ collider (SP$\bar{\text{P}}$S)~\cite{spps} and later refined by the Tevatron experiments~\cite{jes_d0, jes_cdf}. The method is based on transverse momentum conservation and utilizes the \pt-balance in dijet events, back-to-back in azimuth. 

For the measurement of the relative jet energy response, one jet (barrel jet) is required to lie in the central region of the detector ($|\eta|<1.3$) and the other jet (probe jet) at arbitrary $\eta$. The central region is chosen as a reference because of the uniformity of the detector, the small variation of the jet energy response, and because it provides the highest jet \pt-reach. It is also the easiest region to calibrate in absolute terms, using $\gamma$+jet and Z+jet events. The dijet calibration sample is collected as described in Section~\ref{sec:jjsample}. Offline, events are required to contain at least two jets. The two leading jets in the event must be  azimuthally separated  by $\Delta \phi > 2.7 rad$, and one of them must lie in the $|\eta|<1.3$ region.  

The balance quantity $\mathcal{B}$ is defined as:

\begin{equation}
\mathcal{B}=\frac{\pt^{probe}-\pt^{barrel}}{\ptave},
\end{equation}

where \ptave is the average \pt of the two leading jets:

\begin{equation}
  \ptave = \frac{\pt^{barrel}+\pt^{probe}}{2}.
\end{equation}

The balance is recorded in bins of $\eta^{probe}$ and \ptave. In order to avoid a trigger bias, each \ptave bin is populated by events satisfying the conditions of the fully efficient trigger with the highest threshold.

The average value of the $\mathcal{B}$ distribution, $\langle \mathcal{B}\rangle$, in a given $\eta^{probe}$ and \ptave bin, is used to determine the relative response $\mathcal{R_\text{rel}}$: 

\begin{equation}
\mathcal{R_\text{rel}}(\eta^{probe},\ptave)=\frac{2 + \langle \mathcal{B}\rangle}{2 - \langle \mathcal{B}\rangle}.
\end{equation}

The variable $\mathcal{R_\text{rel}}$ defined above is mathematically equivalent to $\langle\pt^{probe}\rangle/\langle\pt^{barrel}\rangle$ for narrow bins of \ptave. The choice of \ptave minimizes the resolution-bias effect (as opposed to binning in $\pt^{barrel}$, which leads to maximum bias) as discussed in Section~\ref{sec:resbias} below. 

A slightly modified version of the dijet \pt-balance method is applied for the measurement of the jet \pt resolution. The use of dijet events for the measurement of the jet \pt resolution was introduced by the D0 experiment at the Tevatron~\cite{d0-asymmetry} while a feasibility study at CMS was presented using simulated events~\cite{JME-09-007}.

In events with at least two jets, the asymmetry variable $\mathcal{A}$ is defined as:

\begin{equation}
\mathcal{A} = \frac{\ptfirst - \ptsecond}{\ptfirst + \ptsecond},
\end{equation}

where \ptfirst and \ptsecond refer to the randomly ordered transverse momenta of the two leading jets. The variance of the asymmetry variable $\sigma_\mathcal{A}$ can be formally expressed as:

\begin{equation}
  \sigma_\mathcal{A}^2 = \left|\frac{\partial\mathcal{A}}{\partial\ptfirst}\right|^2\cdot\sigma^2(\ptfirst) +
  \left|\frac{\partial\mathcal{A}}{\partial\ptsecond}\right|^2\cdot\sigma^2(\ptsecond).
\end{equation}

If the two jets lie in the same $\eta$ region, $\pt\equiv\langle\ptfirst\rangle = \langle\ptsecond\rangle$ and $\sigma(\pt)\equiv\sigma(\ptfirst) = \sigma(\ptsecond)$. The fractional jet \pt resolution is calculated to be:

\begin{equation}
  \frac{\sigma(\pt)}{\pt} = \sqrt{2}\,\sigma_\mathcal{A}.
\end{equation}

The fractional jet \pt resolution in the above expression is an estimator of the true resolution, in the limiting case of no extra jet activity in the event that spoil the \pt balance of the two leading jets. The distribution of the variable $\mathcal{A}$ is recorded in bins of the average \pt of the two leading jets, $\ptave=\left(\ptfirst+\ptsecond\right)/2$, and its variance is proportional to the relative jet \pt resolution, as described above.

%------------------------------- pt balance ----------------------------

\subsection{$\gamma$/Z+jet \pt-Balancing}

The $\gamma$/Z+jet \pt-balancing method is used for the measurement of the jet energy response and the jet \pt resolution with respect to a reference object, which can be a $\gamma$ or a Z boson. The \pt resolution of the reference object is typically much better than the jet resolution and the absolute response $R_{abs}$ is expressed as: 

\begin{equation}
  R_{abs} = \frac{\pt^{jet}}{\pt^{\gamma,Z}}.
\end{equation}

The absolute response variable is recorded in bins of $\pt^{\gamma,Z}$. It should be noted that, because of the much worse jet \pt resolution, compared to the $\gamma$ or Z \pt resolution, the method is not affected by the resolution bias effect (see Section~\ref{sec:resbias}), as it happens in the dijet \pt-balancing method. Also, for the same reason, the absolute response can be defined as above, without the need of more complicated observables, such as the balance $\mathcal{B}$ or the asymmetry $\mathcal{A}$.

%------------------------------- MPF -----------------------------------

\subsection{Missing Transverse Energy Projection Fraction}

The missing transverse energy projection fraction (MPF) method (extensively used at the Tevatron~\cite{jes_d0}) is based on the fact that the $\gamma,Z$+jets events have no intrinsic \vecmet and that, at parton level, the $\gamma$ or Z is perfectly balanced by the hadronic recoil in the transverse plane:

\begin{equation}
\vec{\pt}^{\gamma,Z} + \vec{\pt}^{recoil} = 0.
\end{equation}

For reconstructed objects, this equation can be re-written as:

\begin{equation}
R_{\gamma,Z}\vec{\pt}^{\gamma,Z} + R_{recoil}\vec{\pt}^{recoil} = -\vecmet,
\end{equation}

where $R_{\gamma,Z}$ and $R_{recoil}$ are the detector responses to the $\gamma$ or Z and the hadronic recoil, respectively. 

Solving the two above equations for $R_{recoil}$ gives:

\begin{equation}
R_{recoil}= R_{\gamma,Z} +\frac{\vecmet \cdot \vec{\pt}^{\gamma,Z}}{(\pt^{\gamma,Z})^2}\equiv R_{MPF}.
\end{equation}

This equation forms the definition of the MPF response $R_{MPF}$. The additional step needed is to extract the jet energy response from the measured MPF response. In general, the recoil consists of additional jets, beyond the leading one, soft particles and unclustered energy. The relation $R_{leadjet}=R_{recoil}$ holds to a good approximation if the particles, that are not clustered into the leading jet, have a response similar to the ones inside the jet, or if these particles are in a direction perpendicular to the photon axis. Small response differences are irrelevant if most of the recoil is clustered into the leading jet. This is ensured by vetoing secondary jets in the selected back-to-back $\gamma,Z$+jets events.

The MPF method is less sensitive to various systematic biases compared to the $\gamma,Z$ \pt-balancing method and is used in CMS as the main method to measure the jet energy response, while the $\gamma,Z$ \pt-balancing is used to facilitate a better understanding of various systematic uncertainties and to perform cross-checks.

\subsection{Biases}

All the methods based on data are affected by inherent biases related to detector effects (e.g. \pt resolution) and to the physics properties (e.g. steeply falling jet \pt spectrum). In this Section, the two most important biases related to the jet energy scale and to the \pt resolution measurements are discussed: the resolution bias and the radiation imbalance.

\subsubsection{Resolution Bias}\label{sec:resbias}

The measurement of the jet energy response is always performed by comparison to a reference object. Typically, the object with the best resolution is chosen as a reference object, as in the $\gamma$/Z+jet balancing where the $\gamma$ and the Z objects have much better \pt resolution than the jets. However, in other cases, such as the dijet \pt-balancing, the two objects have comparable resolutions. When such a situation occurs, the measured relative response is biased in favor of the object with the worse resolution. This happens because a reconstructed jet \pt bin is populated not only by jets whose true (particle-level) \pt lies in the same bin, but also from jets outside the bin, whose response has fluctuated high or low. If the jet spectrum is flat, for a given bin the numbers of true jets migrating in and out are equal and no bias is observed. In the presence of a steeply falling spectrum, the number of incoming jets with lower true \pt that fluctuated high is larger and the measured response is systematically higher. In the dijet \pt-balancing, the effect described above affects both jets. In order to reduce the resolution bias, the measurement of the relative response is performed in bins of \ptave, so that if the two jets have the same resolution, the bias is cancelled on average. This is true for the resolution measurement with the asymmetry method where both jets lie in the same $\eta$ region. For the relative response measurement, the two jets lie in general in different $\eta$ regions, and the bias cancellation is only partial.

\subsubsection{Radiation Imbalance}\label{sec:radbias}

The other source of bias is the \pt-imbalance caused by gluon radiation. In general, the measured \pt-imbalance is caused by the response difference of the balancing objects, but also from any additional objects with significant \pt. The effect can by demonstrated as follows: an estimator $\mathcal{R}^{meas}$ of the response of an object with respect to a reference object, is $\mathcal{R}^{meas}=\pt/\pt^{ref}$ where \pt and $\pt^{ref}$ are the measured transverse momenta of the objects. These are related to the true \pt ($p^{true}_{T,ref}$) through the true response: $\pt=R^{true}\cdot\pt^{true}$ and $\pt^{ref}=R^{true}_{ref}\cdot p^{true}_{T,ref}$. In the presence of additional hard objects in the event, $\pt^{true}=p^{true}_{T,ref}-\Delta\pt$, where $\Delta\pt$ quantifies the imbalance due to radiation.  By combining all the above, the estimator $\mathcal{R}^{meas}$ is expressed as: $\mathcal{R}^{meas}=R^{true}/R^{true}_{ref}\left(1-\Delta\pt/p^{true}_{T,ref}\right)$. This relation indicates that the \pt-ratio between two reconstructed objects is a good estimator of the relative response, only in the case where the additional objects are soft, such that $\Delta\pt/p_{T,ref}^{true}\rightarrow 0$. 

The above considerations are important for all \pt-balancing measurements presented in this paper (the dijet \pt-balancing and the $\gamma$/Z+jet \pt-balancing), both for the scale and the resolution determination. Practically, the measurements are performed with a varying veto on an estimator of $a^{true}=\Delta\pt/p_{T,ref}^{true}$ and then extrapolated linearly to $a^{true}=0$. For the dijet \pt-balancing, the estimator of $a^{true}$ is the ratio $\ptthird/\ptave$, while for the $\gamma,Z$+jet \pt-balancing it is the ratio $\ptsecond/\pt^{\gamma,Z}$.  





\input{JEC.tex}
\clearpage
\input{JER.tex}
\clearpage
\section{Summary}\label{sec:summary}

A study of the jet energy response and the \pt-resolution in the CMS detector has been presented. The various measurements were performed using the 2010 dataset of proton-proton collisions at $\sqrt{s}=7\TeV$ corresponding to an integrated luminosity of $36\pbinv$. Three different jet reconstruction methods have been examined: calorimeter jets, jet-plus-track jets, and particle-flow jets, clustered with the anti-$k_T$ algorithm with a distance parameter $R=0.5$.

The jet energy response of all jet types is well understood and good agreement between data and simulation has been observed. The calibration is based on MC simulations, while residual corrections are needed to account for the small differences between data and simulation. The calibration chain also includes an offset correction, which removes the
additional energy inside jets due to pile-up events. Various in situ measurements, which utilize the transverse momentum balance, have been employed to constrain the systematic uncertainty of the jet energy scale. For all jet types, the total energy scale uncertainty is smaller than 3\% for $\pt>50\GeV$ in the region $|\eta|<3.0$. In the forward region $3.0<|\eta|<5.0$, the energy scale uncertainty for calorimeter jets increases to 5\% (Fig.~\ref{fig:finalUncvsPt}).

The jet \pt-resolution has been studied, using the dijet and $\gamma+$jets
samples in both data and simulation. For PF jets in the region
$|\eta|<0.5$ with a $p_T$ of 100 GeV the measured resolution in the data
is better than  $10\%$ (Figs.~\ref{fig:uabs-mcdata-00to05}-\ref{fig:uabs-mcdata-05to50}). The core as well as the tails  of  the jet
\pt-resolution function  have been estimated, and close agreement  is
observed between the  $\gamma+$jets and dijet samples. The core of the
measured  jet  \pt-resolution in data is broader than the one obtained
from  the  simulation, by $10\%$ in the central region and up
to $20\%$ in the forward region. The resolution tails are in
agreement with the simulation within statistical uncertainty.

\section*{Acknowledgements}
\hyphenation{Bundes-ministerium Forschungs-gemeinschaft Forschungs-zentren} We wish to congratulate our colleagues in the CERN accelerator departments for the excellent performance of the LHC machine. We thank the technical and administrative staff at CERN and other CMS institutes. This work was supported by the Austrian Federal Ministry of Science and Research; the Belgium Fonds de la Recherche Scientifique, and Fonds voor Wetenschappelijk Onderzoek; the Brazilian Funding Agencies (CNPq, CAPES, FAPERJ, and FAPESP); the Bulgarian Ministry of Education and Science; CERN; the Chinese Academy of Sciences, Ministry of Science and Technology, and National Natural Science Foundation of China; the Colombian Funding Agency (COLCIENCIAS); the Croatian Ministry of Science, Education and Sport; the Research Promotion Foundation, Cyprus; the Estonian Academy of Sciences and NICPB; the Academy of Finland, Finnish Ministry of Education and Culture, and Helsinki Institute of Physics; the Institut National de Physique Nucl\'eaire et de Physique des Particules~/~CNRS, and Commissariat \`a l'\'Energie Atomique et aux \'Energies Alternatives~/~CEA, France; the Bundesministerium f\"ur Bildung und Forschung, Deutsche Forschungsgemeinschaft, and Helmholtz-Gemeinschaft Deutscher Forschungszentren, Germany; the General Secretariat for Research and Technology, Greece; the National Scientific Research Foundation, and National Office for Research and Technology, Hungary; the Department of Atomic Energy and the Department of Science and Technology, India; the Institute for Studies in Theoretical Physics and Mathematics, Iran; the Science Foundation, Ireland; the Istituto Nazionale di Fisica Nucleare, Italy; the Korean Ministry of Education, Science and Technology and the World Class University program of NRF, Korea; the Lithuanian Academy of Sciences; the Mexican Funding Agencies (CINVESTAV, CONACYT, SEP, and UASLP-FAI); the Ministry of Science and Innovation, New Zealand; the Pakistan Atomic Energy Commission; the State Commission for Scientific Research, Poland; the Funda\c{c}\~ao para a Ci\^encia e a Tecnologia, Portugal; JINR (Armenia, Belarus, Georgia, Ukraine, Uzbekistan); the Ministry of Science and Technologies of the Russian Federation, the Russian Ministry of Atomic Energy and the Russian Foundation for Basic Research; the Ministry of Science and Technological Development of Serbia; the Ministerio de Ciencia e Innovaci\'on, and Programa Consolider-Ingenio 2010, Spain; the Swiss Funding Agencies (ETH Board, ETH Zurich, PSI, SNF, UniZH, Canton Zurich, and SER); the National Science Council, Taipei; the Scientific and Technical Research Council of Turkey, and Turkish Atomic Energy Authority; the Science and Technology Facilities Council, UK; the US Department of Energy, and the US National Science Foundation.
 Individuals have received support from the Marie-Curie programme and the European Research Council (European Union); the Leventis Foundation; the A. P. Sloan Foundation; the Alexander von Humboldt Foundation; the Associazione per lo Sviluppo Scientifico e Tecnologico del Piemonte (Italy); the Belgian Federal Science Policy Office; the Fonds pour la Formation \`a la Recherche dans l'Industrie et dans l'Agriculture (FRIA-Belgium); the Agentschap voor Innovatie door Wetenschap en Technologie (IWT-Belgium); and the Council of Science and Industrial Research, India.

\clearpage

%% **DO NOT REMOVE BIBLIOGRAPHY**
\bibliography{auto_generated}   % will be created by the tdr script.




