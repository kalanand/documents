\section{Jet Transverse Momentum Resolutions}\label{sec:res}

In the following sections, results on jet \pt resolutions are presented, 
extracted from generator-level MC information, and measured 
from the collider data. Unless stated otherwise, CALO, PF and JPT jets 
are corrected for the jet energy scale, as described in the previous section.

The jet \pt resolution is measured from two different samples, 
in both data and MC samples, using methods described in 
Section~\ref{sec:methods}:

\begin{itemize}
\item The dijet asymmetry method, applied to the dijet sample,
\item The photon-plus-jet balance method, applied to the $\gamma+$jet 
sample.
\end{itemize}
 
The dijet asymmetry method exploits momentum conservation in the transverse plane of the dijet system and
is based (almost) exclusively on the measured kinematics of the dijet events. 
This measurement uses two ways  of describing  the jet resolution 
distributions in data and simulated events. The first method 
makes use of a truncated RMS to characterize the core of the distributions.
The second method  employs functional fitting  of the full jet resolution function, 
and is currently limited to a Gaussian approximation for the jet \pt 
probability density.

The $\gamma+$jet balance method exploits the balance in the transverse 
plane between the photon and 
the recoiling jet, and it uses the photon as a 
reference object whose \pt is accurately measured in ECAL. The width
of the $\pt/\pt^{\gamma}$ distribution provides information on 
the jet \pt resolution in a given $\pt^{\gamma}$ bin.  
The resolution is determined independently for both data and simulated events.
The results extracted from $\gamma$+jet \pt balancing provide
useful input for validating the CMS detector simulation, and serve as an independent and complementary cross-check
of the results obtained with the dijet asymmetry method.

In the studies presented in this paper, the resolution broadening from 
extra radiation activity is removed by extrapolating to the ideal case of a two-body process, both in data and in MC.
In addition, the data/MC resolution ratio is derived.
 

 





