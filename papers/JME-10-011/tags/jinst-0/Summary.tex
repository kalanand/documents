\section{Summary}\label{sec:summary}

A study of the jet energy response and the \pt-resolution in the CMS detector has been presented. The various measurements were performed using the 2010 dataset of proton-proton collisions at $\sqrt{s}=7\TeV$ corresponding to an integrated luminosity of $36\pbinv$. Three different jet reconstruction methods have been examined: calorimeter jets, jet-plus-track jets, and particle-flow jets, clustered with the anti-$k_T$ algorithm with a distance parameter $R=0.5$.

The jet energy response of all jet types is well understood and good agreement between data and simulation has been observed. The calibration is based on MC simulations, while residual corrections are needed to account for the small differences between data and simulation. The calibration chain also includes an offset correction, which removes the
additional energy inside jets due to pile-up events. Various in situ measurements, which utilize the transverse momentum balance, have been employed to constrain the systematic uncertainty of the jet energy scale. For all jet types, the total energy scale uncertainty is smaller than 3\% for $\pt>50\GeV$ in the region $|\eta|<3.0$. In the forward region $3.0<|\eta|<5.0$, the energy scale uncertainty for calorimeter jets increases to 5\% (Fig.~\ref{fig:finalUncvsPt}).

The jet \pt-resolution has been studied, using the dijet and $\gamma+$jets
samples in both data and simulation. For PF jets in the region
$|\eta|<0.5$ with a $p_T$ of 100 GeV the measured resolution in the data
is better than  $10\%$ (Figs.~\ref{fig:uabs-mcdata-00to05}-\ref{fig:uabs-mcdata-05to50}). The core as well as the tails  of  the jet
\pt-resolution function  have been estimated, and close agreement  is
observed between the  $\gamma+$jets and dijet samples. The core of the
measured  jet  \pt-resolution in data is broader than the one obtained
from  the  simulation, by $10\%$ in the central region and up
to $20\%$ in the forward region. The resolution tails are in
agreement with the simulation within statistical uncertainty.
