\section{The CMS Detector}\label{sec:detector}

A detailed description of the CMS detector can be found elsewhere~\cite{CMS}. A right-handed coordinate system is used with the origin at the nominal interaction point (IP). The x-axis points to the center of the LHC ring, the y-axis is vertical and points upward, and the z-axis is parallel to the counterclockwise beam direction. The azimuthal angle $\phi$ is measured with respect to the x-axis in the xy-plane and the polar angle $\theta$ is defined with respect to the z-axis, while the pseudorapidity is defined as $\eta=-\ln\left[\tan\left(\theta/2\right)\right]$. The central feature of the CMS apparatus is a superconducting solenoid, of 6\,m internal diameter, that produces a magnetic field of 3.8\,T. Within the field volume are the silicon pixel and strip tracker and the barrel and endcap calorimeters ($|\eta| < 3$), composed of a crystal electromagnetic calorimeter (ECAL) and a brass/scintillator hadronic calorimeter (HCAL). Outside the field volume, in the forward region ($3 < |\eta| < 5$), there is an iron/quartz-fibre hadronic calorimeter. The steel return yoke outside the solenoid is instrumented with gaseous detectors used to identify muons. The CMS experiment collects data using a two-level trigger system, the first-level hardware trigger (L1)~\cite{PTDRI} and the high-level software trigger (HLT)~\cite{HLT}. 
