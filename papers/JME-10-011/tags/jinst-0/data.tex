\label{sec:data}

Data used for the studies presented here were recorded by the CMS
detector during March - December 2010 
in proton-proton collisions at LHC
at $\sqrt{s}=7$~TeV
and correspond to the integrated luminosity of up to
$L_{int}=35.6$~pb$^{-1}$.
For the presented results we employ  dijet $p_T$-average, single-photon, and single-jet 
triggers. For all analyses presented in this paper, 
a veto is applied on  triggers which indicate the occurrence of  beam halo effects.

%The events are further filtered based on their signature in the pixel
%detector: the fraction of high-purity tracks~\cite{bib:tracking900}
%with respect to the total number of tracks was required to be at least
%$20\,\%$ for events with at least ten tracks. 

The events are further filtered requiring the fraction of high-purity 
pixel tracks~\cite{bib:tracking900} to be $>20\%$ for events with at least 
ten pixel tracks.

The presence of at least one well reconstructed  primary vertex (PV) is
required with $|z(\mathrm{PV})|<24$ cm and with  at least four
tracks considered in the vertex fit,
% $n_{\mathrm{dof}}(\mathrm{PV})\geq5.0$.
where $z(\mathrm{PV})$ represents the position of the proton-proton
collision along the beam-line and $z=0$ indicates the center of the CMS detector. In addition
the radial position of the primary vertex has to lie within the beam
pipe  ($\rho(\mathrm{PV})<2.0$ cm).  In order to suppress events in which noise 
is present  in the calorimeter,  additional cuts  are applied on the ratio 
of the  missing  transverse energy to the total transverse energy of the event:
 $E_T^{miss}/\sum{E_T}<0.5$ or $E_T^{miss}<100$ GeV. These additional cuts are applied only
for the analyses which estimate the core, and not the tails, of the jet 
resolution distributions.

CMS has developed jet quality criteria (``Jet ID'') for calorimeter
jets~\cite{bib:jetid} and PF jets~\cite{bib:ichep_pas}
%jetid_pf} 
which are found to retain
the vast majority ($>99\%$) of real jets in the simulation while rejecting most
fake jets arising from calorimeter and/or readout electronics noise in pure noise 
non-collision data samples such
as  cosmic trigger data  or data from triggers on empty bunches during LHC operation. 
Jets used in the analysis are required to satisfy the loose Jet ID criteria;
for JPT jets the calorimeter Jet ID selections are applied.

Photons used in the $\gamma+$jet analysis are required to have  
transverse momentum $p_{T}^{\gamma}>15$ GeV and  $|\eta|<$1.3. Jets used
in the $\gamma+$jet have an $|\eta|<$1.1 requirement. 
This photon sample, collected with single-photon triggers, is dominated by QCD dijet
background, where a jet mimics the  photon. To suppress this background,
the following additional photon isolation and shower shape requirements are applied:

\begin{itemize}

\item
the energy deposited in HCAL within a cone of $\Delta R=0.4$ around the  
photon direction,
 must be smaller than 2.4 GeV or less than $5\%$ of the photon energy ($E_{\gamma}$);

\item
the energy deposited in ECAL within a cone of $\Delta R=0.4$ around the 
photon direction, excluding the energy associated to the photon, must be   
smaller than 3 GeV or less than $5\%$ of the photon energy;

\item
the number of tracks in a cone of $\Delta R=0.35$ around the photon
direction must be less than three, and the total
transverse momentum of the tracks must be less than $10\%$ of the photon
transverse momentum;

\item
the photon cluster major and cluster minor must be in the range of
0.15-0.35, and 0.15-0.3, respectively. Cluster major and minor
are defined as second moments of the energy distribution along
the direction of the maximum and minimum spread of the ECAL
cluster in the $\eta-\phi$ plane.


\end{itemize}
