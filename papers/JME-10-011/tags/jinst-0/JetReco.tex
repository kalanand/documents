\section{Jet Reconstruction}\label{sec:jets}

Jets considered in this paper are reconstructed using the anti-$k_T$ clustering algorithm~\cite{AKT} with a size parameter $R=0.5$ in the $y-\phi$ space. In some cases, jets with a size parameter $R=0.7$ are also considered. The clustering is performed by four-momentum summation. The rapidity $y$ and the transverse momentum \pt of a jet with energy $E$ and momentum $\vec{p}=(p_x,p_y,p_z)$ are defined as $y=\frac{1}{2}\ln\left(\frac{E+p_z}{E-p_z}\right)$ and $\pt=\sqrt{p_x^2+p_y^2}$ respectively. The inputs to the clustering algorithm are the four-momentum vectors of detector energy deposits or of particles in the Monte Carlo (MC) simulations. Detector jets belong to three types, depending on the way the individual contributions from subdetectors are combined: Calorimeter jets, Jet-Plus-Track jets and Particle-Flow jets.  

{\bf Calorimeter (CALO) jets} are reconstructed from energy deposits in the calorimeter towers. A calorimeter tower consists of one or more HCAL cells and the geometrically corresponding ECAL crystals. In the barrel region of the calorimeters, the unweighted sum of one single HCAL cell and 5x5 ECAL crystals form a projective calorimeter tower. The association between HCAL cells and ECAL crystals is more complex in the endcap regions. In the forward region, a different calorimeter technology is employed, using the Cerenkov light signals collected by short and long quartz readout fibers to aid the separation of electromagnetic and hadronic signals. A four-momentum is associated to each tower deposit above a certain threshold, assuming zero mass, and taking as a direction the tower position as seen from the interaction point.

{\bf Jet-Plus-Track (JPT) jets} are reconstructed calorimeter jets whose energy response and resolution are improved by incorporating tracking information, according to the Jet-Plus-Track algorithm~\cite{JME-09-002}. Calorimeter jets are first reconstructed as described above, and then charged particle tracks are associated with each jet, based on the spatial separation between the jet axis and the track momentum vector, measured at the interaction vertex, in the $\eta-\phi$ space. The associated tracks are projected onto the front surface of the calorimeter and are classified as \textit{in-cone} tracks if they point to within the jet cone around the jet axis on the calorimeter surface. The tracks that are bent out of the jet cone because of the CMS magnetic field are classified as \textit{out-of-cone} tracks. The momenta of charged tracks are then used to improve the measurement of the energy of the associated calorimeter jet: for \textit{in-cone} tracks, the expected average energy deposition in the calorimeters is subtracted and the momentum of the tracks is added to the jet energy. For \textit{out-of-cone} tracks the momentum is added directly to the jet energy. The Jet-Plus-Track algorithm corrects both the energy and the direction of the axis of the original calorimeter jet.

The {\bf Particle-Flow (PF) jets} are reconstructed by clustering the four-momentum vectors of particle-flow candidates. The particle-flow algorithm~\cite{PFT-09-001,PFT-10-002} combines the information from all relevant CMS sub-detectors to identify and reconstruct all visible particles in the event, namely muons, electrons, photons, charged hadrons, and neutral hadrons. Charged hadrons, electrons and muons are reconstructed from tracks in the tracker. Photons and neutral hadrons are reconstructed from energy clusters separated from the extrapolated positions of tracks in ECAL and HCAL, respectively. A neutral particle overlapping with charged particles in the calorimeters is identified as a calorimeter energy excess with respect to the sum of the associated track momenta. The energy of photons is directly obtained from the ECAL measurement, corrected for zero-suppression effects. The energy of electrons is determined from a combination of the track momentum at the main interaction vertex, the corresponding ECAL cluster energy, and the energy sum of all bremsstrahlung photons associated with the track. The energy of muons is obtained from the corresponding track momentum. The energy of charged hadrons is determined from a combination of the track momentum and the corresponding ECAL and HCAL energy, corrected for zero-suppression effects, and calibrated for the non-linear response of the calorimeters. Finally, the energy of neutral hadrons is obtained from the corresponding calibrated ECAL and HCAL energy. The PF jet momentum and spatial resolutions are greatly improved with respect to calorimeter jets, as the use of the tracking detectors and of the high granularity of ECAL allows resolution and measurement of charged hadrons and photons inside a jet, which together constitute $\sim$85\% of the jet energy. 

The {\bf Monte Carlo particle jets} are reconstructed by clustering the four-momentum vectors of all stable ($c\tau > 1$ cm) particles generated in the simulation. In particular, there are two types of MC particle jets: those where the neutrinos are excluded from the clustering, and those where both the neutrinos and the muons are excluded. The former are used for the study of the PF and JPT jet response in the simulation, while the latter are used for the study of the CALO jet response (because muons are minimum ionizing particles and therefore do not contribute appreciably to the CALO jet reconstruction).

The {\bf Particle-Flow missing transverse energy} ($\vecmet$), which is needed for the absolute jet energy response measurement, is reconstructed from the particle-flow candidates and is defined as $\vecmet=-\displaystyle\sum_{i}{\left(E_i\sin\theta_i\cos\phi_i\hat{\mathbf{x}}+E_i\sin\theta_i\sin\phi_i\hat{\mathbf{y}}\right)}=\mex\hat{\mathbf{x}}+\mey\hat{\mathbf{y}}$, where the sum refers to all candidates and $\hat{\mathbf{x}},\hat{\mathbf{y}}$ are the unit vectors in the direction of the x and y axes.

