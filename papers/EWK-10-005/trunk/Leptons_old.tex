Events with high-$\et$ electrons are selected online when they pass an
unprescaled L1 trigger filter that requires a coarse-granularity region 
of the ECAL to have $\et > $ 5 or 8~GeV depending on the run period. 
They subsequently must pass an unprescaled High-Level Trigger (HLT)~\cite{HLT}
filter that requires an ECAL cluster with $\et$ well below the offline 
$\et$ threshold of 25 GeV, using the full granularity of the ECAL and $\et$ measurements
corrected using offline calibration~\cite{CMS-PAS-EGM-10-003}. 
High-$\Pt$ muons are recorded online using the Level-1 muon
trigger and the HLT with thresholds of $\Pt>9 \GeV$ or $\Pt>15 \GeV$, 
according to the running periods.

Electrons are identified offline as clusters of ECAL energy deposits
matched to tracks~\cite{GSF} from the silicon tracker. 
The ECAL clustering algorithm is designed to give clusters containing a
large fraction of the energy of the original electron, including energy
radiated along its trajectory. The ECAL clusters must fall in the ECAL fiducial volume
of $|\eta| < 1.44$ for EB clusters or $1.57 < |\eta| < 2.5$ for EE clusters.
The transition region from $1.44 < |\eta| < 1.57$ is excluded as it leads to lower-quality
reconstructed clusters, due mainly to services and cables exiting between the barrel and 
endcap calorimeters.

%\subsection{Muons \label{sec:muonId}}

%Muon used in this analysis are selected according to the
%quality criteria studies in
%Events with high-$\Pt$ muons are recorded online using the Level-1 muon
%trigger and the High-Level Trigger (HLT), which requires muons within $|\eta| < 2.1$ and
%with a thresholds of $\Pt>9 \GeVc$ or $\Pt>15 \GeVc$, according to the running periods. 
Muons must be identified by two different algorithms~\cite{MUONPAS}: one proceeds from 
the inner tracker outwards (``tracker muons''), the other one starts from 
segments in the muon chambers and proceeds inwards (``global muons''). 
Decays in flight of hadrons and punch-through are reducing a cut of $\chi^2/ndof < 10$ 
on a global fit containing tracker and muon detector hits. 
In order to ensure a precise estimate of momentum and impact parameter 
%(the muon momentum resolution is dominated by the inner tracker detector
%for the tranverse momentum range interesting for this measurement)
only tracks with more than 10 hits and at least one hit in the pixel detector are used. 
We require at least two levels of muon stations in the measurement, 
to ensures a good quality momentum estimate at trigger level, and
to further suppresses remaining fake muon candidates.
%For the $\Zmm$ analysis we minimize the cross-corrlation between tracker and muon 
%detectors by drop the $\chi^2/{\mathrm{ndof}}$ and 
%the request that the muon is found by the tracker algorithm.
Cosmics are rejected by requiring a transverse impact parameter distance to the beam spot
position of less than 2 mm.


% Longer version for paper w.r.t. PAS (L.L.)

Events in which hadronic jets mimic an electron or a muon can contaminate
the $\Wo$ and $\Zo$ samples. Such background is suppressed by imposing limits 
on the presence of additional 
tracks and calorimetric deposits near the trajectory of the lepton candidate.
We define isolation variables for the three subsystems:  
$\IECAL = \sum \et(\textrm{ECAL})/\pt^{\ell}$, 
$\IHCAL = \sum \et(\textrm{HCAL})/\pt^{\ell}$ and 
$\ITRK  = \sum \pt(\textrm{tracks})/\pt^{\ell}$,
where  $\pt^{\ell}$ is the transverse momentum of the lepton candidate.
The scalar sums of transverse energy ($\et$) 
and transverse momentum ($\pt$) 
are performed for objects falling within a cone
$\Delta R = \sqrt{(\Delta\eta)^2+(\Delta\phi)^2} < 0.3$ around
the lepton candidate,
the energy deposits and the track associated with the lepton candidate 
being excluded from the sums.  We also define a combined isolation variable,
$\IRelComb =  \IECAL+\IHCAL+\ITRK$.

\par
For both the $\Wen$ and $\Zee$ analyses
an electron candidate is considered isolated if $\ITRK <0.09$, $\IECAL < 0.07$
and $\IHCAL < 0.10$ in the barrel region;  $\ITRK <0.04$, $\IECAL < 0.05$
and $\IHCAL < 0.025$ in the endcap regions. The muon is considered to be 
isolated if $\IRelComb <$~0.10 for $\Wmn$ and $\ITRK<3$~GeV for $\Zmm$.

% Events in which hadronic jets mimic an electron or a muon can contaminate
% the $\Wo$ and $\Zo$ samples. Such background is suppressed by imposing limits
% on the presence of additional 
% tracks and calorimetric deposits near the trajectory of the lepton candidate.
% The additional activity is summed in a cone 
% $\Delta R = \sqrt{(\Delta\eta)^2+(\Delta\phi)^2} < 0.3$ around
% the lepton candidate. For electrons we apply separate cuts on relative ECAL, HCAL 
% and Tracker isolations. For muons we use a relative combined isolation:
% $ \IRelComb = \left( \sum_{\mathrm{tracks}} \pt + \sum_{\mathrm{ECAL}} \Et+ \sum_{\mathrm{HCAL}} \Et \right)/\Pt$,
% except for $\Zmm$, where we use the track isolation: $\ITRK  = \sum_{\mathrm{tracks}} \pt$.

Particles misidentified as electrons are additionally 
suppressed by requiring that the $\eta$ and $\phi$ coordinates
of the track trajectory extrapolated to the ECAL match the $\eta$ and
$\phi$ coordinates of the ECAL cluster, by requiring a narrow ECAL
cluster width in $\eta$, and by limiting the HCAL energy measured in a
cone of $\Delta R < 0.15$ around the ECAL cluster direction.
Electrons from photon conversions are suppressed by requiring 
one hit in the innermost pixel layer for the
reconstructed electron track.  Furthermore, electrons are
rejected when a partner track is found that is consistent with a
photon conversion, based on the opening angle and the separation in
the transverse plane at the point at which the electron and partner
tracks are parallel. More details and studies of electron reconstruction
and identification can be found in Ref.~\cite{CMS-PAS-EGM-10-004}.

An accurate $\MET$ measurement is essential for distinguishing
$\Wo$ signal from QCD backgrounds. We use $\MET$ estimate provided
by the Particle Flow (PF) algorithm which showed the best performance for the
CMS detector. Details on Particle Flow $\MET$ are provided in Ref.~\cite{PFMET}.

