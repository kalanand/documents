\section{Acceptance}
\label{sec:acceptance}

The acceptance $A_\Wo(\mathrm{e})$ for $\Wen$ is defined
as the fraction of simulated $\Wo$ events having an ECAL cluster within
the ECAL fiducial volume with $\Et>25\GeV$.
The ECAL cluster must match the generated electron after final-state radiation
(FSR) within a cone of $\Delta R=0.2$. No matching in energy is required.

%These acceptances are reported in Table~\ref{tab:e-Waecal}.

% \begin{table}[ht]
%   \begin{center}
%   \begin{tabular}{|l|c|c|c|}
%     \hline
%     $A^{\textrm{ECAL}}_\Wo$  & $\Wp$ & $\Wm$ & $\Wpm$ \\
%     \hline\hline
%     EB     & \WEPEBACC & \WEMEBACC & \WEIEBACC \\
%     EE     & \WEPEEACC & \WEMEEACC & \WEIEEACC \\
%     \hline
%     EB+EE  & \WEPACC & \WEMACC & \WEIACC \\
%     \hline
%     \end{tabular}
%   \end{center}
%   \caption{ ECAL acceptances computed from the POWHEG $\Wen$ samples.
%   \label{tab:e-Waecal}}
% \end{table}

There is an inefficiency in the ECAL cluster reconstruction for
electrons direction within the ECAL fiducial volume
due to a small fraction ($0.5\%$) of noisy or malfunctioning towers
removed from the reconstruction. These are
taken into account in the MC simulation, and no uncertainty is assigned to
this purely geometrical inefficiency. The ECAL cluster selection efficiency is
also affected by a bias in the electron energy scale due to
the $25~\GeV$ energy threshold. The related systematic uncertainty is assigned
to the final $\Wo$ and $\Zo$ selection efficiencies.

The acceptance for the $\Zee$ selection, $A_\Zo(\mathrm{e})$,
is defined as the number of simulated events with
two ECAL clusters  with $\Et>25~\GeV$ within the ECAL fiducial volume and
with invariant mass in the range $60<m_{\mathrm{ee}}<120\,\mathrm{GeV}$, divided by the total number
of signal events in the same mass range, with the invariant mass evaluated using
the momenta at generator level before FSR.
The ECAL clusters must match the two simulated electrons after FSR
within cones of $\Delta R<0.2$. No requirement on energy matching is applied.

For the $\Wmn$ analysis, the acceptance $A_\Wo(\mu)$ is defined as the fraction
of simulated $\Wo$ signal events with muons having transverse momentum $\Pt^{\textrm{gen}}$
and pseudorapidity $\eta^{\textrm{gen}}$, evaluated at the generator level
after FSR, within the kinematic selection: $\Pt^{\textrm{gen}}>25~\GeV$ and $|\eta^{\textrm{gen}}|<2.1$.
% Similarily, for $\Wmn$ the muon is required to have a muon with $p_T^{\textrm{gen}}>25\GeV$
% and $|\eta^{\textrm{gen}}|<2.1$.

The acceptance $A_\Zo(\mu)$ for the $\Zmm$ analysis is defined as the number of
simulated $\Zo$ signal events with both muons passing the kinematic selection with momenta evaluated
after FSR, $\Pt^{\textrm{gen}}>20~\GeV$ and $|\eta^{\textrm{gen}}|<2.1$, and with
invariant mass in the range $60<m_{\mu\mu}<120\,\mathrm{GeV}$, divided by the total number
of signal events in the same mass range, with the invariant mass evaluated using
the momenta at generator level before FSR.

Table~\ref{tab:WZlaccgen} presents the acceptances
for $\Wp$, $\Wm$, and inclusive $\Wo$ and $\Zo$ events,
computed from samples simulated with {\sc powheg} using the CT10 PDF,
for the muon and the electron channels. The acceptances are affected by several
theoretical uncertainties,
which are discussed in detail in Section~\ref{sec:theory}.

\begin{table}[htbp] %
\begin{center}
\caption[.] {\label{tab:WZlaccgen}
Acceptances from {\sc powheg} (with CT10 PDF) for $\Wln$ and $\Zll$ final states,
with the MC statistics uncertainties.\\}
\begin{tabular}{|l|c|c|}
\hline
 {\multirow{2}{*}{Process}} &  \multicolumn{2}{c|}{$A_{\mathrm{W,Z}}$}  \\ \cline{2-3}
  & $\ell=\mathrm{e}$ &$\ell=\mu$ \\
\hline\hline
%      $\Wpln$      & \WEPAGEN & \WMPAGEN \\
%      $\Wmln$      & \WEMAGEN & \WMMAGEN\\
%      $\Wln$  & \WEIAGEN & \WMIAGEN \\
     $\Wpln$      & \WEPACC & \WMPAGEN \\
     $\Wmln$      & \WEMACC & \WMMAGEN\\
     $\Wln$  & \WEIACC & \WMIAGEN \\
\hline
% $\Zll$ & \ZEEAGEN & $0.3977 \pm 0.0017$ \\
$\Zll$ & \ZEEACC & $0.3978 \pm 0.0005$ \\
\hline
\end{tabular}
\end{center}
\end{table}


% \begin{table}[htbp]
%   \begin{center}
%   \begin{tabular}{|l|c|}
%     \hline
%     $A^{\textrm{ECAL}}_\Zo$  & $Z \to e^+e^-$ \\
%     \hline\hline
%     EB+EB     & \ZEEBBACC \\
%     EB+EE     & \ZEEBEACC \\
%     EE+EE     & \ZEEEEACC \\
%     \hline
%     all       & \ZEEACC \\
%     \hline
%     \end{tabular}
%   \end{center}
%   \caption{ Signal acceptances computed from POWHEG $Z \to e^+e^-$ samples.
%   \label{tab:e-Zaecal}}
% \end{table}
