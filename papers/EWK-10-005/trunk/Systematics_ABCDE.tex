The systematic uncertainties on the signal extraction using the ABCD method
are summarized in Table~\ref{tab:ABCDEsyst}. The total uncertainty is
taken as the sum in quadrature of the individual components listed in the table,
and corresponds to 0.7$\%$.
The two most important sources of systematic uncertainty arise from the 
modeling of the signal shape. The largest of these (0.53$\%$) comes from 
the uncertainty on $\varepsilon_\mathrm{P}$, dominated by the statistical uncertainties 
in the \TNP method. Uncertainties on the electron energy scale, and hadronic 
recoil response and resolution affect the modeling of the $\MET$ distribution 
for the signal and together give rise to the second largest uncertainty (0.4$\%$).
The uncertainty coming from the modeling of electroweak backgrounds is 
estimated to be 0.2$\%$. The assumption that the fake electron efficiency to 
pass the $\MET$ boundary is independent of the relative track isolation leads 
to a small bias for which a correction is applied. The uncertainty on this 
correction gives rise to a very small error on the yield of 0.07$\%$. This is dominated 
by the uncertainty on the signal contamination of the anti-selected sample used 
to estimate the correction.

\begin{table}[hbtp] %
  \begin{center}
    \caption{Summary of systematic uncertainties for the ABCD method.}
    \label{tab:ABCDEsyst}
    \begin{tabular}{|l|c|}
      \hline
      Source of systematic uncertainty & Value \\
      \hline\hline
      Signal contamination in bias correction & 0.07\% \\
      EWK backgrounds & 0.20\% \\
      Tag-and-probe  & 0.53\% \\
      $\MET$ modeling  & 0.40\% \\
      \hline
      Total & 0.70\% \\
      \hline
    \end{tabular}
  \end{center}
\end{table}

