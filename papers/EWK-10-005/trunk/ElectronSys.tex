
\subsection{Electron Channels}
\label{subsec:ELEsystematics}

\par
The propagation of statistical and systematic uncertainties on the data/simulation
efficiency correction factors ($\rhoeff$)
from the \TNP method (reconstruction, identification, and trigger)
results in uncertainties of $\WEITNPSYST\%$ and $\ZEETNPSYST\%$ for the $\Wen$ and $\Zee$ analyses,
respectively. The uncertainties on the $\Wp$ and $\Wm$ cross sections are larger than that for
the inclusive $\Wo$ because of the larger statistical uncertainty when efficiencies are estimated
per charge. The systematic uncertainty, which depends on the efficiency
under study, is determined by considering alternative signal and background models. 
The size of the systematic uncertainty is 0.3$\%$ for the electron selection efficiencies 
and 1.0$\%$ for the electron reconstruction efficiency. The estimation of the 
trigger efficiency is considered to be background-free so there is no need to 
perform a fit for the signal estimation. Theoretical uncertainties on the 
corrected efficiencies related to the PDF uncertainties and the PDF choice were 
found to be negligible.

\par
The electron energy scale has an impact on the \ET distribution
for the signal. To study this effect, the
energy-scale corrections obtained
from the shift of the $\Zo$ mass peak
(Section~\ref{sec:e-escale}) are applied to electrons in the EB and EE
in simulation (before the $\Et$ requirement)
and  the missing $\Et$ is recomputed.
The obtained variations on the signal yield from the UML fit
are $\WEIESCALESYST\%$ for the inclusive $\Wo$, $\WEPESCALESYST\%$ for the $\Wp$, and
$\WEMESCALESYST\%$ for the $\Wm$ samples and $\WERESCALESYST\%$ on the
$\Wp/\Wm$ ratio.  All the charge-related studies (determination of individual $\Wp$ and $\Wm$
yields and $\Wp/\Wm$ ratio and associated systematic uncertainties)
include data/simulation
%charge misidentification scale factor of $\WPWMISID$, estimated from
charge misidentification scale factors, estimated from
the fraction of same-sign events in the $\Zee$ data and simulated samples.
\par
The energy scale of electrons has an impact on the $\Zo$ yield because of 
the $\Et>25~\GeV$ requirement on the two electrons and the mass window requirement.
Applying the energy-scale corrections
mentioned above to the EB and EE electrons and reprocessing the data, 
the $\Zo$~yield is decreased by 10~events ($\ZEESAMPLE \to \ZEESAMPLEN$).
A systematic uncertainty equal to this decrease of $\ZEEESCALESYST\%$
is assigned to the $\Zo$ signal yield.
%High statistics MC study shows that the effect
%of the energy scale on the $Z$ yield is $0.6\%$.
%This number is consistent with the increase of yields that
%we observe when applying energy scale corrections to electrons in the data,
%both for WP80 and WP95.
The energy-scale uncertainty for the $\Wo$ selection is included in the systematic uncertainty 
described in the previous paragraph. There, the systematic uncertainty 
is larger than that for the $\Zo$ selection because the energy scale also affects the 
$\MET$ shape used for the signal extraction.
The $\Wo$ selection itself is affected by the energy scale at the level of $0.12\%$.
\par
%By applying the additional smearing of the electron energy obtained
%for the same fit to the $\Zee$ data,
%the $\Wo$ yield  is changed by 0.02\% and the $\Zo$ yield, by 0.XX\%.
The \MET shape used in the $\Wo$ fits is also distorted
by energy resolution uncertainties; this induces a change in the $\Wo$ signal yield
of 0.02\%.

\par
%The \MET energy scale is affected by our limited knowledge
%of the intrinsic hadronic recoil response.  However impressive
%progress on this issue has been accomplished by \MET experts
%from studies
%of the hadronic recoil distributions against photons in $\gamma$-jet,
%leptons in $W$ and dileptons in $Z$ events~\cite{metPAS}.
%From the discrepancies found in these data/simulation comparisons,
%we estimate uncertainties due to the \MET energy scale of
%$\WEIMETSYST\%$ for inclusive $W$, $\WEPMETSYST\%$ for $W^+$, $\WEMMETSYST\%$ for
%$W^-$ yields, and of $\WERMETSYST\%$ for the $W^+/W^-$ ratio.



The \MET energy scale is affected by our limited knowledge of the intrinsic hadronic
recoil response. From the discrepancies found in the data/simulation
comparisons (Section~\ref{sec:WsignalMETtemplate}), uncertainties due to
the \MET energy scale are estimated to be 0.3\% for inclusive $\Wo$,
$\Wp$, and $\Wm$ yields, and 0.1\% for the $\Wp/\Wm$ ratio.

\par
The systematic uncertainties on the background subtraction
address the possible difference between the true background distribution
and the modified Rayleigh function that is used in the UML fit.
We make the assumption that any such difference can be accounted
for by an additional $\sigma_2$ parameter (defined in Section~\ref{sec:WQCDbkg}),
which affects the resolution at large values of \MET (below the signal).
The value of $\sigma_2$ is first determined for three samples: the control sample
in the data, the control sample in the QCD simulation, and the
selected sample in the QCD simulation. The values obtained are $\sigma_2=0.0009~\GeV^{-1}$,
$0.0010~\GeV^{-1}$, and $0.0007~\GeV^{-1}$, respectively for $\Wp$ and
$\sigma_2=0.0007~\GeV^{-1}$,
$0.0009~\GeV^{-1}$, and $0.0008~\GeV^{-1}$ for $\Wm$.
The three values of $\sigma_2$ are then fixed in turn, and $\sigma_0$ and $\sigma_1$
are set to their values from data to generate distributions (of the size
of our sample) with
the three-parameter function, which  we then fit with our nominal two-parameter
function. The maximal relative difference in the yields
is quoted as the systematic uncertainty on background subtraction: $\WEIBKGSYST\%$ for inclusive $\Wo$,
$\WEPBKGSYST\%$ for $\Wp$, $\WEMBKGSYST\%$ for $\Wm$, and $\WERBKGSYST\%$ for the ratio.
%
%
The systematic uncertainties of the fixed shape and the ABCD methods, which were also explored 
in order to cross check the extraction of the inclusive $\Wo$ signal yield, were found to 
be 0.40$\%$ and 0.70$\%$, respectively. 

%
%\par
%In the following paragraphs we discuss the systematic uncertainties of 
the fixed shape and the ABCD methods which were also explored in order 
to cross check the extraction of the $\Wen$ signal. 
These uncertainties correspond to the specific methods and are not propagated
to the final cross section measurement reported in this paper. 

The systematic uncertainties on the background subtraction using fixed-shape 
distributions are summarized in Table~\ref{tab:FixedTempSyst}.  
The total uncertainty is taken as the sum in quadrature of 
the values in the table, giving 0.40\%.  
The total uncertainty is dominated by the uncertainty of the correction of the 
signal contamination in the control sample. This uncertainty is evaluated by 
propagating the uncertainty on the measured contamination using the \TNP technique. 
The statistical uncertainty of this evaluation is calculated using 
a large number of shapes in which the number of events is generated 
from a Poisson distribution with the mean equal to the number of events in the nominal shape. 
The signal yield under variation of the requirements used to define the control 
sample was also studied and found to be very stable with an RMS spread  
of 0.12\% for the range of selections considered. In order to take into 
account the observation of small residual correlations that are not corrected for,
an additional systematic uncertainty of 0.35$\%$ is assigned as a conservative estimate of their size.

\begin{table}[htbp] %
  \begin{center}
    \caption{Summary of systematic uncertainties for background modeling using 
fixed-shape distributions.}
    \label{tab:FixedTempSyst}
    \begin{tabular}{|l|c|}
      \hline
      Source of systematic uncertainty & Value \\
      \hline\hline
      MVA Correction & 0.05\% \\
      Signal contamination & 0.15\% \\
      Statistical fluctuations & 0.12\% \\
      Residual correlations & 0.34\% \\
      \hline
      Total & 0.40\% \\
      \hline
    \end{tabular}
  \end{center}
\end{table} 

%
%\par
%The systematic uncertainties on the signal extraction using the ABCD method
are summarized in Table~\ref{tab:ABCDEsyst}. The total uncertainty is
taken as the sum in quadrature of the individual components listed in the table,
and corresponds to 0.7$\%$.
The two most important sources of systematic uncertainty arise from the 
modeling of the signal shape. The largest of these (0.53$\%$) comes from 
the uncertainty on $\varepsilon_\mathrm{P}$, dominated by the statistical uncertainties 
in the \TNP method. Uncertainties on the electron energy scale, and hadronic 
recoil response and resolution affect the modeling of the $\MET$ distribution 
for the signal and together give rise to the second largest uncertainty (0.4$\%$).
The uncertainty coming from the modeling of electroweak backgrounds is 
estimated to be 0.2$\%$. The assumption that the fake electron efficiency to 
pass the $\MET$ boundary is independent of the relative track isolation leads 
to a small bias for which a correction is applied. The uncertainty on this 
correction gives rise to a very small error on the yield of 0.07$\%$. This is dominated 
by the uncertainty on the signal contamination of the anti-selected sample used 
to estimate the correction.

\begin{table}[hbtp] %
  \begin{center}
    \caption{Summary of systematic uncertainties for the ABCD method.}
    \label{tab:ABCDEsyst}
    \begin{tabular}{|l|c|}
      \hline
      Source of systematic uncertainty & Value \\
      \hline\hline
      Signal contamination in bias correction & 0.07\% \\
      EWK backgrounds & 0.20\% \\
      Tag-and-probe  & 0.53\% \\
      $\MET$ modeling  & 0.40\% \\
      \hline
      Total & 0.70\% \\
      \hline
    \end{tabular}
  \end{center}
\end{table}


%

The QCD background in the $\Zee$ channel is estimated, as discussed earlier, 
using the shape information of the relative track isolation distribution. 
The relative uncertainty (approximately 0.14$\%$) of the 
total Z yield is used as the systematic uncertainty. 

