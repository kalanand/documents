In the following paragraphs we discuss the systematic uncertainties of 
the fixed shape and the ABCD methods which were also explored in order 
to cross check the extraction of the $\Wen$ signal. 
These uncertainties correspond to the specific methods and are not propagated
to the final cross section measurement reported in this paper. 

The systematic uncertainties on the background subtraction using fixed-shape 
distributions are summarized in Table~\ref{tab:FixedTempSyst}.  
The total uncertainty is taken as the sum in quadrature of 
the values in the table, giving 0.40\%.  
The total uncertainty is dominated by the uncertainty of the correction of the 
signal contamination in the control sample. This uncertainty is evaluated by 
propagating the uncertainty on the measured contamination using the \TNP technique. 
The statistical uncertainty of this evaluation is calculated using 
a large number of shapes in which the number of events is generated 
from a Poisson distribution with the mean equal to the number of events in the nominal shape. 
The signal yield under variation of the requirements used to define the control 
sample was also studied and found to be very stable with an RMS spread  
of 0.12\% for the range of selections considered. In order to take into 
account the observation of small residual correlations that are not corrected for,
an additional systematic uncertainty of 0.35$\%$ is assigned as a conservative estimate of their size.

\begin{table}[htbp] %
  \begin{center}
    \caption{Summary of systematic uncertainties for background modeling using 
fixed-shape distributions.}
    \label{tab:FixedTempSyst}
    \begin{tabular}{|l|c|}
      \hline
      Source of systematic uncertainty & Value \\
      \hline\hline
      MVA Correction & 0.05\% \\
      Signal contamination & 0.15\% \\
      Statistical fluctuations & 0.12\% \\
      Residual correlations & 0.34\% \\
      \hline
      Total & 0.40\% \\
      \hline
    \end{tabular}
  \end{center}
\end{table} 
