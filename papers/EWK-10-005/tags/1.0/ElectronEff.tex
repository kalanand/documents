
%\subsection{Electron Efficiencies \label{sec:electronEff}}
\par
The electron selection efficiency is the product of three components:~1)~the
reconstruction efficiency,~2)~the
identification and isolation
efficiency,
and~3)~the trigger efficiency.  Efficiencies are evaluated
for the barrel and endcap regions, and for the two possible
electron charges,  separately.
%\par
%The reconstruction efficiency is the probability of finding a reconstructed
%track when the electron cluster is within the ECAL fiducial volume.
%The probe is selected as an ECAL cluster of reconstructed transverse energy
%greater than $25~\GeV$. To reduce backgrounds, which are not insignificant,
%we use a tight selection on the tag and require the probe
%to pass additional loose shower shape and isolation requirements;
%these are known from simulations to be uncorrelated
%with the reconstruction efficiency.
%The measured reconstruction efficiency is the fraction of
%probes reconstructed as electron tracks.
%For the EB and EE electrons we measure a reconstruction efficiency
%of $\WPWIEBEFFRECO$
%and
%$\WPWIEEEFFRECO$, respectively.
%The resulting correction factors are consistent with unity.
%\par
%The efficiency of electron identification, isolation, and conversion
%rejection requirements is estimated relative to the sample of
%reconstructed electrons. The tag selection does not need to be tight,
%and no additional criteria on the probe are imposed.
%In the barrel, we measure a selection efficiency of
%$\WPWIEBEFFID$, to be compared to  $\WPWIEBMCID$ for
%the simulation, resulting in a correction factor of
%$\WPWIEBRID$. In the endcaps, an efficiency of $\WPWIEEEFFID$
%is measured, where $\WPWIEEMCID$ is expected from simulation, 
%resulting in a correction factor of $\WPWIEERID$.
%Finally, we obtain combined L1 and HLT trigger efficiencies
%from identified and isolated electron candidates as probes.
%We measure $\WPWIEBEFFHLT$ in the barrel,
%and $\WPWIEEEFFHLT$ in the endcaps,
%leading to correction factors consistent with unity.
%
%These tag-and-probe efficiencies are confirmed by measurements
%made with a sample of minimum-bias events selected with scintillation
%counters and a sample of events selected by an HLT algorithm that has
%minimum-bias requirements at L1 and a complete emulation of the
%offline ECAL cluster reconstruction.
%
%The charge misidentification for electrons in the simulated $\Wo$ sample
%is $\EFFA{0.XX}{0.XX}$. We infer a data/simulation
%charge misidentification correction factor of $\WPWMISID$
%by comparing the fraction of events with electrons of same electric charge
%in data and simulation samples.
%The charge misidentification for electrons has a strong $\eta$ dependence 
%with a mean value around 0.5(1.5)$\%$ in EB(EE).
%This correction factor is included in the results for $\Wo^{\pm}$
%cross sections, as well as their ratio, and its error propagated to the
%systematic uncertainties on these quantities.
%
The products of all correction factors for the electron selection
are $\WPWIEBR$ for the EB and $\WPWIEER$ for the EE.
%
When combining the correction factors, we take into account the relative
acceptance of electrons from $\Wo$ decays in the EB and EE.
We obtain the efficiency ratio for $\Wen$ events: $\rho_{\Wo} = \WEITNPR$
and separately by charge: $\rho_{\Wp}=\WEPTNPR$ and $\rho_{\Wm}=\WEMTNPR$.

The corrected signal selection efficiencies for the electron cluster in the 
ECAL fiducial volume with $\et>25~\GeV$ are: $\WEIEFF$ for $\Wo$, $\WEPEFF$ 
for $\Wp$ and $\WEMEFF$ for $\Wm$ events. The acceptances of the electron 
cluster in the ECAL fiducial volume with $\et>25~\GeV$ are:
$\WEIACC$ for $\Wo$, $\WEPACC$ for $\Wp$ and $\WEMACC$ for $\Wm$ events.

The charge misidentification for electrons has a strong $\eta$ dependence
with a mean value around 0.5 (1.5)$\%$ in EB (EE).
This correction factor is included in the results for $\Wpm$
cross sections, as well as their ratio, and its error propagated to the
systematic uncertainties on these quantities.

In the $\Zee$ analysis, one electron
candidate is allowed to fail the trigger criteria;
the efficiency ratio is $\rho_{\Zo} = \ZEETNPR$ and
the corrected signal selection efficiency for $\Zee$ events with
both electron clusters in ECAL fiducial volume and $\et>25~\GeV$
is $\ZEEEFF$. The acceptance of those electron clusters is $\ZEEACC$.

%added at the editor request.
%This number is derived from the corrected overall electron
%selection efficiencies, which are $\WPWIEBSELEFF$ and $\WPWIEESELEFF$ in 
%the EB and EE, respectively, and taking into account the 
%expected fractions of $\Zee$ events with
%EB-EB, EB-EE and EE-EE combinations of electrons, which are
%$\ZEBEBFRAC$, $\ZEBEEFRAC$ and $\ZEEEEFRAC$, respectively.


