\section{Introduction}
This note describes the measurement in pp collisions at
$\sqrt{s} = 7~\TeV$ of the inclusive production cross section for W
and Z bosons, observed via their decay to electrons and muons, and
additionally presents selected cross section ratios.

%The production of W and Z bosons decaying to
%charged leptons is an important process to measure at the LHC, for several
%reasons: it is simultaneously a benchmark for lepton
%reconstruction and identification to be used in future analyses, a
%precision test of perturbative QCD and the parton distribution
%functions of the proton (PDFs), a possible estimator of integrated
%luminosity for proton collisions~\cite{Dittmar:1997md}.  
%At the LHC, QCD predictions, in next-to-next-to leading order (NNLO) in the strong
%coupling $\alpha_{\mathrm{S}}$, exist for the matrix elements
%describing inclusive W and Z production.  When combined with
%recent NNLO PDFs, the cross section is predicted with a few percent
%uncertainty~\cite{Martin:2009iq,Martin:2009bu}. 
%The ratio of inclusive cross sections for Ws and Zs, $R_{\mathrm{WZ}}$, and
%the ratio of cross sections for  W$^+$ and W$^-$, $R_{+-}$, are also
%precisely predicted at the same level of accuracy, but do not suffer
%from experimental uncertainties in proton collision luminosity, which
%cancel, along with other uncertainties.

W and the Z production cross sections have been 
previously measured by ATLAS~\cite{WZATLAS:2010} with
an integrated luminosity of 320~nb$^{-1}$ and by 
CMS~\cite{WZCMS:2010} with an integrated luminosity of 2.9~pb$^{-1}$. 
We present here an update with the full luminosity recorded at the LHC 
in 2010 corresponding to 36~pb$^{-1}$.

%A detailed description of CMS and its performance can be found in 
%Ref.~\cite{JINST}. 
%The systematic uncertainty on the luminosity is currently
%11\%~\cite{lumiPAS}.
%
%For W measurements, the missing transverse energy in the event, $\MET$
%is measured using the particle flow method~\cite{PFMET}.
%
%Several samples of simulated events are used to evaluate signal
%and background efficiencies and to validate our analysis techniques.
%Samples of electroweak processes with Z and W~production, both for
%signal and background events, are produced with POWHEG~\cite{Alioli:2008gx, Nason:2004rx, Frixione:2007vw}
%interfaced with the PYTHIA~\cite{Sjostrand:2006za} parton-shower generator using the tune Z2. 
%QCD events with muons, electrons or jets likely to fake electrons in the
%final state are studied with PYTHIA, as well as other minor
%backgrounds in the analysis, like $\ttbar$.  Generated events are
%processed through the full GEANT4~\cite{GEANT4} detector simulation,
%trigger emulation and event reconstruction chain of the CMS
%experiment.

%Several samples of simulated events were used to evaluate signal
%and background efficiencies and to validate the analysis techniques.
%Samples of electroweak processes with Z and W~production, both for
%signal and background events, are produced with POWHEG~\cite{Alioli:2008gx, Nason:2004rx, Frixione:2007vw}
%interfaced with the PYTHIA~\cite{Sjostrand:2006za} parton-shower generator.
%QCD events with muons, electrons or jets likely to fake electrons in the
%final state are studied with PYTHIA, as well as other minor
%backgrounds in the analysis, like $\ttbar$.  Generated events are
%processed through the full GEANT4~\cite{GEANT4} detector simulation,
%trigger emulation and event reconstruction chain of the CMS experiment.


% The note is organized as follows. 
% Section~\ref{sec:leptonId} 
% describes the baseline lepton identification criteria, whereas 
% section~\ref{sec:evtSel} and~\ref{sec:Zmumu} present the relevant
% details of the analysis and signal extraction strategies used to measure
% the $\Wmn$ and $\Zmm$ yields. Section~\ref{sec:muonEff} contains
% a description of the studies carried out to determine the muon efficiencies
% relative to Monte Carlo that enter in the measurements, as well as their
% estimated uncertainty. Sections~\ref{sec:electronId}, \ref{sec:Wenu}, 
% \ref{sec:Zee} and \ref{sec:electronEff} are the corresponding sections 
% for the electron channels. Section~\ref{sec:systematics} 
% discusses the systematic uncertainties for the
% cross sections and ratios, and the final results are 
% reported in Section~\ref{sec:results}.
% A brief conclusion is reported at the end of the note.

