
\subsection{Electron \label{sec:electronId}}

\par
Events with high-$\et$ electrons are selected online when they pass a L1 trigger
filter that requires a coarse-granularity region of the ECAL to have
$\et > 8$~GeV. They subsequently must pass an HLT~\cite{HLT}
filter that requires an ECAL cluster with $\et$ above a threshold, using
the full granularity of the ECAL and $\et$ measurements
corrected using offline calibration~\cite{CMS-PAS-EGM-10-003}. The threshold depends 
on the run period. For the analysis we have used unprescaled triggers.

\par
Electrons are identified offline as clusters of ECAL energy deposits
matched to tracks from the silicon tracker. The ECAL clusters are designed
to collect the largest fraction of the energy of the original electron,
including energy radiated along its trajectory.  They must
fall in the ECAL fiducial volume
of
$|\eta| < 1.44$ for EB
clusters or $1.57 < |\eta| < 2.5$ for EE clusters.
The transition region
from $1.44 < |\eta| < 1.57$
is excluded as it leads to lower-quality
reconstructed clusters, due mainly to
services and cables exiting between the barrel and endcap calorimeters.
Electron tracks are reconstructed using an
algorithm~\cite{GSF} that accounts for possible energy loss due to
bremsstrahlung in the tracker layers.
The energy of an electron candidate with $\et>20~\gev$ is essentially
determined by the ECAL cluster energy, while its momentum direction
is determined by that of the associated track.
Particles misidentified as
electrons are suppressed by requiring that the $\eta$ and $\phi$ coordinates
of the track trajectory extrapolated to the ECAL match the $\eta$ and
$\phi$ coordinates of the ECAL cluster, by requiring a narrow ECAL
cluster width in $\eta$, and by limiting the HCAL energy measured in a
cone of $\Delta R < 0.15$ around the ECAL cluster direction.
\par
Electrons from photon conversions are suppressed by requiring
one hit in the innermost pixel layer
for the
reconstructed electron track.  Furthermore, electrons are
rejected when a partner track is found that is consistent with a
photon conversion, based on the opening angle and the separation in
the transverse plane at the point at which the electron and partner
tracks are parallel.
\par
For both the $\PW$ and $\PZ$ analyses
an electron candidate is considered isolated if $\ITRK <0.09$, $\IECAL < 0.07$
and $\IHCAL < 0.10$ in the barrel region;  $\ITRK <0.04$, $\IECAL < 0.05$
and $\IHCAL < 0.025$ in the endcap regions.
\par
The electron selection criteria were obtained
by optimizing signal and background levels according to
simulation-based studies. The optimization was done for EB
and EE separately.  We use the same criteria for the $\Wen$ and
$\Zee$ channels; these select approximately 81\% of the
reconstructed electrons in the data with clusters in the ECAL fiducial
volume and
$\et>25~\GeV$, and reduce the
fake electron background by
two orders of magnitude.
\par
More details and studies of electron reconstruction and identification
can be found in Ref.~\cite{CMS-PAS-EGM-10-004}.


