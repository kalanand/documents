\section{Systematic uncertainties \label{sec:systematics}}

\par
The largest uncertainty for the cross section measurement comes from
the luminosity measurement~\cite{lumiPAS} which amounts to $\LUMISYST\%$.
\par
Aside from luminosity, one of the main source of experimental uncertainty in our measurements comes from the 
propagation of uncertainties on the efficiency ratios obtained by the tag-and-probe method. 
In the $\Zmm$ analysis, the efficiency uncertainties are mostly
absorbed in the statistical uncertainty of the measurement, via a combined
fit to the yield and the efficiencies.

% General

The $\MET$ energy scale is affected by our limited knowledge of the intrinsic hadronic recoil response. 
We observe minor discrepancies when comparing hadronic recoil distributions in data and simulation, 
and assign an uncertainties of $\WEIMETSYST\%$ and $\WMIMETSYST\%$ for $\Wen$ and $\Wmn$ respectively.
\par
The electron energy scale has an impact on the \ET distribution
for the signal. To study this effect, we apply  the
energy scale corrections obtained
from the shift of the $\Zo$ mass peak to electrons in the barrel and endcap of the electromagnetic calorimeter
in the simulation (before the $\Et$ threshold cut)
and recompute the missing $\Et$. We obtain variations on the signal yield from the fit
of $\WEIESCALESYST\%$ for the inclusive $\Wo$ cross section.  
\par
Dedicated studies indicate a muon momentum scale effect of $(-0.09\pm 0.03)\%$ for 40 GeV muons.
An upper limit on potential resolution effects of 0.12\% is also obtained.
The effect of momentum scale uncertainty has been estimated as \WMISCALESYST\% for $\Wmn$ and
 \ZMMSCALESYST\% for $\Zmm$ respectively.
The uncertainty on the muon pre-triggering inefficiency has been estimated to be
$\ZMMEFFPRET\%$.


\par
In the $\Wen$ channel, the systematic uncertainty
due to background subtraction is obtained by comparing
fits to various background-dominated distributions:
the sample selected with inverted identification criteria
in the data, and the
samples selected
with and without inverted identification criteria in the QCD simulation.
We quantify
the differences in the tails of these three distributions
by an extra parameter in our analytical background function.
Using a set of pseudo-experiments to estimate
the impact of such differences on the results of the nominal fit, we
evaluate the uncertainty due to background subtraction in the $\Wen$
analysis to be $\WEIBKGSYST\%$. 

\par
The background from
fake electrons in the $\Zee$ sample
is estimated from data. The uncertainty on this
background is $\ZEEBKGSYST\%$ of the total $\PZ$ yield.
Uncertainties from the normalization of electroweak and $\ttbar$
backgrounds are negligible in both $W$ and $Z$ channels.

The QCD background shape for the $\Wmn$ analysis is tested by refitting the $\MET$ spectrum
varying the isolation correction factor $\alpha$ within its uncertainty, 
which corresponds to a variation in the signal yield with respect
to the reference template of $\WMIQCDSHAPESYST\%$, that is taken as corresponding uncertainty.
The recoil modeling of the signal shape is also a potential source of
uncertainty. This uncertainty is estimated by refitting the $\MET$
distribution with
the signal shape predicted by the simulation. The variation in the signal
yield with respect
to the reference result is small: $\WMIRECOILSYST\%$.
Uncertainties on the knowledge of the electroweak backgrounds
have a small impact on the $\Wmn$ measurement
section. The assigned systematics, 1.5\% accounts for the
absence of efficiency corrections and
also for remaining differences between PYTHIA and POWHEG approaches.
% Zmm

%The largest source of systematic uncertainty for the \Zmm cross section after 
%luminosity is the theoretical uncertainty which affect the acceptance estimate. 
We have varied the $\Zmm$ fit procedure using different background shape models and binning size
%in order to estimate the uncertainty assigned to the fit model.
and we estimate the corresponding uncertainty to be \ZMMFITSYST\%.
The uncertainty related to the presence of the small background fraction under the peak region in 
the ``golden'' categories amounts to \ZMMBKGSYST\%,
which adds to a total background systematic uncertainty for $\ZMM$ of \ZMMBKGTOTSYST\%.

All theoretical predictions are computed at NNLO with the program 
FEWZ~\cite{Melnikov:2006kv, Melnikov:2006di})
and the MSTW set of PDFs.  The uncertainties are 68\% confidence
levels obtaining by combining the NLO PDF and $\alpha_S$ errors
from the MSTW08, CT10 and NNPDF2.0 groups and adding the
NNLO scale uncertainties in quadrature, as prescribed by the
PDF4LHC working group~\cite{PDF4LHC}. Other contributions to
theoretical uncertainty come from final-state radiation modeling,
resummation and NNLO QCD effects, factorization and renormalization scale dependence,
missing resummation and NNLO QCD and missing electroweak corrections.

Table~\ref{tab:syst} shows a summary of the systematic uncertainties
for the $\Wo$ and $\Zo$ cross section measurements.
Tables~\ref{tab:systMU} and~\ref{tab:systEL}
show the summary of the systematic uncertainties
for the charged cross sections ($\Wp, \Wm$) and ratios ($\Wp/\Wm$, $\Wo/\Zo$).

\begin{table}
   \caption[.]{ \label{tab:syst}
Systematic uncertainties in percent for all inclusive W and Z cross sections.
``n/a'' means the source does not apply.
A common luminosity uncertainty of $\LUMISYST\%$ applies to all channels.}
\begin{center}
\begin {tabular} {|l|c|c|c|c|}
\hline
Source                                   & $\Wen$         & $\Wmn$           & $\Zee$         & $\Zmm$ \\
\hline
Lepton reconstruction \& identification  & \WEITNPSYST    & \WMIEFFSYST      & \ZEETNPSYST    &  n/a \\
Trigger pre-firing                      & n/a            & \WMIEFFPRET      & n/a            & \ZMMEFFPRET \\
Momentum scale \& resolution             & \WEIESCALESYST & \WMISCALESYST    & \ZEEESCALESYST & \ZMMSCALESYST \\
$\MET$ scale \& resolution               & \WEIMETSYST    & \WMIMETSYST      &  n/a           &  n/a   \\
Background subtraction / modeling        & \WEIBKGSYST    & \WMIQCDSHAPESYST & \ZEEBKGSYST    & \ZMMBKGTOTSYST  \\
%Trigger changes throughout 2010          & n/a            & n/a              & n/a            & \ZMMTRIGABSYST \\
\hline
Total experimental                       & \WEEXPSYST    & \WMEXPSYST       & \ZEEEXPSYST    & \ZMMEXPSYST \\
\hline
PDF uncertainty for acceptance           & \WEIPDFACCSYST & \WMIPDFACCSYST   & \ZEEPDFACCSYST & \ZMMPDFACCSYST \\
Other theoretical uncertainties          & \WEITHSYST     & \WMITHSYST       & \ZEETHSYST     & \ZMMTHSYST  \\
\hline
Total theoretical                        & \WEITOTTHSYST & \WMITOTTHSYST & \ZEETOTTHSYST & \ZMMTOTTHSYST \\
\hline
Total                                    & \WEITOTSYST    & \WMITOTSYST      & \ZEETOTSYST    & \ZMMTOTSYST \\
\hline
\end {tabular}
\end{center}
\end{table}

\begin{table}
   \caption[.]{ \label{tab:systMU}
Systematic uncertainties in percent for charged W cross sections and ratios in the
muon channel.
A common luminosity uncertainty of 11\% applies to all cross sections. }
\begin{center}
\begin {tabular} {|l|c|c|c|c|}
\hline
Source       & $\Wp$ ($\mu$) & $\Wm$ ($\mu$) & $\Wp/\Wm$ ($\mu$) & $W/Z$ ($\mu$) \\
         \hline
Lepton reconstruction \& identification  & 0.4 & 0.4 & 1.3 & 0.4 \\
Trigger pre-firing                       & 0.5 & 0.5 & 0   & 0   \\
Momentum scale \& resolution             & $<$ 0.1 & $<$ 0.1 & $<$ 0.1 & $<$ 0.35 \\
$\MET$ scale \& resolution               & 0.2 & 0.2 & 0.0   & 0.2 \\
Background subtraction / modeling        & 0.4 & 0.5 & 0.2 & 0.4 \\
\hline
Total experimental                       & 0.8 & 0.9 & 1.3 & 0.7 \\
\hline
PDF uncertainty for acceptance           & 1.0 & 1.6 & 2.0 & 1.3 \\
Other theoretical uncertainties          & 1.0 & 0.8 & 0.8 & 1.4 \\
\hline
Total theoretical                        & 1.4 & 1.7 & 2.2 & 1.9 \\
\hline
Total                                    & 1.6 & 1.9 & 2.6 & 2.0 \\
\hline
\end {tabular}
\end{center}
\end{table}
%------------------------------------------------------------

\begin{table}
   \caption[.]{ \label{tab:systEL}
Systematic uncertainties in percent for charged W cross sections and ratios in the
electron channel.
A common luminosity uncertainty of 11\% applies to all cross sections. }
\begin{center}
\begin {tabular} {|l|c|c|c|c|}
\hline
Source       & $\Wp$ (e) & $\Wm$ (e) & $\Wp/\Wm$ (e) & $W/Z$ (e) \\
         \hline
Lepton reconstruction \& identification  & 1.7 & 1.5 & 1.6 & 1.0 \\
Momentum scale \& resolution             & 0.5 & 0.6 & 0.1 & 0.2 \\
$\MET$ scale \& resolution               & 0.3 & 0.3 & 0.1 & 0.3 \\
Background subtraction / modeling        & 0.3 & 0.5 & 0.4 & 0.3 \\
\hline
Total experimental                       & 1.9 & 1.7 & 1.7 & 1.1 \\
\hline
PDF uncertainty for acceptance           & 0.8 & 1.3 & 1.6 & 1.1 \\
Other theoretical uncertainties          & 1.0 & 0.7 & 1.2 & 1.2 \\
\hline
Total theoretical                        & 1.3 & 1.5 & 2.1 & 1.6 \\
\hline
Total                                    & 2.3 & 2.3 & 2.7 & 2.0 \\
\hline
\end {tabular}
\end{center}
\end{table}
%------------------------------------------------------------
