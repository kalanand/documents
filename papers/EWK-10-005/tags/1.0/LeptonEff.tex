\par
The efficiencies for lepton reconstruction, identification,
isolation and trigger efficiencies are obtained from data.
Correction factors are the ratios of efficiencies extracted from data and simulations 
($\rho = {\epsilon_{\mathrm{data}}}/{\epsilon_{\mathrm{MC}}}$) 
with a tag-and-probe method exercised on $\Zll$ samples in both data and simulation.
%This procedure adequately removes any systematic uncertainties coming from imperfections in
%the simulation, even though the kinematic distributions
%of leptons in the $\Zll$ sample differ slightly from those
%in the selected $\Wln$ sample.
The tag-and-probe sample for the measurement of a given efficiency
contains events selected with two lepton candidates.
One lepton candidate, called the ``tag,'' satisfies
tight identification and isolation requirements. The other
lepton candidate, called the ``probe,'' is selected with
criteria that depend on the efficiency being measured.
The invariant mass of the tag and probe lepton candidates
must fall in the range $60$--$120~\GeV$.
%The signal yields are obtained for two exclusive subsamples of events
%in which the probe lepton passes or fails the selection criteria considered.
%Fits are performed to the invariant-mass distributions
%of the pass and fail subsamples, including a term that
%accounts for the background.
%The measured efficiency is
%deduced from the relative level of signal in the pass and fail subsamples;
%its uncertainty includes a systematic contribution from the fitting
%procedure.
%\par
%The correction factors are obtained as
%ratios of tag-and-probe efficiencies for
%the data and for the simulation.  They are used to compute
%the signal selection efficiency ratios $\RHO{}$, and
%their uncertainties are  propagated as systematic uncertainties
%on these quantities,
%except in the $\Zmm$ analysis, for which the efficiencies and yields
%are determined simultaneously.
