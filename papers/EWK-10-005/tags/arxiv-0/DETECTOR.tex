\section{The CMS Detector}
\label{sec:detector}
\par
The central feature of the CMS apparatus
is a superconducting solenoid of 6~m internal diameter, providing
a magnetic field of $3.8$~T. Within the field volume are a silicon pixel
and strip tracker, an electromagnetic calorimeter (ECAL),
and a hadron calorimeter (HCAL). Muons are detected
in gas-ionization detectors embedded in the steel return
yoke. In addition to the barrel and endcap detectors, CMS has
extensive forward calorimetry.
\par
A right-handed coordinate system is used in CMS, with the origin at the
nominal interaction point, the $x$-axis pointing to the center of
the LHC ring, the $y$-axis pointing up (perpendicular to the LHC plane),
and the $z$-axis along the anticlockwise-beam direction. The polar
angle $\theta$ is measured from the positive $z$-axis and the
azimuthal angle $\phi$ is measured (in radians) in the $xy$-plane.
The pseudorapidity is given by $\eta = -\ln\tan(\theta/2)$.
\par
The inner tracker measures charged particle trajectories in the
pseudorapidity range $|\eta| < 2.5$.   It consists of $1440$ silicon
pixel and 15\,148 silicon strip detector modules.  It provides an
impact parameter resolution of ${\approx}15\mum$ and a transverse
momentum ($\pt$) resolution of about 1\% for charged particles
with $\pt \approx 40\GeV$.
\par
The electromagnetic calorimeter consists of nearly $76\,000$ lead tungstate
crystals, which provide coverage in pseudorapidity $|\eta| < 1.479$ in a
cylindrical barrel region (EB) and $1.479 < |\eta| < 3.0$ in two endcap
regions (EE).
A preshower detector
consisting of two planes of silicon sensors interleaved with a total of
three radiation lengths of lead is located in front of the EE.
The ECAL has an energy resolution of better than $0.5\%$ for
unconverted photons with transverse energies ($\Et$) above $100\GeV$.
The energy resolution is $3\%$ or better for the range of
electron energies relevant for this analysis.
The hadronic barrel and endcap calorimeters are sampling devices with brass
as the passive material and scintillator as the active material.
The combined calorimeter cells are grouped in projective towers of granularity
$\Delta \eta \times \Delta \phi = 0.087\times0.087$ at central rapidities
and $0.175\times0.175$ at forward rapidities.
The energy of charged pions and other quasi-stable hadrons can be measured with 
the calorimeters (ECAL and HCAL combined) with a resolution of 
$\Delta E/E \simeq 100 \%/\sqrt{E(\GeV)} \oplus 5\%$. 
For charged hadrons, the calorimeter resolution improves on the tracker momentum 
resolution only for $\PT$ in excess of 500~GeV. 
The energy resolution on jets and missing transverse energy is substantially improved with
respect to calorimetric reconstruction by using
the particle flow (PF) algorithm~\cite{PFT} which consists in reconstructing and
identifying each single particle with an optimised combination of all sub-detector
information. This approach exploits the very good tracker momentum resolution
to improve the energy measurement of charged hadrons.

\par
Muons are detected in the pseudorapidity window $|\eta|< 2.4$, with
detection planes based on three technologies: drift tubes, cathode strip
chambers, and resistive plate chambers.  A high-$\pt$ muon originating
from the interaction point produces track segments typically
in three or four muon stations.  Matching these segments to tracks
measured in the inner tracker results in a $\pt$ resolution
between 1 and 2\% for $\pt$ values up to $100$~GeV.
\par
The first level (L1) of the CMS trigger system~\cite{cmsTrigger}, composed of custom
hardware processors, is designed to select the most interesting events
in less than $1\mus$,
using information from the calorimeters
and muon detectors. The High Level Trigger (HLT) processor farm~\cite{HLT} further
decreases the event rate
to a few hundred Hz before data storage. 
%The HLT is divided into two trigger levels (L2 and L3).
A more detailed description of CMS can be found elsewhere~\cite{JINST}.



