\subsection{Muon Channels}
\label{sec:muonSyst}

% Dedicated studies indicate a muon momentum scale effect of $(-0.09\pm 0.03)\%$ for 40 GeV muons.
% An upper limit on potential resolution effects of 0.12\% is also obtained.
% The effect of momentum scale uncertainty has been estimated to be $\WMISCALESYST\%$ for $\Wmn$ and
%  $\ZMMSCALESYST\%$ for $\Zmm$ respectively.

The total uncertainty of 0.9\%  (statistical plus systematic) on the 
correction factors $\rhoeff$ is used as the systematic uncertainty due to muon efficiency 
(reconstruction, identification, selection, isolation,
and trigger) for the $\Wmn$ yield. 
%
% Removed from comments by Isabel (L.L.)
%
%Additionally, the variation of the correction factors with different PDF
%assumptions for the simulated Z samples, used to evaluate the 
%\TNP efficiencies in simulations, has been evaluated
%as $0.01\%$ and therfore has been neglected.
The systematic uncertainty assigned to the efficiencies is evaluated using
a large simulated sample including the Z signal and all potential backgrounds. 
Additional uncertainties are evaluated by varying the initial Z preselection criteria and
the mass window to perform the background subtraction fit, and by using alternative parameterizations to model the 
background. The statistical uncertainties on the fit parameters describing the background correction 
are also included. The effect of the uncertainties due to the choice of PDFs used in the Z simulation is also studied and found to 
be negligible.

%The full difference in correction factors for the positively and negatively charged muons, 1.3\%, 
%is propagated as a systematic uncertainty in the measurement of the $\Wp/\Wm$ cross-section ratio.

A conservative systematic uncertainty of 0.5\%, due to the correction for the trigger prefiring inefficiency 
(Section~\ref{sec:muonEff}), is assigned to both the $\Zmm$ and $\Wmn$ cross-section estimates.

Dedicated studies comparing the peak position and width of the observed Z distribution
with the expected one indicate a muon momentum scale effect of ${\sim} 0.25\%$ for 40 GeV muons.
In order to evaluate the impact on the W cross-section measurement,
the fitting procedure with a new signal distribution where the muon $\Pt$ in the simulations
is modified according to the observed effect, is performed. The difference with respect to the value
quoted above is $\WMISCALESYST$\% for the inclusive W sample, 0.19\% for $\Wp$, and 0.25\% for $\Wm$,
and for the $\Wp/\Wm$ ratio it reduces to 0.06\%.
Muon momentum scale and resolution affect the measurement of the $\Zmm$ cross section 
with a $\ZMMSCALESYST\%$ uncertainty.

The QCD background shape for the W analysis is tested by applying fits to the $\MET$ spectrum
with the two extreme $\MET$ shapes, corresponding to the maximal variations of the correction 
factor, $\alpha$. The variation in the signal yield with respect to that obtained using the reference 
distribution is $\WMIQCDSHAPESYST\%$  for the inclusive W sample, 0.4\% for $\Wp$, 0.5\% for $\Wm$, 
and 0.2\% for the $\Wp/\Wm$ ratio.

The recoil modeling in the signal shape is also a potential source of
uncertainty. This uncertainty is estimated by applying the signal shape predicted by the simulation
to the fits of the $\MET$ distribution. The variation in the signal
yield with respect to the reference result is $\WMIMETSYST\%$.

The systematic uncertainty on the $\Zmm$ signal extraction procedure 
 has been evaluated as follows. The
uncertainty of the fit model is estimated by varying in different ways the background models 
and changing the dimuon mass binning of the various dimuon categories.
Half of the difference between the maximum and minimum fitted yields across all the tested variations
is taken as a systematic uncertainty. This amounts to $\ZMMFITSYST\%$.

The signal shape has been determined assuming that the golden
samples are background-free.
A flat distribution is added as background contribution to the signal shapes
and this produces a relative change in the fitted Z yield 
equal to one third of the introduced background fraction.
An irreducible contamination is known to be present from simulation with the given selection.
It amounts to less than 0.5\%, so a conservative estimate of $\ZMMBKGSYST\%$
systematic uncertainty due to neglecting the background
in the signal shapes used for the fit is assigned.
Adding those two contributions in quadrature, a total 
systematic uncertainty due to the fit method of
$\ZMMBKGTOTSYST\%$ is assigned. 

The stability of the measured Z yields was also checked in 
the two run periods with different trigger thresholds and
the corresponding variation of the signal yield of 0.1\% is taken as
a conservative systematic uncertainty.

