\section{Event Selection}
\label{sec:eventSelection}

The $\Wln$ events are characterized by a prompt, energetic, and
isolated lepton and significant missing transverse energy, $\MET$.  
No requirement on $\MET$ is applied. Rather, the $\MET$ is used as the main 
discriminant variable against backgrounds from QCD events. 

The Z boson decays to leptons (electrons or muons) are selected based on two 
energetic and isolated leptons.
The reconstructed dilepton invariant mass is required to be consistent with
the known Z boson mass. 

\par
The following background processes are considered:
\begin{itemize}
\item  {\sl QCD multijet events.}
Isolation requirements reduce events with leptons produced inside jets.
The remaining background is estimated with a variety of
techniques based on data.  
\item {\sl High-$\Et$ photons.}
For the $\Wen$ channel only, there is a nonnegligible
background contribution coming from  the conversion of a 
photon from the process $\pp\rightarrow\gamma+$jet(s).
\item {\sl Drell--Yan.}
A DY lepton pair 
%
% Proposed by the ARC: what's the best notation ? (L.L.)
%
%($\pp\rightarrow\ell^+\ell^-X$) 
constitutes a background for the $\Wln$ channels 
when one of the two leptons is not reconstructed or does not enter a fiducial region.
%
% The following is not general, applies mainly to electrons: muons
% have ~100% reconstrucion efficiency.
% comment from Isabel.
%
%(not reconstructed or disappears into a non-fiducial region).  
%After the veto of events with
%two reconstructed leptons, this background is small.
%
% not correct: fit with the signal and scaled accordingly (commen from Isabel).
%
% and is estimated using simulations.
\item {\sl $\Wtn$ and $\Ztt$ production.}
A small background contribution comes from W and Z events with one or both $\tau$ decaying
leptonically.  The minimum lepton $\Pt$ requirement tends to suppress
these backgrounds.
%
% same as above...
%
% which are estimated from simulations.
\item {\sl Diboson production.} The production of boson pairs ($\Wo\Wo$, $\Wo\Zo$, $\Zo\Zo$)
is considered a background to the W and Z analysis
because the theoretical predictions for the vector boson production
cross sections used for comparison with data
do not include diboson production.
The background from diboson production
is very small and is estimated using simulations.
\item {\sl Top-quark pairs.}
The background from $\ttbar$ production is quite small and
is estimated from simulations.
\end{itemize}

The backgrounds mentioned in the first two bullets are referred to 
as ``QCD backgrounds'', the Drell--Yan, $\Wtn$, 
and dibosons as "EWK backgrounds", and the last one as "$\ttbar$ background".
For both diboson and $\ttbar$ backgrounds, the NLO cross sections were used.
The complete selection criteria used to reduce the above backgrounds 
are described below.


\subsection{Lepton Isolation}
\label{sec:isolation}

The isolation variables for the tracker and the electromagnetic 
and hadronic ca\-lo\-ri\-me\-ters are defined:
$\ITRK  = \sum_{\mathrm{tracks}} \pt$\,,
$\IECAL = \sum_{\mathrm{ECAL}} \Et$\,, 
$\IHCAL = \sum_{\mathrm{HCAL}} \Et$\,,
where the sums are performed on all objects falling within a cone of aperture
$\Delta R$ = $\sqrt{(\Delta\eta)^2+(\Delta\phi)^2}$ = 0.3 around
the lepton candidate momentum direction.
The energy deposits and the track associated with the lepton candidate 
are excluded from the sums.

\input{ElectronId}

%\subsection{Muons \label{sec:muonId}}

%Muon used in this analysis are selected according to the
%quality criteria studies in
%Events with high-$\Pt$ muons are recorded online using the Level-1 muon
%trigger and the High-Level Trigger (HLT), which requires muons within $|\eta| < 2.1$ and
%with a thresholds of $\Pt>9 \GeVc$ or $\Pt>15 \GeVc$, according to the running periods. 
Muons must be identified by two different algorithms~\cite{MUONPAS}: one proceeds from 
the inner tracker outwards (``tracker muons''), the other one starts from 
segments in the muon chambers and proceeds inwards (``global muons''). 
Decays in flight of hadrons and punch-through are reducing a cut of $\chi^2/ndof < 10$ 
on a global fit containing tracker and muon detector hits. 
In order to ensure a precise estimate of momentum and impact parameter 
%(the muon momentum resolution is dominated by the inner tracker detector
%for the tranverse momentum range interesting for this measurement)
only tracks with more than 10 hits and at least one hit in the pixel detector are used. 
We require at least two levels of muon stations in the measurement, 
to ensures a good quality momentum estimate at trigger level, and
to further suppresses remaining fake muon candidates.
%For the $\Zmm$ analysis we minimize the cross-corrlation between tracker and muon 
%detectors by drop the $\chi^2/{\mathrm{ndof}}$ and 
%the request that the muon is found by the tracker algorithm.
Cosmics are rejected by requiring a transverse impact parameter distance to the beam spot
position of less than 2 mm.



%\subsubsection{$\Wen$}

The event selection requirements for $\Wen$ are then as follows:
\begin{enumerate}
\item
one identified electron within acceptance, satisfying the WP80 set
of identification and isolation criteria,
\item
if a second electron candidate with $\Et > 20~\gev$, within ECAL fiducial 
and satisfying looser identification and isolation criteria (WP95) is present
in the event, the event is rejected.
\end{enumerate}

The number of $\Wen$ candidate events selected
in the data sample is
$\WEISAMPLE$, with
$\WEPSAMPLE$ positrons and
$\WEMSAMPLE$ electrons.


%\input{WmnSel}

%\subsection{$\Zll$ Selection}

%\input{ZeeSel}

%\input{ZmmSel}
