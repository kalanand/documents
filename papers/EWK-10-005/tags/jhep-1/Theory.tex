
\subsection{Theoretical Uncertainties}
\label{sec:theory}

The main theoretical uncertainty on the cross section estimation arises from the computation of the
geometrical and kinematic acceptance of the detector.  Uncertainty due to
the PDF choice, and uncertainties in the PDFs themselves are
studied using the full PDF eigenvector set and comparing among PDFs
provided by the CTEQ, MSTW, and NNPDF groups. For the estimation of 
the acceptance uncertainties, we followed the recipe prescribed by the 
PDF4LHC working group~\cite{PDF4LHC}.

%All theoretical predictions quoted in this section are computed
%at NNLO with the {\sc FEWZ}~\cite{Melnikov:2006kv, Melnikov:2006di} code
%and the MSTW PDF set. 
%The uncertainties correspond to a 68\% confidence
%levels (CL) obtained by combining the NLO PDF and $\alpha_S$ uncertainty
%from the MSTW08~\cite{Martin:2009iq}, CT10~\cite{CTEQ10}, and NNPDF2.1~\cite{NNPDF21} PDFs, and adding the
%NNLO scale uncertainties in quadrature, as prescribed by the
%PDF4LHC working group~\cite{PDF4LHC}.

Systematic uncertainties on the acceptances due to the PDF choice are
reported in Table~\ref{tab:pdfSyst}.
Here $\Delta_{i}$ denotes the uncertainty (68\% confidence level (CL)) within a given set 
$i$ ($i=$ CT10~\cite{CTEQ10}, MSTW08NLO~\cite{Martin:2009iq}, NNPDF2.1~\cite{NNPDF21}). 
The quantity $\Delta_{\mathrm{sets}}$ corresponds to half of the maximum difference between the central values of any pair of sets. 
The final systematic uncertainty (last column) considers half of the maximum
difference between the extreme values (central values plus positive or minus negative 
uncertainties), again for any pair of the three 
sets, plus the remaining $\alpha_S$ uncertainties.
As can be seen from Table~\ref{tab:pdfSyst}, the $\Wm$ acceptance uncertainties 
are larger than the $\Wp$ ones. This is true for each 
PDF set as well as for the total assigned acceptance uncertainty and reflects the 
larger d-quark PDF uncertainties with respect to those for the u quark.
The acceptance estimates obtained using the different PDF sets are summarized in
Table~\ref{tab:pdfAcc}.

\begin{table}[htb] %
\begin{center}
\caption{Systematic uncertainties from the PDF choice on estimated 
acceptances and acceptance correction factors after the analysis selections. }
\label{tab:pdfSyst}
\begin{tabular}{| l | c | c | c | l | c |}
\hline
\centering Quantity & $\Delta_{\mathrm{CTEQ}}$ (\%) & $\Delta_{\mathrm{MSTW}}$ (\%) &  $\Delta_{\mathrm{NNPDF}}$ (\%) & $\Delta_{\mathrm{sets}}$ (\%) & Syst. (\%) \\
\hline
\hline
$\Wp$ acceptance (e) & $\pm 0.5$ & $\pm 0.3$ &  $\pm 0.4$ & $0.2$~{\tiny (NNPDF-MSTW)} & $0.7$ \\
$\Wm$ acceptance (e) & $\pm 0.9$ & $\pm 0.5$ &  $\pm 0.7$ & $0.5$~{\tiny (NNPDF-MSTW)} & $1.2$ \\
$\Wo$ acceptance (e)   & $\pm 0.5$ & $\pm 0.3$ &  $\pm 0.4$ & $0.2$~{\tiny (MSTW-CTEQ)} & $0.6$ \\
$\Zo$ acceptance (e)   & $\pm 0.7$ & $\pm 0.4$ &  $\pm 0.6$ & $0.3$~{\tiny (NNPDF-MSTW)} & $0.9$ \\
$\Wp/\Wm$ correction (e) & $\pm 1.6$ & $\pm 0.5$ &  $\pm 0.7$ & $0.7$~{\tiny (NNPDF-MSTW)} & $1.6$ \\
$\Wo/\Zo$ correction (e) & $\pm 0.6$ & $\pm 0.2$ &  $\pm 0.3$ & $0.2$~{\tiny (NNPDF-MSTW)} & $0.6$ \\
\hline
$\Wp$ acceptance ($\mu$)   & $\pm 0.7$ & $\pm 0.4$ &  $\pm 0.6$ & $0.3$~{\tiny (NNPDF-MSTW)} & $0.9$ \\
$\Wm$ acceptance ($\mu$) & $\pm 1.1$ & $\pm 0.6$ &  $\pm 0.9$ & $0.5$~{\tiny (MSTW-CTEQ)} & $1.5$ \\
$\Wo$ acceptance ($\mu$) & $\pm 0.7$ & $\pm 0.4$ &  $\pm 0.6$ & $0.2$~{\tiny (MSTW-CTEQ)} & $0.8$ \\
$\Zo$ acceptance ($\mu$)   & $\pm 1.0$ & $\pm 0.6$ &  $\pm 0.9$ & $0.2$~{\tiny (NNPDF-MSTW)} & $1.1$ \\
$\Wp/\Wm$ correction ($\mu$) & $\pm 1.9$ & $\pm 0.6$ &  $\pm 0.9$ & $0.8$~{\tiny (NNPDF-MSTW)} & $1.9$ \\
$\Wo/\Zo$ correction ($\mu$) & $\pm 0.8$ & $\pm 0.2$ &  $\pm 0.3$ & $0.2$~{\tiny (NNPDF-CTEQ)} & $0.9$ \\
\hline
\end{tabular}
\end{center}
\end{table}

\begin{table}[htb] %
\begin{center}
\caption{Predictions of the central values of the acceptances and the ratios of 
acceptances for various PDF sets. }
\label{tab:pdfAcc}
\begin{tabular}{| l | c | c | c |}
\hline
\centering Quantity & CTEQ & MSTW &  NNPDF  \\
\hline
\hline
$A_\Wp(\mathrm{e})$ & 0.5017 & 0.5016 & 0.5036 \\
$A_\Wm(\mathrm{e})$ & 0.4808 & 0.4855 & 0.4804 \\
$A_\Wo(\mathrm{e})$ & 0.4933 & 0.4951 & 0.4942 \\
$A_\Zo(\mathrm{e})$ & 0.3876 & 0.3892 & 0.3872 \\
$A_\Wm(\mathrm{e})/A_\Wp(\mathrm{e})$ & 0.9583 & 0.9488 & 0.9626 \\
$A_\Zo(\mathrm{e})/A_\Wo(\mathrm{e})$ & 0.7857 & 0.7853 & 0.7880 \\
\hline
$A_\Wp(\mu)$ & 0.4594 & 0.4587 & 0.4617 \\
$A_\Wm(\mu)$ & 0.4471 & 0.4519 & 0.4472 \\
$A_\Wo(\mu)$ & 0.4543 & 0.4559 & 0.4557 \\
$A_\Zo(\mu)$ & 0.3978 & 0.3990 & 0.3973 \\
$A_\Wm(\mu)/A_\Wp(\mu)$ & 0.9732 & 0.9614 & 0.9778 \\
$A_\Zo(\mu)/A_\Wo(\mu)$ & 0.8756 & 0.8761 & 0.8796 \\
\hline
\end{tabular}
\end{center}
\end{table}

\begin{table}[!ht] %
\begin{center}
\caption{Uncertainties on acceptances due to theoretical assumptions. The different contributions are due to
ISR plus NNLO effects, factorization and renormalization scales, PDF uncertainties, FSR modeling, and EWK corrections.}
\label{tab:th_results}
\begin{tabular}{|l|ccccc|c|}
	\hline
	Quantity & ISR+NNLO & $\mu_R$,$\mu_F$ Scales & PDF & FSR & EWK & Total \\
	\hline\hline
	$\Wp$ acceptance (e)     & $0.63\%$ & $0.77\%$ & $0.7\%$ & $0.17\%$ & $0.14\%$ & $1.2\%$ \\
	$\Wm$ acceptance (e)     & $0.31\%$ & $0.50\%$ & $1.2\%$ & $0.20\%$ & $0.29\%$ & $1.4\%$ \\
	$\Wo$ acceptance (e)       & $0.53\%$ & $0.34\%$ & $0.6\%$ & $0.13\%$ & $0.14\%$ & $0.9\%$ \\
	$\Zo$ acceptance (e)       & $0.84\%$ & $0.39\%$ & $0.9\%$ & $0.54\%$ & $0.84\%$ & $1.6\%$ \\
	$\Wp/\Wm$ correction (e) & $0.32\%$ & $1.14\%$ & $1.6\%$ & $0.26\%$ & $0.25\%$ & $2.0\%$ \\
	$\Wo/\Zo$ correction (e)     & $0.31\%$ & $0.48\%$ & $0.6\%$ & $0.44\%$ & $1.00\%$ & $1.4\%$ \\
	\hline
	$\Wp$ acceptance $(\mu)$     & $0.72\%$ & $0.49\%$ & $0.9\%$ & $0.34\%$ & $0.14\%$ & $1.3\%$ \\
	$\Wm$ acceptance $(\mu)$     & $0.50\%$ & $0.37\%$ & $1.5\%$ & $0.16\%$ & $0.39\%$ & $1.7\%$ \\
	$\Wo$ acceptance $(\mu)$       & $0.65\%$ & $0.44\%$ & $0.8\%$ & $0.21\%$ & $0.13\%$ & $1.1\%$ \\
	$\Zo$ acceptance $(\mu)$       & $1.08\%$ & $0.20\%$ & $1.1\%$ & $0.25\%$ & $1.08\%$ & $1.9\%$ \\
	$\Wp/\Wm$ correction $(\mu)$ & $0.23\%$ & $0.61\%$ & $1.9\%$ & $0.31\%$ & $0.43\%$ & $2.1\%$ \\
	$\Wo/\Zo$ correction $(\mu)$     & $0.43\%$ & $0.38\%$ & $0.9\%$ & $0.27\%$ & $1.22\%$ & $1.6\%$ \\
	\hline
\end{tabular}
\end{center}
\end{table}


Table~\ref{tab:th_results} summarizes the different theoretical uncertainties on the acceptance
due to ISR and NNLO, higher order effects, PDFs, FSR, and missing
EWK contributions.

The baseline MC generator used to simulate the W and Z signals, {\sc POWHEG}, is
accurate up to the NLO in perturbative QCD, and up to
the leading-logarithmic (LL) order  for soft, nonperturbative QCD effects.
A description accurate to just beyond the next to next to LL (NNLL)
can be attained with a resummation procedure~\cite{Collins-1, Collins-2}.
The {\sc ResBos} generator~\cite{ResBos} implements both the resummation and NNLO calculations,
which are missing in the baseline generator, and its predictions for the W boson $\PT$ spectrum 
show 
%remarkable 
agreement with $\mathrm{p}\bar{\mathrm{p}}$ data at 
$\sqrt{s} = 1.96$~TeV~\cite{ResBosComp}. Final state radiation is incorporated 
in {\sc ResBos} via {\sc PHOTOS}~\cite{PHOTOS}.
The effect of soft nonperturbative effects, hard higher-order effects, and initial-state radiation (ISR),
which are not accounted for in the baseline generator, is studied by comparing {\sc ResBos} 
results with {\sc POWHEG},
and the difference is taken as a systematic uncertainty (second column in Table~\ref{tab:th_results}).

Fixed-order cross section calculations depend on the renormalization ($\mu_R$) and factorization
 ($\mu_F$) scales. 
%Differences found when varying $\mu_R$ and $\mu_F$ are taken to indicate the 
%uncertainty on the cross section from missing higher perturbative orders. 
Higher-order virtual 
processes influence the W and Z boson momentum and rapidity distributions. 
%The fixed-order calculations implemented both by generators and integrators lead to an unnatural 
%dependence on the QCD factorization scale that must be quantified. 
{\sc ResBos} fixes $\mu_R$ and $\mu_F$ to the boson mass, so 
{\sc FEWZ}~\cite{Melnikov:2006kv, Melnikov:2006di} code is used to
estimate the effect of scale dependence of NNLO calculations that is quoted as a systematic uncertainty.
The acceptance is computed by varying up and down the renormalization 
and factorization scales within a factor of two, keeping $\mu_R =\mu_F$.
Half of the maximum excursion range from half to twice the central scale value
is taken as a systematic uncertainty (third column in Table~\ref{tab:th_results}). The PDF uncertainties
from Table~\ref{tab:pdfSyst} are reported in the fourth column of Table~\ref{tab:th_results}
and added in quadrature to the other contributions to determine the total theoretical
uncertainties, shown in the last column.

% On top of higher-order QCD corrections, the effect of EWK corrections, not fully 
% implemented in our baseline MC samples, is estimated using the {\sc HORACE} generator~\cite{HORACE-1, HORACE-2, HORACE-3, HORACE-4},
% which implements both FSR and virtual and nonvirtual corrections. 
% Individual effects are separated and the final-state effects
% are then compared to the {\sc PYTHIA} results, as {\sc PYTHIA} is used for FSR in the 
% {\sc POWHEG} event generation. While {\sc PYTHIA} only partially accounts for NLO EWK corrections 
% by generating QED ISR and FSR with a parton shower approximation, HORACE also implements one-loop
% virtual corrections and photon emission from W boson. The difference between the two generators is taken
% as a systematic uncertainty (fifth column in Table~\ref{tab:th_results}). Moreover, FSR is simulated beyond the
% single-photon emission in HORACE using the parton shower method. The difference due to FSR in HORACE and
% PYTHIA is taken as a contribution to the systematic uncertainty (sixth column in Table~\ref{tab:th_results}). 

At the energy scale of weak boson production, NLO EWK corrections have magnitude of comparable order to NNLO QCD effects. 
In the baseline {\sc POWHEG} samples, QED ISR and FSR are simulated using {\sc PYTHIA}
with a parton shower approximation, while virtual corrections and photon emission from W are missing. 
The magnitude of NLO EWK corrections has been estimated using the {\sc HORACE} event generator~\cite{HORACE-1, HORACE-2, HORACE-3, HORACE-4}
which implements both FSR and virtual and nonvirtual corrections. {\sc HORACE} also uses a parton shower approximation to account for FSR 
beyond single photon emission, and has been interfaced to {\sc PYTHIA} in order to generate MC samples with full detector simulation and reconstruction. 
The difference in acceptances between {\sc PYTHIA} and {\sc HORACE} samples, generated enabling FSR simulation only,
is taken as systematic uncertainty due to FSR modelling (fifth column in Table~\ref{tab:th_results}).
The difference in acceptances between {\sc HORACE} samples simulated using the full suite of corrections and
enabling FSR simulation only is taken as systematic uncertainty due to virtual corrections and radiation from W 
(sixth column in Table~\ref{tab:th_results}). 
The effect of QED ISR on the acceptance was found to be negligible comparing {\sc PYTHIA} 
samples generated with QED ISR enabled and disabled. 
