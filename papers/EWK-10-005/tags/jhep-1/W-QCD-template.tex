\subsection{\texorpdfstring{Modeling of the QCD Background and $\Wen$ Signal Yield}{Modeling of the QCD Background and W-> e nu Signal Yield}}
\label{sec:WQCDbkg}

Three signal extraction methods are used, which give consistent
signal yields. The method described in Section~\ref{sec:AnalyticalFunction}
is used to extract the final result.

\subsubsection{Modeling the QCD Background Shape with an Analytical Function}
\label{sec:AnalyticalFunction}

The $\Wen$ signal is extracted using an unbinned maximum likelihood
(UML) fit to the $\MET$ distribution.
%Signal and EWK background
%distributions are derived from simulation and are validated using dedicated studies.

The shape of the $\MET$ distribution for the QCD background is modeled by a parametric function (modified Rayleigh
distribution) whose expression is
\begin{equation}
f_{\mathrm{QCD}}(\MET) = \MET\exp\left(-\frac{\MET{^2}}{2(\sigma_0+\sigma_1 \MET)^{2}}\right)\,.
\label{eq:rayleigh}
\end{equation}
The fit to a control sample, defined by inverting the track-cluster matching selection
variables $\Delta\eta$, $\Delta\phi$, shown in Fig.~\ref{fig:e-inverted}, illustrates
the quality of the description of the background shape by the parameterized function,
including the region of the signal, at high \MET.
\begin{figure}[htbp]
\begin{center}
\includegraphics[width=0.50\textwidth]{figs/fixedMCyield_normal_model.pdf}
\caption{Fit to the background-dominated control sample defined by inverting
the selection on the track-match variables,
while maintaining the rest of the signal selection.
The blue solid line represents the model used to fit the control data sample. This is a Rayleigh
function plus a floating-yield signal shape that accounts for the signal contamination in the
control region. The magenta dashed line shows the Rayleigh function alone with its parameters estimated
from the combined fit.
}\label{fig:e-inverted}
\end{center}
\end{figure}
To study the systematic uncertainties associated with the background shape, the resolution term in
Eq.~(\ref{eq:rayleigh}) was changed by introducing an additional QCD shape parameter $\sigma_2$,
thus: $\sigma_0 + \sigma_1 \MET + \sigma_2 \MET^2$.

The free parameters of the UML fit are the QCD background yield,
the $\Wo$ signal yield, and the background shape
parameters $\sigma_0$ and $\sigma_1$.
The following signal yields are obtained:
$\WEIYIELD$ for the inclusive sample, $\WEPYIELD$ for the $\Wpen$ sample, and
$\WEMYIELD$ for the $\Wmen$ sample. 
The $\sigma_0$ and $\sigma_1$ values obtained from the fit are   
$\sigma_0$~=~8.56~$\pm$~0.15, $\sigma_1$~=~0.130~$\pm$~0.008 
for the $\Wpen$ sample, and $\sigma_0$~=~8.50~$\pm$~0.15, $\sigma_1$~=~0.139~$\pm$~0.008
for the $\Wmen$ sample. 
%with negligible correlation between the $\Wp$ and $\Wm$ yields.
The fit to the inclusive $\Wen$ sample is displayed
in Fig.~\ref{fig:Wen}, while the fits for the charge-specific
channels are displayed in Fig.~\ref{fig:WenPM}.

\begin{figure}[htbp]
\begin{center}
\includegraphics[width=0.48\textwidth]{figs/w_inc_36pb.pdf}
\includegraphics[width=0.48\textwidth]{figs/w_inc_36pb_log.pdf}
\caption{ \label{fig:Wen}
The $\MET$ distribution for the selected $\Wen$ candidates on
a linear scale (left) and on a logarithmic scale (right).
The points with the error bars represent the data. Superimposed are the
contributions obtained with the fit
for QCD background (violet, dark histogram), all other backgrounds
(orange, medium histogram), and signal plus  background (yellow, light histogram).
The orange dashed line is the fitted signal contribution.
}
\end{center}
\end{figure}

\begin{figure}[htbp]
\begin{center}
\includegraphics[width=0.48\textwidth]{figs/wp_36pb.pdf}
\includegraphics[width=0.48\textwidth]{figs/wm_36pb.pdf}
\caption{ \label{fig:WenPM}
The $\MET$ distributions for the selected W$^+$ (left) and W$^-$ (right) candidates.
The points with the error bars represent the data. Superimposed are the contributions
obtained with the fit for QCD background (violet, dark histogram), all other backgrounds
(orange, medium histogram), and signal plus background (yellow, light histogram).
The orange dashed line is the fitted signal contribution.
}
\end{center}
\end{figure}

The Kolmogorov--Smirnov probabilities for the fits to the charge-specific
channels are $\WEPKSPCOR$ for the $\Wp$ sample and
$\WEMKSPCOR$ for the $W^-$ sample.
\begin{figure}[t!]
\begin{center}
\includegraphics[width=0.48\textwidth]{figs/inc_pfmt_withMVA.pdf}
\includegraphics[width=0.48\textwidth]{figs/inc_pfmt_log_withMVA.pdf}
\caption{ \label{fig:WenMT}
The $\MT$ distribution for the selected $\Wen$ candidates on
a linear scale (left) and on a logarithmic scale (right).
The points with the error bars represent the data. Superimposed are the
contributions obtained with the fit
for QCD background (violet, dark histogram), all other backgrounds
(orange, medium histogram), and signal plus  background (yellow, light histogram).
The orange dashed line is the fitted signal contribution.
}
\end{center}
\end{figure}
Figure~\ref{fig:WenMT} shows the distribution for the inclusive $\Wo$ sample
of the transverse mass, defined as
$\MT=\sqrt{2\Pt\MET (1-\cos(\Delta\phi_{\mathrm{l},\MET}))}$,
where $\Delta\phi_{\mathrm{l},\MET}$ is the azimuthal angle between the
lepton and the $\MET$ directions.

\input{Wenu_FixedShapeTemplates.tex}

\label{sec:e-Wsigextr-ABCDE}

\input{ABCDE.tex}

The results of the three signal extraction methods are summarised in Table~\ref{tab:WsignalCollection}.

\begin{table}[htbp] %
\begin{center}
   \caption[.]{ \label{tab:WsignalCollection}
Comparison of $\Wen$ signal extraction methods. The signal yield of each method is presented together with
its statistical uncertainty.
For the fixed shape and the ABCD methods, the ratios of the signal yields with the analytical function method
are also shown taking into account only the uncorrelated systematics between the methods used in the ratios.}
\begin {tabular} {|l|l|c|c|c|}
\hline
\multicolumn{2}{|l|}{Source}        & $\Wen$           & $\Wpen$           & $\Wmen$            \\
\hline\hline
Analytical fun. &yield    & $\WEIYIELD$      &  $\WEPYIELD$      &  $\WEMYIELD$      \\
\hline
\multirow{2}{*}{Fixed shape} & yield                         & $\WEIftYIELD$    &  $\WEPftYIELD$    &  $\WEMftYIELD$      \\
 & ratio & $\rWEIftYIELD$   &  $\rWEPftYIELD$   &  $\rWEMftYIELD$      \\
\hline
\multirow{2}{*}{ABCD} & yield                                & $\WEIabYIELD$    & $\WEPabYIELD$     & $\WEMabYIELD$    \\
 & ratio       & $\rWEIabYIELD$   & $\rWEPabYIELD$    & $\rWEMabYIELD$    \\
\hline
\end {tabular}
\end{center}
\end{table}

