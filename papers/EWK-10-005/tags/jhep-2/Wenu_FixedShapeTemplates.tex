\subsubsection{Modeling the QCD Background Shape with a Fixed Distribution}
\label{sec:e-Wsigextr-FixedTemplate}

In this approach the QCD shape is extracted directly from data using a 
control sample obtained by inverting a subset of the requirements used to select the 
signal. After fixing the shape from data,
only the normalization is allowed to float in the fit.  

The advantage of this approach 
is that detector effects, such as anomalous signals in the ca\-lo\-ri\-me\-ters or 
dead ECAL towers, are automatically reproduced in the QCD shape, since these 
effects are not affected by the selection inversion used to define the control sample.
The track-cluster matching variable $\Delta\eta$ is found
to have the smallest correlation with $\MET$ and is therefore chosen as the one 
to invert in order to suppress the signal and obtain the QCD control sample.  
Requirements on isolation and $H/E$ are the same as for the signal 
selection since these variables show significant correlation with \MET. 
\begin{figure}[h!]
  \begin{center}
    \includegraphics*[angle=-90,width=0.55\textwidth]{figs/antiselShape.pdf}
    \caption{Normalised \MET distribution for QCD and $\gamma$+jet simulated 
events passing the signal selection (solid histogram) compared 
to the normalised distribution for events from all simulated samples passing
the same inverted selection criteria used to obtain the control sample in data
(dashed histogram).}
    \label{fig:antiselShape}
  \end{center}
\end{figure}

The shape of the $\MET$ distribution for QCD and $\gamma$+jet simulated 
events passing the signal selection is compared 
to the \MET distribution for a simulated control sample composed of all simulated samples (signal and 
all backgrounds, weighted according to the 
theoretical production cross sections), after applying the same anti-selection 
as in data (Fig.~\ref{fig:antiselShape}).

\begin{figure}[htb]
  \begin{center}
    \includegraphics*[width=0.5\textwidth]{figs/Wenu_pfMET_lin.pdf}
%    \includegraphics*[width=0.48\textwidth]{figs/Wenu_pfMT_lin.pdf}
    \caption{Result of the fixed-shape fit to the $\MET$ distribution for all W candidates.
The points with the error bars represent the data.   Superimposed are the results
of the maximum likelihood fit for QCD background (violet, dark histogram), other backgrounds
(orange, medium histogram), and signal plus  background (yellow, light histogram).
The orange dashed line (left plot) is the fit contribution from signal.
}
    \label{fig:resultAll}
  \end{center}
\end{figure}

The difference in the $\MET$ distributions from the signal and inverted selections
is found to be predominantly due to two effects, which can be reduced by applying corrections.
The first effect is due to a large difference in the distribution of the output of a multivariate 
analysis (MVA) used for electron identification in the PF algorithm, between the selected events 
and the control sample. The value of the MVA output determines whether an electron candidate 
is treated by the PF algorithm as a genuine electron, or as a superposition of a charged pion and 
a photon, with track momentum and cluster energy each contributing separately to $\MET$. The control 
sample contains a higher fraction of electron candidates in the latter category, resulting in a bias on 
the $\MET$ shape. A correction is derived to account for this.
The second effect comes from the signal 
contamination in the control sample. The size of the contamination (1.17$\%$) is 
measured from data, using the \TNP technique with $\Zee$  
events, by measuring the efficiency for a signal electron to pass the 
control sample selection.


The results of the inclusive fit to the $\MET$ distribution with the fixed QCD background shape
are shown in Fig.~\ref{fig:resultAll}; the only free parameters in the extended 
maximum likelihood fit are the QCD and signal yields.
By applying this second method the following yields are obtained:
$135\,982 \pm 388$ (stat.) for the inclusive sample, $81\,286 \pm 302$ (stat.) for the $\Wpen$ sample, and
$54\,703 \pm 249$ (stat.) for the $\Wmen$ sample.
%with negligible correlation between the $\Wp$ and $\Wm$ yields.
%The quoted yields are from the $\MET$ fits. 
The ratios of the inclusive, $\Wpen$, and $\Wmen$ yields between this 
method and the parameterized
QCD shape method are $0.997 \pm 0.005$, $0.997 \pm 0.005$, and $0.999 \pm 0.005$, respectively, 
considering only the uncorrelated systematic uncertainties between the two methods.

