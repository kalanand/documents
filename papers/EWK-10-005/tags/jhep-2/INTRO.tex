\section{Introduction}
\label{sec:introduction}

This paper describes a measurement carried out by the Compact Muon Solenoid (CMS) Collaboration
of the inclusive production cross sections for W and Z bosons in pp collisions at
$\sqrt{s} = 7\TeV$. The vector bosons are observed  via their decays to electrons and muons.
In addition, selected cross-section ratios are presented. Precise determination
of the production cross sections and their ratios provide an important
test of the standard model (SM) of particle physics.

The production of the electroweak (EWK) gauge bosons in pp collisions 
proceeds mainly via the weak Drell--Yan (DY) process~\cite{DY} consisting of
the annihilation of a quark and an antiquark.
The production process $\pp \ra \Wo + X$ is dominated by  
$\mathrm{u}\bar{\mathrm{d}}\ra\Wp$ and $\mathrm{d}\bar{\mathrm{u}}\ra\Wm$, 
while  $\pp \ra \Zo + X$ is dominated by $\mathrm{u}\bar{\mathrm{u}}$ and
$\mathrm{d}\bar{\mathrm{d}}\ra\Zo$.

Theoretical predictions of the total W and Z production cross sections
are determined from parton-parton cross sections convolved with parton 
distribution functions (PDFs), incorporating higher-order quantum chromodynamics (QCD) effects. 
PDF uncertainties, as well as higher-order QCD and EWK radiative corrections,
limit the precision of current theoretical predictions, which are available at 
next-to-leading order (NLO)~\cite{nlo1, nlo2, nlo3} and next-to-next-to-leading order 
(NNLO)~\cite{nnlo1, nnlo2, nnlo3, nnlo4, nnlo5} in perturbative QCD.

The momentum fractions of the colliding partons $x_1$, $x_2$ are related to the
vector boson masses ($m_{\Wo/\Zo}^{2} = s x_1 x_2$) and
rapidities ($y = \frac{1}{2}\ln(x_1/x_2)$). 
%At central rapidity small fractional momenta of the participating partons are involved,
%$x\sim 0.01$. At larger rapidities one parton should have
%lower $x$ and the other higher $x$; considering the measurable rapidity range, 
%$ | y | \le 2.5$, the values of $x$ remain in the range $10^{-3} \le x \le 0.1$.
Within the accepted rapidity interval, $ | y | \le 2.5$, the values 
of $x$ are in the range $10^{-3} \le x \le 0.1$.

Vector boson production in proton-proton collisions requires at least one sea quark,
while two valence quarks are typical of $\mathrm{p}\bar{\mathrm{p}}$ collisions. 
Furthermore, given the high scale of the process, 
${\hat s} = m_{\Wo/\Zo}^2\sim 10^4\GeV^2$, the gluon is the dominant parton 
in the proton so that the scattering sea quarks are mainly generated by the 
$\mathrm{g} \rightarrow \mathrm{q} \bar{\mathrm{q}}$ splitting process.
For this reason, the precision of the cross section predictions
at the Large Hadron Collider (LHC) depends crucially on the uncertainty 
in the momentum distribution of the gluon.
Recent measurements from HERA~\cite{HERApdf} and the
Tevatron~\cite{TevatronPdf_1, TevatronPdf_2, 
%TevatronPdf_3, 
TevatronPdf_4, TevatronPdf_5, TevatronPdf_6, TevatronPdf_7, TevatronPdf_8, TevatronPdf_9, TevatronPdf_10} 
reduced the PDF uncertainties, leading to more precise
cross-section predictions at the LHC. 

%The leptonic decay modes of the W and Z bosons are used, where the lepton can be 
%either an electron or a muon. 
%The measured average value of W branching fractions to leptonic modes
%is $(10.80\pm0.09)\%$~\cite{PDG}, while the average branching fraction 
%for leptonic decay modes of the Z boson 
%is measured to be $(3.3658\pm0.0023)\%$.

The W and Z production cross sections and their ratios were
previously measured by \mbox{ATLAS}~\cite{WZATLAS:2010} with
an integrated luminosity of 320~nb$^{-1}$ and by 
CMS~\cite{WZCMS:2010} with 2.9~pb$^{-1}$. 
%
% No need to quote conf. note.
%
%ATLAS recently published updated results with 35~pb$^{-1}$ in a conference
%note~\cite{WZATLAS:2011}.
This paper presents an update with the full integrated luminosity recorded by CMS at the LHC 
in 2010, corresponding to 36~pb$^{-1}$.
%%
%% add here the G(W)
%%
The leptonic branching fraction and the width of the W boson can be extracted from the 
measured W/Z cross section ratio 
using the NNLO predictions for the total W and Z cross sections and the measured 
values of the Z boson total and leptonic partial widths~\cite{LEPZ}, together with the SM
prediction for the leptonic partial width of the W. 

This paper is organized as follows: in Section~\ref{sec:detector} the CMS detector is presented,
with particular attention to the subdetectors 
used to identify charged leptons and to infer the presence of neutrinos. 
Section~\ref{sec:samples} describes the data sample and simulation 
used in the analysis. The selection of the W and Z candidate 
events is discussed in Section~\ref{sec:eventSelection}. Section~\ref{sec:acceptance} describes 
the calculation of the geometrical and kinematic acceptances. 
%
% Actually: in the section below we also present signal efficiencies... L.L.
%
The methods used to determine the 
reconstruction, selection, and trigger efficiencies of the leptons within 
the experimental acceptance are presented in Section~\ref{sec:efficiencies}. 
The signal extraction methods for the W and Z channels, as well as 
the background contributions to the
candidate samples, are discussed in Sections~\ref{sec:WsignalExtraction} 
and~\ref{sec:ZsignalExtraction}. Systematic uncertainties are
discussed in Section~\ref{sec:systematics}. The calculation of the 
total cross sections, along with the resulting values of the ratios and derived quantities, 
are summarized in Section~\ref{sec:results}. In the same section we also report the cross 
sections as measured within the fiducial and kinematic acceptance (after final-state QED 
radiation corrections), thereby eliminating the PDF uncertainties from the results.  


