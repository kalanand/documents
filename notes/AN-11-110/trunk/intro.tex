\section{Introduction}
\label{sec:intro}
% ---- ---- ---- ---- ---- ---- ---- ---- ---- ---- ---- ---- ---- ---- ---- ---- ---- ---- ---- ---- ---- ---- ----

The Standard Model (SM) of particle physics successfully describes the majority of high-energy
experimental data~\cite{pdg}. One of the key remaining questions is the origin of the masses of
W and Z bosons.  In the simplest implementation of the SM, it is attributed to the spontaneous
breaking of electroweak symmetry caused by a new scalar field. %~\cite{Higgs1, Higgs2, Higgs3} 
The existence of the associated field quantum, the Higgs boson, has yet to be experimentally confirmed.
Therefore, the search for the Higgs boson is arguably one of the most
important studies being done at the LHC~\cite{lhcmachine}. For Higgs
masses above or near the threshold for decay into two vector bosons,
the decay modes of choice are dominated by those decays because of
their large branching fractions.
%  Looking at the experimentally
% accessible subsequent decay modes of the vector bosons, the relative
% event yields are shown in Table~\ref{tab:intro:evtyield}.
% %%%%%%%%%%%%%%%%%
% \begin{table}[htb]
%   \caption{Higgs Decay Modes into Vector Boson Pairs.}
%   \label{tab:intro:evtyield}
%   \begin{center}
%     \begin{tabular}{l|l|l|c} 
%       \hline \hline
%       Decay Mode & $\mathcal{B}$ & Relative $\mathcal{B}$ & Higgs Mass  Peak   \\\hline \hline
%       $H \rightarrow WW \rightarrow \ell \nu jj$         &0.2    &1      & Yes \\
%       $H \rightarrow WW \rightarrow \ell \nu \ell \nu$   &0.033  &1/6    & No  \\
%       $H \rightarrow ZZ \rightarrow \ell \ell \nu \nu$   &0.0089 &1/22.5 & No  \\
%       $H \rightarrow ZZ \rightarrow \ell \ell jj$        &0.031  &1/6.45 & Yes \\
%       $H \rightarrow ZZ \rightarrow \ell \ell bb$        &0.0067 &1/29.8 & Yes \\
%       $H \rightarrow ZZ \rightarrow \ell \ell \ell \ell$ &0.0045 &1/44.4 & Yes \\
%       \hline \hline
%     \end{tabular}
%   \end{center}
% \end{table}
% %%%%%%%%%%%%%%%%%
It is clear that the events where one $W$ decays leptonically, which
provides the main trigger elements, while the other decays
hadronically have the largest branching fraction and have a
reconstructable Higgs mass peak~\cite{intro2}. 

This note contains the analysis that sets a limit on the Higgs boson cross-section
based on this decay mode. 
The main experimental issue is to control the large $W$ plus jets 
background sufficiently well that the advantages of using this 
final state are realized.

The analysis selects events with one well identified and isolated lepton, 
large missing transverse energy and at least two high \pt jets.
The note is structured as follows. 
A discussion about the data samples used in the analysis and the trigger selections
is presented in Sections~\ref{sec:MCexpectations} and \ref{sec:technicalities}.  
The physics objects reconstruction is discussed in Section~\ref{sec:firstStep}.
The lepton selection and other preselection requirements are described in detail 
in Section~\ref{sec:firstStep} and \ref{sec:dataMCcomparisons}.

A first limit extraction is performed with a few set of selections applied on top of these,
as described in Section~\ref{sec:firstExtraction}.
To enhance the exclusion power of this analysis, 
tighter selections are put in place.

After the preselections,
the signal-over-background ratio is enhanced by means of a selection on a MVA discriminant, 
designed in order to control the background while preserving as much as possible the difference in shape
with respect to the signal (Section~\ref{sec:mvaoptimization}).
The input variables of the MVA exploit the decay angles of the four-body mass system, 
the kinematics of the entire four-body system. For high mass points, a quark-gluon discriminating
variable, already exploited in the CMS search for a Standard Model Higgs boson in its $\ell{}\ell{}jj$ decay,
is also used in parallel with the MVA discriminant.
The MVA variable definition is optimized with dedicated trainings for each Higgs mass hypothesis case, 
for each lepton flavour ($e$, $\mu$) and for each jets multiplicity (2 jets, 3 jets) independently.
In this way, 48 different configurations are obtained.

The main background contaminating the signal region is W+jets.
Its $m_{\ell{}\nu{}jj}$ shape is extrapolated from sidebands in the $m_{jj}$ distribution 
through Monte Carlo based factors (Section~\ref{sec:wjetsBackground}).
Also the QCD shapes come from a data-driven determination, as described in Section~\ref{sec:dataDrivenQCD}.
The remaining background shapes come from the Monte Carlo.
The normalization of the backgrounds in the signal region 
is measured by fitting their Monte Carlo shapes on the same sidebands in the $m_{jj}$ variable 
(Section~\ref{sec:mjjfitfornormal}).
The determination of the $m_{jj}$ distribution for W+jets is described in Section~\ref{sec:wjetsShape}.

The systematic uncertainties present in the signal description are described in Section~\ref{sec:systematics}.
Section~\ref{sec:limitExtraction} describes the obtained limits on the Standard Model Higgs cross section
and Section~\ref{sec:conclusions} closes this work.

