\section{Data Driven QCD Determination}
\label{sec:dataDrivenQCD}
% ---- ---- ---- ---- ---- ---- ---- ---- ---- ---- ---- ---- ---- ---- ---- ---- ---- ---- ---- ---- ---- ---- ----

A background from QCD multijet events comes from 3- or 4-jet events
with one jet passing the lepton criteria as a 'fake'. However, it is
not practical to generate sufficient MC to create a statistically
significant sample that passes the selection criteria. Therefore we
rely on a data-driven approach in which the isolation-inverted samples
from data, which mirror the QCD background, are used instead.
Specifically, we perform a two-component simultaneous fit to data of
the MET distribution in order to obtain the fraction of QCD events in
the data; the two components are a data-based QCD sample and a
MC-based W+Jets sample.

The data sample is constrained to a specific trigger epoch from Run
2011A, comprising approximately 200~$\pbinv$, in which the isolation
requirement in the trigger was loose enough to allow for the inversion
and thereby provide sufficient statistics for the study. The QCD sample is
obtained by inverting the lepton isolation in this data sample to be
$>0.1$ (default selection uses Iso$_{mu}<0.1$ and Iso$_{el}<0.05$).
In order to increase statistics for the QCD sample we also relax the
MET cut from 30~GeV to 20~GeV and (for electrons) the ID requirement
to WP90-like.  The MC W+Jets and target data samples are obtained by
similarly relaxing the MET cut and ID requirements.

The fraction of QCD events in data is then obtained from a
simultaneous fit of the two components on the MET distribution
performed before the MVA cut, again to maintain sufficient statistics.
The W+jets normalization is left free to float, as well as the QCD;
the results are shown in Figure~\ref{fig:QCDTemplateFit_MET}.  We
subsequently adjust the fraction applicable to our analysis by
removing the portion of events for which 20$<$MET$<$30~GeV, and
estimate the fraction of QCD relative to the data as shown in
Table~\ref{tab:qcdfrac}.  A separate study verified
that the fraction of QCD was not sensitive to the MVA selection;
nevertheless, to account for discrepancies in template modeling
(e.g. using W+jets MC as a proxy for all non-QCD processes) and the
fact that this fraction is estimated prior to the MVA cut, a very
large uncertainty is conservatively assumed. The final fraction of QCD
events in data is fed to the $m_{jj}$ fit for determination of the QCD
normalization, and the four-body shape of the QCD distribution is fed
to the final four-body total background determination in preparation
for the limit setting procedure.

Note that the W transverse mass distributions from the data and MC are
statistically consistent, as shown in
Figure~\ref{fig:QCDCutLoosening_MET} for muons; for electrons there's
an insufficient number of MC events to make the comparison.  The MET
for QCD processes is also 'fake'; i.e., it originates from badly
measured jets, and therefore has an exponentially falling spectrum.
By contrast, all other backgrounds exhibit a wide peak at $\sim
35$~GeV from a real neutrino (with the exception of Z+Jets, where the
MET is the result of a poorly measured lepton).

%%  $el_{2J}$ $frac_{QCD}=0.0617\pm 0.00384$,
%%  $el_{3J}$ $frac_{QCD}=0.0213\pm 0.00678$.
%%
%% $\mu_{2J}$ $frac_{QCD}=0.001625\pm 0.004214$,
%% $\mu_{3J}$ $frac_{QCD}=0.0\pm 0.0040797$,


\begin{table}[bthp]
\begin{center}
  \begin{tabular}{l c c}
    \hline  \hline
     & 2 jets & 3 jets \\
    \hline  
    electron  &	6.2 $\pm$ 0.4\% & 2.1 $\pm$ 0.7\% \\
    muon      &	0.2 $\pm$ 0.4\% & 0.0 $\pm$ 0.4\% \\
    \hline  \hline
  \end{tabular}
\end{center}
\caption{\label{tab:qcdfrac} Estimates of the percentage of QCD in data
for the muon and electron datasets after selection, separated into 2- and
3-jet bins.}
\end{table}

\subsection{QCD Uncertainties}
\label{sec:qcd_Uncertainty}
% .... .... .... .... .... .... .... .... .... .... .... .... .... .... .... .... .... .... .... .... .... .... ....

When performing the $m_{jj}$ fit  the QCD yield 
is Gaussian-constrained with a mean given by the value shown in
Table~\ref{tab:qcdfrac}.
In the case of electrons, the error on the QCD fraction
is small and we (conservatively) estimate
the uncertainty to be one half of the expected value. For muons the 
uncertainty is the error on the relative fraction (i.e.,
0.4\% for both 2- and 3-jet bins).
When fitting the sum of electron and muon data, the uncertainties
are combined using the standard error propagation machinery.

\subsection{Cross-Checks}
% .... .... .... .... .... .... .... .... .... .... .... .... .... .... .... .... .... .... .... .... .... .... ....

In order to ensure that our inverted selection provides a consistent representation of
QCD events, 
we fit the QCD with a Raileigh Function: $xe^{-x^2/2(\sigma_0+\sigma_1x)^2}$,
used during the inclusive cross section measurements~\cite{WZCMS:2010}. 
As can be seen from Fig.~\ref{fig:QCDMETRaileighFit},
the function accurately fits the overall shape as well as the parameter
corresponding to the intrinsic MET resolution ($\sigma_0\simeq 10$~GeV).

% we perform the following cross-checks:
% \begin{itemize}
% \item Fit the QCD with a Raileigh Function: $xe^{-x^2/2(\sigma_0+\sigma_1x)^2}$,
% used during the inclusive cross section measurements~\cite{WZCMS:2010}. 
% As can be seen from Fig.~\ref{fig:QCDMETRaileighFit},
% the function accurately fits the overall shape as well as the parameter
% corresponding to the intrinsic MET resolution ($\sigma_0\simeq 10$~GeV).
% \item Compare the W transverse mass shapes for the data sidebands with MET$>20$~GeV vs 
% MET$>30$~GeV (Fig.~\ref{fig:QCDMETCutsWmTShape}). Naturally, events with MET$>30$~GeV do not have the
% same exponential falloff, since they contain a higher percentage of W's.
% \item Examine the impact of setting Iso$>0.1$, rather than Iso$>0.2$.
% We compare the MET (Fig.~\ref{fig:QCDISOCutsMETShape}) and W transverse mass
% (Fig.~\ref{fig:QCDISOCutsWmTShape}) distributions, and conclude that there is no statistically
% significant discrepancy introduced by the looser isolation requirement.
% \end{itemize}


%%%%%%%%%%%%%%%%%%%%%%%%%%%%
%%%%%%%
\begin{figure}[h!] {\centering
\unitlength=0.33\linewidth
\includegraphics[width=0.48\textwidth]{plots/2012_QCD/QCDDataVSMC_Muons2J_MET.pdf}
\caption{ Comparison of the MET shapes for MC vs data-driven muon QCD events in the 2-jet bin. The two are statistically consistent.} 
\label{fig:QCDCutLoosening_MET}
}
\end{figure}
%%%%%%%
%%%%%%%
\begin{figure}[h!] {\centering
\unitlength=0.33\linewidth
\includegraphics[width=0.48\textwidth]{plots/2012_QCD/TemplateFit_MET_mu2j.pdf}
\put(-0.80,0.0){(a)} 
\unitlength=0.33\linewidth
\includegraphics[width=0.48\textwidth]{plots/2012_QCD/TemplateFit_MET_mu3j.pdf}
\put(-0.80,0.0){(b)} \\
\unitlength=0.33\linewidth
\includegraphics[width=0.48\textwidth]{plots/2012_QCD/TemplateFit_MET_el2j.pdf}
\put(-0.80,0.0){(c)} 
\unitlength=0.33\linewidth
\includegraphics[width=0.48\textwidth]{plots/2012_QCD/TemplateFit_MET_el3j.pdf}
\put(-0.80,0.0){(d)} 
\caption{MET distributions fit to the QCD and W$jj$ templates for: (a) muons - 2-jet bin, (b) muons - 3-jet bin, (c) electrons - 2-jet bin, (d) electrons - 3-jet bin.} 
\label{fig:QCDTemplateFit_MET}
}
\end{figure}
%%%%%%%
%%%%%%%
\begin{figure}[h!] {\centering
\unitlength=0.33\linewidth
\includegraphics[width=0.48\textwidth]{plots/2012_QCD/RaileighFitQCD_mu2j.pdf}
\put(-0.80,0.0){(a)} 
\unitlength=0.33\linewidth
\includegraphics[width=0.48\textwidth]{plots/2012_QCD/RaileighFitQCD_mu3j.pdf}
\put(-0.80,0.0){(b)} \\
\unitlength=0.33\linewidth
\includegraphics[width=0.48\textwidth]{plots/2012_QCD/RaileighFitQCD_el2j.pdf}
\put(-0.80,0.0){(c)} 
\unitlength=0.33\linewidth
\includegraphics[width=0.48\textwidth]{plots/2012_QCD/RaileighFitQCD_el3j.pdf}
\put(-0.80,0.0){(d)} 
\caption{QCD MET distributions fitted with a Raileigh Function for: (a) muons - 2-jet bin, (b) muons - 3-jet bin, (c) electrons - 2-jet bin, (d) electrons - 3-jet bin.} 
\label{fig:QCDMETRaileighFit}
}
\end{figure}
%%%%%%%
%%%%%%%
% \begin{figure}[h!] {\centering
% \unitlength=0.33\linewidth
% \includegraphics[width=0.48\textwidth]{plots/2012_QCD/METShapeComp_mu2j.pdf}
% \put(-0.80,0.0){(a)} 
% \unitlength=0.33\linewidth
% \includegraphics[width=0.48\textwidth]{plots/2012_QCD/METShapeComp_mu3j.pdf}
% \put(-0.80,0.0){(b)} \\
% \unitlength=0.33\linewidth
% \includegraphics[width=0.48\textwidth]{plots/2012_QCD/METShapeComp_el2j.pdf}
% \put(-0.80,0.0){(c)} 
% \unitlength=0.33\linewidth
% \includegraphics[width=0.48\textwidth]{plots/2012_QCD/METShapeComp_el3j.pdf}
% \put(-0.80,0.0){(d)} 
% \caption{ QCD W transverse mass shapes with MET$>20$~GeV vs MET$>30$~GeV for: (a) muons - 2-jet bin, (b) muons - 3-jet bin, (c) electrons - 2-jet bin, (d) electrons - 3-jet bin.} 
% \label{fig:QCDMETCutsWmTShape}
% }
% \end{figure}
%%%%%%%
%%%%%%%
% \begin{figure}[h!] {\centering
% \unitlength=0.33\linewidth
% \includegraphics[width=0.48\textwidth]{plots/2012_QCD/ISOShapeComp_MET_mu2j_g01vg02.pdf}
% \put(-0.80,0.0){(a)} 
% \unitlength=0.33\linewidth
% \includegraphics[width=0.48\textwidth]{plots/2012_QCD/ISOShapeComp_MET_mu3j_g01vg02.pdf}
% \put(-0.80,0.0){(b)} \\
% \unitlength=0.33\linewidth
% \includegraphics[width=0.48\textwidth]{plots/2012_QCD/ISOShapeComp_MET_el2j_g01vg02.pdf}
% \put(-0.80,0.0){(c)} 
% \unitlength=0.33\linewidth
% \includegraphics[width=0.48\textwidth]{plots/2012_QCD/ISOShapeComp_MET_el3j_g01vg02.pdf}
% \put(-0.80,0.0){(d)} 
% \caption{ QCD MET shapes with Iso$>0.1$ vs Iso$>0.2$ for: (a) muons - 2-jet bin, (b) muons - 3-jet bin, (c) electrons - 2-jet bin, (d) electrons - 3-jet bin.} 
% \label{fig:QCDISOCutsMETShape}
% }
% \end{figure}
%%%%%%%
%%%%%%%
% \begin{figure}[h!] {\centering
% \unitlength=0.33\linewidth
% \includegraphics[width=0.48\textwidth]{plots/2012_QCD/ISOShapeComp_WmT_mu2j_g01vg02.pdf}
% \put(-0.80,0.0){(a)} 
% \unitlength=0.33\linewidth
% \includegraphics[width=0.48\textwidth]{plots/2012_QCD/ISOShapeComp_WmT_mu3j_g01vg02.pdf}
% \put(-0.80,0.0){(b)} \\
% \unitlength=0.33\linewidth
% \includegraphics[width=0.48\textwidth]{plots/2012_QCD/ISOShapeComp_WmT_el2j_g01vg02.pdf}
% \put(-0.80,0.0){(c)} 
% \unitlength=0.33\linewidth
% \includegraphics[width=0.48\textwidth]{plots/2012_QCD/ISOShapeComp_WmT_el3j_g01vg02.pdf}
% \put(-0.80,0.0){(d)} 
% \caption{ QCD W transverse mass shapes with Iso$>0.1$ vs Iso$>0.2$ for: (a) muons - 2-jet bin, (b) muons - 3-jet bin, (c) electrons - 2-jet bin, (d) electrons - 3-jet bin.} 
% \label{fig:QCDISOCutsWmTShape}
% }
% \end{figure}
%%%%%%%

