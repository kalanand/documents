% Customizable fields and text areas start with % >> below.
% Lines starting with the comment character (%) are normally removed before release outside the collaboration, but not those comments ending lines

% svn info. These are modified by svn at checkout time.
% The last version of these macros found before the maketitle will be the one on the front page,
% so only the main file is tracked.
% Do not edit by hand!
\RCS$Revision: 116895 $
\RCS$HeadURL: svn+ssh://svn.cern.ch/reps/tdr2/notes/AN-11-110/trunk/AN-11-110.tex $
\RCS$Id: AN-11-110.tex 116895 2012-04-18 17:18:47Z dimatteo $
%%%%%%%%%%%%% ptdr definitions %%%%%%%%%%%%%%%%%%%%%
%\input{ptdr-definitions} %These have been replaced by the equivalent style file

%%%%%%%%%%%%%%%  Title page %%%%%%%%%%%%%%%%%%%%%%%%
\cmsNoteHeader{AN-11-110} % This is over-written in the CMS environment: useful as preprint no. for export versions
% >> Title: please make sure that the non-TeX equivalent is in PDFTitle below
\title{Search for the Standard Model Higgs boson in the decay H$\to$WW$\to\ell\nu{}jj$}

% >> Authors
%Author is always "The CMS Collaboration" for PAS and papers, so author, etc, below will be ignored in those cases
%For multiple affiliations, create an address entry for the combination
\address[ttu]{Texas Tech University, Lubbock, Texas, USA}
\address[fnal]{Fermi National Accelerator Laboratory, Batavia, Illinois, USA}
\address[mib]{Milano-Bicocca University and INFN, Milan, Italy}
\address[uva]{University of Virginia, Charlottesville, Virginia, USA}
\address[wsu]{Wayne State University, Detroit, Michigan, USA}
\address[tamu]{Texas A\&M University, College Station, Texas, USA}
\address[nebr]{University of Nebraska at Lincoln, Nebraska, USA}
\address[und]{University of Notre Dame, Notre Dame, Indiana, USA}
\address[roma]{University of Rome, La Sapienza, Rome, Italy}
\address[uerj]{Universidade do Estado do Rio de Janeiro (UERJ), Brazil}
\address[ham]{University of Hamburg, Hamburg, Germany}
\address[pku]{University of Peking, China}
\address[cern]{CERN}


\author[ttu]{Nural~Akchurin}
\author[fnal]{Jake~Anderson}
\author[mib]{Andrea~Benaglia}
\author[fnal]{Andrew Beretvas}
\author[fnal]{Jeffrey~Berryhill}
\author[fnal]{Pushpa~Bhat}
\author[uva]{Sarah~Boutle}
\author[wsu]{Chris~Clarke}
\author[mib]{Fabio~Colombo}
\author[roma]{Daniele~del~Re}
\author[mib]{Leonardo~Di~Matteo}
\author[ttu]{Phil~Dudero}
\author[tamu]{Ricardo~Eusebi}
\author[cern]{Pietro~Govoni}
\author[fnal]{Dan~Green}
\author[uva]{Joey~Goodell}
\author[wsu]{Robert~Harr}
\author[nebr]{Pratima~Jindal}
\author[ham]{Gordon~Kaussen}
\author[wsu]{Kristina~Krylova}
\author[und]{Kevin~Lannon}
\author[ttu]{Sung-Won~Lee}
\author[pku]{Shuai~Liu}
\author[und]{Wuming~Luo}
\author[mib]{Andrea~Massironi}
\author[wsu]{Kellen~McGee}
\author[fnal]{Kalanand~Mishra}
\author[uva]{Chris~Neu}
\author[tamu]{Ilya~Osipenkov}
\author[tamu]{Alexx~Perloff}
\author[roma]{Luca~Pernie}
\author[wsu]{Sasha~Sakharov}
\author[wsu]{Kevin~Siehl} 
\author[und]{Jason~Slaunwhite}
\author[uerj]{Andre~Sznajder}
\author[fnal]{Nhan~V.~Tran} 
\author[uva]{John~Wood}
\author[fnal]{Fan~Yang}
\author[fnal]{Francisco~Yumiceva}

% useful shortcuts
%\renewcommand{\LUMI}{4.7~fb$^{-1}$} %THIS LATEX THING BY NOW DOES NOT WORK

% >> Date
% The date is in yyyy/mm/dd format. Today has been
% redefined to match, but if the date needs to be fixed, please write it in this fashion.
% For papers and PAS, \today is taken as the date the head file (this one) was last modified according to svn: see the RCS Id string above.
% For the final version it is best to "touch" the head file to make sure it has the latest date.
\date{\today}

% >> Abstract
% Abstract processing:
% 1. **DO NOT use \include or \input** to include the abstract: our abstract extractor will not search through other files than this one.
% 2. **DO NOT use %**                  to comment out sections of the abstract: the extractor will still grab those lines (and they won't be comments any longer!).
% 3. **DO NOT use tex macros**         in the abstract: External TeX parsers used on the abstract don't understand them.
\abstract{This note describes the search strategy for the Higgs boson in the H$\to$WW$\to\ell\nu jj$ final state
with the full statistics accumulated during the 2011 LHC run, at the centre-of-mass energy of $\sqrt{s}=7\TeV$.}

% >> PDF Metadata
% Do not comment out the following hypersetup lines (metadata). They will disappear in NODRAFT mode and are needed by CDS.
% Also: make sure that the values of the metadata items are sensible. For APS submissions, they are automatically converted to APS keywords.
\hypersetup{%
pdfauthor={Pietro Govoni, Kalanand Mishra, Fan Yang, Ilya Osipenkov},%
pdftitle={Search for the Standard Model Higgs boson in the decay channel HToWWTolvjj},%
pdfsubject={CMS},%
pdfkeywords={CMS, physics, Higgs}}

\maketitle %maketitle comes after all the front information has been supplied

%%%%%%%%%%%%%%%%%%%%%%%%%%%%%%%%  Begin text %%%%%%%%%%%%%%%%%%%%%%%%%%%%%
%% **DO NOT REMOVE THE BIBLIOGRAPHY** which is located before the appendix.
%% You can take the text between here and the bibiliography as an example which you should replace with the actual text of your document.
%% If you include other TeX files, be sure to use "\input{filename}" rather than "\input filename".
%% The latter works for you, but our parser looks for the braces and will break when uploading the document.
%%%%%%%%%%%%%%%s

\tableofcontents
\clearpage{}
\section{Introduction}
\label{sec:intro}
% ---- ---- ---- ---- ---- ---- ---- ---- ---- ---- ---- ---- ---- ---- ---- ---- ---- ---- ---- ---- ---- ---- ----

The Standard Model (SM) of particle physics successfully describes the majority of high-energy
experimental data~\cite{pdg}. One of the key remaining questions is the origin of the masses of
W and Z bosons.  In the simplest implementation of the SM, it is attributed to the spontaneous
breaking of electroweak symmetry caused by a new scalar field. %~\cite{Higgs1, Higgs2, Higgs3} 
The existence of the associated field quantum, the Higgs boson, has yet to be experimentally confirmed.
Therefore, the search for the Higgs boson is arguably one of the most
important studies being done at the LHC~\cite{lhcmachine}. For Higgs
masses above or near the threshold for decay into two vector bosons,
the decay modes of choice are dominated by those decays because of
their large branching fractions.
%  Looking at the experimentally
% accessible subsequent decay modes of the vector bosons, the relative
% event yields are shown in Table~\ref{tab:intro:evtyield}.
% %%%%%%%%%%%%%%%%%
% \begin{table}[htb]
%   \caption{Higgs Decay Modes into Vector Boson Pairs.}
%   \label{tab:intro:evtyield}
%   \begin{center}
%     \begin{tabular}{l|l|l|c} 
%       \hline \hline
%       Decay Mode & $\mathcal{B}$ & Relative $\mathcal{B}$ & Higgs Mass  Peak   \\\hline \hline
%       $H \rightarrow WW \rightarrow \ell \nu jj$         &0.2    &1      & Yes \\
%       $H \rightarrow WW \rightarrow \ell \nu \ell \nu$   &0.033  &1/6    & No  \\
%       $H \rightarrow ZZ \rightarrow \ell \ell \nu \nu$   &0.0089 &1/22.5 & No  \\
%       $H \rightarrow ZZ \rightarrow \ell \ell jj$        &0.031  &1/6.45 & Yes \\
%       $H \rightarrow ZZ \rightarrow \ell \ell bb$        &0.0067 &1/29.8 & Yes \\
%       $H \rightarrow ZZ \rightarrow \ell \ell \ell \ell$ &0.0045 &1/44.4 & Yes \\
%       \hline \hline
%     \end{tabular}
%   \end{center}
% \end{table}
% %%%%%%%%%%%%%%%%%
It is clear that the events where one $W$ decays leptonically, which
provides the main trigger elements, while the other decays
hadronically have the largest branching fraction and have a
reconstructable Higgs mass peak~\cite{intro2}. 

This note contains the analysis that sets a limit on the Higgs boson cross-section
based on this decay mode. 
The main experimental issue is to control the large $W$ plus jets 
background sufficiently well that the advantages of using this 
final state are realized.

The analysis selects events with one well identified and isolated lepton, 
large missing transverse energy and at least two high \pt jets.
The note is structured as follows. 
A discussion about the data samples used in the analysis and the trigger selections
is presented in Sections~\ref{sec:MCexpectations} and \ref{sec:technicalities}.  
The physics objects reconstruction is discussed in Section~\ref{sec:firstStep}.
The lepton selection and other preselection requirements are described in detail 
in Section~\ref{sec:firstStep} and \ref{sec:dataMCcomparisons}.

A first limit extraction is performed with a few set of selections applied on top of these,
as described in Section~\ref{sec:firstExtraction}.
To enhance the exclusion power of this analysis, 
tighter selections are put in place.

After the preselections,
the signal-over-background ratio is enhanced by means of a selection on a MVA discriminant, 
designed in order to control the background while preserving as much as possible the difference in shape
with respect to the signal (Section~\ref{sec:mvaoptimization}).
The input variables of the MVA exploit the decay angles of the four-body mass system, 
the kinematics of the entire four-body system. For high mass points, a quark-gluon discriminating
variable, already exploited in the CMS search for a Standard Model Higgs boson in its $\ell{}\ell{}jj$ decay,
is also used in parallel with the MVA discriminant.
The MVA variable definition is optimized with dedicated trainings for each Higgs mass hypothesis case, 
for each lepton flavour ($e$, $\mu$) and for each jets multiplicity (2 jets, 3 jets) independently.
In this way, 48 different configurations are obtained.

The main background contaminating the signal region is W+jets.
Its $m_{\ell{}\nu{}jj}$ shape is extrapolated from sidebands in the $m_{jj}$ distribution 
through Monte Carlo based factors (Section~\ref{sec:wjetsBackground}).
Also the QCD shapes come from a data-driven determination, as described in Section~\ref{sec:dataDrivenQCD}.
The remaining background shapes come from the Monte Carlo.
The normalization of the backgrounds in the signal region 
is measured by fitting their Monte Carlo shapes on the same sidebands in the $m_{jj}$ variable 
(Section~\ref{sec:mjjfitfornormal}).
The determination of the $m_{jj}$ distribution for W+jets is described in Section~\ref{sec:wjetsShape}.

The systematic uncertainties present in the signal description are described in Section~\ref{sec:systematics}.
Section~\ref{sec:limitExtraction} describes the obtained limits on the Standard Model Higgs cross section
and Section~\ref{sec:conclusions} closes this work.

               % motivations and brief description of the note  
\input{MCexpectations.tex}      % description of backgrounds and expected cross-section from MC, for signal and background  
\section{Datasets}
\label{sec:technicalities}
% ---- ---- ---- ---- ---- ---- ---- ---- ---- ---- ---- ---- ---- ---- ---- ---- ---- ---- ---- ---- ---- ---- ----

\subsection{Samples used for the analysis}

The data sample used in this analysis was recorded by the CMS experiment in 2012.
Only certified runs and luminosity sections are considered, which means that a good functioning
of all CMS sub-detectors is required. The total statistics analyzed correspond to an integrated
luminosity of 3.5~fb$^{-1}$. % \LUMI{}.

The dataset used for the analysis and the corresponding run ranges are listed in Table~\ref{tab:datasets}.
All samples have been processed using a \texttt{CMSSW\_5\_2\_5} release version.

\begin{table}[htb]
  \begin{center}
  \begin{tabular}{r|r}
  \hline
  Dataset name & Run range \\
  \hline
  /SingleMu/Run2012A-PromptReco-v1/AOD   & 190456-193686  \\
  /SingleElectron/Run2012A-PromptReco-v1/AOD   &            \\ 
  \hline
  /SingleMu/Run2012B-PromptReco-v1/AOD   &  193752-195552  \\
  /SingleElectron/Run2012B-PromptReco-v1/AOD         &  193752-195655    \\
  \hline
  /SingleMu/Run2012A-23May2012-v2/AOD   &  190782-190949  \\
  /SingleElectron/Run2012A-23May2012-v3/AOD   &         \\
  \hline
  \hline
  \end{tabular}
  \end{center}
  \caption{Summary of data samples used and run ranges of applicability.}
  \label{tab:datasets}
\end{table}%

\subsection{Monte Carlo samples}

Standard Model Higgs boson samples, 
as well as samples for a large variety of electroweak and QCD-induced background sources, 
have been generated and showered using different Monte Carlo generators.
To better reproduce the actual data-taking conditions, where there is a significant probability
that more than two protons interact in the same bunch crossing, pile-up (PU) events are
added on top of the hard scattering. Particle interactions with the detector were reproduced through
a detailed description of CMS.

The POWHEG-BOX generator
\cite{Nason:2004rx,Frixione:2007vw,Alioli:2010xd,Nason:2009ai} has
been used to produce signal events, and the showering has been
performed with PYTHIA6 \cite{pythia}. For this analysis, samples with
Higgs mass hypotheses ranging from 180 to 600\GeVcc have been
used.
The background samples used for the studies are listed in
Table~\ref{tab:MCsamples}.

All MC samples considered in this analysis come from the official
``Summer12'' production, with the exception of the ``matching up/down'' and
``scale up/down'' W+Jets samples, which come from the ``Summer11''
production.  Events from Summer12 samples were reconstructed making
use of a \texttt{CMSSW\_5\_2\_X} release version.  
The simulated samples are reweighted to represent the
distribution of number of pp interactions per bunch crossing (pile-up)
as measured in the data.


\begin{sidewaystable}[htb]
  \begin{center}
    \begin{tabular}{|l|} 
      \hline
%%      sample & cross-section (pb) \\
%%      \hline
      /WJetsToLNu\_TuneZ2Star\_8TeV-madgraph-tarball/Summer12-PU\_S7\_START52\_V9-v1/AODSIM   \\
      /WW\_TuneZ2star\_8TeV\_pythia6\_tauola/Summer12-PU\_S7\_START52\_V9-v1/AODSIM   \\
      /WZ\_TuneZ2star\_8TeV\_pythia6\_tauola/Summer12-PU\_S7\_START52\_V9-v1/AODSIM   \\
      /TTJets\_TuneZ2star\_8TeV-madgraph-tauola/Summer12-PU\_S7\_START52\_V9-v1/AODSIM   \\
      /DYJetsToLL\_M-50\_TuneZ2Star\_8TeV-madgraph-tarball/Summer12-PU\_S7\_START52\_V9-v1/AODSIM   \\
      /QCD\_Pt\_20\_MuEnrichedPt\_15\_TuneZ2star\_8TeV\_pythia6/Summer12-PU\_S7\_START52\_V9-v1/AODSIM   \\
      /QCD\_Pt\_20\_30\_EMEnriched\_TuneZ2star\_8TeV\_pythia6/Summer12-PU\_S7\_START52\_V9-v1/AODSIM   \\
%%      /QCD\_Pt\_30\_80\_EMEnriched\_TuneZ2star\_8TeV\_pythia6/Summer12-PU\_S7\_START52\_V9-v1/AODSIM   \\
      /QCD\_Pt\_80\_170\_EMEnriched\_TuneZ2star\_8TeV\_pythia6/Summer12-PU\_S7\_START52\_V9-v1/AODSIM   \\
%%     /QCD\_Pt\_170\_250\_EMEnriched\_TuneZ2star\_8TeV\_pythia6/Summer12-PU\_S7\_START52\_V9-v1/AODSIM   \\
%%      /QCD\_Pt\_250\_350\_EMEnriched\_TuneZ2star\_8TeV\_pythia6/Summer12-PU\_S7\_START52\_V9-v1/AODSIM   \\
%%      /QCD\_Pt\_350\_EMEnriched\_TuneZ2star\_8TeV\_pythia6/Summer12-PU\_S7\_START52\_V9-v1/AODSIM   \\
      /T\_t-channel\_TuneZ2star\_8TeV-powheg-tauola/Summer12-PU\_S7\_START52\_V9-v1/AODSIM   \\
      /T\_tW-channel-DR\_TuneZ2star\_8TeV-powheg-tauola/Summer12-PU\_S7\_START52\_V9-v1/AODSIM   \\
%%      /T\_s-channel-DR\_TuneZ2star\_8TeV-powheg-tauola/Summer12-PU\_S7\_START52\_V9-v1/AODSIM   \\
      /Tbar\_t-channel\_TuneZ2star\_8TeV-powheg-tauola/Summer12-PU\_S7\_START52\_V9-v1/AODSIM   \\
      /Tbar\_tW-channel-DR\_TuneZ2star\_8TeV-powheg-tauola/Summer12-PU\_S7\_START52\_V9-v1/AODSIM   \\
      /Tbar\_s-channel\_TuneZ2star\_8TeV-powheg-tauola/Summer12-PU\_S7\_START52\_V9-v1/AODSIM   \\
      \hline
      /WJetsToLNu\_TuneZ2\_matchingdown\_7TeV-madgraph-tauola/Summer11-PU\_S4\_START42\_V11-v1/AODSIM  \\
      /WJetsToLNu\_TuneZ2\_matchingup\_7TeV-madgraph-tauola/Summer11-PU\_S4\_START42\_V11-v1/AODSIM  \\
      /WJetsToLNu\_TuneZ2\_scaledown\_7TeV-madgraph-tauola/Summer11-PU\_S4\_START42\_V11-v1/AODSIM          \\
      /WJetsToLNu\_TuneZ2\_scaleup\_7TeV-madgraph-tauola/Summer11-PU\_S4\_START42\_V11-v1/AODSIM            \\
      /WToLNu\_1jEnh2\_2jEnh35\_3jEnh40\_4jEnh50\_7TeV-sherpa/Summer11-PU\_S4\_START42\_V11-v1/AODSIM      \\
      \hline 
      /LQ-ggh180\_SIM/qili-New-SQWaT\_PAT\_52X\_v1-290326670ba15ca0752d90668da7d2ec/USER   \\
      /LQ-ggh200\_SIM/dimatteo-SQWaT\_PAT\_52X\_ggH300\_v2-290326670ba15ca0752d90668da7d2ec/USER   \\	
      /LQ-ggh300\_SIM/dimatteo-SQWaT\_PAT\_52X\_ggH300\_v2-290326670ba15ca0752d90668da7d2ec/USER  \\ 
      /LQ-ggh400\_SIM/dimatteo-SQWaT\_PAT\_52X\_ggH400\_v2-290326670ba15ca0752d90668da7d2ec/USER  \\
      /LQ-ggh400\_SIM/dimatteo-SQWaT\_PAT\_52X\_ggH450\_v2-290326670ba15ca0752d90668da7d2ec/USER   \\
      /LQ-ggh500\_SIM/qili-SQWaT\_PAT\_52X\_v1-290326670ba15ca0752d90668da7d2ec/USER   \\
      /LQ-ggh550\_SIM/qili-SQWaT\_PAT\_52X\_v1-290326670ba15ca0752d90668da7d2ec/USER   \\
      /LQ-ggh600\_SIM/qili-SQWaT\_PAT\_52X\_v1-290326670ba15ca0752d90668da7d2ec/USER   \\
%%      /GluGluToHToWWToLNuQQ\_M-*\_7TeV-powheg-pythia6/Fall11-PU\_S6\_START42\_V14B-v1/AODSIM  \\
%%      /GluGluToHToWWToTauNuQQ\_M-*\_7TeV-powheg-pythia6/Fall11-PU\_S6\_START42\_V14B-v1/AODSIM  \\
%%      /VBF\_HToWWToLNuQQ\_M-*\_7TeV-powheg-pythia6/Fall11-PU\_S6\_START42\_V14B-v1/AODSIM \\
%%
     Higgs signal samples for various masses.  \\
      \hline
    \end{tabular}
  \end{center}
  \caption{Summary of Monte Carlo samples used in the analysis.}
  \label{tab:MCsamples}
\end{sidewaystable}

%%%%%%%%%%%%%%%%%%%%%%%%
%%%%%%%%%%%%%%%%%%%%%%%%
%%%%%%%%%%%%%%%%%%%%%%%%
%%\subsection{Plans for signal sample usage for unblinding and beyond}
%%\label{sec:plansforichep}
%%We have produced four 8 TeV FullSim {\sc POWHEG} samples with 
%%Summer12 configuration so far: 180~GeV, 300~GeV, 500~GeV, 
%%and 600~GeV. For the 180~GeV  mass point we have performed a 
%%comparison of the kinematic distributions between 7 TeV and 8 TeV 
%%energies, as described in section~\ref{sec:7and8tevcomparisons}. 
%%The distributions agree well at the level of a few percent, and 
%%excursions are consistent with the uncertainties quoted 
%%in Ref.~\cite{LHCHiggsCrossSectionWorkingGroup:2011ti} and 
%%in Table~\ref{tab:signalPDF}. This study shows that we can use 
%%7 TeV signal samples to set limits on 8 TeV data by simply 
%%scaling the signal strength to 8 TeV cross section.
%%
%%
%%Here is our overall plan:
%%%%%%%%%%
%%\begin{enumerate}
%%\item
%%For unblinding on June 14, we plan to use 8 TeV signal samples for 
%%the four mass points (180~GeV, 300~GeV, 500~GeV, 600~GeV) 
%%that have already been produced and for any additional mass point 
%%that is produced by June 13. We will use 7 TeV samples for the 
%%remaining mass points after rescaling to 8 TeV cross section 
%%and correcting for differences in kinematics as described in 
%%Section~\ref{sec:7and8tevcomparisons}.
%%\item
%%Eventually for approval before ICHEP, we will have 8 TeV samples
%%available for all mass points (except for VBF process, which has small contribution). 
%%We expect that the difference in limits derived from 8 TeV signal 
%%and 7 TeV rescaled signal will be small and adequately covered by 
%%our current systematic uncertainties.
%%\end{enumerate}
%%%%%%%%%%%%%%%%%%%%%%%%%%
%%%%%%%%%%%%%%%%%%%%%%%%%%
%%\subsection{Comparison of signal kinematic distributions at 7 TeV and 8 TeV}
%%\label{sec:7and8tevcomparisons}
%%We have performed a comparison between the 7~TeV and 8~TeV kinematic 
%%distributions for Higgs mass 180~GeV at the generator level after hadronization. 
%%%%We are generating higgs samples and have finished some mass points. 
%%%%Then we compared 7TeV and 8TeV samples in generation level for mh = 180.
%%Samples used for the comparison are listed in Table~\ref{tab:datasetsfortest}.
%%The 8~TeV sample was produced using release version \texttt{CMSSW\_5\_2\_5} 
%%and ``Summer12'' underlying event and pileup conditions.
%%The 7~TeV sample was produced  using release version \texttt{CMSSW\_4\_2\_8\_patch4} 
%%and ``Fall11'' underlying event and pileup conditions.
%%Comparison between 7~TeV and 8~TeV of Higgs \pt and rapidity are shown in 
%%Figures~\ref{fig:higgspt}-\ref{fig:higgseta}. 
%%There is a reasonable agreement between 7~TeV and 8~TeV distributions 
%%even before applying any correction for the energy difference.
%%To account for the energy difference we plot the ratio of 
%%occupancy for 8~TeV relative to 7~TeV as a function of the two 
%%variables, as shown in Figure~\ref{fig:8TeV7TeVratio}.
%%We then use the ratio to re-weight the 7~TeV sample. 
%%The comparison of the two distributions after re-weighting 
%%is also shown in Figures~\ref{fig:higgspt}-\ref{fig:higgseta}. 
%%Similar comparison for the WW invariant mass is shown in 
%%Figures~\ref{fig:fourbodymass}.
%%Figures~\ref{fig:wpluspt}-\ref{fig:wminuseta} compare the 
%%kinematics of the two W bosons and 
%%Figures~\ref{fig:electronpt}-\ref{fig:nutrinopt} show the 
%%comparison for the W daughters, before and after reweighting.
%%All the histograms are normalized to 1000 events to compare the shape.
%%
%%
%%\begin{table}[htb]
%%  \begin{center}
%%  \begin{tabular}{r|r}
%%  \hline
%%  LQ-ggh180\_SIM/zixu-LQ-ggh180\_AODSIM-32e7d5ca409d944a857d457f02b5114b/USER\\
%% \hline
%%  GluGluToHToWWToLNuQQ\_M-180\_7TeV-powheg-pythia6/Fall11-PU\_S6\_START42\_V14B-v1/AODSIM\\
%% \hline
%% \hline
%% \end{tabular}
%% \end{center}
%% \caption{data samples used for reweighting and comparision}
%% \label{tab:datasetsfortest}
%%\end{table}%
%%
%%%ratio
%%\begin{figure}[h!t]
%%  {\centering
%%    \includegraphics[width=0.55\textwidth]{plots/signal_reweight/ratio_180.png}
%%    \caption{ Ratio of 8TeV/7TeV. }
%%    \label{fig:8TeV7TeVratio}}
%%\end{figure}
%%%higgs
%%\begin{figure}[h!t]
%%  {\centering
%%    \includegraphics[width=0.49\textwidth]{plots/signal_reweight/Plots/hpt_nw.png}
%%    \includegraphics[width=0.49\textwidth]{plots/signal_reweight/Plots/hpt.png}
%%    \caption{Comparison of higgs pt, before and after reweighting.}
%%    \label{fig:higgspt}}
%%\end{figure}
%%\begin{figure}[h!t]
%%  {\centering
%%    \includegraphics[width=0.49\textwidth]{plots/signal_reweight/Plots/heta_nw.png}
%%    \includegraphics[width=0.49\textwidth]{plots/signal_reweight/Plots/heta.png}
%%    \caption{Comparison of higgs eta, before and after reweighting.}
%%    \label{fig:higgseta}}
%%\end{figure}
%%\begin{figure}[h!t]
%%  {\centering
%%    \includegraphics[width=0.49\textwidth]{plots/signal_reweight/Plots/mlvjj_nw.png}
%%    \includegraphics[width=0.49\textwidth]{plots/signal_reweight/Plots/mlvjj.png}
%%    \caption{Comparison of four body mass, before and after reweighting. The lvjj mass was reconstructed by adding the lepton, the nutrino, and the 2 jets decayed by higgs.}
%%    \label{fig:fourbodymass}}
%%\end{figure}
%%
%%%wplus
%%\begin{figure}[h!t]
%%  {\centering
%%    \includegraphics[width=0.49\textwidth]{plots/signal_reweight/Plots/wpluspt_nw.png}
%%    \includegraphics[width=0.49\textwidth]{plots/signal_reweight/Plots/wpluspt.png}
%%    \caption{Comparison of wplus pt, before and after reweighting.}
%%    \label{fig:wpluspt}}
%%\end{figure}
%%\begin{figure}[h!t]
%%  {\centering
%%    \includegraphics[width=0.49\textwidth]{plots/signal_reweight/Plots/wpluseta_nw.png}
%%    \includegraphics[width=0.49\textwidth]{plots/signal_reweight/Plots/wpluseta.png}
%%    \caption{Comparison of wplus eta, before and after reweighting.}
%%    \label{fig:wpluseta}}
%%\end{figure}
%%%wminus
%%\begin{figure}[h!t]
%%  {\centering
%%    \includegraphics[width=0.49\textwidth]{plots/signal_reweight/Plots/wminuspt_nw.png}
%%    \includegraphics[width=0.49\textwidth]{plots/signal_reweight/Plots/wminuspt.png}
%%    \caption{Comparison of wminus pt, before and after reweighting.}
%%    \label{fig:wminuspt}}
%%\end{figure}
%%\begin{figure}[h!t]
%%  {\centering
%%    \includegraphics[width=0.49\textwidth]{plots/signal_reweight/Plots/wminuseta_nw.png}
%%    \includegraphics[width=0.49\textwidth]{plots/signal_reweight/Plots/wminuseta.png}
%%    \caption{Comparison of wminus eta, before and after reweighting.}
%%    \label{fig:wminuseta}}
%%\end{figure}
%%
%%%ele
%%\begin{figure}[h!t]
%%  {\centering
%%    \includegraphics[width=0.49\textwidth]{plots/signal_reweight/Plots/elept_nw.png}
%%    \includegraphics[width=0.49\textwidth]{plots/signal_reweight/Plots/elept.png}
%%    \caption{Comparison of electron pt, before and after reweighting.}
%%    \label{fig:electronpt}}
%%\end{figure}
%%\begin{figure}[h!t]
%%  {\centering
%%    \includegraphics[width=0.49\textwidth]{plots/signal_reweight/Plots/eleeta_nw.png}
%%    \includegraphics[width=0.49\textwidth]{plots/signal_reweight/Plots/eleeta.png}
%%    \caption{Comparison of elctron eta, before and after reweighting.}
%%    \label{fig:elctroneta}}
%%\end{figure}
%%%mu
%%\begin{figure}[h!t]
%%  {\centering
%%    \includegraphics[width=0.49\textwidth]{plots/signal_reweight/Plots/mupt_nw.png}
%%    \includegraphics[width=0.49\textwidth]{plots/signal_reweight/Plots/mupt.png}
%%    \caption{Comparison of muon pt, before and after reweighting.}
%%    \label{fig:muonpt}}
%%\end{figure}
%%\begin{figure}[h!t]
%%  {\centering
%%    \includegraphics[width=0.49\textwidth]{plots/signal_reweight/Plots/mueta_nw.png}
%%    \includegraphics[width=0.49\textwidth]{plots/signal_reweight/Plots/mueta.png}
%%    \caption{Comparison of muon eta, before and after reweighting.}
%%    \label{fig:muoneta}}
%%\end{figure}
%%%jet
%%\begin{figure}[h!t]
%%  {\centering
%%    \includegraphics[width=0.49\textwidth]{plots/signal_reweight/Plots/jetpt_nw.png}
%%    \includegraphics[width=0.49\textwidth]{plots/signal_reweight/Plots/jetpt.png}
%%    \caption{Comparison of leading jet pt, before and after reweighting.}
%%    \label{fig:leadingjetpt}}
%%\end{figure}
%%\begin{figure}[h!t]
%%  {\centering
%%    \includegraphics[width=0.49\textwidth]{plots/signal_reweight/Plots/jeteta_nw.png}
%%    \includegraphics[width=0.49\textwidth]{plots/signal_reweight/Plots/jeteta.png}
%%    \caption{Comparison of leading jet eta, before and after reweighting.}
%%    \label{fig:leadingjeteta}}
%%\end{figure}
%%%nutrino
%%\begin{figure}[h!t]
%%  {\centering
%%    \includegraphics[width=0.49\textwidth]{plots/signal_reweight/Plots/nupt_nw.png}
%%    \includegraphics[width=0.49\textwidth]{plots/signal_reweight/Plots/nupt.png}
%%    \caption{Comparison of nutrino pt, before and after reweighting.}
%%    \label{fig:nutrinopt}}
%%\end{figure}
%%
%%%\begin{figure}[h!t]
%%%  {\centering
%%%    \includegraphics[width=0.49\textwidth]{plots/signal_reweight/Plots/nueta_nw.png}
%%%    \includegraphics[width=0.49\textwidth]{plots/signal_reweight/Plots/nueta.png}
%%%    \caption{Comparison of nutrino eta, before and after reweighting.}
%%%    \label{fig:nutrino eta}}
%%%\end{figure}


\clearpage
      % what samples and triggers have been used  
\section{Common Event Selection}
\label{sec:firstStep}
% ---- ---- ---- ---- ---- ---- ---- ---- ---- ---- ---- ---- ---- ---- ---- ---- ---- ---- ---- ---- ---- ---- ----

The final state of the Higgs decay is characterized by a charged lepton, 
large missing energy and two hadronic jets that form a W.
In this section, we first describe the criteria applied to objects
selected in the event and then we describe the requirements made at
the event-level.

\subsection{Object Definitions}

The analysis relies on the standard reconstruction algorithms produced
by the CMS community. 
The PF2PAT procedure was used to coherently define the collection of particle-flow jets, leptons and
MET considered in the event selection.  
The technical details of the software configuration can be found in the group twiki page~\cite{WG_PATtuple_twiki}. 

\subsubsection{Electron Cuts (e+jets)}
\label{sec:electron_cuts}

Electrons are reconstructed using a gaussian-sum filter (GSF)
algorithm \cite{CMS-PAS-EGM-10-004}, and are required to pass electron
ID cuts according to a multi-variate identification
technique~\cite{cite:elemva}.  We also require that selected
electron candidates are isolated. Particle flow-based relative
isolation is defined as
%%%
\begin{equation*}
\mathrm{RelIso_{\mathrm{PF}}} = \frac{I_{\mathrm{CH}}+max(0,I_{\mathrm{NH}}+I_{\mathrm{PHOTON}}-(\mathrm{EA}\cdot\rho))}{E_\mathrm{T}},
\end{equation*} 
%%%

where $I_{\mathrm{CH}}$, $I_{\mathrm{NH}}$ and $I_{\mathrm{PHOTON}}$
are the charged hadron, neutral hadron and photon isolation variables
(using an isolation cone of 0.3). The isolation is corrected for
contributions from pile-up using charged hadron subtraction in the
isolation cone using fastjet algorithm \cite{FastJetPUSubtraction} and
for neutral particles using the effective area correction,
$(\mathrm{EA}\cdot\rho)$ where $\mathrm{EA}$ is the cone effective area
and $\rho$ is the average neutral particle density of the event.

The ID and isolation cuts used are shown in table~\ref{tab:EleID} and
have been tuned to give the same efficiency bin-by-bin with respect to
the working point (WP) used for 2011 analysis.

Additionally, we require
%%%%%%%%%%%%%%%%%%%
\begin{itemize}
\item Electron $E_\mathrm{T} > 30\,\mathrm{GeV}$.
\item Pseudorapidity $|\eta| < 2.5$. There is an exclusion range due
        to the ECAL barrel-endcap transition region, defined by
        $1.4442 < |\eta_{\mathrm{sc}}| < 1.566$, where
        $\eta_{\mathrm{sc}}$ is the pseudorapidity of the ECAL
        supercluster.
%\item Impact parameter: We cut on the absolute value of the impact
%       parameter calculated with respect to the average primary vertex (PV). We
%       require: $d_0(\mathrm{PV}) < 0.02\,\mathrm{cm}.$

%\item In order to make sure that the selected electron and the selected
%jets come from the same hard interaction and not from pile up events,
%we require that the $z$ coordinate of the PV of the event and the $z$
%coordinate of the electron's vertex lie within a distance of
%less than $0.1~\mathrm{cm}$.

\item 
In order to reject events in which the electron candidate actually
originates from a conversion of a photon into an $e^{+}e^{-}$ pair, we
use an approach using the vertex fit probability of fully
reconstructed conversions combined with the requirement that the
number of missed inner tracker layers of the electron track must be
exactly zero (i.e. there are no missed layers before the first hit of
the electron track from the beam line). 
\end{itemize}
%%%%%%%%%%%%%%%%%%%
%%%%%%%%%%%%%%%%%%%
%%%%%%%%%%%%%%%%%%%
\begin{table}[bthp]
\begin{center}
{\footnotesize
\begin{tabular}{|c|c|c|c|}
\hline
Lepton $\eta$ & $|\eta| < 0.8$ & $0.8 < |\eta| < 1.479$ & $1.479 < |\eta| < 2.5$  \\
\hline
ID MVA cut value (tight lepton) & 0.913 & 0.964 & 0.899 \\
Isolation cut value (tight lepton) & 0.105 & 0.178 & 0.150 \\
ID MVA cut value (loose lepton) & 0.877 & 0.811 & 0.707 \\
Isolation cut value (loose lepton) & 0.426 & 0.481 & 0.390 \\
\hline
\end{tabular}
\caption[.]{\label{tab:EleID} Cut values for electron identification
MVA output and for isolation which are tuned to give the same
efficiency as VBTF Working Point (WP) 80, as used for the tight
electron selection, and VBTF Working Point (WP) 90, as used in the
loose electron selection.}}
\end{center}
\end{table}
%%%%%%%%%%%%%%%%%%%
%%%%%%%%%%%%%%%%%%%%%%%%%%%%%%%%%%%%%%%%%%%%%%%%%%%%%%%%%%%%%%%%%%%%%%%%%%%%
%%%%%%%%%%%%%%%%%%%%%%%%%%%%%%%%%%%%%%%%%%%%%%%%%%%%%%%%%%%%%%%%%%%%%%%%%%%%

\subsubsection{Muon Cuts (mu+jets)}
\label{sec:muon_cuts}

Muon candidates are identified by two different 
algorithms~\cite{MUONPAS}: one proceeds from the inner tracker outwards, 
the other one starts from tracks measured in the muon chambers and matches 
and combines them with tracks reconstructed in the inner tracker. 
These selection criteria are summarized below:
%%%%%%%%%%%%%%%%%%%
\begin{itemize}
\item The muon candidate is reconstructed both as a global muon and
as a tracker muon.
\item Number of pixel hits of the Tracker track $\ge 1$;
\item Number of muon system hits of the Global track $\ge 1$;
\item Normalized $\chi^{2}$ of the Global track $< 10.0$.
\item Muon $p_{\mathrm{T}} > 25\,\mathrm{GeV}$.
\item Pseudorapidity $|\eta| < 2.1$.
\item Impact parameter: We cut on the absolute value of the impact
parameter calculated with respect to the primary vertex. We require:
$d_0(\mathrm{PV}) < 0.02\,\mathrm{cm}.$
\item In order to make sure that the selected muon and the selected
jets come from the same hard interaction and not from pile up events,
we require that the $z$ coordinate of the PV of the event and the $z$
coordinate of the muon's inner track vertex lie within a distance of
less than 0.5~cm.
\item The number of tracker layers with hits from the muon track has to be
$N_{\mathrm{layers}} > 5$.
\end{itemize}

The selected muon candidates also have to be isolated. Particle
flow-based relative isolation for muons is defined as
\begin{equation*}
\mathrm{RelIso_{\mathrm{PF}}} = \frac{I_{\mathrm{CH}}+I_{\mathrm{NH}}+I_{\mathrm{PHOTON}}-(0.5~{p}_{T}^\mathrm{sumPU})}{p_\mathrm{T}},
\end{equation*} 

where $I_{\mathrm{CH}}$, $I_{\mathrm{NH}}$ and $I_{\mathrm{PHOTON}}$
are the charged hadron, neutral hadron and photon isolation variables
(using an isolation cone of 0.4). The isolation is corrected for
contributions from pile-up using charged hadron subtraction in the
isolation cone using fastjet algorithm \cite{FastJetPUSubtraction} and
for neutral particles using the DeltaBeta correction, $(0.5
p_T^\mathrm{sumPU})$. We require the muon to have
$\mathrm{RelIso_{\mathrm{PF}}} < 0.12$ in order to be considered
isolated.
%%%%%%%%%%%%%%%%%%%%%%%%%%%%
%%%%%%%%%%%%%%%%%%%%%%%%%%%%
%%%%%%%%%%%%%%%%%%%%%%%%%%%%%%%%%%%%%%%%%%%%%%%%%%%%%%%%%%%%%%%%%%%%%%%%%%%%
%%%%%%%%%%%%%%%%%%%%%%%%%%%%%%%%%%%%%%%%%%%%%%%%%%%%%%%%%%%%%%%%%%%%%%%%%%%%


\subsubsection{Loose Electron}
For the purposes of rejecting events with more than one lepton we
define a loose electron, which has looser cuts. We consider electrons
which have $p_{\mathrm{T}} > 20\,\mathrm{GeV}/c$, $|\eta| < 2.5$,
and which satisfy electron $\mathrm{RelIso_{\mathrm{PF}}}$ and MVA ID
cuts. The cut values for the electron ID and isolation used in the
analysis can be found in table~\ref{tab:EleID}. 
%As in the case of the
%tight electrons, we also require $d_0(\mathrm{PV}) <
%0.02\,\mathrm{cm}$ and that the $z$ coordinate of the PV of the event
%and the $z$ coordinate of the electron's vertex lie within a distance
%of less than $0.1~\mathrm{cm}$.

\subsubsection{Loose Muon}
Additionally, to reject events with more than one lepton, we define a
loose muon, which has looser cuts. We consider all global muons which
have $p_{\mathrm{T}} > 10\,\mathrm{GeV}/c$, $|\eta| < 2.5$, and
$\mathrm{RelIso_{\mathrm{PF}}} < 0.2$ to be loose muons.

\subsubsection{Jet Cuts}
\label{sec:firstStep_jets}

Jets are reconstructed with the anti-KT algorithm \cite{cacciari}, 
starting from the set of objects reconstructed by the particle 
flow \cite{pflow,CMS-PAS-JME-10-003,CMS-PAS-PFT-10-002}.
Jets are corrected such that the measured energy of the jet 
correctly reproduces the energy of the initial particle. 
The CMS standard L2 (relative) correction makes the jet response flat in $\eta$.
The standard L3 (absolute) correction brings the jet closer to the $\PT$ of 
a matched generated jet created using generator level input and a similar 
jet clustering algorithm.
The L2 and L3 corrections are calculated using Monte Carlo, and thus a 
L2L3 residual correction is applied that fixes the discrepancies between 
Monte Carlo and data~\cite{newjes-cms}.
In this analysis we use jets with measured (corrected) $\PT$  
greater than 30~$\gev$. 
We require $|\eta| < 2.4$ so that the jets fall within the
tracker acceptance. Jets from pile-up are identified and removed with PileupJetID tool ~\cite{cite:PileupJetID}.  

%Jets are required to pass a set of loose identification
%criteria; this requirement eliminates jets originating from or being seeded by
%noisy channels in the calorimeter~\cite{Chatrchyan:2009hy}: 
%%%%%%%%%%%%%%
%\begin{itemize}
%\item Fraction of energy due to neutral hadrons $<$ 0.99.
%\item Fraction of energy due to neutral electromagnetic (EM) deposits $<$ 0.99.
%\item Number of constituents $>$ 1.
%\item Number of charged hadron candidates $>$ 0.
%\item Fraction of energy due to charged hadron candidates $>$ 0.
%\item Fraction of energy due to charged EM deposits $<$ 0.99.
%\end{itemize}
%%%%%%%%%%
%All energy fractions are calculated from uncorrected jets.

\par
In order to account for electron and muon objects that
have been reconstructed as jets, we remove from the jet
collection any jet that falls within a
cone of radius $R= 0.5$ of a loose electron or a loose muon. 
This ``cleaning'' procedure is applied in order to ensure that the same
particle is not double counted as two different physics objects.


\subsection{Event-Level Criteria}

The event should have a good primary vertex (PV). This means selecting
the primary vertex with the highest sum of $p_{T}^2$ of the tracks
associated with it and requiring it to have a number of degrees of
freedom (ndof) $\ge 4$, where ndof corresponds to the weighted sum of
the number of tracks used for the construction of the PV. In addition,
the PV must lie in the central detector region of $abs(z) \le
24~\rm{cm}$ and $\rho \le 2~\rm{cm}$ around the nominal interaction
point.

In the e+jets channel, we select events which contain exactly one
tight electron candidate fulfilling the criteria described in
Section~\ref{sec:electron_cuts} and reject events which contain a
loose electron in addition to the tight electron. In this channel we
are only interested in the decay to electron and jets, and we
therefore reject events containing a loose muon.

In the mu+jets channel, we select events which contain exactly one
tight muon candidate whose criteria are described in
Section~\ref{sec:muon_cuts} and reject events which contain an
additional loose muon. In an analoguous way to the e+jets channel, we
reject events containing a loose electron.

We require an event to have missing transverse energy (MET) in excess
of 25(35)~GeV for muons (electrons) and to have a $W$ transverse mass
of at least $30\,\mathrm{GeV}$.  These cuts are designed to reduce the
background from QCD multijet production.

We further require exactly two or three
jets passing the cuts
described in Section~\ref{sec:firstStep_jets}.  
% We require the
% invariant mass of the dijet system formed by the two highest $p_{T}$
% jets in the event to be between $65$ and $95\,\mathrm{GeV}$.

           % to be revised
\input{dataMCcomparisons.tex}
\clearpage{} 
\input{efficiencies.tex}
\input{kineFit.tex}             % to be revised          
\section{First limit extraction}
\label{sec:firstExtraction}
% ---- ---- ---- ---- ---- ---- ---- ---- ---- ---- ---- ---- ---- ---- ---- ---- ---- ---- ---- ---- ---- ---- ----

\subsection{Event selection and limit extraction technique}
% .... .... .... .... .... .... .... .... .... .... .... .... .... .... .... .... .... .... .... .... .... .... ....

We extract a first limit for the Standard Model Higgs boson after applying few additional cuts on
top of the preselections. The detailed description of this limit extraction can be found in a
dedicated Analysis Note \cite{CMS-AN-12-029}.

Events are selected with single lepton triggers throughout the whole data taking, as detailed in
Table~\ref{tab:HLT}. 

The two jets with the highest transverse momentum are chosen as the candidates of the W decay, and
only events where they are compatible with the W mass are selected ($65\GeVcc < m_{jj} < 95\GeVcc$).\\
To further enhance the signal over background ratio, we exploit the fact that Higgs decay products 
tend to be emitted in the central part of the detector, at variance with the behaviour of the W+jets
background. We therefore require events to have the lepton in the barrel ($|\eta|<1.5$), and the two
jets to be closer than 1.5 in $\eta$. Besides, the $|\eta|$ of the di-jet system is required to be
smaller than 3.\\
Finally, the cut on the MET and on the leptonic W transverse mass are set to 30~GeV. 

\begin{table}[htb]
 \begin{center}

%%   \scalebox{0.85}{
   \subfloat[Trigger paths for electrons.
   \label{tab:electronHLT}]{
   \begin{tabular}{c|c|c|l}
     \hline
     period & run range & dataset name & trigger path name                                                                                               \\
     \hline
     (e-i) & 190456 - 191930 & \texttt{Run2012A-PromptReco-v1} & \texttt{HLT\_Ele27\_WP80\_v*}   \\
     (e-ii) & 193829 - 195775 & \texttt{Run2012B-PromptReco-v1} & \texttt{HLT\_Ele27\_WP80\_v*}  \\
     \hline
   \end{tabular}}
%%   }

   \vskip 1cm

%%   \scalebox{0.85}{
   \subfloat[Trigger paths for muons\label{tab:muonHLT}]{
   \begin{tabular}{c|c|c|l}
      \hline
      period & run range & dataset name & trigger path name                     \\
                  \hline \hline
      ($\mu$-i) & 190456 - 191930 & \texttt{Run2012A-PromptReco-v1} & \texttt{HLT\_IsoMu24\_eta2p1\_v*}     \\
      ($\mu$-ii) & 193829 - 195775 & \texttt{Run2012B-PromptReco-v1} & \texttt{HLT\_IsoMu24\_eta2p1\_v*}     \\
      \hline
   \end{tabular}}
%%   }

 \end{center}
 \caption{List of trigger paths used to select events for this analysis, in the final state with
 electrons (a) and with muons (b).}
 \label{tab:HLT}
\end{table}

For each mass hypothesis, the limit on the Standard Model Higgs cross section is obtained from a fit
of the four-body invariant mass spectrum, as the place where the appearance of a Higgs resonance
is expected. The mass fit with a background-only hypothesis and a signal-plus-background hypothesis
to extract the exclusion C.L. is performed within the CMS Combination Tool.

The signal hypothesis is described by a fit on the Monte Carlo invariant mass shape performed for
each mass hypothesis independently. Figure~\ref{fig:signalShape} shows two examples of such fit.

\begin{figure}[htb]
\begin{center}
  \subfigure[]{
  \includegraphics[width=0.45\textwidth]{plots/limitplot/signalShape_MH400.pdf}}
  \subfigure[]{
  \includegraphics[width=0.45\textwidth]{plots/limitplot/signalShape_MH600.pdf}}
  \caption{Higgs boson invariant mass shape, as obtained from a (a) 400\GeVcc and (b) 600\GeVcc
           simulation. All the proper weighting factors taken into account. The fit function is
           constituted by a gaussian core smoothly joined to a high-mass and a low-mass power law
           tail. This smooth function is used as signal input for the  the limit extraction.}
  \label{fig:signalShape}
\end{center}
\end{figure}

To choose the best shape for the background we performed a dedicated study. We identified four
functional families, intended to span the possibilities of the true distribution to which
signal-free data would tend if we disposed of a nearly infinite statistics. The functional forms are:
polynomials ($\mathcal{T}(x;\mu,kT) \times \sum a_i x^i $), exponentials
($\mathcal{T}(x;\mu,kT) \times \sum N_i e^{-\lambda_i x}$), exponentiated polynomials
($\mathcal{T}(x;\mu,kT) \times e^{\sum a_i x^i}$) and power laws
($\mathcal{T}(x;\mu,kT) \times \sum N_i (x-a_i)^{-n_i}$), where all functions are multiplied by
$\mathcal{T}(x;\mu,kT)$, a fermi function intended to model the low invariant-mass turn-on. The
order of the functional forms is chosen after a study of fits on the MC simulation: for each family,
the order order of the fit function is raised until the goodness of the fit significantly improves.
The result is that a third order exponentiated polynomial, a second order exponential,  and a first
order power law are kept as background model candidates. Since a very  large number of degrees of
freedom is needed for the polynomials to well reproduce the invariant mass shape, the whole family
is discarded.

Among the resulting background model candidates, the power law is chosen, being the one that
minimizes the number of degrees of freedom and the bias on the background estimation, as explained
in the following section. To summarize, the background-only hypothesis is parametrized with this
function:

\begin{equation}
f_{\textnormal{bkg}}(x) = \frac{1}{e^{-(x-\mu)/kT} + 1} \cdot N \bigg(\frac{500+a}{x+a}\bigg)^n~,
\label{eq:fit_bkgModel}
\end{equation}

where all parameters are free to float in the fit. 
Figure~\ref{fig:backgroundShape} shows the four-body mass distribution 
of the observed events after the preselections, 
in the sidebands of the $m_{jj}$ spectrum as the analysis is still blind, %FIXME after unblinding
overlapped to a fit of the chosen background-only shape, 
performed outside the combination tool for illustrative purposes.

\begin{figure}[htb]
\begin{center}
 \subfigure[\label{fig:backgroundShape_mu}]{
  \includegraphics[width=0.49\textwidth]{plots/lin_data_mu_attenuatedPowerLaw_lepNuW_m_KF.pdf}
  }
 \subfigure[\label{fig:backgroundShape_e}]{
  \includegraphics[width=0.49\textwidth]{plots/lin_data_e_attenuatedPowerLaw_lepNuW_m_KF.pdf}
  }

  \caption{The four-body mass distribution of the observed events after preselections, for
          muons (a) and electrons (b), with a fit of the background-only shape
          superimposed. The fit is performed outside the combination
          tool for illustrative purposes.
          The fit is performed in the sidebands of the $m_{jj}$ spectrum as the analysis is still blind.%FIXME after unblinding
          }
  \label{fig:backgroundShape}
\end{center}
\end{figure}



\subsection{Uncertainties on the background evaluation}
% .... .... .... .... .... .... .... .... .... .... .... .... .... .... .... .... .... .... .... .... .... .... ....

The statistical uncertainty on the background is derived within the limit calculation from the
uncertainties on the parameters determination in the fit. Additionally, a systematic uncertainty
related to the choice of the specific background model of Equation~\ref{eq:fit_bkgModel} is
accounted for. The set of functions used to estimate the bias was defined in the previous section,
and consists of
$$
\mathcal{S} = \{\textnormal{attExp2,attPL1,attExpPol3}\}~,
$$
where we have:
\begin{equation}
\begin{split}
\textnormal{attExpPol3} & \doteq \mathcal{T}(x;\mu,kT) \times N e^{a_1 x + a_2 x^2 + a_3 x^3} \\
\textnormal{attExp2}    & \doteq \mathcal{T}(x;\mu,kT) \times N_1 \left( e^{-\lambda_1 x} + N_2 e^{-\lambda_2 x} \right) \\
\textnormal{attPL1}     & \doteq \mathcal{T}(x;\mu,kT) \times N \left(\frac{500+a}{x+a}\right)^n
\end{split}
\quad{.}
\end{equation}

To estimate the bias introduced by choosing one of the above functions to fit the mass spectra, the
following procedure has been followed:

\begin{itemize}
\item choose one of the functions above and use it as parent distribution $P(x)$;
\item generate a toy dataset according to $P(x)$ and corresponding to a large amount of accumulated
statistics (equivalent to $\sim 10000\fbinv$). The parameters of $P(x)$ are determined with a fit on
data\footnote{The data sample was preferred to the MC for the determination of the parameters of
the parent functions since the MC statistics is significantly lower than the data, and this can lead
to an artificially larger bias estimation.};
\item fit each toy dataset with all the test fit function $F(x)$ from the list above and and, for
each Higgs mass window $[x_{\textnormal{low}},x_{\textnormal{high}}]$, compute
$\delta(n_B)_{FP} = n_B^{\textnormal{fit}} - n_B^{\textnormal{parent}}$, where
$n_B^{\textnormal{fit}} = \int_{x_{\textnormal{low}}}^{x_{\textnormal{high}}} F(x) dx$ and
$n_B^{\textnormal{parent}} = \int_{x_{\textnormal{low}}}^{x_{\textnormal{high}}} P(x) dx$
\item iterate the procedure for all $P(x)$ in $\mathcal{S}$.
\end{itemize}
In the end, for each combination of a test fit function $F_i(x)$ and a parent function $P_j(x)$,
a value of $\delta(n_B)_{ij}$ is obtained for each of the 8 different Higgs mass hypotheses (the
indices $i$ and $j$ can run on the set of functions $\mathcal{S}$). The distribution of
$\delta(n_B)_{ij}$ obtained when varying the parent function $P(x)$ gives a quantitative
indication of the bias introduced by choosing a given $F(x)$ as the background fit model.

In fact, the bias due to possible difference in shape between the model and data corresponds to
fake signals and dips in the invariant mass spectrum that would systematically affect the
determination of an hypothetical Higgs signal. Therefore, the bias originating from the choice of a
specific background shape is introduced as an additional systematics on the signal, quoting it as a
fraction with respect to the total number of expected signal events. Table~\ref{tab:attPL_bias}
shows the obtained values as a function of the Higgs mass hypothesis.

\begin{table}[h!t]
 \begin{center}
   \begin{tabular}{c|c|c|c}
     \hline
     \multicolumn{2}{c|}{\small{$e\nu{}jj$}} &
     \multicolumn{2}{c} {\small{$\mu{}\nu{}jj$}} \\
     $m_{\textnormal{H}}$ &  unc. & $m_{\textnormal{H}}$ &  unc. \\
     \hline
     \hline
     250 & 54.8\%   &   250 &  9.1\% \\
     300 & 27.2\%   &   300 & 10.9\% \\
     350 & 28.3\%   &   350 & 11.6\% \\
     400 & 29.2\%   &   400 & 12.3\% \\
     450 & 21.8\%   &   400 &  3.7\% \\
     450 &  5.5\%   &   450 & 21.6\% \\
     550 & 38.5\%   &   500 & 41.2\% \\
     600 & 71.6\%   &   550 & 60.8\% \\
     \hline
   \end{tabular}
 \end{center}
 \caption{Signal uncertainty due to the background shape systematic for electron (left) and muon
         (right) final states.}
 \label{tab:attPL_bias}
\end{table}



\subsection{Uncertainties on the signal evaluation}
% .... .... .... .... .... .... .... .... .... .... .... .... .... .... .... .... .... .... .... .... .... .... ....

The sources of systematics on the signal considered in this case are the following:
\begin{itemize}
\item the luminosity uncertainty is considered 2.2\%;
\item the jet energy scale is taken into account by coherently varying all the jets with $\pt>15~\GeV$
      within the error provided by the CMS measurements, and the effect of such a variation is
      considered in the MET. The unclustered MET is conservatively varied by 5\% in a fully
      correlated fashion;
\item the uncertainty on pile-up comes from the limited knowledge of the number of pile-up events in
      data, quoted to be 5\%~\cite{PUScaling};
\item the uncertainty on the theoretical inclusive cross-section is taken from the LHC Cross Section
      Working Group calculations~\cite{LHCHiggsCrossSectionWorkingGroup:2011ti}, and the effects of
      pdfs on the acceptances are calculated according to the recipe provided by the same group;
\item the uncertainty on the Higgs mass shape is considered according the LHC Cross Section Working
      Group recommendations;
\item the uncertainty on the background shape modeling ranges from 2\% to 40\% on the signal rate;
\item trigger and lepton reconstruction efficiencies or scale factors are propagated through the
      analysis according to their measurements uncertainties;
\item the b-tagging uncertainty is propagated in the analysis chain, for the three-jets bin, by using
      the prescriptions from the b-POG and found to be less than a percent over the full mass range.
\end{itemize}
Table~\ref{tab:fitSystematics} summarizes all the sources of uncertainty considered, with the
typical effect on the signal.

\begin{table}[h!]
\begin{center}
\begin{tabular}{l|c}
\hline
source of uncertainty &  impact on signal  \\
\hline
\hline
background modeling                        &  4-70\% \\
Higgs line-shape                           & 10-30\% \\
cross-section                              & 15-20\% \\
luminosity                                 &   2.2\% \\
jet energy scale and MET                   &   2-3\% \\
theo acceptances (pdf)                     &   1-2\% \\
lepton trigger efficiency                  &  $<$1\% \\
lepton efficiency                          &   1-2\% \\
pile-up                                    &  $<$1\% \\
b-tag veto                                 &  $<$1\% \\
\hline
\end{tabular}
\end{center}
\caption{Sources of systematics considered in the fit analysis, with the corresponding value.}
\label{tab:fitSystematics}
\end{table}



\subsection{The obtained limit}
% .... .... .... .... .... .... .... .... .... .... .... .... .... .... .... .... .... .... .... .... .... .... ....

The obtained limit,
with the CLs method, is reported in Figure~\ref{fig:firstLimit}.
To compute the limit value in the mass points where the MC simulation is not available
signal shapes and systematics are interpolated.\\

\begin{figure}[htb]
\begin{center}
   \includegraphics[width=0.6\textwidth]{plots/limitplot/limit_first.pdf}
   \caption{The Standard Model Higgs exclusion limit, obtained by a shape fit just after the preselections.}
 \label{fig:firstLimit}
\end{center}
\end{figure}

The observed 95\% confidence level exclusion limit for the SM Higgs boson production is in the mass range $XXX-YYY\GeV$.

To improve this limit, a tighter selection scheme has been designed, aimed at increasing the
signal-over-background ratio after the selections, together with a full set of data driven
background evaluations.

\clearpage{} % put all the plots until here in the paper
\section{Multivariate optimization}
\label{sec:mvaoptimization}
% ---- ---- ---- ---- ---- ---- ---- ---- ---- ---- ---- ---- ---- ---- ---- ---- ---- ---- ---- ---- ---- ---- ----

In addition to the common preselections listed in Sec.~\ref{sec:firstStep}, the
MVA likelihood analysis applies the following selections:

\begin{itemize}
\item $|\Delta\phi_{\textrm{leading jet,MET}}| > 0.4$ for muons
\item $|\Delta\phi_{\textrm{leading jet,MET}}| > 0.8$ for electrons
\end{itemize}

We adopt the optimization performed on last year's data and MC samples
for masses 170-600~\GeV.  This optimization is documented completely
in the analysis notes AN-2011/110 and AN-2012/008. 

%% For masses above 600~\GeV,
%% we use 8~\TeV signal samples, splitting them in half so that the half that
%% is used for training is statistically independent from the half that is
%% used for limit setting.

An additional point of distinction between the optimization procedures
for the 2011 data and this year's data is the use of the quark-gluon
likelihood discriminant. In 2011, this discriminant was used to
distinguish quark-originated jets (signal) from gluon-originated jets
(W+Jets dominant background), particularly at high mass ($M_{H}\geq
500$~GeV), after applying the MVA output requirement. However, the PDF
for the discriminant for 2012 beam conditions was not yet available as
of this writing, and the PDF based on 2011 beam conditions did not
yield reliable results, so it is not applied.
           % optimization of the MVA discriminant
\section{Data Driven QCD Determination}
\label{sec:dataDrivenQCD}
% ---- ---- ---- ---- ---- ---- ---- ---- ---- ---- ---- ---- ---- ---- ---- ---- ---- ---- ---- ---- ---- ---- ----

A background from QCD multijet events comes from 3- or 4-jet events
with one jet passing the lepton criteria as a 'fake'. However, it is
not practical to generate sufficient MC to create a statistically
significant sample that passes the selection criteria. Therefore we
rely on a data-driven approach in which the isolation-inverted samples
from data, which mirror the QCD background, are used instead.
Specifically, we perform a two-component simultaneous fit to data of
the MET distribution in order to obtain the fraction of QCD events in
the data; the two components are a data-based QCD sample and a
MC-based W+Jets sample.

The data sample is constrained to a specific trigger epoch from Run
2011A, comprising approximately 200~$\pbinv$, in which the isolation
requirement in the trigger was loose enough to allow for the inversion
and thereby provide sufficient statistics for the study. The QCD sample is
obtained by inverting the lepton isolation in this data sample to be
$>0.1$ (default selection uses Iso$_{mu}<0.1$ and Iso$_{el}<0.05$).
In order to increase statistics for the QCD sample we also relax the
MET cut from 30~GeV to 20~GeV and (for electrons) the ID requirement
to WP90-like.  The MC W+Jets and target data samples are obtained by
similarly relaxing the MET cut and ID requirements.

The fraction of QCD events in data is then obtained from a
simultaneous fit of the two components on the MET distribution
performed before the MVA cut, again to maintain sufficient statistics.
The W+jets normalization is left free to float, as well as the QCD;
the results are shown in Figure~\ref{fig:QCDTemplateFit_MET}.  We
subsequently adjust the fraction applicable to our analysis by
removing the portion of events for which 20$<$MET$<$30~GeV, and
estimate the fraction of QCD relative to the data as shown in
Table~\ref{tab:qcdfrac}.  A separate study verified
that the fraction of QCD was not sensitive to the MVA selection;
nevertheless, to account for discrepancies in template modeling
(e.g. using W+jets MC as a proxy for all non-QCD processes) and the
fact that this fraction is estimated prior to the MVA cut, a very
large uncertainty is conservatively assumed. The final fraction of QCD
events in data is fed to the $m_{jj}$ fit for determination of the QCD
normalization, and the four-body shape of the QCD distribution is fed
to the final four-body total background determination in preparation
for the limit setting procedure.

Note that the W transverse mass distributions from the data and MC are
statistically consistent, as shown in
Figure~\ref{fig:QCDCutLoosening_MET} for muons; for electrons there's
an insufficient number of MC events to make the comparison.  The MET
for QCD processes is also 'fake'; i.e., it originates from badly
measured jets, and therefore has an exponentially falling spectrum.
By contrast, all other backgrounds exhibit a wide peak at $\sim
35$~GeV from a real neutrino (with the exception of Z+Jets, where the
MET is the result of a poorly measured lepton).

%%  $el_{2J}$ $frac_{QCD}=0.0617\pm 0.00384$,
%%  $el_{3J}$ $frac_{QCD}=0.0213\pm 0.00678$.
%%
%% $\mu_{2J}$ $frac_{QCD}=0.001625\pm 0.004214$,
%% $\mu_{3J}$ $frac_{QCD}=0.0\pm 0.0040797$,


\begin{table}[bthp]
\begin{center}
  \begin{tabular}{l c c}
    \hline  \hline
     & 2 jets & 3 jets \\
    \hline  
    electron  &	6.2 $\pm$ 0.4\% & 2.1 $\pm$ 0.7\% \\
    muon      &	0.2 $\pm$ 0.4\% & 0.0 $\pm$ 0.4\% \\
    \hline  \hline
  \end{tabular}
\end{center}
\caption{\label{tab:qcdfrac} Estimates of the percentage of QCD in data
for the muon and electron datasets after selection, separated into 2- and
3-jet bins.}
\end{table}

\subsection{QCD Uncertainties}
\label{sec:qcd_Uncertainty}
% .... .... .... .... .... .... .... .... .... .... .... .... .... .... .... .... .... .... .... .... .... .... ....

When performing the $m_{jj}$ fit  the QCD yield 
is Gaussian-constrained with a mean given by the value shown in
Table~\ref{tab:qcdfrac}.
In the case of electrons, the error on the QCD fraction
is small and we (conservatively) estimate
the uncertainty to be one half of the expected value. For muons the 
uncertainty is the error on the relative fraction (i.e.,
0.4\% for both 2- and 3-jet bins).
When fitting the sum of electron and muon data, the uncertainties
are combined using the standard error propagation machinery.

\subsection{Cross-Checks}
% .... .... .... .... .... .... .... .... .... .... .... .... .... .... .... .... .... .... .... .... .... .... ....

In order to ensure that our inverted selection provides a consistent representation of
QCD events, 
we fit the QCD with a Raileigh Function: $xe^{-x^2/2(\sigma_0+\sigma_1x)^2}$,
used during the inclusive cross section measurements~\cite{WZCMS:2010}. 
As can be seen from Fig.~\ref{fig:QCDMETRaileighFit},
the function accurately fits the overall shape as well as the parameter
corresponding to the intrinsic MET resolution ($\sigma_0\simeq 10$~GeV).

% we perform the following cross-checks:
% \begin{itemize}
% \item Fit the QCD with a Raileigh Function: $xe^{-x^2/2(\sigma_0+\sigma_1x)^2}$,
% used during the inclusive cross section measurements~\cite{WZCMS:2010}. 
% As can be seen from Fig.~\ref{fig:QCDMETRaileighFit},
% the function accurately fits the overall shape as well as the parameter
% corresponding to the intrinsic MET resolution ($\sigma_0\simeq 10$~GeV).
% \item Compare the W transverse mass shapes for the data sidebands with MET$>20$~GeV vs 
% MET$>30$~GeV (Fig.~\ref{fig:QCDMETCutsWmTShape}). Naturally, events with MET$>30$~GeV do not have the
% same exponential falloff, since they contain a higher percentage of W's.
% \item Examine the impact of setting Iso$>0.1$, rather than Iso$>0.2$.
% We compare the MET (Fig.~\ref{fig:QCDISOCutsMETShape}) and W transverse mass
% (Fig.~\ref{fig:QCDISOCutsWmTShape}) distributions, and conclude that there is no statistically
% significant discrepancy introduced by the looser isolation requirement.
% \end{itemize}


%%%%%%%%%%%%%%%%%%%%%%%%%%%%
%%%%%%%
\begin{figure}[h!] {\centering
\unitlength=0.33\linewidth
\includegraphics[width=0.48\textwidth]{plots/2012_QCD/QCDDataVSMC_Muons2J_MET.pdf}
\caption{ Comparison of the MET shapes for MC vs data-driven muon QCD events in the 2-jet bin. The two are statistically consistent.} 
\label{fig:QCDCutLoosening_MET}
}
\end{figure}
%%%%%%%
%%%%%%%
\begin{figure}[h!] {\centering
\unitlength=0.33\linewidth
\includegraphics[width=0.48\textwidth]{plots/2012_QCD/TemplateFit_MET_mu2j.pdf}
\put(-0.80,0.0){(a)} 
\unitlength=0.33\linewidth
\includegraphics[width=0.48\textwidth]{plots/2012_QCD/TemplateFit_MET_mu3j.pdf}
\put(-0.80,0.0){(b)} \\
\unitlength=0.33\linewidth
\includegraphics[width=0.48\textwidth]{plots/2012_QCD/TemplateFit_MET_el2j.pdf}
\put(-0.80,0.0){(c)} 
\unitlength=0.33\linewidth
\includegraphics[width=0.48\textwidth]{plots/2012_QCD/TemplateFit_MET_el3j.pdf}
\put(-0.80,0.0){(d)} 
\caption{MET distributions fit to the QCD and W$jj$ templates for: (a) muons - 2-jet bin, (b) muons - 3-jet bin, (c) electrons - 2-jet bin, (d) electrons - 3-jet bin.} 
\label{fig:QCDTemplateFit_MET}
}
\end{figure}
%%%%%%%
%%%%%%%
\begin{figure}[h!] {\centering
\unitlength=0.33\linewidth
\includegraphics[width=0.48\textwidth]{plots/2012_QCD/RaileighFitQCD_mu2j.pdf}
\put(-0.80,0.0){(a)} 
\unitlength=0.33\linewidth
\includegraphics[width=0.48\textwidth]{plots/2012_QCD/RaileighFitQCD_mu3j.pdf}
\put(-0.80,0.0){(b)} \\
\unitlength=0.33\linewidth
\includegraphics[width=0.48\textwidth]{plots/2012_QCD/RaileighFitQCD_el2j.pdf}
\put(-0.80,0.0){(c)} 
\unitlength=0.33\linewidth
\includegraphics[width=0.48\textwidth]{plots/2012_QCD/RaileighFitQCD_el3j.pdf}
\put(-0.80,0.0){(d)} 
\caption{QCD MET distributions fitted with a Raileigh Function for: (a) muons - 2-jet bin, (b) muons - 3-jet bin, (c) electrons - 2-jet bin, (d) electrons - 3-jet bin.} 
\label{fig:QCDMETRaileighFit}
}
\end{figure}
%%%%%%%
%%%%%%%
% \begin{figure}[h!] {\centering
% \unitlength=0.33\linewidth
% \includegraphics[width=0.48\textwidth]{plots/2012_QCD/METShapeComp_mu2j.pdf}
% \put(-0.80,0.0){(a)} 
% \unitlength=0.33\linewidth
% \includegraphics[width=0.48\textwidth]{plots/2012_QCD/METShapeComp_mu3j.pdf}
% \put(-0.80,0.0){(b)} \\
% \unitlength=0.33\linewidth
% \includegraphics[width=0.48\textwidth]{plots/2012_QCD/METShapeComp_el2j.pdf}
% \put(-0.80,0.0){(c)} 
% \unitlength=0.33\linewidth
% \includegraphics[width=0.48\textwidth]{plots/2012_QCD/METShapeComp_el3j.pdf}
% \put(-0.80,0.0){(d)} 
% \caption{ QCD W transverse mass shapes with MET$>20$~GeV vs MET$>30$~GeV for: (a) muons - 2-jet bin, (b) muons - 3-jet bin, (c) electrons - 2-jet bin, (d) electrons - 3-jet bin.} 
% \label{fig:QCDMETCutsWmTShape}
% }
% \end{figure}
%%%%%%%
%%%%%%%
% \begin{figure}[h!] {\centering
% \unitlength=0.33\linewidth
% \includegraphics[width=0.48\textwidth]{plots/2012_QCD/ISOShapeComp_MET_mu2j_g01vg02.pdf}
% \put(-0.80,0.0){(a)} 
% \unitlength=0.33\linewidth
% \includegraphics[width=0.48\textwidth]{plots/2012_QCD/ISOShapeComp_MET_mu3j_g01vg02.pdf}
% \put(-0.80,0.0){(b)} \\
% \unitlength=0.33\linewidth
% \includegraphics[width=0.48\textwidth]{plots/2012_QCD/ISOShapeComp_MET_el2j_g01vg02.pdf}
% \put(-0.80,0.0){(c)} 
% \unitlength=0.33\linewidth
% \includegraphics[width=0.48\textwidth]{plots/2012_QCD/ISOShapeComp_MET_el3j_g01vg02.pdf}
% \put(-0.80,0.0){(d)} 
% \caption{ QCD MET shapes with Iso$>0.1$ vs Iso$>0.2$ for: (a) muons - 2-jet bin, (b) muons - 3-jet bin, (c) electrons - 2-jet bin, (d) electrons - 3-jet bin.} 
% \label{fig:QCDISOCutsMETShape}
% }
% \end{figure}
%%%%%%%
%%%%%%%
% \begin{figure}[h!] {\centering
% \unitlength=0.33\linewidth
% \includegraphics[width=0.48\textwidth]{plots/2012_QCD/ISOShapeComp_WmT_mu2j_g01vg02.pdf}
% \put(-0.80,0.0){(a)} 
% \unitlength=0.33\linewidth
% \includegraphics[width=0.48\textwidth]{plots/2012_QCD/ISOShapeComp_WmT_mu3j_g01vg02.pdf}
% \put(-0.80,0.0){(b)} \\
% \unitlength=0.33\linewidth
% \includegraphics[width=0.48\textwidth]{plots/2012_QCD/ISOShapeComp_WmT_el2j_g01vg02.pdf}
% \put(-0.80,0.0){(c)} 
% \unitlength=0.33\linewidth
% \includegraphics[width=0.48\textwidth]{plots/2012_QCD/ISOShapeComp_WmT_el3j_g01vg02.pdf}
% \put(-0.80,0.0){(d)} 
% \caption{ QCD W transverse mass shapes with Iso$>0.1$ vs Iso$>0.2$ for: (a) muons - 2-jet bin, (b) muons - 3-jet bin, (c) electrons - 2-jet bin, (d) electrons - 3-jet bin.} 
% \label{fig:QCDISOCutsWmTShape}
% }
% \end{figure}
%%%%%%%

             % description of the QCD DD technique
\section{Modeling of background shapes in \texorpdfstring{$m_{jj}$}{dijet invariant mass} }
\label{sec:modelShape}

We utilize parameterizations of the shapes for the background.  These
are determined from fully corrected MC.  There are 3 major
contributions to the background, V+jets (W and Z), top
($t\overline{t}$ and single $t$), and diboson (WW and WZ).  These are
used in the fit to the $m_{jj}$ spectrum with the hadronic W signal
region removed.  The next sections detail the shapes used for each
background component.

\subsection{diboson background shape}
\label{sec:dibosonShape}

The diboson component is modeled as a sum of two Gaussian
distributions whose means differ by the W/Z mass difference.  The
widths of the peaking shapes is also the same fraction of the mean.
There is also a tail for poorly reconstructed diboson events consiting
of an exponential decay times a error function turn on.  The values of
the parameters are determined from MC and fixed in the fit.
\begin{equation}
\mathcal{F}_\text{diboson} = f_W\mathcal{G}_W + f_Z\mathcal{G}_Z + (1-f_W-f_Z)\mathcal{F}_\text{comb}
\end{equation}

\subsection{top background shape}
\label{sec:topShape}

The top component is model as a peaking component and a combinatoric
component.  For Higgs mass hypotheses below 450 GeV the combinatoric
background is a exponential times an error function for 450 GeV and
higher the combinatoric component is just a second Gaussian.  Again
the values of the parameters is determined from MC and fixed in the
fit.
\begin{equation}
\mathcal{F}_\text{top} = f_\text{peak}\mathcal{G} +
(1-f_\text{peak})\mathcal{F}_\text{comb}
\end{equation}

\subsection{V+jets background shape}
\label{sec:wjetsShape}
 
We employ an empirical description of the V+jets shape.  This
description is a kinematic turn on and a power law tail:
\begin{equation}
  \mathcal{F}_{W+\text{jets}} = \text{erf}(m_{jj}; m_0, \sigma)\times\left[(m_{jj})^{-\alpha-\beta\ln(m_{jj}/\sqrt{s})}\right]\,,
\end{equation}
where $m_0$ is the value of the turn on and $\sigma$ is the width of
this turn on.  The parameters $m_0$, $\sigma$, $\alpha$ and $\beta$
are determined in the fit to the data after the MVA cut.  There are
some constraints on the parameters from the MC, but because of the
relatively low statistics of the MC samples they are not overly
constraining.

% This nominal fit shape is not particularly well suited for all of the
% mass points.  In the 2-jet channels for masses from 180 GeV and below
% we use the MC morphing technique used in the study of the W+2 jets
% mass spectrum analysis documented in CMS AN-2011/266, Section 12.
% These lower mass points do not suffer from the lack of statistics as
% they are background rich, particularly in W+jets background.  We also
% use the MC morphing technique for 3-jet mass points from 200 GeV and
% below.  We use the following parameterization for 2-jet mass points
% 190 and 200 GeV.

% \begin{equation}
%   \mathcal{F}_{W+\text{jets low mass, 2 jets}} = \text{erf}(m_{jj}; m_0, \sigma)\times(m_{jj})^{-\alpha}\times\exp(m_{jj}\tau)\,,
% \end{equation}

% where the parameters $m_0$, $\sigma$, $\alpha$ and $\tau$ are
% determined in the fit.  In the 3-jet channels we use the
% parameterization

% \begin{equation}
%   \mathcal{F}_{W+\text{jets low mass, 3 jets}} = (m_{jj})^{-\alpha-\beta\ln(m_{jj}/\sqrt(s))}\times\exp(m_{jj}\tau)
% \end{equation}

% for masses 250 and 300 GeV.  These functional line shapes are
% motivated by the W+jets MC, however their parameters are derived
% strictly from data in the $m_{jj}$ sidebands around the W mass.
                % W+jets two-body mass shape
\clearpage{} % put all the plots until here in the paper
\section{The data-driven four-body mass background determination}
\label{sec:wjetsBackground}
% ---- ---- ---- ---- ---- ---- ---- ---- ---- ---- ---- ---- ---- ---- ---- ---- ---- ---- ---- ---- ---- ---- ----


\subsection{Determination of the W+jets shape in \texorpdfstring{$m_{\ell{}\nu{}jj}$}{Four-body Invariant Mass}}
% .... .... .... .... .... .... .... .... .... .... .... .... .... .... .... .... .... .... .... .... .... .... ....
\label{sec:alphaExtraction}

The four-body mass shape of the W+jets background in the signal region
is estimated in a data driven way from sidebands in the $m_{jj}$ region.

For each of the 48 working points of the MVA optimization,
three regions in $m_{jj}$ are looked at:
\begin{itemize}
\item lower sideband region (SBL):  $m_{jj} \in$ [55,65]~GeV
\item signal region: $m_{jj} \in $ [65,95]~GeV
\item upper sideband region (SBH): $m_{jj} \in$ [95,115] for $M_H<$250~GeV, [95,200]~GeV  for $M_H\ge$250~GeV, 
\end{itemize}

In the Monte Carlo,
the $m_{\ell\nu jj}$ shapes in the upper and lower sidebands
are compared to the one in the signal region,
to find the best mixture of the first two that reproduce the latter.
Therefore an $\alpha$ parameter is searched for, such that:
\begin{equation}
m_{WW_i}^{j} = (1-\alpha^j) \cdot m_{SBH_i}^j + \alpha^j \cdot  m_{SBL_i}^j~,
\label{EqnAlpha}
\end{equation}
where the index $j$ refers to each of the 48 mass spectra and $i$ to
the bins in the four-body mass.  In this way, the technique is largely
data driven but the precise extrapolation depends on the Monte Carlo
model. The value for the best alpha in $W+$jets MC and the $\chi^2/$NDF
scan of the shapes are shown in Figs.~\ref{fig:mcalphacheck_2j350mu}
as an example for the SM Higgs mass of 350~GeV for the 2-jet
$W\to\mu\nu$ category.

%
\begin{figure}[!t]
  \centering
  \includegraphics[width=0.49\textwidth]{plots/anaexample/2j350mu-Alpha-mcget-MVAgt_60_Range_12_300-780_SB_55-65_95-200.pdf}
  \includegraphics[width=0.49\textwidth]{plots/anaexample/2j350mu-Alpha-mcsca-MVAgt_60_Range_12_300-780_SB_55-65_95-200.pdf}
  \caption{\label{fig:mcalphacheck_2j350mu}The optimal alpha value
    from $W+$jets MC for the SM Higgs mass of 350~GeV for the 2-jet
    $W\to\mu\nu$ category.}
\end{figure}
%

\subsection{Determination of the normalization from fits to dijet mass}
\label{sec:mjjfitfornormal}
% .... .... .... .... .... .... .... .... .... .... .... .... .... .... .... .... .... .... .... .... .... .... ....

We extract the background yields from an unbinned maximum likelihood
fit to the dijet invariant mass distribution $m_{jj}$, after the selection on the MVA discriminant,
excluding the signal region ($65~{\mbox{GeV}} < m_{jj} < 95~{\mbox{GeV}}$). 
% Events in the signal region are later used to set Higgs exclusion limits. 
In this fit, all the backgrounds are considered. 
The W+jets shape is described as described in Section~\ref{sec:wjetsShape},
the QCD shape as described in Section~\ref{sec:dataDrivenQCD}, 
while the other background shapes come from the simulation.
The dijet mass spectrum fit fixes therefore the yields of the physics processes. 
Table~\ref{tab:mjj_shapes_and_normalization} shows in a schematic view how the 
shape of each component is determined, and what constraints 
are applied to fit for the normalization. 

\begin{table}[!ht]
  \begin{center}
 \caption{Determination of the $m_{jj}$ shape and normalization.}  
 \label{tab:mjj_shapes_and_normalization} 
 \begin{tabular} {l  c  c c c }
   \hline \hline
   Process                &    Shape                         &  Shape syst.           & Normalization   &  Norm. syst.\\  \hline
   W+jets                 &    data                      &  ---  & Unconstrained   &  Unconstrained \\
   diboson                &    MC                            &  JES                   & Constrain: NLO        &  Gauss $\sigma =10\%$ \\ 
   $t\bar{t}$ &    MC                            &  JES                   & Constrain: NLO        &  Gauss $\sigma =6.3\%$  \\ 
   single top & MC & JES & Constrain:NLO & Gauss $\sigma=5\%$ \\
   Z+jets                 &    MC                            &  JES                   & Constrain: NLO        &  Gauss $\sigma =4.3\%$  \\
   QCD                    &    data                          &  JES                   & Constrain: MET fit in data  &  Sec.~\ref{sec:qcd_Uncertainty}  \\\hline \hline
 \end{tabular}
\end{center}
\end{table}


%% The resulting fit for the SM Higgs mass of 350~GeV for the 2-jet
The resulting fit for the SM Higgs mass of 300~GeV for the 2-jet
$W\to\mu\nu$ category is shown in Figs~\ref{fig:mjj_mH350} with the
event yields of the different physics processes and the chisquared of
the fit. 
The electroweak diboson yield is robustly determined with the expected mean mass and mass resolution. 

\begin{figure}[!t]
  \centering
%%  \includegraphics[width=0.49\textwidth]{plots/anaexample/H350_Mjj_Muon_2jets_Stacked}
%%  \includegraphics[width=0.49\textwidth]{plots/anaexample/H350_Mjj_Muon_2jets_Pull}
  \includegraphics[width=0.49\textwidth]{plots/anaexample/H300_Mjj_Muon_2jets_Stacked}
  \includegraphics[width=0.49\textwidth]{plots/anaexample/H300_Mjj_Muon_2jets_Pull}
  \caption{\label{fig:mjj_mH350}For the SM Higgs mass of
%%  350~GeV for the 2-jet $W\to\mu\nu$ category, the distribution of
    300~GeV for the 2-jet $W\to\mu\nu$ category, the distribution of
    the dijet invariant mass $m_{jj}$ is shown on the left. The pull
    distribution computed as [(Data - Fit)/ Fit uncertainty] is shown
    on the right. } %% The signal region is blinded.}
\end{figure}


\subsection{Use of four-body mass to extract Higgs limits}
\label{sec:mlvjjforlimit}
% .... .... .... .... .... .... .... .... .... .... .... .... .... .... .... .... .... .... .... .... .... .... ....


Having determined the yield of the ensemble of physics processes using
the Dijet sidebands the four body mass spectrum where the dijet mass
lies in the W mass window is then explored.  For the W+jets
background, the WW four body mass is derived from the four body mass
distributions of the high and low dijet sidebands extrapolated into
the W mass window using the alpha value derived from the Monte Carlo
models as was described in Sec.~\ref{sec:alphaExtraction}.  The W+jets
shape derived from the sidebands is then smoothed with a parametric
fit to an exponential shape, modulated with a turn-on when necessary.
The distribution of the extrapolated W+jets background in the signal
region is reported for four working points in
Figure~\ref{fig:Wjets_dd_example}.  The black dots represent the
extrapolated background, while the red line shows the fitting function
and the blue shaded band the error from the fit.  All other background
categories use the $m_{\ell\nu jj}$ shape predicted by their
respective MC samples.
\begin{figure}[!t]
  \centering
%%     \includegraphics[width=0.45\textwidth]{plots/2012_WJetsShape/H190_Mlvjj_Electron_2jets_WpJShape.pdf}
%%     \includegraphics[width=0.45\textwidth]{plots/2012_WJetsShape/H190_Mlvjj_Electron_3jets_WpJShape.pdf}
     \includegraphics[width=0.45\textwidth]{plots/2012_WJetsShape/H200_Mlvjj_Electron_2jets_WpJShape.pdf}
     \includegraphics[width=0.45\textwidth]{plots/2012_WJetsShape/H200_Mlvjj_Electron_3jets_WpJShape.pdf}
%%     \includegraphics[width=0.45\textwidth]{plots/2012_WJetsShape/H350_Mlvjj_Muon_2jets_WpJShape.pdf}
%%     \includegraphics[width=0.45\textwidth]{plots/2012_WJetsShape/H350_Mlvjj_Muon_3jets_WpJShape.pdf}
%%    \includegraphics[width=0.45\textwidth]{plots/2012_WJetsShape/H200_Mlvjj_Muon_2jets_WpJShape.pdf}
%%    \includegraphics[width=0.45\textwidth]{plots/2012_WJetsShape/H200_Mlvjj_Muon_3jets_WpJShape.pdf}
    \includegraphics[width=0.45\textwidth]{plots/2012_WJetsShape/H300_Mlvjj_Muon_2jets_WpJShape.pdf}
    \includegraphics[width=0.45\textwidth]{plots/2012_WJetsShape/H300_Mlvjj_Muon_3jets_WpJShape.pdf}
    \includegraphics[width=0.45\textwidth]{plots/2012_WJetsShape/H500_Mlvjj_Muon_2jets_WpJShape.pdf}
    \includegraphics[width=0.45\textwidth]{plots/2012_WJetsShape/H500_Mlvjj_Muon_3jets_WpJShape.pdf}
    \caption{\label{fig:Wjets_dd_example} 
    The distribution of the extrapolated background in the signal region is reported for
    the Higgs mass hypotheses of 200~GeV (top row), 300~GeV (middle row) and 500~GeV (bottom
    row). The left and right columns display results for the muon 2jet and 3jet cases, respectively.
    The points represent the extrapolated points, while the blue line shows the fitting function and
    the blue dashed lines are the uncertainty on the shape.}
\end{figure}

The four-body mass distributions of background events in the signal regions,
derived from the $m_{jj}$ data sidebands, are shown in Figures~\ref{fig:m4_dd_example200}-\ref{fig:m4_dd_example500}.
%%The data in the signal region is blinded/not shown.

\begin{figure}[!t]
  \centering
%%  \subfigure[electron 2 jets, 190 GeV Higgs]{
%%    \includegraphics[width=0.45\textwidth]{plots/2012_FOURBSHAPES/H190_Mlvjj_Electron_2jets_Stacked}
%%    \includegraphics[width=0.45\textwidth]{plots/2012_FOURBSHAPES/H190_Mlvjj_Electron_2jets_Stacked_log}
  \subfigure[electron 2 jets, 200 GeV Higgs]{
      \includegraphics[width=0.45\textwidth]{plots/2012_FOURBSHAPES/H200_Mlvjj_Electron_2jets_Stacked}
      \includegraphics[width=0.45\textwidth]{plots/2012_FOURBSHAPES/H200_Mlvjj_Electron_2jets_Stacked_log}
%%      \includegraphics[width=0.3\textwidth]{plots/2012_FOURBSHAPES/H200_Mlvjj_Electron_2jets_Pull}
}

%%\subfigure[electron 3 jets, 190 GeV Higgs]{
%%      \includegraphics[width=0.45\textwidth]{plots/2012_FOURBSHAPES/H190_Mlvjj_Electron_3jets_Stacked}
%%      \includegraphics[width=0.45\textwidth]{plots/2012_FOURBSHAPES/H190_Mlvjj_Electron_3jets_Stacked_log}
\subfigure[electron 3 jets, 200 GeV Higgs]{
      \includegraphics[width=0.45\textwidth]{plots/2012_FOURBSHAPES/H200_Mlvjj_Electron_3jets_Stacked}
      \includegraphics[width=0.45\textwidth]{plots/2012_FOURBSHAPES/H200_Mlvjj_Electron_3jets_Stacked_log}
%      \includegraphics[width=0.3\textwidth]{plots/2012_FOURBSHAPES/H200_Mlvjj_Electron_3jets_Pull}
}
  \caption{The four-body mass distribution of data-driven background events in the signal regions for mass
  point M=200~GeV.}
  \label{fig:m4_dd_example200}
\end{figure}

\begin{figure}[!t]
  \centering
%% \subfigure[muon 2 jets, 350 GeV Higgs] {
%%      \includegraphics[width=0.3\textwidth]{plots/2012_FOURBSHAPES/H350_Mlvjj_Muon_2jets_Stacked}
%%      \includegraphics[width=0.3\textwidth]{plots/2012_FOURBSHAPES/H350_Mlvjj_Muon_2jets_Stacked_log}
\subfigure[muon 2 jets, 300 GeV Higgs] {
     \includegraphics[width=0.45\textwidth]{plots/2012_FOURBSHAPES/H300_Mlvjj_Muon_2jets_Stacked}
     \includegraphics[width=0.45\textwidth]{plots/2012_FOURBSHAPES/H300_Mlvjj_Muon_2jets_Stacked_log}
%%     \includegraphics[width=0.3\textwidth]{plots/2012_FOURBSHAPES/H350_Mlvjj_Muon_2jets_Pull}
}

%% \subfigure[muon 3 jets, 350 GeV Higgs] {
%%      \includegraphics[width=0.3\textwidth]{plots/2012_FOURBSHAPES/H350_Mlvjj_Muon_3jets_Stacked}
%%      \includegraphics[width=0.3\textwidth]{plots/2012_FOURBSHAPES/H350_Mlvjj_Muon_3jets_Stacked_log}
\subfigure[muon 3 jets, 300 GeV Higgs] {
     \includegraphics[width=0.45\textwidth]{plots/2012_FOURBSHAPES/H300_Mlvjj_Muon_3jets_Stacked}
     \includegraphics[width=0.45\textwidth]{plots/2012_FOURBSHAPES/H300_Mlvjj_Muon_3jets_Stacked_log}
%%     \includegraphics[width=0.3\textwidth]{plots/2012_FOURBSHAPES/H350_Mlvjj_Muon_3jets_Pull}
}
  \caption{The four-body mass distribution of data-driven background events in the signal regions for mass
  point M=300~GeV.}
  \label{fig:m4_dd_example300}
\end{figure}

\begin{figure}[!t]
  \centering
\subfigure[muon 2 jets, 500 GeV Higgs] {
     \includegraphics[width=0.45\textwidth]{plots/2012_FOURBSHAPES/H500_Mlvjj_Muon_2jets_Stacked}
     \includegraphics[width=0.45\textwidth]{plots/2012_FOURBSHAPES/H500_Mlvjj_Muon_2jets_Stacked_log}
}

\subfigure[muon 3 jets, 500 GeV Higgs] {
     \includegraphics[width=0.45\textwidth]{plots/2012_FOURBSHAPES/H500_Mlvjj_Muon_3jets_Stacked}
     \includegraphics[width=0.45\textwidth]{plots/2012_FOURBSHAPES/H500_Mlvjj_Muon_3jets_Stacked_log}
}
  \caption{The four-body mass distribution of data-driven background events in the signal regions for mass
  point M=500~GeV.}
  \label{fig:m4_dd_example500}
\end{figure}

%The full set of plots for the analysis on the 48 working points can be
%found in the CMS analysis note AN-2012/024, Section 15.

\clearpage
           % determination of the backgrounds contamination

The signal efficiency is estimated using simulation, considering both 
the $\rm{q}\bar{\rm{q}}\to\WW$ and $\mathrm{gg} \to \WW$ processes. 
Correction factors for the lepton reconstruction and identification
efficiency are determined using $\dyll$ events~\cite{wzxs} recorded with
dedicated unbiased triggers.
These factors depend on the lepton $\pt$ 
and $|\eta|$ and are typically around 0.96 for electrons and 0.98 for muons.
The trigger efficiency for signal events is measured to be approximately 98\%
for both $\rm{q}\bar{\rm{q}}\to\WW$ and $\mathrm{gg} \to \WW$ final states.

The uncertainties on lepton momentum scale and resolution, 
$\met$ modeling and jet energy scale are applied to the reconstructed objects in 
simulated events by smearing and scaling the relevant observables and propagating 
the effects to the kinematic variables used in the analysis. 
The uncertainty assigned to the pile-up corresponds to shifting the mean number
of pile-up events per beam crossing up and down by one interaction, and amounts to $2.3\%$.

The uncertainty on the acceptance due to variations in the parton distribution functions and the
value of $\alpha_{s}$ was found to be 2.3\% by following the {\sc pdf4lhc} prescription
~\cite{PDF4LHC,LHCHiggsCrossSectionWorkingGroup:2011ti}. 
The effect of higher-order corrections was found to be 1.5\% by varying the QCD renormalisation 
($\mu_R$) and factorisation ($\mu_F$) scales using the \textsc{mcfm} program~\cite{MCFM}.

The $\WW$ jet veto efficiencies in data are estimated from simulation, and multiplied by a 
data-to-simulation scale factor derived from $\dyll$ events in the $\Z$ peak: 
$\epsilon_{\WW}^{\rm data} = \epsilon_{\WW}^{\rm MC} \times \epsilon_{\Z}^{\rm data}/\epsilon_{\Z}^{\rm MC}.$
The uncertainty is thus factorized into the uncertainty on the $\Z$ efficiency in data and the
uncertainty on the ratio of the $\WW$ efficiency to the \Z\ efficiency in simulation
($\epsilon_{\WW}^{\rm MC}/\epsilon_{\Z}^{\rm MC}$). The former, which is statistically dominated, 
is 0.3\%. Theoretical uncertainties due to higher order corrections contribute most to the 
WW/Z efficiency ratio uncertainty, which is estimated to be 4.6\% for $\WW$ production. The 
data-to-simulation correction factor is found to be close to one.

The uncertainties on the $\Wjets$ and top background predictions 
are 36\% and 15\%, respectively, as described in Section~\ref{sec:backgrounds}. 

The theoretical uncertainties on the $\WZ$ and $\ZZ$ cross sections are calculated 
by varying the QCD renormalisation ($\mu_R$) and factorisation ($\mu_F$) scales using 
the \textsc{mcfm} program~\cite{MCFM}. The effect of variations in the parton distribution 
functions and the value of $\alpha_{s}$ on the predicted cross sections were derived 
by following the same prescription as for the signal acceptance.
Including the experimental uncertainties gives a systematic uncertainty of around 10\%.
The total uncertainty on the background estimations is about 13\%, which is dominated by 
the systematic uncertainties on the normalisation of the top quark and $\Wjets$ 
backgrounds. The luminosity measurement uncertainty is $4.4\%$~\cite{lumiPAS}.

A summary of the uncertainties taken into account in this analysis is given in
Table~\ref{tab:wwSystematics}.

\begin{table}[!ht]
\begin{center}
\caption{Relative systematic uncertainties on the estimated signal and
background yields, in percent. Data-driven background uncertainties
are separated in statistical component of the control sample and systematic
component of the extrapolation, (stat.) $\bigoplus$ (syst.).
\label{tab:wwSystematics}}
{\scriptsize
\begin{tabular}{l|c|c|c|c|c|c|c|c}
\hline
\multirow{9}{*}{} & $\rm{q}\bar{\rm{q}}$ & gg       & top & W+jets & WZ  & Z$/\gamma*$   & W+$\gamma$ & W+$\gamma^{*}$ \\
                  & $\to$ WW             & $\to$ WW &     &        & +ZZ & $\to\ell\ell$ &            &                \\
\hline
luminosity                   & 4.4 &  4.4 &  -                &  -                & 4.4 &  -                & 4.4 &  -  \\
trigger efficiency           & 1.5 &  1.5 &  -                &  -                & 1.5 &  -                & 1.5 &  -  \\
lepton id efficiency         & 2.0 &  2.0 &  -                &  -                & 2.0 &  -                & 2.0 &  -  \\
muon momentum scale          & 1.5 &  1.5 &  -                &  -                & 1.5 &  -                & 1.5 &  -  \\
electron energy scale        & 2.5 &  2.5 &  -                &  -                & 1.9 &  -                & 2.0 &  -  \\
$\met$ resolution            & 2.0 &  2.0 &  -                &  -                & 2.0 &  -                & 2.0 &  -  \\
jet veto efficiency          & 4.7 &  4.7 &  -                &  -                & 4.7 &  -                & 4.7 &  -  \\
pile-up                      & 2.3 &  2.3 &  -                &  -                & 2.3 &  -                & 2.3 &  -  \\
top normalisation            &   - &    - & 10 $\bigoplus$ 15 &  -                & -   &  -                & -   &  -  \\
W+jets normalisation         &   - &    - &  -                & 19 $\bigoplus$ 36 & -   &  -                & -   &  -  \\
Z normalisation              &   - &    - &  -                &  -                & -   & 15 $\bigoplus$ 24 & -   &  -  \\
W+$\gamma$ normalisation     &   - &    - &  -                &  -                & -   &  -                & 30  &  -  \\
W+$\gamma^{*}$ normalisation &   - &    - &  -                &  -                & -   &  -                & -   & 30  \\
PDFs                         & 2.3 &  0.8 &  -                &  -                & 5.9 &  -                & -   &  -  \\
higher order corrections     & 1.5 & 30.0 &  -                &  -                & 3.3 &  -                & -   &  -  \\
sample statistics            & 1.1 &  3.1 &  -                &  -                & 4.1 &  -                & 8.4 & 8.4 \\
\hline
\end{tabular}}
\end{center}
\end{table}
               % systematics on the signal
\section{Extraction of the limit on the cross-section}
\label{sec:limitExtraction}
% ---- ---- ---- ---- ---- ---- ---- ---- ---- ---- ---- ---- ---- ---- ---- ---- ---- ---- ---- ---- ---- ---- ----

One strength of this analysis, with data-driven estimations of the QCD
and $W+$jets backgrounds (using sidebands and $\alpha$), is that we
can treat the full mass range, 170-600 GeV, by the same methods.  We
use the ``Higgs Combination'' package \cite{cite:combine} for setting
exclusion limits. This package is a RooStats\cite{cite:roostats}-based
statistical analysis toolset recommended by the CMS Higgs PAG.  Inputs
to the limit setter are binned graphs and histograms for data,
background and signal Monte Carlo.  The Higgs mass points are: 170,
180, 190, 200, 250, 300, 350, 400, 450, 500, 550 and 600~GeV. Standard
Model Higgs cross sections $\times$ branching ratios for different
mass points are summarized in Table~\ref{tab:signals} and shown in
Figure~\ref{fig:higgsXSBR}.


To set limits we use the full shape information of the $m_{\ell\nu
jj}$ distribution.  The four-body mass window is set roughly by the
position and width of the signal distributions for the different Higgs
mass points. All of the distributions are segregated by lepton flavor
and whether they meet the 2-jet or 3-jet requirements, which represent
independent channel inputs to the limit setter.

The systematics described in Section~\ref{sec:systematics} are treated
as follows when being input to the limit setter:
\begin{itemize}
\item The main background systematics are total background shape
uncertainty and total background normalization uncertainty. 
% The
% 1-sigma up- and down-fluctuated background shape inputs to the limit
% setter are shown in Appendix~\ref{app:limitShapes}, along with the
% nominal shapes.  
Both shape and normalization uncertainties are
treated as uncorrelated across all channels, since they are derived
from fits performed on independent sample sets.
\item JES, MET uncertainty, and pileup are considered negligible for
signal and are otherwise subsumed in the normalization/shape
uncertainties for background, so they are omitted from the limit setter
inputs.
\item Uncertainties on the signal deriving from lepton reconstruction
and selection as well as trigger efficiency are treated as 100\%
correlated across the same-flavor lepton channels, but uncorrelated
between electron and muon channels.
\item Uncertainties on the signal deriving from parton distribution
functions (Table~\ref{tab:signalPDF}), luminosity, and theoretical
cross-section uncertainty are treated as 100\% correlated across all
channels. Since the PDF and cross-section uncertainties for the gluon
fusion process are uniformly worse than those for the vector boson
fusion process, the uncertainties for the former are taken as
applicable to the summed signal yields.
\item Uncertainties on the signal deriving from the final selection
efficiency are treated as uncorrelated across different channels, but as
correlated between quark-quark and glu-glu signal processes for the same
channel.
\end{itemize}

The limit setter is then set to utilize
the ``asymptotic CL$_{s}$''
\cite{cite:asympcls1,cite:asympcls2} method. 
%The final exclusion limit will be calculated with 
%the ``HybridNew CL$_{s}$'' technique.
The resulting median
expected limit with 1- and 2-sigma error bands are plotted.
The limit plots for all four channels combined (electron 2-jets, 
electron 3-jets, muon 2-jets, and muon 3-jets) are shown in
Fig.~\ref{fig:limitsetup:combinedlimit}. 
The background uncertainty is the limiting factor in this analysis.
%%%%%%%%%%%%%%%%%%%%
\begin{figure}[htb] 
  \begin{center}
      \includegraphics[width=0.8\textwidth]{plots/2012_LIMITS/limit_4chan_fullsyst_asymp.pdf}
%       \includegraphics[width=0.6\linewidth]{plots/2012_LIMITS/fullCLs_limits.png}
    \caption{The combined Higgs exclusion limit from all four channels 
      after including all systematic uncertainties.}
    \label{fig:limitsetup:combinedlimit}
  \end{center}
\end{figure}
%%%%%%%%%%%%%%%%%%%%%%%%%%%%%%%%%%%%%%%%
%%%%%%%%%%%%%%%%%%%%%%%%%%%%%%%%%%%%%%%%
%%%%%%%%%%%%%%%%%%%%%%%%%%%%%%%%%%%%%%%%

Using all four channels combined we exclude the Standard Model 
Higgs boson in the mass range 325--410~GeV, to a confidence level
of 95\%. 

\section{Conclusions}
\label{sec:conclusions}
% ---- ---- ---- ---- ---- ---- ---- ---- ---- ---- ---- ---- ---- ---- ---- ---- ---- ---- ---- ---- ---- ---- ----

This study presents the search for a Standard Model Higgs boson, 
in its decay into two W bosons, 
out of which one decays hadronically and the other one decays leptonically.
The major background that contaminates the signal region is due to W+jets, 
and it is measured in a data-driven way from sidebands in the invariant mass
of the hadronically decaying W.
This allows us to keep the relative systematic uncertainties, 
which are dominated by the background shape particularly at low mass, to within
a few percent for each Higgs mass point. 
Our observed 95\% confidence level
exclusion limit for the SM Higgs boson production is in the mass range
320--420~GeV. For the mass range (170--510~GeV)
we are sensitive to production rate that is within two
times the SM cross section.            
 \clearpage{}
\bibliography{auto_generated}   % will be created by the tdr script.
\appendix
% \input{appendixExtrapWjets.tex}           
% \input{appendixFourBodyShapes.tex}           
% \input{appendixLimitShapes.tex}           

% % examples of appendices. **DO NOT PUT \end{document} at the end
 \clearpage
 \appendix
% \section{Additional plots}


 \begin{figure*}[h!]
     \includegraphics[width=0.49\textwidth]{figs/Wjj_Mjj_Muon_2jets_Stacked.pdf}
     \includegraphics[width=0.49\textwidth]{figs/Wjj_Mjj_Electron_2jets_Stacked.pdf}
     \includegraphics[width=0.49\textwidth]{figs/Wjj_Mjj_Muon_2jets_Subtracted.pdf}
     \includegraphics[width=0.49\textwidth]{figs/Wjj_Mjj_Electron_2jets_Subtracted.pdf}
   \caption{(upper row) Distribution of the invariant mass spectrum of
     the two jets observed in data in muon plus 2 jets (left) and
     electron plus 2 jets (right) categories.  Overlaid are the
     template distributions used in the likelihood fit to the measured
     \mjj distibution, with their relative normalization as obtained
     from the fit.  The region between the vertical dotted lines is
     excluded in the fit.  Depicted is the number of events per GeV,
     \textit{i.e.}, the raw event count can be obtained by multiplying
     with the bin width.  (lower row) The same distribution after
     subtraction of all SM components except the electroweak diboson
     WW/WZ.  Error bars correspond to the statistical uncertainty.
     The band represents the systematic uncertainty in the sum of the
     SM components. }
 \end{figure*}





 \begin{figure*}[h!]
   \includegraphics[width=0.49\textwidth]{figs/Wjj_Mjj_Muon_3jets_Stacked.pdf}
   \includegraphics[width=0.49\textwidth]{figs/Wjj_Mjj_Electron_3jets_Stacked.pdf}
   \includegraphics[width=0.49\textwidth]{figs/Wjj_Mjj_Muon_3jets_Subtracted.pdf}
   \includegraphics[width=0.49\textwidth]{figs/Wjj_Mjj_Electron_3jets_Subtracted.pdf}
     \caption{(upper row) Distribution of the invariant mass spectrum
       of the leading two jets observed in data in muon plus 3 jets
       (left) and electron plus 3 jets (right) categories.  Overlaid
       are the template distributions used in the likelihood fit to
       the measured \mjj distibution, with their relative
       normalization as obtained from the fit.  The region between the
       vertical dotted lines is excluded in the fit.  Depicted is the
       number of events per GeV, \textit{i.e.}, the raw event count
       can be obtained by multiplying with the bin width.  (lower row)
       The same distribution after subtraction of all SM components
       except the electroweak diboson WW/WZ.  Error bars correspond to
       the statistical uncertainty.  The band represents the
       systematic uncertainty in the sum of the SM components.  }
 \end{figure*}




 \begin{figure*}[h!t]
   \includegraphics[width=0.3\textwidth]{figs/mu_W_muon_pt.pdf}
   \includegraphics[width=0.3\textwidth]{figs/mu_W_mt.pdf}
   \includegraphics[width=0.3\textwidth]{figs/mu_event_met_pfmet.pdf}
   \caption{Comparison of the distributions in data and MC for the
     muon plus jets event sample after event selection (left) of the
     transverse momentum of the muon candidate, (middle) of the
     transverse momentum of the reconstructed W candidate, (right) of
     the missing transverse energy. The error bars on the data points
     are statistical only.  The relative normalization of the various
     MC samples are taken from the result of the fit to the \mjj
     spectrum.  }
 \end{figure*}


 \begin{figure*}[h!t]
   \includegraphics[width=0.3\textwidth]{figs/elec_W_electron_et.pdf}
   \includegraphics[width=0.3\textwidth]{figs/el_W_mt.pdf}
   \includegraphics[width=0.3\textwidth]{figs/el_event_met_pfmet.pdf}
   \caption{Comparison of the distributions in data and MC for the
     electron plus jets event sample after event selection (left) of
     the transverse energy of the electron candidate, (middle) of the
     transverse momentum of the reconstructed W candidate, (right) of
     the missing transverse energy. The error bars on the data points
     are statistical only.  The relative normalization of the various
     MC samples are taken from the result of the fit to the \mjj
     spectrum.  }
 \end{figure*}



 \begin{figure*}[h!t]
   \includegraphics[width=0.49\textwidth]{figs/mu_jetld_pt.pdf}
   \includegraphics[width=0.49\textwidth]{figs/mu_jetnt_pt.pdf}
   \includegraphics[width=0.49\textwidth]{figs/mu_dijet_pt.pdf}
   \includegraphics[width=0.49\textwidth]{figs/mu_deltaRjj.pdf}
   \caption{Comparison of the distributions in data and MC for the
     muon plus jets event sample after event selection (upper left) of
     the transverse momentum of the leading jet, (upper right) of the
     transverse momentum of the second leading jet, (lower left) of
     the dijet transverse momentum, (lower right) of the dijet
     distance parameter $\Delta R= \sqrt{(\Delta \eta_{jj})^2 +
       (\Delta \phi_{jj})^2}$. The error bars on the data points are
     statistical only.  The relative normalization of the various MC
     samples are taken from the result of the fit to the \mjj
     spectrum.  }
 \end{figure*}

 \begin{figure*}[h!t]
   \includegraphics[width=0.49\textwidth]{figs/el_jetld_pt.pdf}
   \includegraphics[width=0.49\textwidth]{figs/el_jetnt_pt.pdf}
   \includegraphics[width=0.49\textwidth]{figs/el_dijet_pt.pdf}
   \includegraphics[width=0.49\textwidth]{figs/el_deltaRjj.pdf}
   \caption{Comparison of the distributions in data and MC for the
     electron plus jets event sample after event selection (upper
     left) of the transverse momentum of the leading jet, (upper
     right) of the transverse momentum of the second leading jet,
     (lower left) of the dijet transverse momentum, (lower right) of
     the dijet distance parameter $\Delta R= \sqrt{(\Delta
       \eta_{jj})^2 + (\Delta \phi_{jj})^2}$. The error bars on the
     data points are statistical only.  The relative normalization of
     the various MC samples are taken from the result of the fit to
     the \mjj spectrum.  }
 \end{figure*}


%%% DO NOT ADD \end{document}!
%\clearpage{}

%%--------------------------------------------------
\section{Study of QCD shape and normalization in MC}
Since there are not enough QCD events passing all of the cuts, we relax some of them. Specifically, the new requirements are
\begin{itemize}
\item MET$>20.0$~GeV for both muons and electrons.
\item Isolation$>0.3$ for both muons and electrons.
\item $WP80$ restrictions are removed for electrons.
\end{itemize}
This approach does not alter the shapes of the W transverse mass distributions
(Figure~\ref{fig:QCDCutLoosening}) and provides a sufficient number of
events (300-500) to construct the templates. Furthermore, since the
W transverse mass distribution should be the same for all processes except
QCD, we perform a fit to the data using the W$jj$, QCD templates -
Figure~\ref{fig:QCDTemplateFit} and fix the fraction of QCD relative
to W$jj$. After accounting for acceptances of the above cuts these
are: $\mu$ $frac_{QCD}=0.0053\pm 0.00050$, $el$ $frac_{QCD}=0.0386\pm
0.00486$.

%%%%%%%%%%%%%%%%%%%%%%%%%%%%
%%%%%%%
\begin{figure}[h!] {\centering
\unitlength=0.33\linewidth
\includegraphics[width=0.48\textwidth]{figs/QCDCutLoosening_mu.pdf}
\put(-0.80,0.0){(a)} 
\unitlength=0.33\linewidth
\includegraphics[width=0.48\textwidth]{figs/QCDCutLoosening_el.pdf}
\put(-0.80,0.0){(b)} 
\caption{ Comparison of the $W_{mT}$ shapes before and after loosening the (MET, Isolation and $WP80$) cuts for (a) muon QCD, (b) electron QCD distribution. The two are statistically consistent.} 
\label{fig:QCDCutLoosening}}
\end{figure}
%%%%%%%
%%%%%%%
\begin{figure}[h!] {\centering
\unitlength=0.33\linewidth
\includegraphics[width=0.48\textwidth]{figs/QCDTemplateFit_mu.pdf}
\put(-0.80,0.0){(a)} 
\unitlength=0.33\linewidth
\includegraphics[width=0.48\textwidth]{figs/QCDTemplateFit_el.pdf}
\put(-0.80,0.0){(b)} 
\caption{$W_{mT}$ fit using QCD and W$jj$ templates for (a) muon, (b) electron data.} 
\label{fig:QCDTemplateFit}}
\end{figure}
%%%%%%%
  
