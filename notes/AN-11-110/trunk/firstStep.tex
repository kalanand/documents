\section{Common Event Selection}
\label{sec:firstStep}
% ---- ---- ---- ---- ---- ---- ---- ---- ---- ---- ---- ---- ---- ---- ---- ---- ---- ---- ---- ---- ---- ---- ----

The final state of the Higgs decay is characterized by a charged lepton, 
large missing energy and two hadronic jets that form a W.
In this section we first describe the criteria applied to objects
selected in the event and then we describe the requirements made at
the event-level.

\subsection{Object Definitions}

The analysis relies on the standard reconstruction algorithms produced
by the CMS community. 
The PF2PAT procedure was used to coherently define the collection of particle-flow jets, leptons and
MET considered in the event selection.  
The technical details of the software configuration can be found in the group twiki page~\cite{WG_PATtuple_twiki}. 

\subsubsection{Electron Cuts (e+jets)}
\label{sec:electron_cuts}

Electrons are reconstructed using a gaussian sum 
filter (GSF) algorithm \cite{CMS-PAS-EGM-10-004},
and are required to pass electron ID cuts according 
to the simple cut-based electron ID~\cite{simplecutbasedelectronid}, 
with the ``VBTF Working Point 70''. 
The GSF algorithm accounts for possible energy loss due to
bremsstrahlung in the tracker layers.
The energy of an electron candidate with $\et>35~\gev$ is essentially
determined by the ECAL cluster energy, while its momentum direction
is determined by that of the associated track.
The simple cut based electron ID relies on three shower
shape variables with different cut values for the barrel and
the endcap regions, as shown in Table~\ref{tab:EleID}.

Additionally, we require
%%%%%%%%%%%%%%%%%%%
\begin{itemize}
\item Electron $E_\mathrm{T} > 35\,\mathrm{GeV}$.
\item Pseudorapidity $|\eta| < 2.5$. There is an exclusion range due
        to the ECAL barrel-endcap transition region, defined by
        $1.4442 < |\eta_{\mathrm{sc}}| < 1.566$, where
        $\eta_{\mathrm{sc}}$ is the pseudorapidity of the ECAL
        supercluster.
\item Impact parameter: We cut on the absolute value of the impact
       parameter calculated with respect to the average beamspot. We
       require:  $d_0(\mathrm{Bsp}) < 0.02\,\mathrm{cm}.$   
\item The selected electron candidates have to be isolated simultaneously in
the tracker, and in the electromagnetic and hadronic calorimeters.  Combined
relative isolation is defined as
%%%
\begin{equation*}
\mathrm{RelIso_{\mathrm{Comb}}} = \frac{I_{\mathrm{Trk}}+I_{\mathrm{EM}}+I_{\mathrm{had}}}{E_\mathrm{T}}.
\end{equation*} 
%%%
The electron candidate is required to have 
$\mathrm{RelIso_{\mathrm{Comb}}} < 0.05$ in order 
to be considered isolated. 
A pile-up offset subtraction in the isolation cone 
using fastjet algorithm \cite{FastJetPUSubtraction} is applied.
We also veto the presence of a second loose lepton in the event.
\item 
In order to reject events in which the electron candidate actually
originates from a conversion of a photon into an $e^{+}e^{-}$ pair, we
require the number of missed inner tracker layers of the electron
track to be exactly zero (i.e. there are no missed layers before the
first hit of the electron track from the beam line). In addition, we
reject any event in which the selected electron is flagged as a
conversion, \textit{i.e.}, an electron that has a 
distance of the partner track $|$\texttt{dist}$|$ $< 0.02$~mm and an
opening angle $|$\texttt{dcot}$|$ $< 0.02$~\cite{ConversionRejection}.
\end{itemize}
%%%%%%%%%%%%%%%%%%%
%%%%%%%%%%%%%%%%%%%
%%%%%%%%%%%%%%%%%%%
\begin{table}[bthp]
\begin{center}
{\footnotesize
\begin{tabular}{|c|c|c|c|c|}
\hline
ID Variable & WP70 Barrel & WP70 Endcaps & WP95 Barrel & WP95 Endcaps  \\
\hline
$\sigma_{i\eta i\eta}$ & 0.01 & 0.03 & 0.01 & 0.03 \\
$\phi_{\mathrm{SC}} - \phi_{\mathrm{trk}}$ & 0.03 & 0.02 & 0.8 & 0.7 \\
$\eta_{\mathrm{SC}} - \eta_{\mathrm{trk}}$ & 0.004 & 0.005 & 0.007 & 0.01 \\
\hline
\end{tabular}
\caption[.]{\label{tab:EleID} Cut values for electron identification
variables for VBTF Working Point (WP) 70 (barrel and endcap), as used
for the tight electron selection, and VBTF Working Point (WP) 95
(barrel and endcap), as used in the loose electron selection.}}
\end{center}
\end{table}
%%%%%%%%%%%%%%%%%%%
%%%%%%%%%%%%%%%%%%%%%%%%%%%%%%%%%%%%%%%%%%%%%%%%%%%%%%%%%%%%%%%%%%%%%%%%%%%%
%%%%%%%%%%%%%%%%%%%%%%%%%%%%%%%%%%%%%%%%%%%%%%%%%%%%%%%%%%%%%%%%%%%%%%%%%%%%

\subsubsection{Muon Cuts (mu+jets)}
\label{sec:muon_cuts}

Muon candidates are identified by two different 
algorithms~\cite{MUONPAS}: one proceeds from the inner tracker outwards, 
the other one starts from tracks measured in the muon chambers and matches 
and combines them with tracks reconstructed in the inner tracker. 
These selection criteria are summarized below:
%%%%%%%%%%%%%%%%%%%
\begin{itemize}
\item The muon candidate is reconstructed both as a global muon and
as a tracker muon.
\item The number of hits of the muon track in the silicon tracker has
to be $N_{\mathrm{Hits}} > 10$.
\item Number of pixel hits of the Tracker track $\ge 1$;
\item Number of muon hits of the Global track $\ge 2$;
\item Normalized $\chi^{2}$ of the Global track $< 10.0$.
\item Muon $p_{\mathrm{T}} > 25\,\mathrm{GeV}$.
\item Pseudorapidity $|\eta| < 2.1$.
\item Impact parameter: We cut on the absolute value of the impact
parameter calculated with respect to the beamspot. We require:
$d_0(\mathrm{Bsp}) < 0.02\,\mathrm{cm}.$
\item In order to make sure that the selected muon and the selected
jets come from the same hard interaction and not from pile up events,
we require that the $z$ coordinate of the PV of the event and the $z$
coordinate of the muon's inner track vertex lie within a distance of
less than 1~cm.
\item The selected muon candidates also have to be isolated.
We require the muon to be isolated simultaneously in the
tracker, and in the electromagnetic and hadronic calorimeters.  
The muon candidate is required to have
$\mathrm{RelIso_{\mathrm{Comb}}} < 0.1$ in order to be considered
isolated.
We also veto the presence of a second loose lepton in the event.
\end{itemize}

%%%%%%%%%%%%%%%%%%%%%%%%%%%%
%%%%%%%%%%%%%%%%%%%%%%%%%%%%
%%%%%%%%%%%%%%%%%%%%%%%%%%%%%%%%%%%%%%%%%%%%%%%%%%%%%%%%%%%%%%%%%%%%%%%%%%%%
%%%%%%%%%%%%%%%%%%%%%%%%%%%%%%%%%%%%%%%%%%%%%%%%%%%%%%%%%%%%%%%%%%%%%%%%%%%%


\subsubsection{Loose Electron}
For the purposes of rejecting events with more than one lepton we
define a loose electron, which has looser cuts. We consider electrons
which have $p_{\mathrm{T}} > 15\,\mathrm{GeV}/c$, $|\eta| < 2.5$, and
$\mathrm{RelIso_{\mathrm{Trk}}} < 0.2$ and which satisfy electron ID
cuts according to ``VBTF Working Point 95'' to be ``loose''. The cut
values for the electron ID variables used in the analysis can be found
in table~\ref{tab:EleID}.

\subsubsection{Loose Muon}
Additionally, to reject events with more than one lepton, we define a
loose muon, which has looser cuts. We consider all global muons which
have $p_{\mathrm{T}} > 10\,\mathrm{GeV}/c$, $|\eta| < 2.5$, and
$\mathrm{RelIso_{\mathrm{Trk}}} < 0.2$ to be loose muons.

\subsubsection{Jet Cuts}
\label{sec:firstStep_jets}

Jets are reconstructed with the anti-KT algorithm \cite{cacciari}, 
starting from the set of objects reconstructed by the particle 
flow \cite{pflow,CMS-PAS-JME-10-003,CMS-PAS-PFT-10-002}.
Jets are corrected such that the measured energy of the jet 
correctly reproduces the energy of the initial particle. 
The CMS standard L2 (relative) correction makes the jet response flat in $\eta$.
The standard L3 (absolute) correction brings the jet closer to the $\PT$ of 
a matched generated jet created using generator level input and a similar 
jet clustering algorithm.
The L2 and L3 corrections are calculated using Monte Carlo, and thus a 
L2L3 residual correction is applied that fixes the discrepancies between 
Monte Carlo and data~\cite{newjes-cms}.
In this analysis we use jets with measured (corrected) $\PT$  
greater than 30~$\gev$. 
We require $|\eta| < 2.4$ so that the jets fall within the
tracker acceptance.  
Jets are required to pass a set of loose identification
criteria; this requirement eliminates jets originating from or being seeded by
noisy channels in the calorimeter~\cite{Chatrchyan:2009hy}: 
%%%%%%%%%%%%%%
\begin{itemize}
\item Fraction of energy due to neutral hadrons $<$ 0.99.
\item Fraction of energy due to neutral EM deposits $<$ 0.99.
\item Number of constituents $>$ 1.
\item Number of charged hadrons candidates $>$ 0.
\item Fraction of energy due to charged hadrons candidates $>$ 0.
\item Fraction of energy due to charged EM deposits $<$ 0.99.
\end{itemize}
%%%%%%%%%%
All energy fractions are calculated from uncorrected jets.

\par
In order to account for electron and muon objects that
have been reconstructed as jets, we remove from the jet
collection any jet that falls within a
cone of radius $R= 0.3$ of a loose electron or a loose muon. 
This ``cleaning'' procedure is applied in order to ensure that the same
particle is not double counted as two different physics objects.
We require exactly two or three jets in the event.


\subsection{Event-Level Criteria}

The event should have a good primary vertex (PV). This means selecting
the primary vertex with the highest sum of $p_{T}^2$ of the tracks
associated with it and requiring it to have a number of degrees of
freedom (ndof) $\ge 4$, where ndof corresponds to the weighted sum of
the number of tracks usd for the construction of the PV. In addition,
the PV must lie in the central detector region of $abs(z) \le
24~\cm{rm}$ and $\rho \le 2~\rm{cm}$ around the nominal interaction
point.

In the e+jets channel, we select events which contain exactly one
tight electron candidate fulfilling the criteria described in
Section~\ref{sec:electron_cuts} and reject events which contain a
loose electron in addition to the tight electron. In this channel we
are only interested in the decay to electron and jets, and we
therefore reject events containing a loose muon.

In the mu+jets channel, we select events which contain exactly one
tight muon candidate whose criteria are described in
Section~\ref{sec:muon_cuts} and reject events which contain an
additional loose muon. In an analoguous way to the e+jets channel, we
reject events containing a loose electron.

In both channels we require an event to have missing transverse energy
(MET) in excess of $30(25)\,\mathrm{GeV}$ in the electron (muon)
channel and to have a $W$ transverse mass of $50\,\mathrm{GeV}$.
These cuts are designed to reduce the background from QCD multijet
production.

We further require exactly two or three jets passing the cuts
described in Section~\ref{sec:firstStep_jets}.  
% We require the
% invariant mass of the dijet system formed by the two highest $p_{T}$
% jets in the event to be between $65$ and $95\,\mathrm{GeV}$.

