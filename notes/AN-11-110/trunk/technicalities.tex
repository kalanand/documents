\section{Datasets}
\label{sec:technicalities}
% ---- ---- ---- ---- ---- ---- ---- ---- ---- ---- ---- ---- ---- ---- ---- ---- ---- ---- ---- ---- ---- ---- ----

\subsection{Samples used for the analysis}

The data sample used in this analysis was recorded by the CMS experiment in 2011.
Only certified runs and luminosity sections are considered, which means that a good functioning
of all CMS sub-detectors is required. The total statistics analyzed correspond to an integrated
luminosity of 5.0~fb$^{-1}$. % \LUMI{}.

% The analysis relies on centrally-produced Primary Datasets (PDs), each of which consists of a collection
% of High Level Trigger (HLT) paths. As explained in Section~\ref{sec:trigger}, single-lepton triggers
% are used to select the events interesting for this analysis in both
% the $e$ and $\mu$ channels, up to an instantaneous luminosity $L \sim 1\cdot10^{33}\percms$. At higher
% luminosities, we move to cross electron+di-jet triggers for the $e$ channel to cope with the
% unsustainable single-electron HLT rate. Therefore, the analysis relies on the so-called ``SingleElectron''
% and ``SingleMu'' PDs for the first data-taking period, and on the ``ElectronHad'' and ``SingleMu'' ones 
% for the most runs. 

The dataset used for the analysis and the corresponding run ranges are listed in Table~\ref{tab:datasets}.
All samples have been processed using a \texttt{CMSSW\_4\_2\_X} release version.

\begin{table}[htb]
  \begin{center}
  \begin{tabular}{r|r}
  \hline
  Dataset name & Run range \\
  \hline
  /EG/Run2010A-Apr21ReReco-v1/AOD   & 136033 - 144114  \\
  /Mu/Run2010A-Apr21ReReco-v1/AOD   &            \\ 
  \hline
  /Electron/Run2010B-Apr21ReReco-v1/AOD   &  144919 - 149442  \\
  /Mu/Run2010B-Apr21ReReco-v1/AOD         &             \\
  \hline
  /SingleElectron/Run2011A-May10ReReco-v1/AOD   & 160431 - 163869 \\
  /SingleMu/Run2011A-May10ReReco-v1/AOD         &                 \\
  \hline                                     
  /SingleElectron/Run2011A-PromptReco-v4/AOD    & 165088 - 167913 \\
  /SingleMu/Run2011A-PromptReco-v4/AOD          &                 \\
  \hline                                     
  /SingleElectron/Run2011A-05Aug2011-v1/AOD     & 170826 - 172619 \\
  /SingleMu/Run2011A-05Aug2011-v1/AOD           &                 \\
  \hline                                     
  /SingleElectron/Run2011A-PromptReco-v6/AOD    & 172620 - 173692 \\
  /SingleMu/Run2011A-PromptReco-v6/AOD          &                 \\
  \hline                                     
  /SingleElectron/Run2011B-PromptReco-v1/AOD    & 175832 - 180252 \\
  /SingleMu/Run2011B-PromptReco-v1/AOD          &                 \\
  \hline                                     
%  /ElectronHad/Run2011B-PromptReco-v1/AOD       & 175832 - 177053 \\
%  /SingleMu/Run2011B-PromptReco-v1/AOD          &                 \\
%  \hline                                     
%  /ElectronHad/Run2011B-PromptReco-v1/AOD       & 177074 - 177515 \\
%  /SingleMu/Run2011B-PromptReco-v1/AOD          &                 \\  
%  \hline
%  \texttt{/SingleElectron/Run2011A-May10ReReco-v1/AOD} & 160404 - 163869 \\
%  \texttt{/SingleMu/Run2011A-May10ReReco-v1/AOD}       & 160404 - 163869 \\
%  \hline
%  \texttt{/SingleElectron/Run2011A-PromptReco-v4/AOD} & 165088 - 165620 \\
%  \texttt{/SingleMu/Run2011A-PromptReco-v4/AOD}       & 165088 - 165620 \\
%  \hline
%  \texttt{/ElectronHad/Run2011A-PromptReco-v4/AOD} & 165970 - 167784 \\
%  \texttt{/SingleMu/Run2011A-PromptReco-v4/AOD}    & 165970 - 167784 \\
  \hline
  \end{tabular}
  \end{center}
  \caption{Summary of data samples used and run ranges of applicability.}
  \label{tab:datasets}
\end{table}%



\subsection{Monte Carlo samples}

Standard Model Higgs boson samples, 
as well as samples for a large variety of electroweak and QCD-induced background sources, 
have been generated and showered using different Monte Carlo generators.
To better reproduce the actual data-taking conditions, where there is a significant probability
that more than two protons interact in the same bunch crossing, pile-up (PU) events are
added on top of the hard scattering. Particle interactions with the detector were reproduced through
a detailed description of CMS.

The POWHEG-BOX generator \cite{Nason:2004rx,Frixione:2007vw,Alioli:2010xd,Nason:2009ai} has been used to
produce signal events, and the showering has been performed with PYTHIA6 \cite{pythia}. For this analysis, 
samples with Higgs mass hypotheses ranging from 160 to 600\GeVcc have been used.
 
The background samples used for the studies are listed in Table~\ref{tab:MCsamples},
together with the equivalent luminosity available for the study.

All MC samples considered in this analysis come from the official ``Fall11'' production. 
Events were reconstructed making use of a \texttt{CMSSW\_4\_2\_X} release version. 
The simulated samples are reweighted to represent the distribution of number of 
pp interactions per bunch crossing (pile-up) 
as measured in the data.

\begin{sidewaystable}[htb]
  \caption{ MC samples for background, signal and systematic study}
  \label{tab:datasets:mcstat}
  \begin{center}
    \begin{tabular}{l|l} 
      \hline
      sample & cross-section (pb) \\
      \hline
      /WJetsToLNu\_TuneZ2\_7TeV-madgraph-tauola/Fall11-PU\_S6\_START42\_V14B-v1/AODSIM    \\
      /WW\_TuneZ2\_7TeV\_pythia6\_tauola/Fall11-PU\_S6\_START42\_V14B-v1/AODSIM     \\
      /WZ\_TuneZ2\_7TeV\_pythia6\_tauola/Fall11-PU\_S6\_START42\_V14B-v1/AODSIM     \\
      /TTJets\_TuneZ2\_7TeV-madgraph-tauola/Fall11-PU\_S6\_START42\_V14B-v2/AODSIM    \\
      /QCD\_Pt-20\_MuEnrichedPt-15\_TuneZ2\_7TeV-pythia6/Fall11-PU\_S6\_START42\_V14B-v2/AODSIM  \\      
      /DYJetsToLL\_TuneZ2\_M-50\_7TeV-madgraph-tauola/Fall11-PU\_S6\_START42\_V14B-v1/AODSIM   \\
      /Tbar\_TuneZ2\_s-channel\_7TeV-powheg-tauola/Fall11-PU\_S6\_START42\_V14B-v1/AODSIM     \\
      /Tbar\_TuneZ2\_t-channel\_7TeV-powheg-tauola/Fall11-PU\_S6\_START42\_V14B-v1/AODSIM     \\
      /Tbar\_TuneZ2\_tW-channel-DS\_7TeV-powheg-tauola/Fall11-PU\_S6\_START42\_V14B-v1/AODSIM    \\
      /T\_TuneZ2\_s-channel\_7TeV-powheg-tauola/Fall11-PU\_S6\_START42\_V14B-v1/AODSIM     \\
      /T\_TuneZ2\_t-channel\_7TeV-powheg-tauola/Fall11-PU\_S6\_START42\_V14B-v1/AODSIM     \\
      /T\_TuneZ2\_tW-channel-DS\_7TeV-powheg-tauola/Fall11-PU\_S6\_START42\_V14B-v1/AODSIM     \\
      \hline
%%       /QCD\_Pt-20to30\_EMEnriched\_TuneZ2\_7TeV-pythia/Summer11-PU\_S4\_START42\_V11-v1/AODSIM\\
%%       /QCD\_Pt-30to80\_EMEnriched\_TuneZ2\_7TeV-pythia/Summer11-PU\_S4\_START42\_V11-v1/AODSIM\\
%%       /QCD\_Pt-80to170\_EMEnriched\_TuneZ2\_7TeV-pythia6/Summer11-PU\_S4\_START42\_V11-v1/AODSIM \\
%%       /QCD\_Pt-20to30\_BCtoE\_TuneZ2\_7TeV-pythia6/Summer11-PU\_S4\_START42\_V11-v1/AODSIM \\
%%       /QCD\_Pt-30to80\_BCtoE\_TuneZ2\_7TeV-pythia6/Summer11-PU\_S4\_START42\_V11-v1/AODSIM \\
%%       /QCD\_Pt-80to170\_BCtoE\_TuneZ2\_7TeV-pythia/Summer11-PU\_S4\_START42\_V11-v1/AODSIM   \\
%%      \hline
      /WJetsToLNu\_TuneZ2\_matchingdown\_7TeV-madgraph-tauola/Summer11-PU\_S4\_START42\_V11-v1/AODSIM  \\
      /WJetsToLNu\_TuneZ2\_matchingup\_7TeV-madgraph-tauola/Summer11-PU\_S4\_START42\_V11-v1/AODSIM  \\
      /WJetsToLNu\_TuneZ2\_scaledown\_7TeV-madgraph-tauola/Summer11-PU\_S4\_START42\_V11-v1/AODSIM          \\
      /WJetsToLNu\_TuneZ2\_scaleup\_7TeV-madgraph-tauola/Summer11-PU\_S4\_START42\_V11-v1/AODSIM            \\
      /WToLNu\_1jEnh2\_2jEnh35\_3jEnh40\_4jEnh50\_7TeV-sherpa/Summer11-PU\_S4\_START42\_V11-v1/AODSIM      \\
      \hline 
      /GluGluToHToWWToLNuQQ\_M-*\_7TeV-powheg-pythia6/Fall11-PU\_S6\_START42\_V14B-v1/AODSIM  \\
      /VBF\_HToWWToLNuQQ\_M-*\_7TeV-powheg-pythia6/Fall11-PU\_S6\_START42\_V14B-v1/AODSIM \\
     Higgs mass from 170 to 600~GeV  &  \\
      \hline
    \end{tabular}
  \end{center}
  \caption{Summary of Monte Carlo samples used in the analysis.}
  %FIXME add the corresponding cross-sections
  \label{tab:MCsamples}
\end{sidewaystable}

\clearpage

% \begin{table}[htb]
%   \begin{center}
%   \begin{tabular}{c|r|r}
%   \hline
%   Channel & Dataset name & \lumi (\fbinv) \\
%   \hline
%   W+jets                        & \footnotesize{\texttt{/WJetsToLNu\_TuneZ2\_7TeV-madgraph-tauola}} & 0.48 \\
%   Z+jets                        & \footnotesize{\texttt{/DYJetsToLL\_TuneZ2\_M-50\_7TeV-madgraph-tauola}} & 0.76 \\
%   WW                            & \footnotesize{\texttt{/WWtoAnything\_TuneZ2\_7TeV-pythia6-tauola}} & 47.90 \\
%   WZ                            & \footnotesize{\texttt{/WZtoAnything\_TuneZ2\_7TeV-pythia6-tauola}} & 116.00 \\
%   ZZ                            & \footnotesize{\texttt{/ZZtoAnything\_TuneZ2\_7TeV-pythia6-tauola}} & 357.00 \\
%   t$\bar{\textnormal{t}}$+jets  & \footnotesize{\texttt{/TTJets\_TuneZ2\_7TeV-madgraph-tauola}} & 7.39 \\
%   t+jets ($t$-channel)          & \footnotesize{\texttt{/TToBLNu\_TuneZ2\_s-channel\_7TeV-madgraph}} & 332.00 \\
%   t+jets ($s$-channel)          & \footnotesize{\texttt{/TToBLNu\_TuneZ2\_t-channel\_7TeV-madgraph}} & 23.10 \\
%   t+jets ($t$W-channel)         & \footnotesize{\texttt{/TToBLNu\_TuneZ2\_tW-channel\_7TeV-madgraph}} & 46.20 \\
%   QCD                           & \footnotesize{\texttt{/QCD\_Pt-20to30\_EMEnriched\_TuneZ2\_7TeV-pythia6}} & 0.01 \\
%   QCD                           & \footnotesize{\texttt{/QCD\_Pt-30to80\_EMEnriched\_TuneZ2\_7TeV-pythia6}} & 0.02 \\
%   QCD                           & \footnotesize{\texttt{/QCD\_Pt-80to170\_EMEnriched\_TuneZ2\_7TeV-pythia6}} & 0.06 \\
%   QCD                           & \footnotesize{\texttt{/QCD\_Pt-20to30\_BCtoE\_TuneZ2\_7TeV-pythia6/}} & 0.02 \\  
%   QCD                           & \footnotesize{\texttt{/QCD\_Pt-30to80\_BCtoE\_TuneZ2\_7TeV-pythia6/}} & 0.01 \\  
%   QCD                           & \footnotesize{\texttt{/QCD\_Pt-80to170\_BCtoE\_TuneZ2\_7TeV-pythia6/}} & 0.06 \\    
%   QCD                           & \footnotesize{\texttt{/QCD\_Pt-20\_MuEnrichedPt-15\_TuneZ2\_7TeV-pythia6}} & 0.35 \\
%   \hline
%   \end{tabular}
%   \end{center}
%   \caption{Summary of Monte Carlo samples used and the corresponding equivalent integrated luminosity
%   for each sample.}
%   \label{tab:MCsamples}
% \end{table}
