\section{Physics objects reconstruction}
\label{sec:reco}
% ---- ---- ---- ---- ---- ---- ---- ---- ---- ---- ---- ---- ---- ---- ---- ---- ---- ---- ---- ---- ---- ---- ----


The analysis relies on the standard reconstruction algorithms produced by the CMS community.

Electrons are reconstructed with the gaussian sum filter algorithm \cite{CMS-AN-2010/472},
and are required to survive the WP90 identification selections \cite{CMS-AN-2010/264, muIDtwiki, eleIDtwiki}.
The selections applied are listed in Table \ref{tab:eleID90}.

\begin{table}[hbt]
  \begin{center}
  \begin{tabular}{l|l|l}
  \hline
  varible & barrel limit & endcap limit \\
  \hline
  track isolation                       & 0.10   & 0.10  \\    
  EM isolation                          & 0.10   & 0.10  \\    
  HAD isolation                         & 0.10   & 0.10 \\     
  combined isolation (PU subtracted)    & 0.085  & 0.051  \\          
  $\sigma_{i\eta i\eta}$                & 0.01   & 0.01  \\           
  $\Delta\varphi$                       & 0.071   & 0.047  \\  
  $\Delta\eta$                          & 0.007  & 0.011 \\  
  \hline
  \end{tabular}
  \end{center}
  \caption{The identification selections applied on electrons at preselection level.
           \textbf{FIXME check the cuts}}
  \label{tab:eleID90}
\end{table}%

A pile-up offset subtraction in the isolation cone using fastjet algorithm \cite{Cacciari2008119} is applied.

In addition to the identification and isolation criteria, we also use the following criteria to reject
contributions from photon conversions \cite{CMS-AN-2009/178, CMS-AN-2009/159}:
\begin{itemize}
 \item no missing hits in the pixel detector
 \item distance of the partner d $>$ 0.02 or $\Delta cot\theta$ $>$ 0.02
\end{itemize}


The electron track is required to be compatible with the reconstructed primary vertex
asking the logitudinal impact parameter $dz$ and 
the transverse impact parameter $d_{0}$ to be less than
500 $\mu$m.
Additional electron selections are summarized in Table \ref{tab:eleSelAdd}.

\begin{table}[hbt]
  \begin{center}
  \begin{tabular}{l|l}
  \hline
        & Selection \\
  \hline
  Impact parameter                      & $d_{0}<$ 500 $\mu$m and $dz<$ 500 $\mu$m \\    
  Missing hits                          & 0  \\    
  $\Delta cot\theta$                    & $>$ 0.02 \\
  Dist                                  & $>$ 0.02 \\
  \hline
  \end{tabular}
  \end{center}
  \caption{Photon conversion rejection and primary vertex compatibility for electrons.
           \textbf{FIXME check the cuts}}
  \label{tab:eleSelAdd}
\end{table}%



Muons obtained from the CMS reconstruction \cite{PAS-MUO-2010-002} 
are required to survive the selections defined in Table \ref{tab:muID}.

\begin{table}[hbt]
  \begin{center}
  \begin{tabular}{l|l}
  \hline
  variable & limit \\
  \hline
  combined isolation (PU subtracted)       & 0.15 \\        
  $\chi^2$/ndof                            & 10   \\     
  nb. valid tk hits                        & 10   \\              
  nb. valid muon hits                      & 0    \\           
  isTrackerMuon                            & 1    \\   
  isStandaloneMuon                         & 1    \\      
  isGlobalMuon                             & 1    \\      
  \hline
  \end{tabular}
  \end{center}
  \caption{The identification selections applied on muons at preselection level.\textbf{FIXME check the cuts}}
  \label{tab:muID}
\end{table}%


The muon track is required to be compatible with the reconstructed primary vertex
asking the logitudinal impact parameter $dz$ to be less than 500 $\mu$m and 
the transverse impact parameter $d_{0}$ to be less than 100 $\mu$m.
Additional muon selections are summarized in Table \ref{tab:muSelAdd}.

\begin{table}[hbt]
  \begin{center}
  \begin{tabular}{l|l}
  \hline
        & Selection \\
  \hline
  Impact parameter                      & $d_{0}<$ 100 $\mu$m and $dz<$ 500 $\mu$m \\    
  \hline
  \end{tabular}
  \end{center}
  \caption{Muon primary vertex compatibility.\textbf{FIXME check the cuts}}
  \label{tab:muSelAdd}
\end{table}%

Jets are reconstructed with the anti-KT algorithm \cite{cacciari}, 
starting from the set of objects reconstructed by the particle flow \cite{pflow1,pflow2}.
The L2 and L3 corrections \cite{newjes-cms} are applied to them in the simulated samples,
while also the residual corrections \cite{newjes-cms} are applied on data.

