\section{Analysis overview}
\label{sec:overview}

The broad framework of the analysis is shown in
Fig.~\ref{fig:dfdlevel0}.  The figure is intended to depict the flow
of major data elements and how they are combined and transformed by
analysis processes into the final result, the measured signal
cross-section with uncertainties. The physics object reconstruction
and selection are broadly grouped into a phase of the analysis labeled
pre-selection, the output of which is a set of ntuples for iterative
analysis.  A parallel path is followed by the QCD preselection, since
the QCD background estimates are data-driven and utilize a different
set of selections (Section~\ref{sec:qcd}).

The analysis ``proper'' includes the event selection followed by a
detailed study of the 2-body invariant mass of the major backgrounds,
which is covered in more detail in Sections~\ref{sec:wjetsShape}
and~\ref{sec:mjj_fit}. Once these shapes have been optimally
determined the total yields are fed into the cross-section calculation
(Section~\ref{sec:results}).

%%%%%%%%%%%%%%%%%%%
\begin{figure}[bthp]
\begin{center}
\includegraphics[width=\textwidth]{figs/mjjdfd0.pdf}
\caption{\label{fig:dfdlevel0}A schematic of the major elements of
the analysis in terms of data flows and transforms.}
\end{center}
\end{figure}
%%%%%%%%%%%%%%%%%%%
