 The variables most often used in analyses involving jets are usually the jet direction and the momentum transverse to the beam ($p_T$). 
However, being  the jets composite objects, their masses and internal substructure contain additional information.
One strong motivation for studies of the internal substructure of jets is that at the LHC particles such as $W$ and $Z$ bosons and top quarks are produced abundantly with significant Lorentz boosts. The same may also be true for new particles produced at the LHC. When
 such particles decay hadronically, the products tend to be collimated in a small area of the detector. For sufficiently large boosts, the resulting hadrons can be clustered into a single jet. Substructure studies offer a technique to extract these single jets of interest from the overall jet background. Such techniques have been found promising for boosted W decay identification, Higgs searches and boosted top identification amongst others \cite{jetsub}. However, many of these promising approaches have never been tested with collision data and rely on the assumption that the internal structure of jets is well modelled by parton-shower Monte Carlo approaches. It is therefore important to measure some of the relevant variables in a sample of jets to verify the expected features. First results on a QCD enriched samples of boosted jets have been presented by ATLAS \cite{atlasJS}. We present here similar studies on the jet mass and substructure in dijet events. 


The observable we measure is the differential cross section with respect to the jet mass,
corrected for detector inefficiency and resolution effects. This is a doubly
differential distribution in the average transverse momentum ($\pt$) of
the highest two $\pt$ jets ($\pt^{AVG} = ({\pt}_1 + {\pt}_2) / 2$), and 
the average jet mass of the highest two $\pt$ jets ($m_J^{AVG} = ({m_J}_1 + {m_J}_2) / 2$): 

\begin{equation}
\label{eq:dsigmadmjet}
\frac{d\sigma}{dm_J^{AVG} d\pt^{AVG}}
\end{equation}

In order to remove known discrepancies of theoretical predictions of the $\pt^{AVG}$,
the measurement is presented by comparing the differential cross section of the
jet mass in a given $\pt^{AVG}$ range, normalized to the number of events in that
range. This is a probability distribution function in the jet mass $m_J^{AVG}$, 
with units of 1/GeVcc. 

This is formally equivalent to the following expression

\begin{equation}
\label{eq:pdf_mjet}
PDF(m_J^{AVG}) = \frac{1}{\frac{d\sigma}{d\pt^{AVG}}} \times \frac{d\sigma}{d\pt^{AVG} dm_J^{AVG}}. 
\end{equation}

For convenience, if we define 

\begin{equation}
\hat{\sigma} = \frac{d\sigma}{d\pt^{AVG}}
\end{equation}

then Equation~\ref{eq:pdf_mjet} becomes

\begin{equation}
\label{eq:pdf_mjet_simple}
PDF(m_J^{AVG}) = \frac{1}{\hat{\sigma}} \times \frac{d\hat{\sigma}}{dm_J^{AVG}}. 
\end{equation}

The results presented in this paper are measurements
of the value of Eq.~\ref{eq:pdf_mjet_simple} for 
jets with and without various grooming algorithms applied, 
corrected for detector inefficiency and resolution to the
particle level, and are compared to several theoretical
predictions. 

