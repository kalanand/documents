\label{sec:cms_detector}


The CMS detector~\cite{:2008zzk}
is a general-purpose device, however it has
many features particularly suited for reconstruction of 
energetic jets,specifically, the finely segmented electromagnetic
and hadronic calorimeters, and the charged particle tracking.
The charged particles are reconstructed by the inner tracker,
immersed in a $3.8$~T axial magnetic field; the inner tracker consists
of three layers and two endcap disks of pixel sensors, and ten
barrel layers and twelve endcap disks of silicon strips.  This
arrangement results
in a full azimuthal coverage within $|\eta| < 2.5$, where $\eta$
is the pseudorapidity and is defined as $\eta = -\ln\tan(\theta/2)$.
The CMS uses a polar coordinate system, with the 
$z$ axis coinciding with the axis of symmetry of the CMS detector,
and oriented in the counterclockwise proton direction; here $\theta$ 
is the polar angle defined with respect to the positive $z$ axis.
The pseudorapidity is an approximation for the full rapidity $y$, and
the approximation is exact for massless particles. Since many of the
particles we use are not massless, we use the full rapidity $y$ which
is defined as $y = \frac{1}{2} \frac {E + p_{z}/c}{E - p_{z}/c}$.

A lead-tungstate crystal electromagnetic calorimeter (ECAL) and 
a brass-scintillator hadronic calorimeter (HCAL) surround the tracking
volume and allow photon, electron and jet reconstruction up to $|\eta|=3$.
The ECAL and HCAL cells are grouped into towers projecting radially 
outward from the interaction region.  In the central region ($|\eta|<1.74$)
the towers have dimensions $\Delta\eta = \Delta\phi = 0.087$; however,
at higher $\eta$, the $\Delta\eta$ and $\Delta\phi$ widths increase.  
ECAL and HCAL
cell energies above the noise suppression thresholds are combined within
each tower to define the calorimeter tower energy, and the towers are further
combined into clusters, which are then identified as jets.  For an improved
jet reconstruction, the tracking and calorimeter information is combined 
in an algorithm called particle-flow~\cite{particleflow}, which is described
below.

