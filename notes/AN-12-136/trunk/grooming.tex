\subsection{Sequential jet-clustering algorithms}
\label{sec:algos}

In this analysis, jets are clustered with sequential jet-clustering algorithms. 
In these algorithms, the four-vectors of input particles are combined pairwise, 
via four-vector addition, until a final jet
is found. Specifically, for each pair of input particles $i$ and $j$,
two quantities are computed, the first is a distance measure between
the two particles, and the second is the so-called ``beam distance''
of each particle:

\begin{eqnarray}
\label{eq:dij}
d_{ij} &=& \mathrm{min}({\pt}_i^{2n},{\pt}_j^{2n}) \Delta R_{ij}^2 / R^2 \\
d_{iB} &=& {\pt}_i^{2n}
\end{eqnarray}

where $\pt$ is the transverse momentum, 
$\Delta R = \sqrt{(\Delta \eta_{ij})^2 + (\Delta\phi_{ij})^2 }$
is the angular distance between the two particles $i$ and $j$,
and $R$ is an order-unity parameter chosen for the algorithm.

The value for $n$ is a parameter of the algorithm in question and
governs the shape of the jets.
The first such sequential combination algorithm is called the
$k_{\mathrm{T}}$ (KT) algorithm and has $n=1$. These jets are typically irregularly
shaped and are useful for reconstructing lower momentum jets~\cite{ktalg}. 
The second such algorithm
is called the anti-$k_{\mathrm{T}}$ (AK) algorithm and has $n=-1$. This algorithm
behaves like an idealized cone algorithm, and are used extensively
at the LHC experiments and elsewhere~\cite{ktalg}. The third such
algorithm is called the Cambridge-Aachen (CA) algorithm and has $n=0$. This
algorithm uses only angular information, and like the $k_\mathrm{T}$ algorithm
has irregularly-shaped jets. The CA algorithm is very useful for distinguishing
jet substructure~\cite{CAcambridge,CAaachen}.

Jet grooming - elimination of uncorrelated UE/PU radiation from a target jet - is useful irrespective of the specific 
boosted particle search and can even be applied for slow-moving heavy particles that decay to well-separated jets. 
We consider in this analysis three different forms of grooming: trimming, pruning and filtering. 

There are choices of what jet algorithm (KT, AK, or CA) can be used by all of these grooming
algorithms. These algorithms can use different jet algorithms for jet finding and the
substructure determination. We have chosen to cluster our jets with the anti-$k_{\mathrm{T}}$
algorithm with $R=0.7$ (AK7), as these are extensively studied at CMS. 
Comparisons of AK jets with $R=0.5$ (AK5) AND $R=0.8$ (AK8)
are also investigated, 
as well as
with the CA algorithm with $R$=0.8 (CA8) and $R$=1.2 (CA12). The latter
two are compared because of their usage in other CMS analyses~\cite{EXO-11-006,HIG-11-XX}. 
After the initial jet clustering, the choice of algorithm
for the substructure determination depends on the algorithm chosen and is described
in detail below. 

\subsection{Pruning algorithm}

Pruning was introduced by Ellis, Vermilion and Walsh~\cite{pruning}-\cite{pruning2}. 
In our implementation, after
the jets are clustered with the AK7 algorithm, the pruning algorithm reclusters the constituents
of the AK7 jet with the CA algorithm, with extra
veto conditions applied in addition to the standard conditions in Equation~\ref{eq:dij}. 
The particle is vetoed if either of the following two conditions are met:

\begin{eqnarray}
z_{ij} & = & \frac{\mathrm{min}({\pt}_i,{\pt}_j)}{{\pt}_{i+j}} < z_{\mathrm{cut}} \\
\Delta R_{ij} & < & D_{\mathrm{cut}} = \alpha \times \frac{m_J}{{\pt}_J}
\end{eqnarray}

where ${\pt}_i$ and ${\pt}_j$ are the transverse momentum of the constituents,
${\pt}_{i+j}$ is the transverse momentum of the four-vector sum of those constituents,
$m_J$ and ${\pt}_J$ are the mass and transverse momentum of the AK7 jet, 
and $z_{\mathrm{cut}}$ and $\alpha$ are parameters of the algorithm, 
chosen to be 0.1 and 0.5, respectively. 

\subsection{Trimming algorithm}

Trimming is a technique that ignores particles within a jet that fall below 
a dynamic $\pt$ threshold. It was introduced by Krohn, Thaler and Wang in~\cite{trimming}. 
Trimming reclusters the jets's constituents with the $k_{\mathrm{T}}$ 
algorithm with a radius 
$R_{sub}$ and then accepts only the subjets that have 
${\pt}_{sub} > f_{cut}$, where $f_{cut}$ is taken proportional 
either to the jet's $\pt$ or to the event's total$H_T$.
The values $R_{sub}$ and $f_{cut}$ are parameters of the algorithm,
taken to be 0.2 and 0.03, respectively. 



\subsection{Filtering algorithm}

The filtering procedure aims to identify relatively hard, symmetric splittings in a jet that contribute significantly to the jet invariant mass. This procedure is taken from recent Higgs search studies~\cite{boostedHiggs}. The parameters are tuned to maximise sensitivity to a Standard Model Higgs boson decaying to $b\bar{b}$, but this procedure is suitable generally for identifying two-body decay processes. The effect of the procedure is to search for jets where the clustering process combined two relatively low mass objects to make a much more massive object. This indicates the presence of a heavy particle decay. The procedure then attempts to retain only the constituents believed to be related to the decay of this particle.
The identification strategy proposed in~\cite{boostedHiggs} uses the inclusive, longitudinally invariant CA algorithm to flexibly adapt to the fact that the two-jet angular separation varies significantly with 
the heavy particle $\pt$ and decay orientation. In this algorithm the angular distance 
$\Delta R^2_{ij} = (\Delta \eta_{ij} )^2 + (\Delta \phi_{ij} )^2$, 
where $y$ is the pseudorapidity and $\phi$ the azimuthal angle, is calculated between all 
pairs of objects $i$ and $j$. The closest pair is combined into a single object, the set of distances is 
updated, and the procedure is repeated until all objects are separated by a $\Delta R_{ij} > R$, where $R$ 
is a parameter of the algorithm. This provides a hierarchical structure for the clustering, like the 
$k_{\mathrm{T}}$ algorithm but in angles rather than in relative transverse momenta.
%Each stage in the clustering combines two objects $i$ and $j$ to make another object $j$. 
We use the definition $v = \frac{min({\pt}_i^2,{\pt}_j^2) }{m^2_{i+j}} \Delta R^2$. The procedure takes a
jet to be the object $j$ and applies the following:
\begin{enumerate}
\item Undo the last clustering step of $i+j$ to get $i$ and $j$. These are ordered such that their mass has the property $m_{i} > m_{j}$. If $i+j$ cannot be unclustered (i.e. it is a single particle) then it is not a suitable candidate, so discard this jet.
\item  If the splitting has $m_{i}/m_{i+j} < \mu$ (large change in jet mass) and $v > v_{cut}$ (fairly symmetric) then continue, otherwise redefine $i+j$ as $i$ and go back to step 1. Both $\mu$ and $v_{cut}$ are parameters of the algorithm.
\item  Recluster the constituents of the jet with the CA algorithm with an $R$-parameter of $R_{filt} =\mbox{min} (0.3, \Delta R^2_{ij} /2)$ finding $n$ new subjets $s_1, s_2 ...s_n$ ordered in descending $\pt$.
\item  Redefine the jet as the sum of subjet four-momenta $\sum_{i=1}^{min(n,3)} s_i$
\end{enumerate}

\noindent The algorithm parameters $\mu$ and $v_{cut}$ are taken as 0.67 and 0.09 respectively.
The $\mu$ cut attempts to identify a hard structure in the distribution of energy in the jet, which would imply the decay of a heavy particle. The cut on $v$ further helps by suppressing very asymmetric decays of the type favoured by splittings of quarks and gluons.
Steps 3 and 4 filter out some of the particles in the candidate jet, the aim being to retain particles relevant to the hard process while reducing the contribution from effects like underlying event and pile-up. The 4-vector after step 4 can be treated like a new jet. This new jet has a $\pt$ and mass less than or equal to those of the original jet.

