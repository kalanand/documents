This analysis uses standard physics objects provided by the PAT framework 
and approved by the relevant POGs.  This section describes the reconstruction,
identification, and selection of electrons, muons, jets, and
missing transverse energy. 
Events are reconstructed using the particle-flow reconstruction algorithm~\cite{particleflow},
which attempts to reconstruct all stable particles in an event by combining information from
all subdetectors. The algorithm categorizes all particles into five types: muons,
electrons, photons, charged and neutral hadrons. The resulting particle flow candidates are passed
to each jet clustering algorithm to create "particle flow jets".
 Since pile-up affects all of these physics objects,
we begin with a description of the primary vertex selection and the methods 
applied to mitigate the effects of pile-up.

\subsection{Primary vertex selection and pile-up treatment}

Primary vertices are identified using tracks clustered with the Deterministic
Annealing algorithm~\cite{PVDA}.  Reconstructed primary vertices are required to
have a $z$ position within $24\cm$ of the nominal detector center, a radial
position within $2\cm$ of the beamspot, and the vertex fit must include more 
than four degrees of freedom.  The primary vertex with the largest value of
$\sum_i {\pt}_i^2$ is selected, where ${\pt}_i$ is the transverse momentum of 
the $i$th track in the vertex.

The 2011 data sample contains a significant number of additional interactions 
per bunch crossing, an effect known as pile-up (PU).  The average number
of PU interactions in each triggered event is roughly given by the average number 
of reconstructed primary vertices, which was a little less than six for the first 
$2\fbinv$ of data collected, increasing to well over ten for the later running.  
Over the course of 2011 LHC operation, in-time PU as well as out-of-time pile-up
increased.  Figure~\ref{fig:PVreweight} shows the distribution of primary vertices
in Z$(\mu\mu)$+ jet  events.

PU affects jet momentum reconstruction, and therefore the reconstruction of the jet 
mass.  It also affects the MET reconstruction, 
lepton isolation and b-tagging.  There are two distinct approaches to address all these 
effects (apart from MET):
\begin{itemize}
 \item {\bf PFnoPU}: also known as Charged Hadron Subtraction (CHS), 
       PFPU is an algorithm embedded in the PF2PAT processing chain that attempts to 
       filter all charged hadrons that do not appear to originate from the primary 
       interaction.  This approach is very effective but only works in the pseudorapidity 
       region covered by the Tracker, and only for in-time PU.  Algorithms for tagging b 
       jets are not impacted, since they apply their own track pre-filtering that is also 
       designed to be PU-resistant.
 \item {\bf Fastjet}: is an external software package from which CMSSW takes virtually 
       all its jet reconstruction services~\cite{FastJet}.  In particular it provides the 
       means to calculate the momentum density per unit area $\rho$ due to PU for each 
       event, which can be used to subtract the contamination of jets and lepton isolation 
       cones based on their respective areas.  These methods are therefore referred 
       to as ''Fastjet Subtraction.'' 
\end{itemize}
Ideally, charged hadrons from PU interactions are filtered from the event first before
the application of Fastjet. In this analysis, both the PFnoPU and Fastjet Subtraction 
methods are applied consistently in the reconstruction and identification of jets,
and in the calculation of lepton isolation.

The standard reweighting technique~\cite{PUreweight} is used in this analysis, with
different weighting applied for 2011A and 2011B.  We use {\tt Fall 11} Monte Carlo, which
has a pile-up profile closer to that in the full data.
%A variation of the mean of the measured PU distribution by $\pm 0.5$ would 
%yield a reasonable systematic uncertainty of that measurement, and such a variation 
%could potentially cover all systematic uncertainties of the analysis due to PU, after 
%appropriate techniques were applied and validated to correct physics objects for 
%PU effects.


\subsection{Jet selection}
\label{sec:jet_cuts}

The jet selection used in this analysis was the ``loose'' jet
identification. 


\subsection{Missing transverse energy}

The use of missing transverse energy is central to the analyses presented in this
note.  It is critical in the reconstruction of the \WtoLN\ decays, and
is used in the $Z$+jet channels to increase the purity of the selection.
 For the offline analysis, missing transverse 
energy is computed from the list of particle-flow objects with the method described 
in~\cite{CMS-PAS-JME-10-003}.  The vector \VEtmiss\ is calculated as the negative of 
the vectorial sum of transverse momenta of all particle-flow objects identified in the 
event, and the magnitude of this vector is referred to as pfMET.  

