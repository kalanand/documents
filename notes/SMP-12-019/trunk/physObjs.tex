This analysis uses standard physics objects provided by the PAT framework 
and approved by the relevant POGs.  This section describes the reconstruction,
identification, and selection of electrons, muons, jets, and
missing transverse energy. 
Events are reconstructed using the particle-flow reconstruction algorithm~\cite{particleflow},
which attempts to reconstruct all stable particles in an event by combining information from
all subdetectors. The algorithm categorizes all particles into five types: muons,
electrons, photons, charged and neutral hadrons. The resulting particle flow candidates are passed
to each jet clustering algorithm to create "particle flow jets".
 Since pile-up affects all of these physics objects,
we begin with a description of the primary vertex selection and the methods 
applied to mitigate the effects of pile-up.

\subsection{Primary vertex selection and pile-up treatment}

Primary vertices are identified using tracks clustered with the Deterministic
Annealing algorithm~\cite{PVDA}.  Reconstructed primary vertices are required to
have a $z$ position within $24\cm$ of the nominal detector center, a radial
position within $2\cm$ of the beamspot, and the vertex fit must include more 
than four degrees of freedom.  The primary vertex with the largest value of
$\sum_i {\pt}_i^2$ is selected, where ${\pt}_i$ is the transverse momentum of 
the $i$th track in the vertex.

The 2011 data sample contains a significant number of additional interactions 
per bunch crossing, an effect known as pile-up (PU).  The average number
of PU interactions in each triggered event is roughly given by the average number 
of reconstructed primary vertices, which was a little less than six for the first 
$2\fbinv$ of data collected, increasing to well over ten for the later running.  
Over the course of 2011 LHC operation, in-time PU as well as out-of-time pile-up
increased.  Figure~\ref{fig:PVreweight} shows the distribution of primary vertices
in Z$(\mu\mu)$+ jet  events.

PU affects jet momentum reconstruction, and therefore the reconstruction of the jet 
mass.  It also affects the MET reconstruction, 
lepton isolation and b-tagging.  There are two distinct approaches to address all these 
effects (apart from MET):
\begin{itemize}
 \item {\bf PFnoPU}: also known as Charged Hadron Subtraction (CHS), 
       PFPU is an algorithm embedded in the PF2PAT processing chain that attempts to 
       filter all charged hadrons that do not appear to originate from the primary 
       interaction.  This approach is very effective but only works in the pseudorapidity 
       region covered by the Tracker, and only for in-time PU.  Algorithms for tagging b 
       jets are not impacted, since they apply their own track pre-filtering that is also 
       designed to be PU-resistant.
 \item {\bf Fastjet}: is an external software package from which CMSSW takes virtually 
       all its jet reconstruction services~\cite{FastJet}.  In particular it provides the 
       means to calculate the momentum density per unit area $\rho$ due to PU for each 
       event, which can be used to subtract the contamination of jets and lepton isolation 
       cones based on their respective areas.  These methods are therefore referred 
       to as ''Fastjet Subtraction.'' 
\end{itemize}
Ideally, charged hadrons from PU interactions are filtered from the event first before
the application of Fastjet. In this analysis, both the PFnoPU and Fastjet Subtraction 
methods are applied consistently in the reconstruction and identification of jets,
and in the calculation of lepton isolation.

The standard reweighting technique~\cite{PUreweight} is used in this analysis, with
different weighting applied for 2011A and 2011B.  We use {\tt Fall 11} Monte Carlo, which
has a pile-up profile closer to that in the full data.
%A variation of the mean of the measured PU distribution by $\pm 0.5$ would 
%yield a reasonable systematic uncertainty of that measurement, and such a variation 
%could potentially cover all systematic uncertainties of the analysis due to PU, after 
%appropriate techniques were applied and validated to correct physics objects for 
%PU effects.


\subsection{Electron selection}
\label{sec:electron_cuts}
Electrons are reconstructed using a gaussian sum 
filter (GSF) algorithm \cite{CMS-PAS-EGM-10-004},
and are required to pass electron ID cuts according 
to the simple cut-based electron ID~\cite{simplecutbasedelectronid}, 
with the ``VBTF Working Point 95' ($Z(ee)$) or ``VBTF Working Point 70'' ($W(e\nu)$). 
The GSF algorithm accounts for possible energy loss due to
bremsstrahlung in the tracker layers.
The energy of an electron candidate with $\et>30~\gev$ is essentially
determined by the ECAL cluster energy, while its momentum direction
is determined by that of the associated track.
The simple cut based electron ID relies on three shower
shape variables with different cut values for the barrel and
the endcap regions. The three variables are:
%%%%%%%%%%%%%%%%%%%
\begin{itemize}
\item $\sigma_{i\eta i\eta}$, the supercluster $\eta$ width.
\item $\eta_{\mathrm{SC}} - \eta_{\mathrm{trk}}$: Difference between
      the $\eta$ of the supercluster (SC) and the $\eta$ of the track,
      extrapolated from the vertex.
\item $\phi_{\mathrm{SC}} - \phi_{\mathrm{trk}}$: Difference between
      the $\phi$ of the supercluster and the $\phi$ of the track,
      extrapolated from the vertex.
\end{itemize}
%%%%%%%%%%%%%%%%%%%
The cut values used in the analysis can be found in
Table~\ref{tab:EleID}.
%%%%%%%%%%%%%%%%%%%
\begin{table}[htbp!]
\begin{center}
{\footnotesize
\begin{tabular}{|c|c|c|c|c|}
\hline
ID Variable & WP70 Barrel & WP70 Endcaps & WP95 Barrel & WP95 Endcaps  \\
\hline
$\sigma_{i\eta i\eta}$ & 0.01 & 0.03 & 0.01 & 0.03 \\
$\phi_{\mathrm{SC}} - \phi_{\mathrm{trk}}$ & 0.03 & 0.02 & 0.8 & 0.7 \\
$\eta_{\mathrm{SC}} - \eta_{\mathrm{trk}}$ & 0.004 & 0.005 & 0.007 & 0.01 \\
\hline
\end{tabular}
\caption[.]{\label{tab:EleID} Cut values for electron identification
variables for VBTF Working Point (WP) 70 (barrel and endcap), as used
for the $W(e\nu)$ electron selection, and VBTF Working Point (WP) 95
(barrel and endcap), as used in the $Z(ee)$ electron selection.}}
\end{center}
\end{table}
%%%%%%%%%%%%%%%%%%%

Additionally, we require
%%%%%%%%%%%%%%%%%%%
\begin{itemize}
\item Electron $E_\mathrm{T} > 30 (20),\mathrm{GeV}$ for $W(e\nu)$ ($Z(ee)$).
\item Pseudorapidity $|\eta| < 2.5$. There is an exclusion range due
        to the ECAL barrel-endcap transition region, defined by
        $1.4442 < |\eta_{\mathrm{sc}}| < 1.566$, where
        $\eta_{\mathrm{sc}}$ is the pseudorapidity of the ECAL
        supercluster.
\item Impact parameter: We cut on the absolute value of the impact
       parameter calculated with respect to the average beamspot. We
       require:
\begin{equation*}
 d_0(\mathrm{Bsp}) < 0.02\,\mathrm{cm}.    
\end{equation*}
\item The selected electron candidates have to be isolated simultaneously in
the tracker, and in the electromagnetic and hadronic calorimters.  Combined
relative isolation is defined as
%%%
\begin{equation*}
\mathrm{RelIso_{\mathrm{Comb}}} = \frac{I_{\mathrm{Trk}}+I_{\mathrm{EM}}+I_{\mathrm{had}}}{E_\mathrm{T}}.
\end{equation*} 
%%%
The electron candidate is required to have 
$\mathrm{RelIso_{\mathrm{Comb}}} < 0.05$ in order 
to be considered isolated. 
A pile-up offset subtraction in the isolation cone 
using fastjet algorithm \cite{FastJetPUSubtraction} is applied.
\item 
In order to reject events in which the electron candidate actually
originates from a conversion of a photon into an $e^{+}e^{-}$ pair, we
require the number of missed inner tracker layers of the electron
track to be exactly zero (i.e. there are no missed layers before the
first hit of the electron track from the beamline). In addition, we
reject any event in which the selected electron is flagged as a
conversion, \textit{i.e.}, an electron that has a 
distance of the partner track $|$\texttt{dist}$|$ $< 0.02$~mm and an
opening angle $|$\texttt{dcot}$|$ $< 0.02$~\cite{ConversionRejection}.
\end{itemize}
%%%%%%%%%%%%%%%%%%%
%%%%%%%%%%%%%%%%%%%%%%%%%%%%%%%%%%%%%%%%%%%%%%%%%%%%%%%%%%%%%%%%%%%%%%%%%%%%
%%%%%%%%%%%%%%%%%%%%%%%%%%%%%%%%%%%%%%%%%%%%%%%%%%%%%%%%%%%%%%%%%%%%%%%%%%%%
\subsection{Muon selection}
\label{sec:muon_cuts}
Muons are obtained from the CMS reconstruction \cite{MUONPAS}.
Muon candidates are identified by two different 
algorithms~\cite{MUONPAS}: one proceeds from the inner tracker outwards, 
the other one starts from tracks measured in the muon chambers and matches 
and combines them with tracks reconstructed in the inner tracker. 
Muons from decays in flight of hadrons and punch-through particles are 
reduced by applying a cut on $\chi^2/dof$
of a global fit including tracker and muon detector hits.
In order to ensure a precise estimate of the momentum and impact parameter
only tracks with more than 10 hits in the inner tracker and at least 
one hit in the pixel detector are used.
We require hits in at least two muon detection layers in the measurement,
to ensure a good quality momentum estimate at the trigger level, and
to suppress further any remaining fake muon candidates.
Cosmic muons are rejected by imposing a maximum allowed transverse impact parameter 
distance to the beam spot position.
These criteria are summarized below:
%%%%%%%%%%%%%%%%%%%
\begin{itemize}
\item The muon candidate is reconstructed both as a global muon and
as a tracker muon.
\item The number of hits of the muon track in the silicon tracker has
to be $N_{\mathrm{Hits}} > 10$.
\item Number of pixel hits of the Tracker track $\ge 1$;
\item Number of muon hits of the Global track $\ge 2$;
\item Normalized $\chi^{2}$ of the Global track $< 10.0$.
\item Muon $p_{\mathrm{T}} > 25\,\mathrm{GeV}$.
\item Pseudorapidity $|\eta| < 2.1$.
\item Impact parameter: We cut on the absolute value of the impact
parameter calculated with respect to the beamspot. We require:
\begin{equation*}
 d_0(\mathrm{Bsp}) < 0.02\,\mathrm{cm}.
\end{equation*}
\item In order to make sure that the selected muon and the selected
jets come from the same hard interaction and not from pile up events,
we require that the $z$ coordinate of the PV of the event and the $z$
coordinate of the muon's inner track vertex lie within a distance of
less than 1~cm.
\item The selected muon candidates also have to be isolated.
We require the muon to be isolated simultaneusly in the
tracker, and in the electromagnetic and hadronic calorimeters.  
This ``combined relative isolation'' is defined as
\begin{equation*}
\mathrm{RelIso_{\mathrm{Comb}}} = \frac{I_{\mathrm{Trk}}+I_{\mathrm{EM}}+I_{\mathrm{had}}}{p_\mathrm{T}}.
\end{equation*} 
The muon candidate is required to have
$\mathrm{RelIso_{\mathrm{Comb}}} < 0.1$ in order to be considered
isolated.
\end{itemize}

Tighter isolation is not necessary in the muon mode, as the QCD background is
observed to be smaller than that found in $W(e\nu)$ mode.

%%%%%%%%%%%%%%%%%%%



\subsection{Missing transverse energy}

The use of missing transverse energy is central to the analyses presented in this
note.  It is critical in the reconstruction of the \WtoLN\ decays, and
is used in the $Z$+jet channels to increase the purity of the selection.
 For the offline analysis, missing transverse 
energy is computed from the list of particle-flow objects with the method described 
in~\cite{CMS-PAS-JME-10-003}.  The vector \VEtmiss\ is calculated as the negative of 
the vectorial sum of transverse momenta of all particle-flow objects identified in the 
event, and the magnitude of this vector is referred to as pfMET.  

