
\subsection{Dijet trigger selection}
\label{sec:dataSampleAndEventSelection}

The dijet data sample was collected in 2011 at $\sqrt{s}=7$ TeV and corresponds to
an integrated luminosity of \intlumi. 
Events are collected with single-jet high level triggers (HLT). The online jet reconstruction 
used only calorimetric information and the resulting HLT 
jets had typically worse energy resolution than the offline ones.  
 %Events are triggered by a single
%\ifpas
%calorimeter jet.
%\else
%{\tt CaloJet}.
%\fi

As the instantaneous luminosity increased with time,
different thresholds are used in different time periods.
%We use a trigger strategy such that the triggers are
%assigned based on the $\pt^{AVG}$ evaluated using anti-$k_t$ jets with $R$=0.7. 

\label{sec:trigAssignment}
The triggers that were used in the dijet analysis have partially overlapping phase space. 
The lower-threshold triggers have high prescales in order to accommodate higher
trigger rates, and the $\pt$ selections on these triggers are only one-sided,
hence lower-threshold triggers have overlapping phase space with higher-threshold
triggers. To uniquely assign events to a specific trigger, the
strategy used is to compute the trigger efficiency as a function
of $\pt^{AVG}$ and select a region in the trigger plateau, then
assign one trigger per $\pt^{AVG}$ region. The trigger assignment is based on the ungroomed jet $\pt$. Table~\ref{TriggerTurnOns}
shows the trigger threshold for each trigger.

\begin{table}[h]
  \centering
  \begin{tabular}{ |c|c|}
    \hline 
trigger $\pt$ threshold (\GeV) & $\pt^{AVG}$ region (\GeV) \\ 
\hline
%60 & 0-150   \\
%100& 150-220 \\
190& 220-300  \\
240& 300-450  \\
370& $>$450 \\
   \hline 
  \end{tabular}
  \caption{Trigger efficiency turn-on values for all jet samples.\label{TriggerTurnOns}}
\end{table}

\subsection{V+jets trigger selection}


Several triggers are used to collect events consistent with
the topology of a vector boson decaying leptonically, to electrons or muons. For the W + jet channels the trigger
paths consist of several single-lepton triggers with lepton
identification criteria applied. Leptons are also required to be isolated from other
tracks and calorimeter energy depositions to maintain an acceptable
trigger rate. For the W$(\mu\nu)$ channel, the trigger thresholds for the
muon $\pt$ are in the range of 17 to 40 GeV.
The higher thresholds are used for the periods of higher instantaneous
 luminosity. The combined trigger efficiency is
 around 90\% for signal events that pass all offline requirements,
 described in the following.
 For the W$(e\nu)$ channel the electron $\pt$ threshold ranges
 from 25 to 65 GeV. Lower $\pt$ threshold electron triggers in combination with online dijet requirements are available but are not used in this analysis to avoid the modeling of the trigger response for groomed jets.

To preferentially select only
 real W events, the single electron triggers typically also require
 minimum thresholds on the missing transverse energy (MET) and the transverse mass
 $m_T$ of the electron plus MET system.

 The combined efficiency for these triggers for signal events
 that pass the final offline selection criteria is $>95\%$.


 The Z$(\mu\mu)$ channel uses the same single-muon triggers as the
 W$(\mu\nu)$ channel. For the Z$(ee)$ channel, di-electron triggers
 with lower $\pt$ thresholds (17 and 8 GeV) and isolation
 requirements are used. These triggers are 99\% efficient for all
 Z$(ee)$ plus jet events that pass the final offline selection criteria.

\subsection{Transverse momentum bin assignment}
\label{sec:ptBinAssignment}


%The jet mass is correlated with the energy of the jet. As such,
%it is necessary to measure the jet mass in bins of transverse 
%momentum of the jet $\pt$. 
%The average jet $\pt$ is used as a variable of interest ($\pt^{AVG}$)
%for the dijet analysis, while the leading jet $\pt$ is used as a
%variable of interest for the V+jets analysis.

We measure the jet
mass distribution for different jet $\pt$ regions. For the dijet analysis, 
$\pt^{AVG}$ is used, while the leading jet $\pt$ is used for
the V+jets analysis.

The $\pt$ binning is 
identical between the two analyses, and shown in Table~\ref{tab:ptBins}.
In this case, to assure that the appropriate $\pt$ dependence
of the jet mass is captured for the various grooming algorithm,
the $\pt$ is evaluated for each of the grooming techniques. 


\begin{table}[h]
  \centering
  \begin{tabular}{ |c|c|}
    \hline 
    Bin & $\pt$ (\GeV) \\ 
    \hline
%    1 & 50-125 \GeV \\
    1 & 125-150 \\
    2 & 150-220  \\
    3 & 220-300  \\
    4 & 300-450  \\
    5 & 450-500  \\
    6 & 500-600  \\
    7 & 600-800  \\
    8 & 800-1000  \\
    9& 1000-1500  \\
%    11& $>$1500  \\
   \hline 
  \end{tabular}
  \caption{Jet $\pt$ bins. \label{tab:ptBins}}
\end{table}




