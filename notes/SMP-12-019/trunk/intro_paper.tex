 The variables most often used in analyses involving jets are the jet direction and the momentum transverse to the beam ($\pt$). 
However, being composite objects, their masses contain additional information.
One strong motivation for studies of the jet mass is that at the Large Hadron Collider (LHC), particles such as W and Z bosons and top quarks are often produced with significant Lorentz boosts, and the mass of such jets can be used to discriminate them from lighter objects. The same may also be true for new particles produced at the LHC. When such particles decay hadronically, the products tend to be collimated in a small area of the detector. For sufficiently large boosts, the resulting hadrons can be clustered into a single jet. Jet grooming techniques are able to separate these single jets of interest from the overall jet background. Such techniques have been found promising for boosted W decay identification, Higgs searches and boosted top identification amongst others \cite{jetsub}. 
Among the main advantages promised by the use of these algorithm is the improved signal to background separation of the jet mass and the intrinsic robustness to pile-up effects, that are going to be a greater challenge in the future high luminosity LHC runs. 

However, many of these promising approaches have never been tested with collision data either at the Tevatron or at the LHC, 
and rely on the assumption that the jet mass is well modelled by Monte Carlo approaches at Leading Order (LO) or Next-to-Leading Order (NLO) matched to Parton Showers to describe the higher order radiation. 
Furthermore, many recent theoretical works in QCD have focused on computations of the jet mass, including
predictions using advances in an effective field theory of jets 
(soft colinear effective theory, or SCET)
\cite{Kelley:2011aa,KhelifaKerfa:2011zu,Hornig:2011tg,Li:2011hy,Bauer:2011uc,Kelley:2011tj,Chien:2010kc,Schwartz:2007ib,Fleming:2007qr,Dasgupta:2001sh,Bauer:2006mk,Bauer:2006qp,Hornig:2011iu,Kelley:2011ng,Jouttenus:2009ns,Ellis:2009wj,Ellis:2010rwa,Cheung:2009sg,Kelley:2010qs,Banfi:2010pa,Bauer:2001yt}.



Such studies will allow: $i)$ to understand to what degree the current parton shower Monte Carlo simulations correctly simulate the modeling of internal jet structure, $ii)$ provide first benchmark studies for the use of these algorithms in boosted Higgs or new physics searches by investigating some of the most relevant background for such analyses, and $iii)$ to provide observational data to be compared to recent computations of the jet mass. 

%It is therefore important to measure some of the relevant variables in a sample of jets to verify the expected features. First results on 
QCD jet mass studies have been presented by ATLAS \cite{atlasJS} and CDF\cite{cdfJS}. We present here a similar measurement of the jet mass in samples of dijet and boosted V+jet events, where V=W, Z, using a data sample corresponding to an integrated luminosity of 5 fb$^{-1}$, collected in 2011 by the Compact Muon Solenoid (CMS) experiment at a center-of-mass energy of 7 TeV. The jet mass after several jet grooming techniques has also been measured. The grooming techniques that we have investigated are the filtering~\cite{boostedHiggs}, trimming~\cite{trimming}, and pruning~\cite{pruning,pruning2} techniques. These are discussed in detail below. 
 
The observable we measure is a double differential cross section with respect to the jet mass and transverse momentum,
corrected for detector inefficiency and resolution effects:

\begin{equation}
\label{eq:dsigmadmjetvjets}
\frac{d^2\sigma}{dm_J d\pt}
\end{equation}

For the dijet analysis, $\pt$ and $m_J$ are replaced by 
the average transverse momentum and jet mass of
the highest two $\pt$ jets: $\pt^{AVG} = ({\pt}_1 + {\pt}_2) / 2$ and
$m_J^{AVG} = ({m_J}_1 + {m_J}_2) / 2$.

%\begin{equation}
%\label{eq:dsigmadmjet}
%\frac{d^2\sigma}{dm_J^{AVG} d\pt^{AVG}}
%\end{equation}


The measurements are presented by comparing the differential cross section of the
jet mass in a given transverse momentum range, normalized to the number of events in that
range.  This is formally equivalent to the following expression

\begin{equation}
\label{eq:pdf_mjet}
\frac{1}{\frac{d\sigma}{d\pt}} \times \frac{d^2\sigma}{d\pt dm_J}. 
\end{equation}

that can be also expressed as

\begin{equation}
\label{eq:pdf_mjet_simple}
 \frac{1}{\hat{\sigma}} \times \frac{d\hat{\sigma}}{dm_J}. 
\end{equation}

\noindent where $\hat{\sigma} = d\sigma/d\pt$.


The results presented in this paper are measurements
of the distribution of Eq.~\ref{eq:pdf_mjet_simple} for 
jets with and without various grooming algorithms applied, 
corrected for detector inefficiency and resolution to the
particle level, and are compared to several theoretical
predictions. 

The paper is organized as follows: after a brief description of the CMS detector and the data and Monte Carlo (MC) simulated samples used (Sec.~\ref{sec:cms_detector}), we give details about the data collection (Sec.~\ref{sec:dataSampleAndEventSelection}) and reconstruction methods used (Sec.~\ref{sec:reconstruction}), focusing in particular on the jet clustering algorithms studied (Sec.~\ref{sec:algos}). The event selection is then described (Sec.~\ref{sec:evsel_paper}), and the effect of pileup on the jet mass in this event selection is investigated (Sec.~\ref{sec:pileup}). We then discuss the unfolding procedure being applied to the jet mass spectra (Sec.~\ref{sec:unfolding_paper}), and the uncertainties considered (Sec.~\ref{sec:systematics}). In Sections~\ref{sec:dijetresults} and ~\ref{sec:vjetresults} we present the results for the dijet and V+jet analyses, respectively. Some final observations and remarks on the results presented are included in Sec.~\ref{sec:summary}.   
 
