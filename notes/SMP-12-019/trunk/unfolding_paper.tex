
\label{sec:unfolding_paper}
The measured spectrum of a physical observable, like the jet mass distribution, is usually distorted by	detector effects, such as finite resolution and limited acceptance. Moreover, in this analysis the chosen bin size is comparable to the resolution, so there is a significant migration of events generated in one jet bin mass and ending up in a different bin of reconstructed jet mass. A comparison of the measured mass spectrum with that predicted at generator level requires that we remove these effects to obtain the true underlying mass spectrum. There are several possible ways to achieve the unfolding of detector effects on measured spectra. We use an unfolding procedure described by G.~D.~Agostini in~\cite{agostini}. Repeated application of Bayes theorem is used 
to invert the response matrix. Regularization is achieved by stopping iterations before reaching the ``true'' (but wildly fluctuating) inverse. The regularization parameter is just the number of iterations.
 In principle, this has to be tuned to prevent the statistical fluctuations being interpreted as structure in the true distribution, according to the sample statistics and binning. In practice, the results are fairly insensitive to the precise setting used and four iterations are usually sufficient. A trivial bin-by-bin unfolding technique is used as a cross check, and consistent results are observed where expected. 

