\subsection{Event reconstruction}
\label{evrecosection}

\ifnpas
All data are reconstructed using CMSSW 4.2.x.
\fi
\label{sec:preselection}
\label{sec:reconstruction}
Event data are reconstructed using the particle-flow reconstruction algorithm~\cite{particleflow},
which attempts to reconstruct all stable particles in an event by combining information from
all subdetectors. The algorithm categorizes all particles into five types: muons,
electrons, photons, charged and neutral hadrons. The resulting particle flow candidates are passed
to each jet clustering algorithm described in Sec.~\ref{sec:algos} to create "particle flow jets", 
as implemented in FastJet version 3.0.1 \cite{fastjet1,fastjet2}.
A detailed  description of the jet algorithms used in this analysis is reported below. 

%An extra correction is applied to the data to account for a residual
%nonlinearity that is not observed in the simulation. No pileup
%corrections are applied. 
Charged hadrons identified as pileup are removed from the inputs to the jet clustering algorithms.
The charged hadrons are classified as belonging to a 
pileup vertex when they are used to reconstruct a vertex that is not
the highest $\pt$ primary vertex. The primary vertices are
reconstructed with a deterministic annealing filter (DAF) techique from the tracks in
the event. There are also quality criteria
placed on the primary vertex, in that it must contain at least four
degrees of freedom in the spatial fit (roughly corresponding to at least four
tracks), and satisfy $\chi^2 / ndof < 8$. It must also be within the
physical region of the pixel detector. 



Electron reconstruction requires the matching of an energy cluster in the 
ECAL with a track in the silicon tracker~\cite{CMS-PAS-EGM-10-004}.  
Identification criteria based on the ECAL shower shape, track-ECAL cluster 
matching, and consistency with the primary vertex are imposed. Additional 
requirements are imposed to remove electrons produced by photon conversions. 
In this analysis, electrons are considered in the pseudorapidity range 
$|\eta|<2.5$, excluding the $1.44<|\eta|<1.57$ transition region between the 
ECAL barrel and endcap.
Muons are reconstructed using two algorithms~\cite{CMS-PAS-MUO-10-004}: 
one in which tracks in the silicon tracker are matched to signals in 
the muon chambers, and another in which a global track fit is performed 
seeded by signals in the muon system. The muon candidates used in the 
analysis are required to be reconstructed successfully by both algorithms. 
Further identification criteria are imposed on the muon candidates to reduce 
the fraction of tracks misidentified as muons. These include the number of 
measurements in the tracker and the muon system, the fit quality of the 
 muon track, and its consistency with the primary vertex.

Charged leptons from W and Z boson decays are expected to be isolated from 
other activity in the event. For each lepton candidate, a cone 
is constructed around the track direction at the
 event vertex. The scalar sum of the transverse energy of each 
 reconstructed particle compatible with the primary vertex and contained 
 within the cone is calculated excluding the contribution from the 
lepton candidate itself. If this sum exceeds approximately 10\% of the 
candidate $p_T$ the lepton is rejected; the exact requirement depends 
on the lepton $\eta$, $p_T$ and flavor.
Muons (electrons) are required to have a $p_T$, greater than 30 GeV (80 GeV). 
The high threshold for the electron selection ensures a high trigger efficiency.

%The very high offline threshold on the eletron momentum is motivated by the criteria to avoid turn-off effects in the trigger efficiency for the single electron trigger. 

Furthermore, charged leptons with an isolation
of 15\% (muons) or 20\% (electrons) of the lepton transverse momentum 
are removed from consideration for the jet clustering algorithm, 
where the isolation is defined as the
 energy from charged particles, neutral particles, 
and photons (counted separately). 


After the removal of charged hadrons from pileup vertices and isolated
leptons, only the
neutral component of pileup remains
that is not interesting for the jet algorithm. 
The remaining particle flow candidates are used as input to the jet
clustering algorithm. Energy corrections are applied to the resulting
jets as follows. 

% After these subtractions, aside from the true constituents of the jets,
% this leaves only the neutral component of pileup,
% This is removed by applying a residual

The remaining neutral hadron component of pileup is removed by
applying a residual area-based correction as described in
Ref.~\cite{jetarea_fastjet,jetarea_fastjet_pu}. 
The mean $\pt$ per unit area
is computed with the $k_{\mathrm T}$ algorithm with the ``active area'' method,
with a distance parameter of 0.6, and the
jet energy is corrected by the amount of pileup expected in the jet
area. The ``active area'' method adds a large number of infinitely
soft ``ghost'' particles to the clustering sequence to examine which
jet they are clustered into, and the area is computed by the set of
points for each jet. This is done for all of the grooming algorithms
in the same way as the standard jets. The active area of the
groomed jets is smaller than the active area of the ungroomed jets,
and hence the pileup correction is smaller. 
Due to the different responses in the endcap
and barrel calorimeters, the jet corrections are $\eta$-dependent. The amount of energy expected from underlying
event is added back into the jet. 


The pileup-subtracted jet four momenta are finally 
corrected for nonlinearities in $\eta$ and $\pt$
with simulated data, with a residual $\eta$-dependent correction added 
to correct for the difference in simulated and true responses~\cite{citeJEC}.  
The jet corrections used are specifically derived for the ungroomed
jet algorithms, and also applied to the groomed jet algorithms with an
additional systematic uncertainty described below.


% These corrections are then
% applied as 
% \begin{equation}
% \pt^{PU-corr} = \pt^{RAW} \times \left( 1 - F(\eta) \times \frac{A_{jet} (\rho - <\rho_{UE}>)}{<A_{jet}> <(\rho_{PU,NPV=1})>} \right)
% \end{equation}
% The function $F(\eta)$ is an $\eta$-dependent correction designed to correct for
% different responses between the barrel and endcap calorimeters.
% It is derived from an assumption of a linear dependence of jet
% momenta as a function of the number of reconstructed primary vertices, 
% and then computed over a range of jet pseudorapidities. 
% The area of the jet $A_{jet}$ is computed as described in Ref~\cite{fastjet_area},
% and depends ont he jet algorithm. The mean $\pt$ per unit area for underlying
% event ($\rho_{UE}$) is added back into the jet, since the cachement area approach
% subtracts this as well (which should, however, be added in the jet momentum measurement
% to be consistent with theoretical predictions). The average jet area $<A_{jet}>$ is
% derived for the algorithm in question in a dijet sample, and $<\rho_{PU,NPV=1}>$ is the
% expected average mean $\pt$ per unit area in events with exactly one primary vertex. 


An accurate MET measurement is essential for distinguishing the W signal from QCD backgrounds. We use the MET measured in the event using the full particle-flow reconstruction. The MET resolution, measured as a function of the sum $E_T$ ($\sum E_T$) of the particle-flow~\cite{particleflow} objects in the event, varies from 4\% at $\Sjm E_T$ =60 GeV to 10\% at $\sum E_T$ =350 GeV~\cite{metJINST}. In this analysis, we require MET $>$50 GeV. 
