\section{Introduction}
\label{sec:intro}
% ---- ---- ---- ---- ---- ---- ---- ---- ---- ---- ---- ---- ---- ---- ---- ---- ---- ---- ---- ---- ---- ---- ----

The Standard Model (SM) of particle physics successfully describes the
majority of high-energy experimental data~\cite{pdg}. One of the key
remaining questions is the origin of the masses of W and Z bosons.  In
the simplest implementation of the SM, it is attributed to the
spontaneous breaking of electroweak symmetry caused by a new scalar
field. %~\cite{Higgs1, Higgs2, Higgs3} The existence of the associated
field quantum, the Higgs boson, has yet to be experimentally
confirmed.  Therefore, the search for the Higgs boson is arguably one
of the most important studies being done at the
LHC~\cite{lhcmachine}. For Higgs masses above or near the threshold
for decay into two vector bosons, the decay modes of choice are
dominated by those decays because of their large branching fractions.
It is clear that the events where one $W$ decays leptonically, which
provides the main trigger elements, while the other decays
hadronically have the second highest branching fraction and have a
reconstructable Higgs mass peak~\cite{intro2}.

This note contains the analysis that sets a limit on the Higgs boson
cross-section based on this decay mode, performed on data acquired by
CMS at $\sqrt{s}~=$~8~TeV during the year 2012. Previous searches
performed with similar procedures are documented in \cite{HIG-12-021}
(5.1\fbinv collected in 2012) and \cite{HIG-12-003} (5.0\fbinv
collected in 2011).  The analysis selects events with one well
identified and isolated lepton, large missing transverse energy and at
least two high \pt jets.  Therefore, the main experimental issue is to
control the large $W$ plus jets background sufficiently well that the
advantages of using this final state are realized.

With respect to the earlier 2012 analysis, the available sample
continues to be acquired this year by single lepton triggers, allowing
to relax the selection on the tranverse mass of the leptonic W boson.
Physics objects are selected with the techniques available in 2012, in
particular jet identification is applied to reduce the effect of
pile-up. Much of the analysis techniques are unchanged with respect to
\cite{HIG-12-021}.

Differences from \cite{HIG-12-021} include the relaxation of electron
identification requirements to enhance the acceptance of signal and
minor adjustments to the fitting procedure associated with the
availability of simulation samples. A substantial difference is the
inclusion of a Higgs lineshape reweighting procedure that improves the
shape of the mass distribution of the Higgs boson in simulation
samples, as well as the inclusion of a reweighting procedure for
effects of interference from SM diboson production on the Higgs
production cross section and mass distribution. These procedures
follow the recommendation from the Higgs Cross Section Working Group
\cite{Dittmaier:2012vm} and allow us to reduce some of the theoretical
uncertainties that were previously necessary.

The note is structured in a manner similar to previous notes.  A
discussion about the data samples used in the analysis and the trigger
selections is presented in Sections~\ref{sec:MCexpectations} and
\ref{sec:technicalities}.  The physics objects reconstruction is
discussed in Section~\ref{sec:firstStep}.  The lepton selection and
other preselection requirements are described in detail in
Section~\ref{sec:firstStep} and \ref{sec:dataMCcomparisons}.

After the preselections, the signal-over-background ratio is enhanced
by means of a selection on a MVA discriminant, designed in order to
control the background while preserving as much as possible the
difference in shape with respect to the signal
(Section~\ref{sec:mvaoptimization}).  The input variables of the MVA
exploit the decay angles of the four-body mass system, the kinematics
of the entire four-body system. The MVA variable definition is
optimized with dedicated trainings for each Higgs mass hypothesis case
for each lepton flavour ($e$, $\mu$), and for each jet multiplicity (2
jets, 3 jets) independently.  In this way, 48 different configurations
are obtained.

The main background contaminating the signal region is W+jets.  Its
$m_{\ell{}\nu{}jj}$ shape is extrapolated from sidebands in the
$m_{jj}$ distribution through Monte Carlo based factors
(Section~\ref{sec:wjetsBackground}).  Also the QCD shapes come from a
data-driven determination, as described in
Section~\ref{sec:dataDrivenQCD}.  The remaining background shapes come
from the Monte Carlo.  The normalization of the backgrounds in the
signal region is measured by fitting their Monte Carlo shapes on the
same sidebands in the $m_{jj}$ variable
(Section~\ref{sec:mjjfitfornormal}).  The determination of the
$m_{jj}$ distribution for W+jets is described in
Section~\ref{sec:wjetsShape}.

The systematic uncertainties present in the signal description are
described in Section~\ref{sec:systematics}.
Section~\ref{sec:limitExtraction} describes the obtained limits on the
Standard Model Higgs cross section and Section~\ref{sec:conclusions}
closes this work.

