\section{Multivariate optimization}
\label{sec:mvaoptimization}
% ---- ---- ---- ---- ---- ---- ---- ---- ---- ---- ---- ---- ---- ---- ---- ---- ---- ---- ---- ---- ---- ---- ----

In addition to the common preselections listed in Sec.~\ref{sec:firstStep}, the
MVA likelihood analysis applies the following selections:

\begin{itemize}
\item $|\Delta\phi_{\textrm{leading jet,MET}}| > 0.4$ for muons
\item $|\Delta\phi_{\textrm{leading jet,MET}}| > 0.8$ for electrons
\end{itemize}

We adopt the optimization performed on last year's data and MC samples
for masses 170-600~\GeV.  This optimization is documented completely
in the analysis notes AN-2011/110 and AN-2012/008. 

%% For masses above 600~\GeV,
%% we use 8~\TeV signal samples, splitting them in half so that the half that
%% is used for training is statistically independent from the half that is
%% used for limit setting.

An additional point of distinction between the optimization procedures
for the 2011 data and this year's data is the use of the quark-gluon
likelihood discriminant. In 2011, this discriminant was used to
distinguish quark-originated jets (signal) from gluon-originated jets
(W+Jets dominant background), particularly at high mass ($M_{H}\geq
500$~GeV), after applying the MVA output requirement. However, the PDF
for the discriminant for 2012 beam conditions was not yet available as
of this writing, and the PDF based on 2011 beam conditions did not
yield reliable results, so it is not applied.
