\section{Conclusions}

We have used $358$~pb$^{-1}$ of CMS data to measure the dijet mass spectrum
with the following eta cuts on the two leading jets: $\mid \Delta\eta \mid < 1.3$ and 
$\mid \eta \mid < 2.5$. 
The shape of the measured dijet mass spectrum 
is in good agreement with a QCD prediction from PYTHIA and the full simulation of 
the CMS detector.


We have performed direct searches for high-mass dijet resonances in the
dijet mass distribution. The dijet mass data are well fit by a simple parameterization. 
There is no significant evidence for new particle production
in the data.  We set 95\% confidence level upper limits on the cross section for
a dijet resonance, applicable to any narrow resonance producing the following
specific pairs of partons:  $qq$, $qg$, and $gg$.  We have compared our cross section
limits with the expected cross sections from several such existing models. 
Including statistical uncertainties only, we exclude at the 95\% confidence level 
the following models of new new particles: string resonances in the mass
range $0.9 < M(S) < 3.8$ TeV, excited quarks in the mass range $0.9<M(q*)<2.3$ TeV,
axigluons and colorons in the mass range 
$0.9<M(A)<2.3$ TeV, and $E_6$ diquarks in the mass range
$0.9<M(D)<2.7$ TeV extending previously 
published limits from a dijet mass search on all models. 


\section{Acknowledgments}

We would like to thank Can Kilic and Scott Thomas for their work on the string resonance
cross section, both at LHC and the Tevatron, Bob Cousins for helping us define the
statistical model, and Gena Kukartsev, Kalanand Mishra and many others for their 
work on the statistics in RooStat.
