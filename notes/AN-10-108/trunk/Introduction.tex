\section{Introduction}
\label{chap:Intro}

In this note we document our measurement of the dijet mass
distribution and our search for dijet resonances in pp Collisions at
$\sqrt{s}=7$ TeV.

\subsection{History of Document}

This note documents our dijet mass resonance search with 1.01 fb$^{-1}$ using Widejets, CaloJets and 
PFjets, for the paper .  It should be read along with the Widejets
analysis note~\cite{CMS_AN_2011/242} which contains additional details of the Widejet analysis
that are not contained in this note. In the note it desciribed the jet with a term `Fatjet'.
However, we use the term 'Widejets'.
This note updates a previous note (AN-2011/203 v3) which documented the search with 358 pb$^{-1}$ from 2011 but did not use WideJets.
It has a similar format and content as the note which documents the 2010 search~\cite{CMS_AN_2010/108}.

\subsection{Motivation}
Our experimental motivation is to make a first measurement of the dijet mass distribution
and see whether it agrees with expectations or contains new physics beyond the Standard Model (SM).
The LHC is a parton-parton collider in a previously
unexplored energy region. If new parton-parton resonances exist at sufficiently low mass 
then the LHC will produce 
them copiously. These resonances necessarily also decay to partons giving two jets in the 
final state.  One theoretical motivation is that the SM has important unanswered questions:
Why do quarks come in different flavors?
Why are the quarks arranged in generations? Why are there four different forces? How do we unify gravitation with the other forces?
Why is gravity so weak? Models that try to address these questions often predict short-lived particles that
can decay to two partons: dijet resonances.

\subsection{Models}
We search for processes producing narrow resonances, $X$, decaying to dijets 
as illustrated in fig.~\ref{feyn}:
$pp \rightarrow X \rightarrow$ jet + jet (inclusive).


\begin{figure}[hbt]
  \begin{center}
      \includegraphics[width=0.5\textwidth]{Figures/resonanceDiagram.pdf}
    \caption{ Feynman Diagram of dijet resonance. The initial state
and final state both contain two partons (quarks, antiquarks or
gluons) and the intermediate state contains an $s$-channel resonance
$X$.}
    \label{feyn}
  \end{center}
\end{figure}


We perform a generic search that we can apply to any model.
Here we introduce some models, say a few words about the cross section, 
and explicitly list the partons involved in production and decay~\cite{CMS_AN_2006-070}. 
Excited states of 
composite quarks~\cite{Baur:1987ga} are strongly produced giving large cross sections 
($qg\rightarrow q^*$). Axigluons~\cite{Bagger:1987fz} 
or colorons~\cite{Chivukula:1996yr} 
from an additional color interaction are also strongly produced, but require an 
antiquark in the initial state ($q\bar{q} \rightarrow A$ or $C$) 
slightly reducing the cross section compared to excited quarks.  
Diquarks~\cite{Hewett:1988xc} from 
superstring-inspired $E_6$ grand unified models are produced with
electromagnetic coupling from the valence quarks of the proton 
($ud \rightarrow D$). 
The cross section for $E_6$ diquarks at high mass is the largest 
of all the models considered, because at high parton momentum the 
probability of finding a quark 
in the proton is significantly larger than the probability of finding a 
gluon or antiquark. 
Randall Sundrum gravitons~\cite{RS}, with coupling $k/M_{PL}=0.1$, from a model of 
large extra dimensions are produced 
from gluons or quark-antiquark pairs in the initial state ($q\bar{q},gg \rightarrow G$).
Heavy $W$ bosons~\cite{Eichten:1984eu} inspired by left-right
symmetric grand unified models have electroweak couplings 
and require antiquarks for their
production($q_1 \bar{q_2} \rightarrow W^{\prime}$), giving small cross sections.  
Heavy $Z$ bosons~\cite{Eichten:1984eu} 
inspired by grand-unified models are widely anticipated by theorists, 
but they are electroweakly produced, 
and require an antiquark in the initial state($q\bar{q} \rightarrow Z^{\prime}$), 
so their production cross section is around the lowest of the models considered.
The model with the largest cross section is a recent model of string resonances, 
Regge excitations of the quarks and gluons in open string theory, which
includes resonances in three parton-parton channels (predominantly $qg$ at LHC but also some $q\bar{q}$ and $gg$)
with multiple spin states and quantum numbers~\cite{Anchordoqui:2008di,Cullen:2000ef}.
Table~\ref{table:models} summarizes some properties of these models, and the string resonance model
is discused in detail in Appendix~\ref{appString}.


\begin{table}[th]
\centering
\normalsize
       \begin{tabular}{|c|c|c|c|c|c|}
        \hline
        Model Name 	& X & Color 	& $J^{P}$ 	& $\Gamma /(2M)$ & Chan		\\
        \hline
	Excited Quark 	& q*& Triplet 	& $1/2^+$ 	& $0.02$ 	& qg		\\
        E$_{6}$ Diquark & D & Triplet 	& $0^+$ 	& $0.004$ 	& qq		\\
        Axigluon 	& A & Octet 	& $1^+$ 	& $0.05$ 	& $q\bar{q}$	\\
        Coloron 	& C & Octet 	& $1^-$ 	& $0.05$ 	& $q\bar{q}$	\\
        RS Graviton 	& G & Singlet 	& $2^+$ 	& $0.01$ 	& $q\bar{q}$ , gg	\\
        Heavy W 	& W'& Singlet 	& $1^-$ 	& $0.01$ 	& $q\bar{q}$	\\
        Heavy Z 	& Z'& Singlet 	& $1^-$ 	& $0.01$ 	& $q\bar{q}$	\\
	String          & S & mixed     & mixed         & $0.003-0.037$  & $qg$, $q\bar{q}$, $gg$ \\
        \hline
        \end{tabular}
 	\caption{Properties of Some Resonance Models}
	\label{table:models}
	\end{table}



\subsection{Summary of Experimental Technique}
Our experimental technique starts with a measurement of the 
inclusive process $pp \rightarrow$ jet + jet + anything.  Inclusive means we
measure processes containing at least two jets in the final state, but the 
events are allowed to contain additional jets, or anything else.
The dijet in the event is simply the two highest pt jets, the leading jets.
Within the standard model our dataset is expected to be overwhelmingly
dominated by the $2\rightarrow 2$ process of hard parton scatters, with 
additional radiation off the initial
and final state partons naturally giving additional jets. We do not cut
away events that contain this radiation, which would reduce signals that 
also have similar amounts of radiation, 
and unnecessarily restrict signals to a narrow topology.  The 
events can also contain additional particles, such as leptons or photons, 
but this will occur very rarely in the standard model.  Finally, even more
rarely within the standard model, the two leading CaloJets in the event can
result from electrons, photons or taus producing energy in the calorimeter,
and we do not exclude these insignificant contributions to our sample either.  
Our dijet selection is then open to many signals of new physics including 
high pt jets, leptons and photons. However, our selection is optimized for 
signals in the $2\rightarrow 2$ parton scattering process, and is
overwhelmingly dominated by the signal background of dijets from QCD within 
the standard model.


Our experimental method to search for dijet resonances utilizes the
dijet mass spectrum measured from the two leading jets in the data.
If a dijet resonance exists, it should appear in the dijet mass
spectrum as a bump. First we compare the dijet mass spectrum to QCD
predictions from PYTHIA to see if they agree, but we do not use QCD to
model our background in the search.  We use a smooth parameterization
to model our background. We fit the dijet mass spectrum with a smooth
parameterization and see whether we can get a good fit.  The fit
probability tells us whether the data is smooth, which is the first
test for the presence of a resonance.  We look at the difference
between the data and the fit, and estimate the significance of the
largest upward fluctuation in the data interpreted as a narrow
resonance.  If there is no significant evidence for dijet resonances,
we proceed to set limits. The dijet resonance shape for generic
di-parton resonances containing qq, qg and gg partons were simulated
using PYTHIA, and used as resonance signals. To calculate the upper
cross section limit for this dijet resonance shape in our data, we use
a binned maximum likelihood method. The method gives a Poisson
likelihood as a function of the cross section.  We convolute the
statistical likelihood distribution with our Gaussian systematic
uncertainty and find the 95\% confidence level upper limit on the
cross section. This gives cross section limits for generic narrow qq,
qg and gg resonances, independent of any specific resonance model.
The upper limit on the cross section is then compared with the
predicted cross section for a few benchmark models to obtain mass
limits on particular models. This experimental method is basically the
same as that employed by the dijet resonance searches at the
Tevatron~\cite{Aaltonen:2008dn,Abazov:2003tj,Abe:1997hm} and at
ATLAS~\cite{ATLAS_Search}.
