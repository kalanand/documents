\section{Conclusions}

We have used 2.875 pb$^{-1}$ of CMS data to measure the dijet mass spectrum
with the following eta cuts on the two leading jets: $\mid \Delta\eta \mid < 1.3$ and 
$\mid \eta \mid < 2.5$. The event with the largest observed dijet mass is at
 $m=2.05$ TeV. The measured dijet mass spectrum 
is in good agreement with a QCD prediction from PYTHIA and the full simulation of 
the CMS detector.


We have performed direct searches for high-mass dijet resonances in the
dijet mass distribution. The dijet mass data is well fit by a simple parameterization. There is no significant evidence for new particle production
in the data.  We set 95\% confidence level upper limits on the cross section for
a dijet resonance, applicable to any narrow resonance producing the following
specific pairs of partons:  $qq$, $qg$, and $gg$.  We have compared our cross section
limits with the expected cross sections from several existing models. 
We exclude at the 95\% confidence level new particles predicted 
in the following specific models: string resonances with mass less than 2.50~TeV, excited quarks with mass less than 1.58~TeV, 
and axigluons, colorons and $E_6$ diquarks in specific mass intervals, extending previously 
published limits on all models.


\section{Acknowledgments}

We would like to thank Can Kilic for his work on the string resonance
cross section, both at LHC and the Tevatron, exotica conveners Greg Landsberg and Chris Hill for their
careful reading of the note and suggestions for improvement, and the analysis review committee 
Bob Cousins, Valerie Halyo, and Jesus Marco for their effort and valuable suggestions.
