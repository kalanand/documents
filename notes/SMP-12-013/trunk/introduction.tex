Studies of $\WW$ production test the description of electroweak and
strong interactions in the standard model (SM).
Next-to-leading order (NLO) calculations~\cite{MCFM} of $\WW$ production in pp
collisions at $\sqrt{s} = 8~\TeV$ predict a cross section of
%
\begin{equation}
\sigma^{{\rm NLO}} (\mathrm{gg}\to\WW + \mathrm{qq}\to\WW) = \nloCrossSection.\nonumber
\end{equation}

The dominant $\WW$ production mechanisms are the s-channel and
t-channel $\rm{q}\bar{\rm{q}}$ annihilation diagrams. The gluon-gluon fusion
contributes $\sim$3\%. The $\W\W\gamma$ and $\W\W\Z$ triple-gauge boson couplings (TGCs) in
the s-channel are sensitive to possible new physics processes at a higher mass
scale. The presence of anomalous TGCs would change the $\WW$ production rate or
kinematic distributions from the SM predictions. In addition, $\WW$ production
is an important background source for new particle searches, e.g. Higgs
boson searches~\cite{HiggsPAS2012}. In fact, the selection applied in this paper
closely follows the selection applied in the Higgs boson search in the event
category with zero counted jets.
The Higgs boson preselection for masses above 140~\GeV and the selection
applied in this paper differ only in the trailing lepton minimum transverse momentum.
This requirement is tightened from 10~\GeV to 20~\GeV
to further suppress the background from $\Wjets$ events, where the trailing 
lepton is a misidentified jet. With this selection a
SM Higgs boson with a mass of 110 (130)~\GeV would enhance the $\WW$ signal by
about $0.3\%$ $(7\%)$.

This paper reports a measurement of the $\WW$ production cross section in pp
collisions at $\sqrt{s} = 8~\TeV$, performed on a dataset corresponding to an
integrated luminosity of $\usedLumi$, collected during 2012.
