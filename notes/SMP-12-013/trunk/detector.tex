While the CMS detector is described in detail elsewhere~\cite{CMSdetector}, the
key components for this analysis are summarized here.
A superconducting solenoid occupies the
central region of the CMS detector, providing an axial magnetic
field of 3.8~Tesla parallel to the beam direction.
The silicon pixel and strip
tracker, the crystal electromagnetic calorimeter and the brass/scintillator hadron
calorimeter are located within the solenoid. A quartz-fiber
Cherenkov calorimeter extends the coverage to $|\eta| <$ 5.0, where pseudorapidity
is defined as $\eta=-{\rm ln}[\tan{(\theta/2)}]$,
and $\theta$ is the polar angle of the trajectory of the particle
with respect to the beam direction. Muons are measured
in gas detectors embedded in the iron return yoke outside the
solenoid.
The first level of the CMS trigger system, composed of custom
hardware processors, is designed to select the most interesting events
in less than 3 $\mu$s using information from the calorimeters and muon
detectors. The High Level Trigger processor farm further
decreases the event rate to a few hundred Hz, before data storage.

This measurement is based on $\WW$ candidate events in which
both bosons decay leptonically. The experimental signature consists of
two isolated, high transverse momentum ($\pt$), 
oppositely-charged leptons (electrons or muons) and large
missing transverse energy ($\met$, defined as the modulus of the negative vector sum of the 
transverse momenta of all reconstructed particles, charged or neutral, in the event) due to the undetected neutrinos.
Several SM processes lead to backgrounds to the $\WW$ sample.
These include $\Wjets$, $\wgamma$, and ${\rm QCD}$ multi-jet events where at
least one of the jets is misidentified as a lepton, top production
($\ttbar$ and $\tw$), Drell-Yan $\dyll$, and diboson
production ($\WZ$, and $\ZZ$).

A number of Monte Carlo event generators are used to simulate the signal and
backgrounds. The $\rm{q}\bar{\rm{q}}\to\WW$ signal, $\Wjets$, $\WZ$,
$\W\gamma^{(*)}$, Drell-Yan and $\ttbar$ processes are generated using the
\textsc{madgraph}~\cite{madgraph} event generator. The $\mathrm{gg} \to \WW$
signal component, which is expected to contribute 3\% of the total $\WW$
production rate~\cite{MCFM}, is generated using \textsc{gg2ww}~\cite{ggww}.
The \textsc{powheg} program~\cite{powheg} provided event samples for the $\tw$
process, and the remaining processes are generated using \textsc{pythia}~\cite{pythia}.
For leading-order generators, the default set of parton distribution functions
(PDF) used to produce these samples is \textsc{cteq6l}~\cite{cteq66}, while 
\textsc{ct10}~\cite{ct10} is used for next-to-leading order generators.
NLO calculations are used for background cross sections.
For all processes, the detector response is simulated using a detailed
description of the CMS detector, based on the \textsc{geant4}
package~\cite{Agostinelli:2002hh}.

The simulated samples are reweighted according to the distribution of number of pp interactions 
per bunch crossing (pile-up) as measured in the data.
