Three final states are considered:
$\Elp\Elm$, $\Mp\Mm$ and $\Elpm\Mmp$.
W$\to \ell \nu$ ($\ell = \mathrm{e},\mu$) decays are the main signal
components, while $\W\to\tau\nu_{\tau}$ events with leptonic $\tau$
decays are included, although the analysis is not optimised for this final state.
The events are selected by trigger conditions requiring
the presence of one or two high $\pt$ electrons or muons.

Two oppositely-charged lepton candidates are required, both with $\pt > 20~\GeV$.
Muon candidates~\cite{muonpas} are identified using a selection close to
the one described in Ref.~\cite{Chatrchyan:2011tz}, while
electron candidates are selected using a multivariate
approach, that exploits correlations among the selection variables
described in~\cite{egmpas} to improve identification performance. 
The lepton candidates are required to be compatible with the primary vertex of the
event, which is chosen as the vertex with highest $\sum \pt^2$ of its associated tracks. 
This criterion provides the correct assignment for the
primary vertex in more than 99\% of both signal and
background events for the pile-up distribution observed in the data. 

Charged leptons from $\W$ boson decays are usually isolated
from other activity in the event. For each lepton candidate, a cone
is constructed around the track direction at the event vertex.  The scalar
sum of the transverse energy of each reconstructed
particle~\cite{PFT-09-001} compatible with the
chosen primary vertex and contained within the cone is calculated
excluding the contribution from the lepton candidate itself. Electron candidates
are rejected if this sum exceeds 15\% of the candidate $\pt$.
Cones of several widths are constructed around the track direction at the event vertex 
to account for the differences in the differential energy deposit between a muon and a jet 
faking a lepton. This information is combined using a multivariate approach.
For both electrons and muons a correction is applied to account for the contribution to 
the energy in the isolation cone due to the several pp interactions per bunch crossing.
A median energy density ($\rho$) is determined event by event and the pile-up contribution to 
the transverse energy is estimated as the product of $\rho$ and an effective isolation cone area. 
This contribution is  subtracted~\cite{Cacciari:subtraction} from the transverse energy in the 
isolation cone.

Jets are reconstructed from calorimeter and tracker information using
the particle-flow technique~\cite{PFT-09-001,jetpas}. The anti-$\mathrm{k_T}$
clustering algorithm~\cite{antikt} with distance parameter $\mathrm{
R}=0.5$ is used, as implemented in the \textsc{fastjet}
package~\cite{Cacciari:fastjet1,Cacciari:fastjet2}. The jet energy is corrected
for pile-up in a manner similar to the correction of the energy inside a lepton
isolation cone. Jet energy corrections are also applied as a function of the jet
$\Et$ and $\eta$~\cite{cmsJEC}. 

The properties of the hard jets are modified by particles from pile-up interactions;
a combinatoric background from low-\pt jets from pile-up interactions that get 
clustered into high-\pt jets may also arise.
A multivariate selection is applied to separate jets coming from the 
primary interaction from those reconstructed from energy deposits associated with 
pile-up. The discrimination is based on the differences in the jet shapes,
in the relative multiplicity of charged and neutral components and in the different 
fraction of transverse momentum carried by the hardest components.
Tracks associated with a jet are required to be compatible with the primary vertex.

To reduce the background from top quark decays, events with one or more jet
surviving the jet selection criteria and with corrected $\Et > 30~\GeV$ and
$|\eta|< 4.7$ are rejected. To further suppress the top quark background, a
top tagging technique based on soft-muon and b-jet
tagging~\cite{btag1,btag2} is applied. The first method vetoes events containing
muons from b-quarks. The second method uses b-jet tagging applied to jets with
$15 < \Et < 30~\GeV$: this technique is based on tracks with large impact parameter within
jets. The combined rejection efficiency for top events is about 50\%.

A {\it projected $\met$} is defined
as the component of $\met$ transverse to the closest lepton if it is closer
than $\pi/2$ in azimuthal angle, and the full $\met$ otherwise.
This observable more efficiently rejects $\dytt$ background events, where the $\met$
is preferably aligned with the leptons, as well as $\dyll$ events with mismeasured
$\met$ associated with poorly reconstructed leptons or jets.
Since the {\it projected $\met$} resolution is deteriorated by pile-up,
the minimum of two different observables is used: the first includes all particle candidates in the
event~\cite{PFT-09-001}, while the second uses only the charged
particle candidates associated to the primary vertex. This exploits
the correlation between the two observables in events with significant real
{\it $\met$}, as in the signal, and the lack of correlation otherwise, as
in Drell-Yan events.  
In order to reduce the Drell-Yan contamination, the {\it projected $\met$} is
required to be above 45~$\GeV$ in the $\Elp\Elm$ and $\Mp\Mm$ final states.
For the $\Elpm\Mmp$ final state, which has lower
contamination from $\dyll$ decays, the threshold is reduced to 20 $\GeV$.
These requirements remove more than 99\% of the Drell-Yan background.
To reduce the contamination from $\dyll$ events in which the $\Z$ boson recoils
against a jet, the angle in the transverse plane between the dilepton system and
the most energetic jet with $\Et > 15~\GeV$ is required to be smaller than 165 degrees.
This selection is applied only in the $\Elp\Elm$ and $\Mp\Mm$
final states when
the leading jet has $E_T>$~15 $\GeV$.
To further reduce the Drell-Yan background in the
$\Elp\Elm$ and $\Mp\Mm$ final states,
events with a dilepton mass within $\pm 15$~\GeV of
the \Z mass are rejected.
Events with dilepton masses below 12~$\GeV$ are also rejected
to suppress contributions from low-mass resonances. The same
requirement is also
applied in the $\Elpm\Mmp$ final state. 
Finally, the transverse momentum of the dilepton system, $\pt^{\ell\ell}$,
is required to be above 45~$\GeV$ to reduce both Drell-Yan and 
events containing jets misidentfied as leptons.

To reduce the background from other diboson processes, such as $\WZ$ and $\ZZ$
production, any event that has an additional third lepton passing the
identification and isolation requirements is rejected. 
$\wgamma$ production, in which the photon converts, is suppressed
by rejecting electrons consistent with a photon conversion.

