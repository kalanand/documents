
Counting the number of events in the signal region,
the $\WW$ yield is calculated by subtracting the estimated contributions of the various 
SM background processes.  The signal efficiency times acceptance averaged over all 
lepton flavors including $\tau$s is found to be  $(3.22 \pm 0.22~\rm{(total)})\%$.

The total background yield is $275.2\pm 14.9~(\rm{stat.}) \pm 31.2~(\rm{syst.})$ 
events and the total number of
events observed is $1111$. Using the W $\to \ell \nu$ branching ratio of $(0.1080 \pm 0.0009)$ 
from Ref.~\cite{pdg}, the $\WW$ production cross section in pp collision data at 
$\sqrt{s} = 8~\TeV$, is calculated to be

\begin{displaymath}
\sigma_{\rm{WW}} = \measuredCrossSection.
\end{displaymath}

The statistical uncertainty is due to the total number of observed events.
The systematic uncertainty includes both the statistical and systematic 
uncertainties on the background prediction, as well as the uncertainty 
on the signal efficiency. This measurement is consistent with the 
SM expectation of~\nloCrossSection~\cite{MCFM}. The difference between the
measured and the theoretical value is $12.6 \pm 7.3$~pb, equivalent to
$(22 \pm 13)\%$ of the theoretical value. (Experimental and theoretical
uncertainties have been added in quadrature.)

%%% We have also measured the WW cross section in the di-lepton acceptance region,
%%% defined as follows: an event is accepted if the two leptons from the W bosons
%%% (be it electron, muon or tau) satisfy the $\pt > 20~\GeV$ and $|\eta| < 2.5$
%%% requirements. The acceptance defined this way has an efficiency of
%%% $\epsilon_{\rm{qq}} = 54.4$\% for the qq~$\to$~WW sample and
%%% $\epsilon_{\rm{gg}} = 61.8$\% for the gg~$\to$~WW sample. Using these values,
%%% together with $f_{\rm{gg}} = 0.03$, the WW fiducial cross section is
%%% %
%%% \begin{eqnarray}
%%%   \hat{\sigma}_{\rm{WW}} &=& \sigma_{\rm{WW}} \cdot {\rm{BR}}({\rm{WW}}\to\ell\nu\ell\nu) \cdot \left[f_{\rm{gg}}\cdot\epsilon_{\rm{gg}} + (1 - f_{\rm{gg}})\cdot\epsilon_{\rm{qq}}\right]\nonumber\\
%%%   ~\nonumber\\
%%%   ~ &=& 3.96 \pm 0.16~(\rm{stat.}) \pm 0.33~(\rm{syst.}) \pm 0.20~(\rm{lumi.})~\rm{pb.}\nonumber
%%% \end{eqnarray}

Finally, the measurement of the WW production cross section is interpreted in terms of the
ratio with the Z production cross section. The Z process is measured using events
passing the same lepton selection as the WW measurement, that fall within the Z
mass window in the ${\rm e}^+{\rm e}^-$ and $\mu^+\mu^-$ final states.
The theoretical expectation of the ratio $\sigma_{\rm{WW}}/\sigma_{\rm{Z}}$ at 8~TeV is
\mbox{$(1.72 \pm 0.08~(\rm{scale}) \pm 0.01~\rm({PDF}))\times 10^{-3}$}, where the
Z cross section at NNLO is calculated using \textsc{fewz}~\cite{FEWZ}. The efficiency
and the acceptance of the Z selection are obtained using the \textsc{madgraph} MC
sample, and the extrapolation factor to the $60~\GeV < m_{\ell\ell} < 120~\GeV$
mass range is obtained using \textsc{fewz}. The ratio of the $\WW$ 
efficiency times acceptance to the $\rm{Z}$ efficiency times acceptance is found 
to be $0.112 \pm 0.010~(\rm{syst.})$. The
systematic uncertainty on the efficiency ratio
takes into account the correlation between 
those uncertainties that are common to both terms.
By using the branching fraction for
${\rm{Z}}\to {\rm ee}/\mu\mu$ of $(6.73 \pm 0.008)\%$ from Ref.~\cite{pdg}, 
the WW to Z cross section ratio is found to be,

\begin{equation}
  \frac{\sigma(\rm{WW})}{\sigma(\rm{Z})} = (2.09 \pm 0.18~(\rm{syst.}) \pm 0.09~(\rm{stat.})) \times 10^{-3}.
\end{equation}

The systematic uncertainty includes the theoretical and experimental systematic uncertainties 
on the ratio of the efficiency times acceptance for the two processes, as well
as the total uncertainty due to the subtraction of background processes.
The statistical uncertainty is due to the total number of observed events.
The difference between the measured value of $\sigma_{\rm{WW}}/\sigma_{\rm{Z}}$
and the theoretical one is $(22\pm 13)\%$, neglecting correlations between 
theoretical uncertainties that contribute to both values.

