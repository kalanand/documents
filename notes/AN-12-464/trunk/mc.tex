%%%%%%%%%%%%%%%%%%%%%%%%%%%%%%%%%%%%%%%%%%%%%%%%%%%%%%%%%%%%%%%%%%%%%%%%%%
%%%%%%%%%%%%%%%%%%%%%%%%%%%%%%%%%%%%%%%%%%%%%%%%%%%%%%%%%%%%%%%%%%%%%%%%%%
\clearpage{}
\section{Signal and background expectations from CMS simulation}
\label{sec:MCexpectations}
% ---- ---- ---- ---- ---- ---- ---- ---- ---- ---- --
\subsection {The signal}
Table~\ref{tab:signals} lists the production cross-section for the WW and 
WZ signals and their overall reconstruction efficiency.
%%%%%%%%%%%%%%%%%%%%%%%%%%%%%
\begin{table}[bthp]
\begin{center}

{\footnotesize
\begin{tabular}{l|llllll}
\hline
\hline
        Quantity                        & $\mu$ anti-btag & el anti-btag & $\mu$ b-tag & el btag & $\mu$ boosted & el boosted \\
\hline	
$\sigma_{WW}$(\unit{pb})         & $57.2$           & $57.2$        & $57.2$        &   $57.2$       & $57.2$        & $57.2$       \\
$\sigma_{WZ}$(\unit{pb})         & $22.9$           & $22.9$        & $22.9$        &   $22.9$       & $22.9$        & $22.9$       \\
$\sigma_{WW+WZ}$(\unit{pb})      & $80.1$           & $80.1$        & $80.1$        &   $80.1$       & $80.1$        & $80.1$       \\
	$\mathcal{A}\varepsilon$        & $1.64\times 10^{-3}$           & $1.44\times 10^{-3}$         & $1.33\times 10^{-4}$         & $1.14\times 10^{-4}$         & $2.02\times 10^{-4}$        & $1.76\times 10^{-4}$       \\
\hline
\hline
\end{tabular}
}

\end{center}
\caption{\label{tab:signals}
The cross section for the WW and WZ signals~\cite{Campbell:2011bn} and their overall reconstruction efficiency, which includes
the W leptonic branching fraction and the W/Z hadronic branching fraction, for the ``at-least-one b-jet'' and ``no b-jet'' final states. The 
acceptance and selection cuts are described in section~\ref{sec:reco} of this note. The systematic uncertainty 
for each $A\times\varepsilon$ is about 4\% and is described in section~\ref{sec:syst}.}
\end{table}
%%%%%%%%%%%%%%%%%%%%%%%%%%%%%
%%%%%%%%%%%
\subsection {The backgrounds}
%%%%%%%%%%%
All processes yielding one lepton, two or more jets, 
and missing transverse energy are possible sources 
of background. The most relevant ones are:
%%%%%%%%%%%
\begin{itemize}
  \item W+jets: this is the production of a single W vector boson 
        in association with quarks or gluons that mimic the final state 
        signature. 
        Because of its large cross section, it is by far the most important 
	background to the analysis.
  \item Drell-Yan Z/$\gamma^{*}$+jets: this is the production of 
        a single Z/$\gamma^{*}$ boson in association with quarks or gluons, 
	where one lepton escapes undetected because of acceptance or 
        inefficiency effects, and the hadronic activity mimics the final 
	 state signature.
%  \item ZZ: in case one Z decays hadronically, and one lepton is not identified by the detector,
%              this sample contributes to the backgrounds.
  \item $t\bar{t}$: top quarks pairs are produced at LHC via the gluon fusion process
              $gg\to{}t\bar{t}$ or via QCD quark annihilation $q\bar{q}\to{}t\bar{t}$.
              The semi-leptonic component, in which one W decays hadronically,
              is reduced by requiring exactly 2 or 3 jets,
              while the fully leptonic decay mode is reduced by requiring only 
	      one good lepton in the event.
              However, because of acceptance and inefficiencies, 
              this background still contaminates the signal.
  \item Single top production: it proceeds through three separate channels:
       \begin{enumerate}
         \item t-channel: top is produced after a quark-gluon interaction 
               with the exchange of a virtual W.
         \item s-channel: top is produced in association with an anti-bottom, 
               after the annihilation of a pair of quarks in a weak vertex.
         \item tW-channel: top is produced in association with a charged vector boson in a weak process, 
               from a gluon-bottom pair in the initial state.
       \end{enumerate}
%       The first two can be distinguished from signal thanks to a selection on the energy of b-quarks, 
%       which are quite different from signal tag quarks, while tW-channel has got a missing jet.
  \item QCD multi-jet events generate a background 
       because of the non-negligible probability of jets to be reconstructed as leptons.
\end{itemize}
%%%%%%%%%%%%%%%%%%%%%%
The cross section for the backgrounds, multiplied by the branching ratio when meaningful, 
are reported in Table~\ref{tab:bkg_XS}. 
%%%%%%%%%%%
\begin{table}[htbp!]
  \begin{center}
  \begin{tabular}{c|c}
  \hline  \hline
  Channel & Cross-section (pb) \\
  \hline
  W+jets                        & $37509$ \\
  Z+jets                        & $3503.7$ \\
%  ZZ                            & $8.3$\\
  t$\bar{\textnormal{t}}$+jets  & $225.2$ \\
  t+jets ($s$-channel)          & $5.55$ \\
  t+jets ($t$-channel)          & $87.1$\\
  t+jets ($t$W-channel)         & $22.2$ \\
%%  QCD (ele enriched)            & $6740000 \pm 100\%$\\
  QCD (mu enriched)             & $134680$\\
  \hline  \hline
  \end{tabular}
  \end{center}
  \caption{The cross section for the backgrounds, multiplied by the branching ratio when meaningful.}
  \label{tab:bkg_XS}
\end{table}%
%%%%%%%%%%%
%%%%%%%%%%%%%%%%%%%%%%%%%%%%%%%%%%%%%%%%%%%%
\section{Datasets}
\label{sec:technicalities}
% ---- ---- ---- ---- ---- ---- ---- ---- ---- ---- ---- ---- --
\subsection{Data samples}
The data samples used in this analysis were recorded by the CMS experiment in 2012.
Only certified runs and luminosity sections are considered, which means that a good functioning
of all CMS sub-detectors is required. The total statistics analyzed correspond to an integrated
luminosity of about $19.5~\fbinv$.
% The analysis relies on centrally-produced Primary Datasets (PDs), each of which consists of a collection
% of High Level Trigger (HLT) paths. As explained in Section~\ref{sec:trigger}, single-lepton triggers
% are used to select the events interesting for this analysis in both
% the $e$ and $\mu$ channels, up to an instantaneous luminosity $L \sim 1\cdot10^{33}\percms$. At higher
% luminosities, we move to cross electron+di-jet triggers for the $e$ channel to cope with the
% unsustainable single-electron HLT rate. Therefore, the analysis relies on the so-called ``SingleElectron''
% and ``SingleMu'' PDs for the first data-taking period, and on the ``ElectronHad'' and ``SingleMu'' ones 
% for the most runs. 
The datasets used for the analysis and the corresponding run ranges are listed in Table~\ref{tab:datasets}.
All data samples were reconstructed using a \texttt{CMSSW\_5\_3\_X} release version.
%%%%%%%%%%%
\begin{table}[htbp!]
  \begin{center}
  \begin{tabular}{r|r}
  \hline  \hline
  Dataset name & Run range \\
  \hline
  /SingleMu/Run2012A-13Jul2012-v1/AOD         & 190456-193621  \\
  /SingleElectron/Run2012A-13Jul2012-v1/AOD   &            \\
  \hline
  /SingleMu/Run2012A-recover-06Aug2012-v1/AOD        &  190782-190949  \\
  /SingleElectron/Run2012A-recover-06Aug2012-v1/AOD  &     \\
  \hline
  /SingleMu/Run2012B-13Jul2012-v1/AOD         &  193833-196531  \\
  /SingleElectron/Run2012B-13Jul2012-v1/AOD   &         \\
  \hline
  /SingleMu/Run2012C-24Aug2012-v1/AOD       & 198022-198913 \\
  /SingleElectron/Run2012C-24Aug2012-v1/AOD & \\
  \hline
  /SingleMu/Run2012C-PromptReco-v2/AOD       & 198934-203746 \\
  /SingleElectron/Run2012C-PromptReco-v2/AOD & \\
  \hline
  /SingleMu/Run2012D-PromptReco-v1/AOD       & 203894-208686 \\
  /SingleElectron/Run2012D-PromptReco-v1/AOD & \\
  \hline  \hline
  \end{tabular}
  \end{center}
  \caption{Summary of the data samples used and the run ranges of applicability.}
  \label{tab:datasets}
\end{table}%
%%%%%%%%%%%
%%%%%%%%%%%%%%%%%%%%%%%%%%%%%%%%%%%%%%%%%%%%
\subsection{Monte Carlo samples}
Samples for a large variety of electroweak and QCD-induced background sources, 
as well as for the electroweak diboson signal 
have been generated and showered using different Monte Carlo generators.
To better reproduce the actual data-taking conditions, where there is a significant probability
that more than two protons interact in the same bunch crossing, pile-up (PU) events are
added on top of the hard scattering. Particle interactions with the detector were reproduced through
a detailed description of CMS.


The diboson events were produced with PYTHIA6 \cite{pythia}.
The background and signal samples used for the studies are listed in Table~\ref{tab:MCsamples},
together with the equivalent luminosity available for the study.
All background MC samples considered in this analysis come from 
the official ``Summer12'' production. Events were
reconstructed making use of a \texttt{CMSSW\_5\_3\_X} release version. 
% The pile-up scenario used for
% these samples consists of a flat distribution of PU events up to ten additional interactions on top
% of the hard scattering plus a poissonian tail with a mean of ten interactions; only pile-up occurring 
% in the same bunch crossing of the main event (in time PU) was considered.
%%%%%%%%%%%%%%%%%%%%%%%%%%%%%%%%%
\begin{table}[htb]
  \begin{center}
  \scriptsize
    \begin{tabular}{l c} 
      \hline\hline
      sample & $\approx$ \lumi (\fbinv)\\
      \hline
      /W2JetsToLNu\_TuneZ2Star\_8TeV-madgraph/Summer12\_DR53X-PU\_S10\_START53\_V7A-v1/AODSIM & 20\\
      /W3JetsToLNu\_TuneZ2Star\_8TeV-madgraph/Summer12\_DR53X-PU\_S10\_START53\_V7A-v1/AODSIM & 30\\
      /W4JetsToLNu\_TuneZ2Star\_8TeV-madgraph/Summer12\_DR53X-PU\_S10\_START53\_V7A-v1/AODSIM & 60\\
      /WJetsToLNu\_matchingdown\_8TeV-madgraph-tauola/Summer12\_DR53X-PU\_S10\_START53\_V7A-v1/AODSIM & 1\\
      /WJetsToLNu\_matchingup\_8TeV-madgraph-tauola/Summer12\_DR53X-PU\_S10\_START53\_V7A-v1/AODSIM & 1\\
      /WJetsToLNu\_scaledown\_8TeV-madgraph-tauola/Summer12\_DR53X-PU\_S10\_START53\_V7A-v1/AODSIM & 1\\
      /WJetsToLNu\_scaleup\_8TeV-madgraph-tauola/Summer12\_DR53X-PU\_S10\_START53\_V7A-v2/AODSIM & 1\\ 
      /WJetsToLNu\_PtW-100\_TuneZ2star\_8TeV-madgraph/Summer12\_DR53X-PU\_S10\_START53\_V7A-v1/AODSIM & 33\\
%      /WJetsToLNu\_TuneZ2Star\_8TeV-madgraph-tarball/Summer12\_DR53X-PU\_S10\_START53\_V7A-v1/AODSIM        & 2.5\\
      /TTJets\_MassiveBinDECAY\_TuneZ2star\_8TeV-madgraph-tauola/Summer12\_DR53X-PU\_S10\_START53\_V7A-v1/AODSIM            & 23\\
      /TT\_CT10\_TuneZ2star\_8TeV-powheg-tauola/Summer12\_DR53X-PU\_S10\_START53\_V7A-v1/AODSIM & 107\\                              
      /DYJetsToLL\_M-50\_TuneZ2Star\_8TeV-madgraph-tarball/Summer12\_DR53X-PU\_S10\_START53\_V7A-v1/AODSIM  & 11\\
      /Tbar\_s-channel\_TuneZ2star\_8TeV-powheg-tauola/Summer12\_DR53X-PU\_S10\_START53\_V7A-v1/AODSIM     & 82\\
      /Tbar\_t-channel\_TuneZ2star\_8TeV-powheg-tauola/Summer12\_DR53X-PU\_S10\_START53\_V7A-v1/AODSIM     & 83\\
      /Tbar\_tW-channel-DR\_TuneZ2star\_8TeV-powheg-tauola/Summer12\_DR53X-PU\_S10\_START53\_V7A-v1/AODSIM & 41\\
      /T\_s-channel\_TuneZ2star\_8TeV-powheg-tauola/Summer12\_DR53X-PU\_S10\_START53\_V7A-v1/AODSIM        & 81\\
      /T\_t-channel\_TuneZ2star\_8TeV-powheg-tauola/Summer12\_DR53X-PU\_S10\_START53\_V7A-v1/AODSIM        & 93\\
      /T\_tW-channel-DR\_TuneZ2star\_8TeV-powheg-tauola/Summer12\_DR53X-PU\_S10\_START53\_V7A-v1/AODSIM    & 103\\
      \hline
      /WW\_TuneZ2star\_8TeV\_pythia6\_tauola/Summer12\_DR53X-PU\_S10\_START53\_V7A-v1/AODSIM  & 92  \\
      /WZ\_TuneZ2star\_8TeV\_pythia6\_tauola/Summer12\_DR53X-PU\_S10\_START53\_V7A-v1/AODSIM  & 224  \\
      \hline\hline
    \end{tabular}
  \end{center}
  \caption{Summary of Monte Carlo samples used in the analysis for background and signal modeling and for systematic studies.}
  \label{tab:datasets:mcstat}
  %FIXME add the corresponding cross-sections
  \label{tab:MCsamples}
\end{table}
%%%%%%%%%%%%%%%%%%%%%%%%%%%%%%%%%
\clearpage
%%%%%%%%%%%%%%%%%%%%%%%%%%%%%%%%%%%%%%%%%%%%%%%%%%
