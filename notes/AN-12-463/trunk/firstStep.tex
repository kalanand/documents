\section{Common Event Selection}
\label{sec:firstStep}
% ---- ---- ---- ---- ---- ---- ---- ---- ---- ---- ---- ---- ---- ---- ---- ---- ---- ---- ---- ---- ---- ---- ----

The final state of the Higgs decay is characterized by a charged lepton, 
large missing energy and two hadronic jets that form a W.
In this section, we first describe the criteria applied to objects
selected in the event and then we describe the requirements made at
the event-level.

\subsection{Object Definitions}

The analysis relies on the standard reconstruction algorithms produced
by the CMS community. 
The PF2PAT procedure was used to coherently define the collection of particle-flow jets, leptons and
MET considered in the event selection.  
The technical details of the software configuration can be found in the group twiki page~\cite{WG_PATtuple_twiki}. 

\subsubsection{Electron Cuts (e+jets)}
\label{sec:electron_cuts}

Electrons are reconstructed using a gaussian-sum filter (GSF)
algorithm \cite{CMS-PAS-EGM-10-004}, and are required to pass electron
ID cuts according to a multi-variate identification
technique~\cite{cite:elemva}.  We also require that selected
electron candidates are isolated. Particle flow-based relative
isolation is defined as
%%%
\begin{equation*}
\mathrm{RelIso_{\mathrm{PF}}} = \frac{I_{\mathrm{CH}}+max(0,I_{\mathrm{NH}}+I_{\mathrm{PHOTON}}-(\mathrm{EA}\cdot\rho))}{E_\mathrm{T}},
\end{equation*} 
%%%

where $I_{\mathrm{CH}}$, $I_{\mathrm{NH}}$ and $I_{\mathrm{PHOTON}}$
are the charged hadron, neutral hadron and photon isolation variables
(using an isolation cone of 0.3). The isolation is corrected for
contributions from pile-up using charged hadron subtraction in the
isolation cone using fastjet algorithm \cite{FastJetPUSubtraction} and
for neutral particles using the effective area correction,
$(\mathrm{EA}\cdot\rho)$ where $\mathrm{EA}$ is the cone effective area
and $\rho$ is the average neutral particle density of the event.

The ID and isolation cuts used are shown in table~\ref{tab:EleID} and
have been tuned to give the same efficiency bin-by-bin with respect to
the working point (WP) used for 2011 analysis.

Additionally, we require
%%%%%%%%%%%%%%%%%%%
\begin{itemize}
\item Electron $E_\mathrm{T} > 30\,\mathrm{GeV}$.
\item Pseudorapidity $|\eta| < 2.5$. There is an exclusion range due
        to the ECAL barrel-endcap transition region, defined by
        $1.4442 < |\eta_{\mathrm{sc}}| < 1.566$, where
        $\eta_{\mathrm{sc}}$ is the pseudorapidity of the ECAL
        supercluster.
%\item Impact parameter: We cut on the absolute value of the impact
%       parameter calculated with respect to the average primary vertex (PV). We
%       require: $d_0(\mathrm{PV}) < 0.02\,\mathrm{cm}.$

%\item In order to make sure that the selected electron and the selected
%jets come from the same hard interaction and not from pile up events,
%we require that the $z$ coordinate of the PV of the event and the $z$
%coordinate of the electron's vertex lie within a distance of
%less than $0.1~\mathrm{cm}$.

\item 
In order to reject events in which the electron candidate actually
originates from a conversion of a photon into an $e^{+}e^{-}$ pair, we
use an approach using the vertex fit probability of fully
reconstructed conversions combined with the requirement that the
number of missed inner tracker layers of the electron track must be
exactly zero (i.e. there are no missed layers before the first hit of
the electron track from the beam line). 
\end{itemize}
%%%%%%%%%%%%%%%%%%%
%%%%%%%%%%%%%%%%%%%
%%%%%%%%%%%%%%%%%%%
\begin{table}[bthp]
\begin{center}
{\footnotesize
\begin{tabular}{|c|c|c|c|}
\hline
Lepton $\eta$ & $|\eta| < 0.8$ & $0.8 < |\eta| < 1.479$ & $1.479 < |\eta| < 2.5$  \\
\hline
ID MVA cut value (tight lepton) & 0.913 & 0.964 & 0.899 \\
Isolation cut value (tight lepton) & 0.105 & 0.178 & 0.150 \\
ID MVA cut value (loose lepton) & 0.877 & 0.811 & 0.707 \\
Isolation cut value (loose lepton) & 0.426 & 0.481 & 0.390 \\
\hline
\end{tabular}
\caption[.]{\label{tab:EleID} Cut values for electron identification
MVA output and for isolation which are tuned to give the same
efficiency as VBTF Working Point (WP) 80, as used for the tight
electron selection, and VBTF Working Point (WP) 90, as used in the
loose electron selection.}}
\end{center}
\end{table}
%%%%%%%%%%%%%%%%%%%
%%%%%%%%%%%%%%%%%%%%%%%%%%%%%%%%%%%%%%%%%%%%%%%%%%%%%%%%%%%%%%%%%%%%%%%%%%%%
%%%%%%%%%%%%%%%%%%%%%%%%%%%%%%%%%%%%%%%%%%%%%%%%%%%%%%%%%%%%%%%%%%%%%%%%%%%%

\subsubsection{Muon Cuts (mu+jets)}
\label{sec:muon_cuts}

Muon candidates are identified by two different 
algorithms~\cite{MUONPAS}: one proceeds from the inner tracker outwards, 
the other one starts from tracks measured in the muon chambers and matches 
and combines them with tracks reconstructed in the inner tracker. 
These selection criteria are summarized below:
%%%%%%%%%%%%%%%%%%%
\begin{itemize}
\item The muon candidate is reconstructed both as a global muon and
as a tracker muon.
\item Number of pixel hits of the Tracker track $\ge 1$;
\item Number of muon system hits of the Global track $\ge 1$;
\item Normalized $\chi^{2}$ of the Global track $< 10.0$.
\item Muon $p_{\mathrm{T}} > 25\,\mathrm{GeV}$.
\item Pseudorapidity $|\eta| < 2.1$.
\item Impact parameter: We cut on the absolute value of the impact
parameter calculated with respect to the primary vertex. We require:
$d_0(\mathrm{PV}) < 0.02\,\mathrm{cm}.$
\item In order to make sure that the selected muon and the selected
jets come from the same hard interaction and not from pile up events,
we require that the $z$ coordinate of the PV of the event and the $z$
coordinate of the muon's inner track vertex lie within a distance of
less than 0.5~cm.
\item The number of tracker layers with hits from the muon track has to be
$N_{\mathrm{layers}} > 5$.
\end{itemize}

The selected muon candidates also have to be isolated. Particle
flow-based relative isolation for muons is defined as
\begin{equation*}
\mathrm{RelIso_{\mathrm{PF}}} = \frac{I_{\mathrm{CH}}+max(0,I_{\mathrm{NH}}+I_{\mathrm{PHOTON}}-(0.5~{p}_{T}^\mathrm{sumPU}))}{p_\mathrm{T}},
\end{equation*} 

where $I_{\mathrm{CH}}$, $I_{\mathrm{NH}}$ and $I_{\mathrm{PHOTON}}$
are the charged hadron, neutral hadron and photon isolation variables
(using an isolation cone of 0.4). The charged hadron isolation variable uses tracks from the primary vertices only.
The neutral hadron and photon isolation variables are corrected for
contributions from pile-up using the DeltaBeta correction, $(0.5
p_T^\mathrm{sumPU})$. We require the muon to have
$\mathrm{RelIso_{\mathrm{PF}}} < 0.12$ in order to be considered
isolated.
%%%%%%%%%%%%%%%%%%%%%%%%%%%%
%%%%%%%%%%%%%%%%%%%%%%%%%%%%
%%%%%%%%%%%%%%%%%%%%%%%%%%%%%%%%%%%%%%%%%%%%%%%%%%%%%%%%%%%%%%%%%%%%%%%%%%%%
%%%%%%%%%%%%%%%%%%%%%%%%%%%%%%%%%%%%%%%%%%%%%%%%%%%%%%%%%%%%%%%%%%%%%%%%%%%%


\subsubsection{Loose Electron}
For the purposes of rejecting events with more than one lepton we
define a loose electron, which has looser cuts. We consider electrons
which have $p_{\mathrm{T}} > 20\,\mathrm{GeV}/c$, $|\eta| < 2.5$,
and which satisfy electron $\mathrm{RelIso_{\mathrm{PF}}}$ and MVA ID
cuts. The cut values for the electron ID and isolation used in the
analysis can be found in table~\ref{tab:EleID}. 
%As in the case of the
%tight electrons, we also require $d_0(\mathrm{PV}) <
%0.02\,\mathrm{cm}$ and that the $z$ coordinate of the PV of the event
%and the $z$ coordinate of the electron's vertex lie within a distance
%of less than $0.1~\mathrm{cm}$.

\subsubsection{Loose Muon}
Additionally, to reject events with more than one lepton, we define a
loose muon, which has looser cuts. We consider all global muons which
have $p_{\mathrm{T}} > 10\,\mathrm{GeV}/c$, $|\eta| < 2.5$, and
$\mathrm{RelIso_{\mathrm{PF}}} < 0.2$ to be loose muons.

\subsubsection{Jet Cuts}
\label{sec:firstStep_jets}

Jets are reconstructed with the anti-KT algorithm \cite{cacciari},
starting from the set of objects reconstructed by the particle flow
\cite{pflow,CMS-PAS-JME-10-003,CMS-PAS-PFT-10-002}.  Jets are
corrected such that the measured energy of the jet correctly
reproduces the energy of the initial particle.  The CMS standard L1
pile-up correction removes the energy contributions coming from pile-up
events. The standard L2
(relative) correction makes the jet response flat in $\eta$.  The
standard L3 (absolute) correction brings the jet closer to the $\PT$
of a matched generated jet created using generator level input and a
similar jet clustering algorithm.  The L2 and L3 corrections are
calculated using Monte Carlo, and thus a L2L3 residual correction is
applied that fixes the discrepancies between Monte Carlo and
data~\cite{newjes-cms}.  In this analysis we use jets with measured
(corrected) $\PT$ greater than 30~$\gev$.  We require $|\eta| < 2.4$
so that the jets fall within the tracker acceptance. Jets from pile-up
are identified and removed with PileupJetID tool
~\cite{cite:PileupJetID}.  To reduce the contribution of jets from top
decays, we require that each jet that falls within the tracker
acceptance not be b-tagged as defined by the CSV loose b-tagging
working point.

%Jets are required to pass a set of loose identification
%criteria; this requirement eliminates jets originating from or being seeded by
%noisy channels in the calorimeter~\cite{Chatrchyan:2009hy}: 
%%%%%%%%%%%%%%
%\begin{itemize}
%\item Fraction of energy due to neutral hadrons $<$ 0.99.
%\item Fraction of energy due to neutral electromagnetic (EM) deposits $<$ 0.99.
%\item Number of constituents $>$ 1.
%\item Number of charged hadron candidates $>$ 0.
%\item Fraction of energy due to charged hadron candidates $>$ 0.
%\item Fraction of energy due to charged EM deposits $<$ 0.99.
%\end{itemize}
%%%%%%%%%%
%All energy fractions are calculated from uncorrected jets.

\par
In order to account for electron and muon objects that have been
reconstructed as jets, we remove from the jet collection any jet that
falls within a cone of radius $R= 0.5$ of a loose electron or a loose
muon.  This ``cleaning'' procedure is applied in order to ensure that
the same particle is not double counted as two different physics
objects.


\subsection{Event-Level Criteria}

The event should have a good primary vertex (PV). This means selecting
the primary vertex with the highest sum of $p_{T}^2$ of the tracks
associated with it and requiring it to have a number of degrees of
freedom (ndof) $\ge 4$, where ndof corresponds to the weighted sum of
the number of tracks used for the construction of the PV. In addition,
the PV must lie in the central detector region of $abs(z) \le
24~\rm{cm}$ and $\rho \le 2~\rm{cm}$ around the nominal interaction
point.

In the e+jets channel, we select events which contain exactly one
tight electron candidate fulfilling the criteria described in
Section~\ref{sec:electron_cuts} and reject events which contain a
loose electron in addition to the tight electron. In this channel we
are only interested in the decay to electron and jets, and we
therefore reject events containing a loose muon.

In the mu+jets channel, we select events which contain exactly one
tight muon candidate whose criteria are described in
Section~\ref{sec:muon_cuts} and reject events which contain an
additional loose muon. In an analoguous way to the e+jets channel, we
reject events containing a loose electron.

We require an event to have missing transverse energy (MET) in excess
of 25(35)~GeV for muons (electrons) and to have a $W$ transverse mass
of at least $30\,\mathrm{GeV}$.  These cuts are designed to reduce the
background from QCD multijet production.

% We further require exactly two or three
% jets passing the cuts
% described in Section~\ref{sec:firstStep_jets}.  
% We require the
% invariant mass of the dijet system formed by the two highest $p_{T}$
% jets in the event to be between $65$ and $95\,\mathrm{GeV}$.

