\section{Modeling of background shapes in \texorpdfstring{$m_{jj}$}{dijet invariant mass} }
\label{sec:modelShape}

We utilize parameterizations of the shapes for the background.  These
are determined from fully corrected MC.  There are 3 major
contributions to the background, V+jets (W and Z), top
($t\overline{t}$ and single $t$), and diboson (WW and WZ).  These are
used in the fit to the $m_{jj}$ spectrum with the hadronic W signal
region removed.  The next sections detail the shapes used for each
background component.

\subsection{diboson background shape}
\label{sec:dibosonShape}

The diboson component is modeled as a sum of two Gaussian
distributions whose means differ by the W/Z mass difference.  The
widths of the peaking shapes is also the same fraction of the mean.
There is also a tail for poorly reconstructed diboson events consiting
of an exponential decay times a error function turn on.  The values of
the parameters are determined from MC and fixed in the fit.
\begin{equation}
\mathcal{F}_\text{diboson} = f_W\mathcal{G}_W + f_Z\mathcal{G}_Z + (1-f_W-f_Z)\mathcal{F}_\text{comb}
\end{equation}

\subsection{top background shape}
\label{sec:topShape}

The top component is model as a peaking component and a combinatoric
component.  For Higgs mass hypotheses below 450 GeV the combinatoric
background is a exponential times an error function for 450 GeV and
higher the combinatoric component is just a second Gaussian.  Again
the values of the parameters is determined from MC and fixed in the
fit.
\begin{equation}
\mathcal{F}_\text{top} = f_\text{peak}\mathcal{G} +
(1-f_\text{peak})\mathcal{F}_\text{comb}
\end{equation}

\subsection{V+jets background shape}
\label{sec:wjetsShape}
 
We employ an empirical description of the V+jets shape.  This
description is a kinematic turn on and a power law tail:
\begin{equation}
  \mathcal{F}_{W+\text{jets}} = \text{erf}(m_{jj}; m_0, \sigma)\times\left[(m_{jj})^{-\alpha-\beta\ln(m_{jj}/\sqrt{s})}\right]\,,
\end{equation}
where $m_0$ is the value of the turn on and $\sigma$ is the width of
this turn on.  The parameters $m_0$, $\sigma$, $\alpha$ and $\beta$
are determined in the fit to the data after the MVA cut.  There are
some constraints on the parameters from the MC, but because of the
relatively low statistics of the MC samples they are not overly
constraining.

% This nominal fit shape is not particularly well suited for all of the
% mass points.  In the 2-jet channels for masses from 180 GeV and below
% we use the MC morphing technique used in the study of the W+2 jets
% mass spectrum analysis documented in CMS AN-2011/266, Section 12.
% These lower mass points do not suffer from the lack of statistics as
% they are background rich, particularly in W+jets background.  We also
% use the MC morphing technique for 3-jet mass points from 200 GeV and
% below.  We use the following parameterization for 2-jet mass points
% 190 and 200 GeV.

% \begin{equation}
%   \mathcal{F}_{W+\text{jets low mass, 2 jets}} = \text{erf}(m_{jj}; m_0, \sigma)\times(m_{jj})^{-\alpha}\times\exp(m_{jj}\tau)\,,
% \end{equation}

% where the parameters $m_0$, $\sigma$, $\alpha$ and $\tau$ are
% determined in the fit.  In the 3-jet channels we use the
% parameterization

% \begin{equation}
%   \mathcal{F}_{W+\text{jets low mass, 3 jets}} = (m_{jj})^{-\alpha-\beta\ln(m_{jj}/\sqrt(s))}\times\exp(m_{jj}\tau)
% \end{equation}

% for masses 250 and 300 GeV.  These functional line shapes are
% motivated by the W+jets MC, however their parameters are derived
% strictly from data in the $m_{jj}$ sidebands around the W mass.
