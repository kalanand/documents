\section{Introduction}
\label{sec:intro}
% ---- ---- ---- ---- ---- ---- ---- ---- ---- ---- ---- ---- ---- ---- ---- ---- ---- ---- ---- ---- ---- ---- ----
The Standard Model (SM) of electroweak
interactions~\cite{StandardModel67_1,StandardModel67_2,StandardModel67_3}
relies on the existence of the Higgs boson, H, a scalar particle
associated with the field responsible for spontaneous electroweak
symmetry
breaking~\cite{Englert:1964et,Higgs:1964ia,Higgs:1964pj,Guralnik:1964eu,Higgs:1966ev,Kibble:1967sv}.
The observation of a Higgs boson with a mass of 125
GeV~\cite{Chatrchyan:2012ufa,Aad:2012tfa} is consistent with the
theoretical constraint coming from the unitarization of WW scattering
at high
energies~\cite{Dicus:1992vj,Veltman:1976rt,Lee:1977eg,Lee:1977yc,Passarino:1990hk,Chanowitz:1985hj,Duncan:1985vj,Dicus:1986jg,Bagger:1995mk,Ballestrero:2009vw}.
However, there is still a possibility that the newly discovered
particle has no connection to the electroweak symmetry breaking
mechanism~\cite{Low:2011gn,Low:2012rj}; thus, it is important to
continue searching for the SM Higgs boson in the high mass regime to
further confirm the properties of the new boson at 125~GeV. In
addition, several popular scenarios, such as general two-Higgs-doublet
models (for a review see~\cite{Branco:2011iw}) or models in which the
SM Higgs boson mixes with a heavy electroweak
singlet~\cite{Patt:2006fw,PhysRevD.77.035005,PhysRevD.77.117701,PhysRevD.79.095002,Bock201044,Fox:2011qc,Englert:2011yb,PhysRevD.85.035008,Batell:2011pz,Englert:2011aa,Gupta:2011gd,Dolan:2012ac,Bertolini:2012gu,Batell:2012mj},
predict the existence of additional resonances at high mass, with
couplings similar to the SM Higgs boson.

For Higgs masses above or near the threshold for decay into two vector
bosons, the decay modes of choice are dominated by those decays
because of their large branching fractions.  It is clear that the
events where one $W$ decays leptonically, which provides the main
trigger elements, while the other decays hadronically have the second
highest branching fraction and have a reconstructable Higgs mass
peak~\cite{intro2}.

This note contains the analysis that sets a limit on the Higgs boson
cross-section based on this decay mode, performed on data acquired by
CMS at $\sqrt{s}~=$~8~TeV during the year 2012. Previous searches
performed with similar procedures are documented in \cite{HIG-12-046}
(12\fbinv collected in 2012) and \cite{HIG-12-003} (5\fbinv
collected in 2011).  The analysis selects events with one
well-identified and isolated lepton, large missing transverse energy
and at least two high \pt jets.  Therefore, the main experimental
issue is to control the large $W$ plus jets background sufficiently
well that the advantages of using this final state are realized.

Physics objects are selected with the techniques available in 2012, in
particular jet identification is applied to reduce the effect of
pile-up. Much of the analysis techniques are unchanged with respect to
\cite{HIG-12-046}. Differences from \cite{HIG-12-046} include the
increase in integrated luminosity, conversion from exclusive to
inclusive jet binning (elimination of n-jet categories), the folding
of the multijet data-driven background estimation for the electron
channel into the V+jets background estimation, and a new procedure
that simultaneously fits two and four-body distributions and extracts
the upper limits using the Higgs ``combine'' package.

The note is structured in a manner similar to previous notes.  A
discussion about the data samples used in the analysis and the trigger
selections is presented in Sections~\ref{sec:MCexpectations} and
\ref{sec:technicalities}.  The physics objects reconstruction is
discussed in Section~\ref{sec:firstStep}.  The lepton selection and
other preselection requirements are described in detail in
Section~\ref{sec:firstStep} and \ref{sec:dataMCcomparisons}.

After the preselections, the signal-over-background ratio is enhanced
by means of a selection on a MVA discriminant, designed in order to
control the background while preserving as much as possible the
difference in shape with respect to the signal
(Section~\ref{sec:mvaoptimization}).  The input variables of the MVA
exploit the decay angles of the four-body mass system, the kinematics
of the entire four-body system. The MVA variable definition is
optimized with dedicated trainings for each Higgs mass hypothesis case
and for each lepton flavour ($e$, $\mu$).  In this way, 24 different
configurations are obtained.

The main background contaminating the signal region is W+jets; its
contribution is determined from data.  Minor backgrounds include
$t\bar{t}$, single top, and diboson, and are determined using Monte
Carlo. The procedures for extracting the $m_{jj}$ and
$m_{\ell{}\nu{}jj}$ shapes and normalizations for these backgrounds
are described in Sections~\ref{sec:modelShape}
and~\ref{sec:wjetsBackground}.
%Also the QCD shapes come from a data-driven determination, as described in
%Section~\ref{sec:dataDrivenQCD}.  

The systematic uncertainties present in the signal description are
described in Section~\ref{sec:systematics}.
Section~\ref{sec:limitExtraction} describes the obtained limits on the
Standard Model Higgs cross section and Section~\ref{sec:conclusions}
closes this work.

