% Customizable fields and text areas start with % >> below.
% Lines starting with the comment character (%) are normally removed before release outside the collaboration, but not those comments ending lines

% svn info. These are modified by svn at checkout time.
% The last version of these macros found before the maketitle will be the one on the front page,
% so only the main file is tracked.
% Do not edit by hand!
\RCS$Revision: 110852 $
\RCS$HeadURL: svn+ssh://svn.cern.ch/reps/tdr2/notes/AN-12-021/trunk/AN-12-021.tex $
\RCS$Id: AN-12-021.tex 110852 2012-03-15 14:20:08Z kalanand $
%%%%%%%%%%%%% local definitions %%%%%%%%%%%%%%%%%%%%%
% This allows for switching between one column and two column (cms@external) layouts
% The widths should  be modified for your particular figures. You'll need additional copies if you have more than one standard figure size.
\newlength\cmsFigWidth
\ifthenelse{\boolean{cms@external}}{\setlength\cmsFigWidth{0.85\columnwidth}}{\setlength\cmsFigWidth{0.4\textwidth}}
\ifthenelse{\boolean{cms@external}}{\providecommand{\cmsLeft}{top}}{\providecommand{\cmsLeft}{left}}
\ifthenelse{\boolean{cms@external}}{\providecommand{\cmsRight}{bottom}}{\providecommand{\cmsRight}{right}}
%%%%%%%%%%%%%%%  Title page %%%%%%%%%%%%%%%%%%%%%%%%
\cmsNoteHeader{AN-12-021} % This is over-written in the CMS environment: useful as preprint no. for export versions
% >> Title: please make sure that the non-TeX equivalent is in PDFTitle below
\title{Computation of Offline \& Trigger Efficiencies for Standard Model Higgs boson search in H$\to$WW$\to\ell\nu jj$ decay}

% >> Authors
%Author is always "The CMS Collaboration" for PAS and papers, so author, etc, below will be ignored in those cases
%For multiple affiliations, create an address entry for the combination


\address[ttu]{Texas Tech University, Lubbock, Texas, USA}
\address[fnal]{Fermi National Accelerator Laboratory, Batavia, Illinois, USA}
\address[tamu]{Texas A\&M University, College Station, Texas, USA}
\address[nebr]{University of Nebraska at Lincoln, Nebraska, USA}
\address[uerj]{Universidade do Estado do Rio de Janeiro (UERJ), Brazil}

\author[ttu]{Nural~Akchurin}
\author[fnal]{Jake~Anderson}
\author[fnal]{Andrew Beretvas}
\author[fnal]{Jeffrey~Berryhill}
\author[fnal]{Pushpa~Bhat}
\author[ttu]{Phil~Dudero}
\author[tamu]{Ricardo~Eusebi}
\author[fnal]{Dan~Green}
\author[nebr]{Pratima~Jindal}
\author[ttu]{Sung-Won Lee}
\author[fnal]{Kalanand~Mishra}
\author[tamu]{Ilya~Osipenkov}
\author[tamu]{Alexx~Perloff}
\author[uerj]{Andre~Sznajder}
\author[fnal]{Nhan~V.~Tran}
\author[fnal]{Fan~Yang}
\author[fnal]{Francisco~Yumiceva}


% >> Date
% The date is in yyyy/mm/dd format. Today has been
% redefined to match, but if the date needs to be fixed, please write it in this fashion.
% For papers and PAS, \today is taken as the date the head file (this one) was last modified according to svn: see the RCS Id string above.
% For the final version it is best to "touch" the head file to make sure it has the latest date.
\date{\today}

% >> Abstract
% Abstract processing:
% 1. **DO NOT use \include or \input** to include the abstract: our abstract extractor will not search through other files than this one.
% 2. **DO NOT use %**                  to comment out sections of the abstract: the extractor will still grab those lines (and they won't be comments any longer!).
% 3. **DO NOT use tex macros**         in the abstract: External TeX parsers used on the abstract don't understand them.
\abstract{

We present a study of the lepton efficiency for offline 
reconstruction and selection (isolation, ID) and for online 
triggering by the CMS High-level and Level-1 trigger system. 
The efficiencies are computed using
a ``Tag \& Probe'' technique on the Drell-Yan Z$\to\ell\ell$ events in
both data and Monte Carlo.  
The ratios of data versus Monte Carlo efficiencies for
various steps of lepton reconstruction/selection are derived. 
If such
a ratio is found to be statistically inconsistent with unity, it is
applied as a scale factor correction to the MC samples.
The trigger efficiencies are always computed for 
leptons passing the offline analysis level selection criteria.
For electron data we also compute efficiency for 
the W transverse mass cut ($> 50$~GeV) applied in the 
lowest $p_T$ single electron trigger in most epochs of data taking.
Additionally, we have considered the possibility to use 
electron + 2 jets + missing $H_T$ trigger for which we compute 
factorized efficiency for the lepton leg, jet leg, and missing $H_T$ 
leg.
This analysis note complements the documentation of 
AN-11/110 and AN-12/008.
}

% >> PDF Metadata
% Do not comment out the following hypersetup lines (metadata). They will disappear in NODRAFT mode and are needed by CDS.
% Also: make sure that the values of the metadata items are sensible and are in plain text (no TeX! -- for \sqrt{s} use sqrt(s) -- this will show with extra quote marks in the draft version but is okay).

\hypersetup{%
pdfauthor={Jake Anderson, Phil Dudero, 
Pratima Jindal, Kalanand Mishra, Ilya Osipenkov, Fan Yang},%
pdftitle={Multivariate optimization and background estimation for the Standard Model Higgs boson search in HtoWWtoellnujj decay},%
pdfsubject={CMS},%
pdfkeywords={CMS, physics, software, computing}}

\maketitle %maketitle comes after all the front information has been supplied

% >> Text
%%%%%%%%%%%%%%%%%%%%%%%%%%%%%%%%  Begin text %%%%%%%%%%%%%%%%%%%%%%%%%%%%%
%% **DO NOT REMOVE THE BIBLIOGRAPHY** which is located before the appendix.
%% You can take the text between here and the bibiliography as an example which you should replace with the actual text of your document.
%% If you include other TeX files, be sure to use "\input{filename}" rather than "\input filename".
%% The latter works for you, but our parser looks for the braces and will break when uploading the document.
%%%%%%%%%%%%%%%
\tableofcontents
\newpage
%\section{Introduction}
%\section{Introduction}

Jets are the experimental signatures of quarks and gluons produced in high-energy processes such as hard scattering of partons in proton-proton collisions. The detailed understanding of both the jet energy scale and of the transverse momentum resolution is of crucial importance for many physics analyses, and it is an important component of the systematic uncertainty. This paper presents studies for the determination of the energy scale and resolution of jets, performed with the Compact Muon Solenoid (CMS) at the CERN Large Hadron Collider (LHC), on proton-proton collisions at $\sqrt{s}=7\TeV$, using a data sample corresponding to an integrated luminosity of $36\pbinv$.

The paper is organized as follows: Section~\ref{sec:detector} describes briefly the CMS detector, while Section~\ref{sec:jets} describes the jet reconstruction methods considered here. Sections~\ref{sec:data} and~\ref{sec:methods} present the data samples and the experimental techniques used for the various measurements. The jet energy calibration scheme is discussed in Section~\ref{sec:jec} and the jet transverse momentum resolution is presented in Section~\ref{sec:res}. 

%\newpage
%%%%%%%%%%%%%%%%%%%%%%%%%%%%%%%%%%%%%%%%%%%%%%%%%%%%%%%%%%%%%%%%%%%%
%%%%%%%%%%%%%%%%%%%%%%%%%%%%%%%%%%%%%%%%%%%%%%%%%%%%%%%%%%%%%%%%%%%%
%%%%%%%%%%%%%%%%%%%%%%%%%%%%%%%%%%%%%%%%%%%%%%%%%%%%%%%%%%%%%%%%%%%%

\section{Recap of the event selection}
\label{sec:reco}
The event selection criteria are described in detail in AN-11/110. 
Here we summarize them in brief.

%%%%%%%%%%%%%%%%%%%%%%%%%%%%
\subsection{Electron selection}
\label{sec:electron_cuts}
Electrons are reconstructed using a gaussian sum 
filter (GSF) algorithm \cite{CMS-PAS-EGM-10-004},
and are required to pass electron ID cuts according 
to the simple cut-based electron ID~\cite{simplecutbasedelectronid}, 
with the ``VBTF Working Point 70''. 
The GSF algorithm accounts for possible energy loss due to
bremsstrahlung in the tracker layers.
The energy of an electron candidate with $\et>30~\gev$ is essentially
determined by the ECAL cluster energy, while its momentum direction
is determined by that of the associated track.
The simple cut based electron ID relies on three shower
shape variables with different cut values for the barrel and
the endcap regions, as shown in Table~\ref{tab:EleID}.

Additionally, we require
%%%%%%%%%%%%%%%%%%%
\begin{itemize}
\item Electron $E_\mathrm{T} > 30\,\mathrm{GeV}$.
\item Pseudorapidity $|\eta| < 2.5$. There is an exclusion range due
        to the ECAL barrel-endcap transition region, defined by
        $1.4442 < |\eta_{\mathrm{sc}}| < 1.566$, where
        $\eta_{\mathrm{sc}}$ is the pseudorapidity of the ECAL
        supercluster.
\item Impact parameter: We cut on the absolute value of the impact
       parameter calculated with respect to the average beamspot. We
       require:  $d_0(\mathrm{Bsp}) < 0.02\,\mathrm{cm}.$   
\item The selected electron candidates have to be isolated simultaneously in
the tracker, and in the electromagnetic and hadronic calorimters.  Combined
relative isolation is defined as
%%%
\begin{equation*}
\mathrm{RelIso_{\mathrm{Comb}}} = \frac{I_{\mathrm{Trk}}+I_{\mathrm{EM}}+I_{\mathrm{had}}}{E_\mathrm{T}}.
\end{equation*} 
%%%
The electron candidate is required to have 
$\mathrm{RelIso_{\mathrm{Comb}}} < 0.05$ in order 
to be considered isolated. 
A pile-up offset subtraction in the isolation cone 
using fastjet algorithm \cite{FastJetPUSubtraction} is applied.
We also veto the presence of a second loose lepton in the event.
\item 
In order to reject events in which the electron candidate actually
originates from a conversion of a photon into an $e^{+}e^{-}$ pair, we
require the number of missed inner tracker layers of the electron
track to be exactly zero (i.e. there are no missed layers before the
first hit of the electron track from the beamline). In addition, we
reject any event in which the selected electron is flagged as a
conversion, \textit{i.e.}, an electron that has a 
distance of the partner track $|$\texttt{dist}$|$ $< 0.02$~mm and an
opening angle $|$\texttt{dcot}$|$ $< 0.02$~\cite{ConversionRejection}.
\end{itemize}
%%%%%%%%%%%%%%%%%%%
%%%%%%%%%%%%%%%%%%%
%%%%%%%%%%%%%%%%%%%
\begin{table}[bthp]
\begin{center}
{\footnotesize
\begin{tabular}{|c|c|c|c|c|}
\hline
ID Variable & WP70 Barrel & WP70 Endcaps & WP95 Barrel & WP95 Endcaps  \\
\hline
$\sigma_{i\eta i\eta}$ & 0.01 & 0.03 & 0.01 & 0.03 \\
$\phi_{\mathrm{SC}} - \phi_{\mathrm{trk}}$ & 0.03 & 0.02 & 0.8 & 0.7 \\
$\eta_{\mathrm{SC}} - \eta_{\mathrm{trk}}$ & 0.004 & 0.005 & 0.007 & 0.01 \\
\hline
\end{tabular}
\caption[.]{\label{tab:EleID} Cut values for electron identification
variables for VBTF Working Point (WP) 70 (barrel and endcap), as used
for the tight electron selection, and VBTF Working Point (WP) 95
(barrel and endcap), as used in the loose electron selection.}}
\end{center}
\end{table}
%%%%%%%%%%%%%%%%%%%
%%%%%%%%%%%%%%%%%%%%%%%%%%%%%%%%%%%%%%%%%%%%%%%%%%%%%%%%%%%%%%%%%%%%%%%%%%%%
%%%%%%%%%%%%%%%%%%%%%%%%%%%%%%%%%%%%%%%%%%%%%%%%%%%%%%%%%%%%%%%%%%%%%%%%%%%%
\subsection{Muon selection}
\label{sec:muon_cuts}
Muon candidates are identified by two different 
algorithms~\cite{MUONPAS}: one proceeds from the inner tracker outwards, 
the other one starts from tracks measured in the muon chambers and matches 
and combines them with tracks reconstructed in the inner tracker. 
These selection criteria are summarized below:
%%%%%%%%%%%%%%%%%%%
\begin{itemize}
\item The muon candidate is reconstructed both as a global muon and
as a tracker muon.
\item The number of hits of the muon track in the silicon tracker has
to be $N_{\mathrm{Hits}} > 10$.
\item Number of pixel hits of the Tracker track $\ge 1$;
\item Number of muon hits of the Global track $\ge 2$;
\item Normalized $\chi^{2}$ of the Global track $< 10.0$.
\item Muon $p_{\mathrm{T}} > 25\,\mathrm{GeV}$.
\item Pseudorapidity $|\eta| < 2.1$.
\item Impact parameter: We cut on the absolute value of the impact
parameter calculated with respect to the beamspot. We require:
$d_0(\mathrm{Bsp}) < 0.02\,\mathrm{cm}.$
\item In order to make sure that the selected muon and the selected
jets come from the same hard interaction and not from pile up events,
we require that the $z$ coordinate of the PV of the event and the $z$
coordinate of the muon's inner track vertex lie within a distance of
less than 1~cm.
\item The selected muon candidates also have to be isolated.
We require the muon to be isolated simultaneusly in the
tracker, and in the electromagnetic and hadronic calorimeters.  
The muon candidate is required to have
$\mathrm{RelIso_{\mathrm{Comb}}} < 0.1$ in order to be considered
isolated.
We also veto the presence of a second loose lepton in the event.
\end{itemize}
%%%%%%%%%%%%%%%%%%%
%%%%%%%%%%%%%%%%%%%%%%%%%%%%%%%%%%%%%%%%%%%%%%%%%%%%%%%%%%%%%%%%%%%%%%%%%%%%
%%%%%%%%%%%%%%%%%%%%%%%%%%%%%%%%%%%%%%%%%%%%%%%%%%%%%%%%%%%%%%%%%%%%%%%%%%%%
\subsection{Jet selection}
\label{sec:firstStep_jets}
Jets are reconstructed with the anti-KT algorithm \cite{cacciari}, 
starting from the set of objects reconstructed by the particle 
flow \cite{pflow,CMS-PAS-JME-10-003,CMS-PAS-PFT-10-002}.
Jets are corrected such that the measured energy of the jet 
correctly reproduces the energy of the initial particle. 
The CMS standard L2 (relative) correction makes the jet response flat in $\eta$.
The standard L3 (absolute) correction brings the jet closer to the $\PT$ of 
a matched generated jet created using generator level input and a similar 
jet clustering algorithm.
The L2 and L3 corrections are calculated using Monte Carlo, and thus a 
L2L3 residual correction is applied that fixes the discrepancies between 
Monte Carlo and data~\cite{newjes-cms}.
In this analysis we use jets with measured (corrected) $\PT$  
greater than 30~$\gev$. 
We require $|\eta| < 2.4$ so that the jets fall within the
tracker acceptance.  
Jets are required to pass a set of loose identification
criteria; this requirement eliminates jets originating from or being seeded by
noisy channels in the calorimeter~\cite{Chatrchyan:2009hy}: 
%%%%%%%%%%%%%%
\begin{itemize}
\item Fraction of energy due to neutral hadrons $<$ 0.99.
\item Fraction of energy due to neutral EM deposits $<$ 0.99.
\item Number of constituents $>$ 1.
\item Number of charged hadrons candidates $>$ 0.
\item Fraction of energy due to charged hadrons candidates $>$ 0.
\item Fraction of energy due to charged EM deposits $<$ 0.99.
\end{itemize}
%%%%%%%%%%
All energy fractions are calculated from uncorrected jets.

\par
In order to account for electron and muon objects that
have been reconstructed as jets, we remove from the jet
collection any jet that falls within a
cone of radius $R= 0.3$ of a loose electron or a loose muon. 
This ``cleaning'' procedure is applied in order to ensure that the same
particle is not double counted as two different physics objects.
We require exactly two or three jets in the event.

%%%%%%%%%%%%%%%%%%%%%%%%%%%%%%%%%%%%%%%%%%%%%%%%%%%%%%
%%%%%%%%%%%%%%%%%%%%%%%%%%%%%%%%%%%%%%%%%%%%%%%%%%%%%%
\subsection{Missing transverse energy requirement}
\label{sec:MET}
An accurate MET measurement is essential for distinguishing
the $\Wo$ signal from QCD backgrounds. 
We use the MET estimate provided by the Particle Flow algorithm.
PF MET showed the best performance
among several MET algorithms~\cite{PFMET}.
The MET is computed as the vector sum of all PF objects.
A good agreement is found between the MET
distributions of $\Wln$ events in data and simulation~\cite{metPAS}.
The resolution for inclusive multi-jet samples and for
$\Wln$ events is also well reproduced by the simulation.  
A relative broadening of a few percent is observed in the data compared to MC,  
and has a negligible impact on the
extraction of the W yields~\cite{WZCMS:2010}.

We require the event to have missing transverse energy
MET in excess of 25~GeV in case of muon sample and 
30 GeV in case of electron sample.
We also require W transverse mass $ > 50\,\mathrm{GeV}$.
These cuts are designed to reduce the background
from QCD multijet production.
%%%%%%%%%%%%%%%%%%%%%%%%%%%%%%%%%%%%%%%%%%%%%%%%%%%%%%
%%%%%%%%%%%%%%%%%%%%%%%%%%%%%%%%%%%%%%%%%%%%%%%%%%%%%%
%%%%%%%%%%%%%%%%%%%%%%%
%%%%%%%%%%%%%%%%%%%%%%%%%%%%%%%%%%%%%%%%%%%%%%%%%%%%%%
%%%%%%%%%%%%%%%%%%%%%%%%%%%%%%%%%%%%%%%%%%%%%%%%%%%%%%
%%%%%%%%%%%%%%%%%%%%%%%%%%%%%%%%%%%%%%%%%%%%%%%%%%%%%%
%%%%%%%%%%%%%%%%%%%%%%%%%%%%%%%%%%%%%%%%%%%%%%%%%%%%%%
\section{Lepton reconstruction/selection efficiencies}
\label{sec:lepeff}
Lepton reconstruction and selection efficiencies are computed using
a ``Tag \& Probe'' technique on the Drell-Yan Z$\to\ell\ell$ events in
both data and MC~\cite{tagnprobe}.  The ratios of data versus MC efficiencies for
various steps of lepton reconstruction/selection are given below. If such
a ratio is found to be statistically inconsistent with unity, it is
applied as a scale factor correction to the MC samples.
%%%%%%%%%%%%%%%%%%
\subsection{Muon reconstruction/isolation corrections}
%%%%%%%%%%%
The efficiency scale factor for muon isolation is given in
Table~\ref{tab:muonEffsRecoToIso_ScaleFactors}.  The scale factor is
statistically consistent with unity throughout.  Results consistent
with these were also obtained by the Drell-Yan group using
Z$\rightarrow\mu\mu$ events.
Figure~\ref{fig:muonisoeffsf} shows this scale factor variation in $m_{jj}$.
%%%%%%%%%%%%%%%%%%%%%%%%%%%%%

%%%%%%%%%%%%%%%%%%%%%%%%%%%%%
\begin{table}[bthp]
\begin{center}
  \begin{tabular}{l l c | l c}
    \hline  \hline
    $p_T$ range (GeV) & $\eta$ range  &
    $\frac{\epsilon_{\rm{Data}}}{\epsilon_{\rm{MC}}}$ & 
    $\eta$ range  & $\frac{\epsilon_{\rm{Data}}}{\epsilon_{\rm{MC}}}$\\
    \hline  
    25--30 &    -2.1-- -1.5 &   1.00  & 1.5--2.1   & 1.00 \\
           &    -1.5-- -1.0 &   0.99  & 1.0--1.5   & 1.00 \\
           &    -1.0-- -0.5 &   1.00  & 0.5--1.0   & 1.00 \\
           &    -0.5--  0.0 &   1.00  & 0.0--0.5   & 1.00 \\
    \hline  
    30--35 &    -2.1-- -1.5 &   1.00  & 1.5--2.1   & 1.00 \\
           &    -1.5-- -1.0 &   0.99  & 1.0--1.5   & 1.00 \\
           &    -1.0-- -0.5 &   1.00  & 0.5--1.0   & 1.00 \\
           &    -0.5--  0.0 &   1.00  & 0.0--0.5   & 1.00 \\
    \hline  
    35--40 &    -2.1-- -1.5 &   0.99  & 1.5--2.1   & 1.00 \\
           &    -1.5-- -1.0 &   0.99  & 1.0--1.5   & 1.00 \\
           &    -1.0-- -0.5 &   1.00  & 0.5--1.0   & 1.00 \\
           &    -0.5--  0.0 &   1.00  & 0.0--0.5   & 1.00 \\
    \hline  
    40--45 &    -2.1-- -1.5 &   1.00  & 1.5--2.1   & 1.00 \\
           &    -1.5-- -1.0 &   0.99  & 1.0--1.5   & 1.00 \\
           &    -1.0-- -0.5 &   1.00  & 0.5--1.0   & 1.00 \\
           &    -0.5--  0.0 &   1.00  & 0.0--0.5   & 1.00 \\
    \hline  
    45--50 &    -2.1-- -1.5 &   1.00  & 1.5--2.1   & 1.00 \\
           &    -1.5-- -1.0 &   0.99  & 1.0--1.5   & 1.00 \\
           &    -1.0-- -0.5 &   1.00  & 0.5--1.0   & 1.00 \\
           &    -0.5--  0.0 &   1.00  & 0.0--0.5   & 1.00 \\
    \hline  
    50--200&    -2.1-- -1.5 &   1.00  & 1.5--2.1   & 1.00 \\
           &    -1.5-- -1.0 &   0.99  & 1.0--1.5   & 1.00 \\
           &    -1.0-- -0.5 &   1.00  & 0.5--1.0   & 1.00 \\
           &    -0.5--  0.0 &   1.00  & 0.0--0.5   & 1.00 \\
    \hline  \hline
  \end{tabular}
\end{center}
\caption{\label{tab:muonEffsRecoToIso_ScaleFactors}
Muon isolation efficiency data/MC scale factors. The statistical uncertainties were found
to be negligible, while the systematic uncertainty is $\sim$1\%.}
\end{table}
%%%%%%%%%%%%

%%%%%%%%%%%%
\subsection{Electron reconstruction scale factors}
The efficiency scale factor for electron reconstruction is given in 
Table~\ref{tab:eleEffsSCToReco_ScaleFactors}.
The scale factor is statistically consistent with unity throughout.
Figure~\ref{fig:electronRecoIDeffsf} shows this scale factor variation in $m_{jj}$.
%%%%%%%%%%%
%%%%%%%%%%%%%%%%%%%%%%%%%%%%%
\begin{table}[bthp]
\begin{center}
  \begin{tabular}{l l c | l c}
    \hline  \hline
    $p_T$ range (GeV) & $\eta$ range  &
    $\frac{\epsilon_{\rm{Data}}}{\epsilon_{\rm{MC}}}$ & 
    $\eta$ range  & $\frac{\epsilon_{\rm{Data}}}{\epsilon_{\rm{MC}}}$\\
    \hline  
    35--40 &    -2.5-- -1.5 & 1.0038 $\pm$ 0.0043 & 1.5--2.5 & 1.0135 $\pm$ 0.0040 \\
           &    -1.5-- 0.0  & 0.9987 $\pm$ 0.0016 & 0.0--1.5 & 0.9935 $\pm$ 0.0016 \\
    \hline  
    40--45 &    -2.5-- -1.5 & 1.0002 $\pm$ 0.0070 & 1.5--2.5 & 1.0111 $\pm$ 0.0034 \\
           &    -1.5-- 0.0  & 0.9951 $\pm$ 0.0012 & 0.0--1.5 & 0.9941 $\pm$ 0.0012 \\
    \hline  
    45--50 &    -2.5-- -1.5 & 1.0202 $\pm$ 0.0021 & 1.5--2.5 & 1.0170 $\pm$ 0.0080 \\
           &    -1.5-- 0.0  & 0.9941 $\pm$ 0.0014 & 0.0--1.5 & 0.9967 $\pm$ 0.0013 \\
    \hline  
    50--200&    -2.5-- -1.5 & 1.0287 $\pm$ 0.0049 & 1.5--2.5 & 1.0421 $\pm$ 0.0092 \\
           &    -1.5-- 0.0  & 0.9805 $\pm$ 0.0130 & 0.0--1.5 & 0.9989 $\pm$ 0.0018 \\
    \hline  \hline
  \end{tabular}
\end{center}
\caption{\label{tab:eleEffsSCToReco_ScaleFactors}
Electron reconstruction efficiency data/MC scale factors. 
The uncertainties are statistical only.
The systematic uncertainty is $\sim$1\%.
}
\end{table}
%%%%%%%%%%%%%%%%%%%%%%%%%%%%%
%%%%%%%%%%%%%%%%%%%%
\begin{figure}[h!t]
  {\centering
  \includegraphics[width=0.48\textwidth]{figs/effPlots/fig_eff_mu_RecoToIso_ScaleFactors.pdf}
   \caption{Luminosity weighted average efficiency scale factors (data/MC) for muon isolation.}
\label{fig:muonisoeffsf}}
\end{figure}
%%%%%%%%%%%%%%%%%%%%
%%%%%%%%%%%%%%%%%%%%
\begin{figure}[h!t]
  {\centering
  \subfigure[]{
  \includegraphics[width=0.48\textwidth]{figs/effPlots/fig_eff_ele_SCToReco_ScaleFactors.pdf}
   }
   \subfigure[]{
  \includegraphics[width=0.48\textwidth]{figs/effPlots/fig_eff_ele_RecoToID_ScaleFactors.pdf}
   }
   \caption{Luminosity weighted average efficiency scale factors (data/MC) for electron 
   reconstruction, \textit{i.e.}, super cluster $\to$ GSF electron (a) and electron ID (b).}
\label{fig:electronRecoIDeffsf}}
\end{figure}
%%%%%%%%%%%%%%%%%%%%
%%%%%%%%%%%%%%%%%%%%%%%%%%%%%
\subsection{Electron selection (isolation and ID) scale factors}
The efficiency scale factor for electron selection is given in 
Table~\ref{tab:eleEffsRecoToWP80_ScaleFactors}.
The scale factor is statistically consistent with unity in the ECAL barrel and in the endcaps 
within systematic uncertainties.
Figure~\ref{fig:electronRecoIDeffsf} shows this scale factor variation in $m_{jj}$.
%%%%%%%%%%%
%\verbatiminput{eleEffsRecoToWP80_ScaleFactors.txt}
%%%%%%%%%%%%%%%%%%%%%%%%%%%%%
\begin{table}[bthp]
\begin{center}
  \begin{tabular}{l l c | l c}
    \hline  \hline
    $p_T$ range (GeV) & $\eta$ range  &
    $\frac{\epsilon_{\rm{Data}}}{\epsilon_{\rm{MC}}}$ & 
    $\eta$ range  & $\frac{\epsilon_{\rm{Data}}}{\epsilon_{\rm{MC}}}$\\
    \hline  
    35--40 &    -2.5-- -1.5 & 0.9545 $\pm$ 0.0055 & 1.5--2.5 & 0.9607 $\pm$ 0.0053 \\
           &    -1.5-- 0.0  & 0.9910 $\pm$ 0.0024 & 0.0--1.5 & 0.9960 $\pm$ 0.0025 \\
    \hline  
    40--45 &    -2.5-- -1.5 & 0.9661 $\pm$ 0.1567 & 1.5--2.5 & 0.9648 $\pm$ 0.0024 \\
           &    -1.5-- 0.0  & 0.9946 $\pm$ 0.0019 & 0.0--1.5 & 0.9892 $\pm$ 0.0877 \\
    \hline  
    45--50 &    -2.5-- -1.5 & 0.9672 $\pm$ 0.0050 & 1.5--2.5 & 0.9729 $\pm$ 0.0051 \\
           &    -1.5-- 0.0  & 0.9938 $\pm$ 0.0773 & 0.0--1.5 & 0.9917 $\pm$ 0.0022 \\
    \hline  
    50--200&    -2.5-- -1.5 & 0.9836 $\pm$ 0.0066 & 1.5--2.5 & 0.9813 $\pm$ 0.0068 \\
           &    -1.5-- 0.0  & 0.9915 $\pm$ 0.0030 & 0.0--1.5 & 0.9857 $\pm$ 0.0030 \\
    \hline  \hline
  \end{tabular}
\end{center}
\caption{\label{tab:eleEffsRecoToWP80_ScaleFactors}
Electron selection efficiency data/MC scale factors. The uncertainties are statistical only.
The systematic uncertainty is $\sim$1\%.}
\end{table}
%%%%%%%%%%%%%%%%%%%%%%%%%%%%%%%%%%%%%%%%%%%%%%%%%%%%%%%%%%%%%%%%%%%%
%%%%%%%%%%%%%%%%%%%%%%%%%%%%%%%%%%%%%%%%%%%%%%%%%%%%%%%%%%%%%%%%%%%%
%%%%%%%%%%%%%%%%%%%%%%%%%%%%%%%%%%%%%%%%%%%%%%%%%%%%%%%%%%%%%%%%%%%%

\section{Trigger selection}
\label{sec:trigger}
%%%%%%%%%%%%%%%%%%%%%%%%%%%%%%%
%%%%%%%%%%%%%%%%%%%%%%%%%%%%%%%
\subsection{Run2010: Runs 136033--149442}
\begin{itemize}
\item
Muon data:\\
     Mu9 OR Mu11 OR Mu13 OR Mu15\_v* OR Mu17\_v* OR Mu24\_v*  
\item
Electron data:\\   
     Ele10\_* OR Ele15\_* OR Ele17\_* 
\end{itemize}
%%%%%%%%%%%%%%%%%%%%%%%%%%%%%%%
%%%%%%%%%%%%%%%%%%%%%%%%%%%%%%%
\subsection{Run2011A: Menus 5E32 (Runs: 160404--163869), 
1E33 (Runs:165088--166967), and 1.4E33 (Runs:167039--167913)}
\begin{itemize}
\item
Muon data:\\
     IsoMu17\_v* OR Mu30\_v* \\
Note: We really needed to OR in the nonisolated muon 
trigger as it recovers about half of the offline-isolated 
muons rejected by IsoMu, increasing the trigger efficiency 
by ~5\%. 
\item
Electron data:\\   
Ele27\_CaloIdVT\_CaloIsoT\_TrkIdT\_TrkIsoT\_v* \, \, \, \textcolor{red}{5E32 epoch}\\
Ele25\_WP80\_PFMT40\_v1 \, \, \, \textcolor{red}{1E33 epoch}\\
Ele27\_WP80\_PFMT50\_v* \, \, \, \textcolor{red}{1.4E33 epoch}
\end{itemize}
%%%%%%%%%%%%%%%%%%%%%%%%%%%%%%%
%%%%%%%%%%%%%%%%%%%%%%%%%%%%%%%
\subsection{Run2011A:Menu 2E33, Runs 170249--173198}
\begin{itemize}
\item
Muon data:\\
     (IsoMu17\_v13 OR IsoMu20\_v8 OR IsoMu24\_v8) \, \, OR \, \, (Mu30\_v7 OR Mu40\_v5)\\

Note: This epoch was complicated because Mu30, IsoMu17, 
and IsoMu20 were all prescaled for brief periods, so we 
could either break it down into sub-epochs or lump them 
together. We chose the latter because it is predominantly 
IsoMu17 and the sub-epoch lumi accounting is painful.   
\item
Electron data:\\   
Ele27\_WP80\_PFMT50\_v*
\end{itemize}
%%%%%%%%%%%%%%%%%%%%%%%%%%%%%%%
%%%%%%%%%%%%%%%%%%%%%%%%%%%%%%%
\subsection{Run2011A:Menu 3E33, Runs: 173236--173692}
\begin{itemize}
\item
Muon data:\\
HLT\_IsoMu20\_v9 OR HLT\_Mu40\_eta2p1\_v1
\item
Electron data:\\
Ele27\_WP80\_PFMT50\_v* 
\end{itemize}
%%%%%%%%%%%%%%%%%%%%%%%%%%%%%%%
%%%%%%%%%%%%%%%%%%%%%%%%%%%%%%%
\subsection{Run2011B: Menu 3E33, Runs: 175832--178380}
\begin{itemize}
\item
Muon data:\\
    (IsoMu30\_eta2p1\_v3  OR IsoMu24\_eta2p1\_v3  OR IsoMu24\_v9 OR IsoMu20\_v9) \\
     OR \\  
    (Mu40\_eta2p1\_v1  OR  HLT\_Mu40\_v6)
\item
Electron data:\\  
Ele27\_WP80\_PFMT50\_v* OR Ele27\_WP70\_PFMT50\_v*
\end{itemize}
%%%%%%%%%%%%%%%%%%%%%%%%%%%%%%%
%%%%%%%%%%%%%%%%%%%%%%%%%%%%%%%
\subsection{Run2011B: Menu 5E33, Runs: 178420--180252}
\begin{itemize}
\item
Muon data:\\
       (IsoMu30\_eta2p1\_v6 OR IsoMu24\_eta2p1\_v6 OR IsoMu24\_v12 OR \\
       IsoMu30\_eta2p1\_v7 OR IsoMu24\_eta2p1\_v7 OR IsoMu24\_v13) \\
       OR \\
      (Mu40\_eta2p1\_v4 OR  Mu40\_v9) \, \, \,
      \textcolor{red}{(v1.4, 178420-179889)} \\
       OR (Mu40\_eta2p1\_v5 OR  Mu40\_v10) \, \, \,
      \textcolor{red}{(v2.2, 179959--180252)}
\item
Electron data:\\
 Ele32\_WP70\_PFMT50\_v*
\end{itemize}
%%%%%%%%%%%%%%%%%%%%%%%%%%%%%%%%%%%%%%%%%%%%%%%%%%%%%%%%%%%
\section{Trigger efficiency computation}
\label{sec:trigeff}
%%%%%%%%%%%%%%%%%%%%%%%%%%%%%%%%%%%%%%%%%%%%%%%%%%%%%%%%%%%
The efficiency of the single lepton triggers are computed 
using tag \& probe technique from Z$\to\ell^+\ell^-$ events.
The procedure is straightforward and is described in detail 
in \cite{tagnprobe} and \cite{eleceff}.
%%%%%%%%%%%%%%%%%%%%%%%%%%%%%%%%%%%%%%%%%%%%%%%%%%%%%%%%%%
\subsection{Muon trigger efficiency table}
\label{sec:trigeff_mu}
The luminosity weighted average (LWA) trigger efficiency 
for single muon triggers in data is given 
in Table~\ref{tab:muonEffsIsoToHLT_data_LP_LWA}. 
The efficiency is slowly varying
with changes in lepton transverse momentum and rapidity. 
The efficiency is typically
about 90\%.
%%%%%%%%%%%
%%%%%%%%%%%%%%%%%%%%%%%%%%%%%
\begin{table}[bthp]
\begin{center}
  \begin{tabular}{l l c | l c}
    \hline  \hline
    $p_T$ range (GeV) & $\eta$ range  & $\epsilon_{\rm{Data}}$ & 
    $\eta$ range  & $\epsilon_{\rm{Data}}$\\
    \hline  
    25--30 &    -2.1-- -1.5 &   0.8490 $\pm$ 0.0032  & 1.5--2.1   & 0.8457 $\pm$ 0.0033 \\
           &    -1.5-- -1.0 &   0.8725 $\pm$ 0.0032  & 1.0--1.5   & 0.8628 $\pm$ 0.0032 \\
           &    -1.0-- -0.5 &   0.9057 $\pm$ 0.0026  & 0.5--1.0   & 0.8999 $\pm$ 0.0027 \\
           &    -0.5--  0.0 &   0.9211 $\pm$ 0.0022  & 0.0--0.5   & 0.9251 $\pm$ 0.0022 \\
    \hline  
    30--35 &    -2.1-- -1.5 &   0.8797 $\pm$ 0.0031  & 1.5--2.1   & 0.8768 $\pm$ 0.0031 \\
           &    -1.5-- -1.0 &   0.9136 $\pm$ 0.0030  & 1.0--1.5   & 0.9016 $\pm$ 0.0031 \\
           &    -1.0-- -0.5 &   0.9397 $\pm$ 0.0025  & 0.5--1.0   & 0.9387 $\pm$ 0.0025 \\
           &    -0.5--  0.0 &   0.9579 $\pm$ 0.0022  & 0.0--0.5   & 0.9556 $\pm$ 0.0021 \\
    \hline  
    35--40 &    -2.1-- -1.5 &   0.8816 $\pm$ 0.0027  & 1.5--2.1   & 0.8894 $\pm$ 0.0026 \\
           &    -1.5-- -1.0 &   0.9142 $\pm$ 0.0025  & 1.0--1.5   & 0.9008 $\pm$ 0.0026 \\
           &    -1.0-- -0.5 &   0.9385 $\pm$ 0.0022  & 0.5--1.0   & 0.9385 $\pm$ 0.0021 \\
           &    -0.5--  0.0 &   0.9571 $\pm$ 0.0019  & 0.0--0.5   & 0.9546 $\pm$ 0.0019 \\
    \hline  
    40--45 &    -2.1-- -1.5 &   0.8878 $\pm$ 0.0024  & 1.5--2.1   & 0.8902 $\pm$ 0.0024 \\
           &    -1.5-- -1.0 &   0.9221 $\pm$ 0.0021  & 1.0--1.5   & 0.9076 $\pm$ 0.0022 \\
           &    -1.0-- -0.5 &   0.9443 $\pm$ 0.0020  & 0.5--1.0   & 0.9457 $\pm$ 0.0019 \\
           &    -0.5--  0.0 &   0.9622 $\pm$ 0.0018  & 0.0--0.5   & 0.9617 $\pm$ 0.0018 \\
    \hline  
    45--50 &    -2.1-- -1.5 &   0.8922 $\pm$ 0.0029  & 1.5--2.1   & 0.8934 $\pm$ 0.0028 \\
           &    -1.5-- -1.0 &   0.9202 $\pm$ 0.0027  & 1.0--1.5   & 0.9069 $\pm$ 0.0027 \\
           &    -1.0-- -0.5 &   0.9458 $\pm$ 0.0024  & 0.5--1.0   & 0.9437 $\pm$ 0.0025 \\
           &    -0.5--  0.0 &   0.9625 $\pm$ 0.0023  & 0.0--0.5   & 0.9615 $\pm$ 0.0023 \\
    \hline  
    50--200&    -2.1-- -1.5 &   0.8920 $\pm$ 0.0031  & 1.5--2.1   & 0.8903 $\pm$ 0.0032 \\
           &    -1.5-- -1.0 &   0.9178 $\pm$ 0.0030  & 1.0--1.5   & 0.9041 $\pm$ 0.0030 \\
           &    -1.0-- -0.5 &   0.9419 $\pm$ 0.0027  & 0.5--1.0   & 0.9424 $\pm$ 0.0028 \\
           &    -0.5--  0.0 &   0.9606 $\pm$ 0.0026  & 0.0--0.5   & 0.9604 $\pm$ 0.0025 \\
    \hline  \hline
  \end{tabular}
\end{center}
\caption{\label{tab:muonEffsIsoToHLT_data_LP_LWA}
Muon trigger efficiency in data (luminosity 
weighted average). The uncertainties are statistical only.
The systematic uncertainty is $\sim$1\%.}
\end{table}
%%%%%%%%%%%%
%%%%%%%%%%%
\subsection{Electron trigger efficiency table}
\label{sec:trigeff_eleEle27}
The luminosity weighted average trigger efficiency for the 
electron triggers is shown in
Table~\ref{tab:eleEffsHLTEle}.  
The value is typically about 99\% in the barrel and 92--97\% in 
the endcaps, and is weakly dependent on the electron
$p_T$ and pseudorapidity.
The efficiency for the W transverse mass leg is shown in 
Table~\ref{tab:eleEffsHLTEleMT}.  
The average value is 91.74 $\pm$ 0.07\% with a large variation depending on 
whether the electron is in the barrel or endcaps.
Fortunately for us this is just an overall scaling effect in the 
$m_{jj}$ and $m_{\ell\nu jj}$ distributions.
The impact of this turnon on the dijet invariant mass or WW invariant mass templates
(used in Higgs analysis) is negligible as shown in 
Fig~\ref{fig:singleElehlteffMT}.
%%%%%%%%%%%
%%%%%%%%%%%%%%%%%%%%%%%%%%%%%
\begin{table}[bthp]
\begin{center}
  \begin{tabular}{l l c | l c}
    \hline  \hline
    $p_T$ range (GeV) & $\eta$ range  & $\epsilon_{\rm{Data}}$ & 
    $\eta$ range  & $\epsilon_{\rm{Data}}$\\
    \hline  
    35--40 &    -2.5-- -1.5 & 0.92 $\pm$ 0.01 & 1.5--2.5 & 0.91 $\pm$ 0.01 \\
           &    -1.5-- 0.0  & 0.99 $\pm$ 0.01 & 0.0--1.5 & 0.99 $\pm$ 0.01 \\
    \hline  
    40--45 &    -2.5-- -1.5 & 0.96 $\pm$ 0.01 & 1.5--2.5 & 0.96 $\pm$ 0.01 \\
           &    -1.5-- 0.0  & 0.99 $\pm$ 0.01 & 0.0--1.5 & 0.99 $\pm$ 0.01 \\
    \hline  
    45--50 &    -2.5-- -1.5 & 0.97 $\pm$ 0.01 & 1.5--2.5 & 0.97 $\pm$ 0.01 \\
           &    -1.5-- 0.0  & 0.99 $\pm$ 0.01 & 0.0--1.5 & 0.99 $\pm$ 0.01 \\
    \hline  
    50--200&    -2.5-- -1.5 & 0.93 $\pm$ 0.01 & 1.5--2.5 & 0.92 $\pm$ 0.01 \\
           &    -1.5-- 0.0  & 0.98 $\pm$ 0.01 & 0.0--1.5 & 0.98 $\pm$ 0.01 \\
    \hline  \hline
  \end{tabular}
\end{center}
\caption{\label{tab:eleEffsHLTEle}
Electron trigger efficiency in data (luminosity weighted average). 
The statistical uncertainties are negligible.
The quoted uncertainties are systematic.}
\end{table}
%%%%%%%%%%%%%%%%%%%%%%%%%%%%%
\begin{table}[bthp]
\begin{center}
  \begin{tabular}{l l c | l c}
    \hline  \hline
    Offline $m_T$ range (GeV) & electron $\eta$ range  & $\epsilon_{\rm{Data}}$ & 
    electron $\eta$ range  & $\epsilon_{\rm{Data}}$\\
    \hline  
     50--55 &	-2.5-- -1.5 & 0.3580 $\pm$ 0.0167 &	+1.5--2.5 & 0.3580 $\pm$ 0.0167 \\ 
            &	-1.5--0.0 & 0.7315 $\pm$ 0.0129  &	+0.0--1.5 & 0.7315 $\pm$ 0.0129 \\
    \hline
     55--60 &	-2.5-- -1.5 & 0.4796 $\pm$ 0.0165  &	+1.5--2.5 & 0.4796 $\pm$ 0.0165 \\ 
            &	-1.5--0.0 & 0.8151 $\pm$ 0.0112  &	+0.0--1.5 & 0.8151 $\pm$ 0.0112 \\ 
    \hline
     60--65 &	-2.5-- -1.5 & 0.6073 $\pm$ 0.0144  &	+1.5--2.5 & 0.6073 $\pm$ 0.0144 \\ 
            &	-1.5--0.0 & 0.9035 $\pm$ 0.0085  &	+0.0--1.5 & 0.9035 $\pm$ 0.0085 \\ 
    \hline
     65--70 &	-2.5-- -1.5 & 0.7473 $\pm$ 0.0100 &	+1.5--2.5 & 0.7473 $\pm$ 0.0100 \\  
            &	-1.5--0.0 & 0.9548 $\pm$ 0.0047  &	+0.0--1.5 & 0.9548 $\pm$ 0.0047  \\
    \hline
     70--75 &	-2.5-- -1.5 & 0.8256 $\pm$ 0.0069  &	+1.5--2.5 & 0.8256 $\pm$ 0.0069 \\ 
            &	-1.5--0.0 & 0.9756 $\pm$ 0.0036  &	+0.0--1.5 & 0.9756 $\pm$ 0.0036 \\ 
    \hline
     75--80 &	-2.5-- -1.5 & 0.8711 $\pm$ 0.0060  &	+1.5--2.5 & 0.8711 $\pm$ 0.0060 \\ 
            &	-1.5--0.0 & 0.9866 $\pm$ 0.0034  &	+0.0--1.5 & 0.9866 $\pm$ 0.0034 \\
    \hline
     80--85 &	-2.5-- -1.5 & 0.9047 $\pm$ 0.0059  &	+1.5--2.5 & 0.9047 $\pm$ 0.0059  \\
            &	-1.5--0.0 & 0.9934 $\pm$ 0.0034  &	+0.0--1.5 & 0.9934 $\pm$ 0.0034 \\ 
    \hline
     85--90 &	-2.5-- -1.5 & 0.9308 $\pm$ 0.0061  &	+1.5--2.5 & 0.9308 $\pm$ 0.0061 \\ 
            &	-1.5--0.0 & 0.9958 $\pm$ 0.0038  &	+0.0--1.5 & 0.9958 $\pm$ 0.0038  \\
    \hline
     90--95 &	-2.5-- -1.5 & 0.9415 $\pm$ 0.0068  &	+1.5--2.5 & 0.9415 $\pm$ 0.0068  \\
            &	-1.5--0.0 & 0.9975 $\pm$ 0.0046  &	+0.0--1.5 & 0.9975 $\pm$ 0.0046  \\
    \hline
     95--100 &	-2.5-- -1.5 & 0.9441 $\pm$ 0.0080 &	+1.5--2.5 & 0.9441 $\pm$ 0.0080  \\ 
             &	-1.5--0.0 & 0.9973 $\pm$ 0.0057  &	+0.0--1.5 & 0.9973 $\pm$ 0.0057  \\
    \hline
     100--110 &	-2.5-- -1.5 & 0.9358 $\pm$ 0.0074  &	+1.5--2.5 & 0.9358 $\pm$ 0.0074 \\ 
              &	-1.5--0.0 & 0.9980 $\pm$ 0.0059  &	+0.0--1.5 & 0.9980 $\pm$ 0.0059 \\ 
    \hline
     110--120 &	-2.5-- -1.5 & 0.9120 $\pm$ 0.0109  &	+1.5--2.5 & 0.9120 $\pm$ 0.0109 \\
              &	-1.5--0.0 & 0.9963 $\pm$ 0.0101  &	+0.0--1.5 & 0.9963 $\pm$ 0.0101  \\
    \hline
     120--140 &	-2.5-- -1.5 & 0.8721 $\pm$ 0.0117  &	+1.5--2.5 & 0.8721 $\pm$ 0.0117  \\
              &	-1.5--0.0 & 0.9950 $\pm$ 0.0123  &	+0.0--1.5 & 0.9950 $\pm$ 0.0123  \\
    \hline
     140--180 &	-2.5-- -1.5 & 0.8311 $\pm$ 0.0153  &	+1.5--2.5 & 0.8311 $\pm$ 0.0153  \\
              &	-1.5--0.0 & 0.9899 $\pm$ 0.0171  &	+0.0--1.5 & 0.9899 $\pm$ 0.0171  \\
    \hline
     180--240 &	-2.5-- -1.5 & 0.8011 $\pm$ 0.0266  &	+1.5--2.5 & 0.8011 $\pm$ 0.0266  \\
              &	-1.5--0.0 & 0.9915 $\pm$ 0.0290  &	+0.0--1.5 & 0.9915 $\pm$ 0.0290  \\
    \hline
     240--300 &	-2.5-- -1.5 & 0.8110 $\pm$ 0.0710  &	+1.5-+2.5 & 0.8110 $\pm$ 0.0710 \\ 
              &	-1.5--0.0 & 1.0000 $\pm$ 0.0829  &	+0.0--1.5 & 1.0000 $\pm$ 0.0829  \\
    \hline  \hline
  \end{tabular}
\end{center}
\caption{\label{tab:eleEffsHLTEleMT}
Efficiency for W transverse mass cut ($> 50$~GeV for most epochs) in HLT 
for single electron trigger in data (luminosity weighted average). 
The uncertainties are all inclusive.}
\end{table}
%%%%%%%%%%%%%%%%%%%%%%%%%%%%%
%%%%%%%%%%%%%%%%%%%%
\begin{figure}[h!t]
  {\centering
  \subfigure[]{
  \includegraphics[width=0.48\textwidth]{figs/effPlots/WMt50TriggerEfficiency.png}
  }   
\vspace*{1mm} \\
  \subfigure[]{
  \includegraphics[width=0.48\textwidth]{figs/effPlots/fig_eff_HLTWMT50_template.png}
   }
   \subfigure[]{
   \includegraphics[width=0.48\textwidth]{figs/effPlots/fig_eff_HLTWMT50_template4body.png}
   }
   \caption{Luminosity weighted average trigger efficiency in the 
   %first 200 pb${}^{-1}$ of 2011 electron data (single electron $HLT_Ele27$) 
     electron data for W transverse mass leg as a function 
   of $m_{jj}$ (a). 
   The effect of this efficiency correction on W+jets shape is shown for 
   $m_{jj}$ (b) and $m_{\ell\nu jj}$ (c) templates.}
\label{fig:singleElehlteffMT}}
\end{figure}
%%%%%%%%%%%%%%%%%%%%
%%%%%%%%%%%%%%%%%%%%%%%%%%%%%%%%%%%%%%%%%%%%%%%%%%%%%%%%%%%%%%%%%%%%
%%%%%%%%%%%%%%%%%%%%%%%%%%%%%%%%%%%%%%%
\subsection{Effect of trigger efficiency on shapes}
As shown in Figs.~\ref{fig:muonhlteff}-\ref{fig:singleElehlteff}, 
the overall trigger efficiency with respect to the analysis selection criteria
is uniform~\footnote{Except for the Electron+2Jet+MHT cross-object trigger, for 
which the efficiency has significant variations in the kinematic variables of 
interest. We are currently working to take these variations into account.} 
across various trigger epochs within the systematic uncertainty. 
Since the trigger efficiency is flat, it does not alter the 
di-jet mass shape, other than introducing a 
global factor that will be absorbed in the normalization of the fit.
Therefore, we have followed the strategy not to apply any trigger 
correction in Monte Carlo. 
We compute the systematic error due to the efficiency 
uncertainty by recomputing the envelope for the MC shape templates 
and propagating these templates to the $m_{jj}$ fit as described in 
a later section. 

%%%%%%%%%%%%%%%%%%%%
\begin{figure}[h!t]
  {\centering
  \subfigure[]{
  \includegraphics[width=0.48\textwidth]{figs/effPlots/fig_eff_HLTMu.pdf}
  }   
\vspace*{1mm} \\
  \subfigure[]{
  \includegraphics[width=0.48\textwidth]{figs/effPlots/fig_eff_HLTMu_template.pdf}
   }
   \subfigure[]{
   \includegraphics[width=0.48\textwidth]{figs/effPlots/fig_eff_HLTMu_template4body.pdf}
   }
   \caption{Luminosity weighted average trigger efficiency in the muon data as a function 
   of $m_{jj}$ (a). 
   The effect of this efficiency correction on W+jets shape is shown for 
   $m_{jj}$ (b) and $m_{\ell\nu jj}$ (c) templates.}
\label{fig:muonhlteff}}
\end{figure}
%%%%%%%%%%%%%%%%%%%%
%%%%%%%%%%%%%%%%%%%%
\begin{figure}[h!t]
  {\centering
  \subfigure[]{
  \includegraphics[width=0.48\textwidth]{figs/effPlots/fig_eff_HLTEle27_May10ReReco.pdf}
  }   
\vspace*{1mm} \\
  \subfigure[]{
  \includegraphics[width=0.48\textwidth]{figs/effPlots/fig_eff_HLTEle27_May10ReReco_template.pdf}
   }
   \subfigure[]{
   \includegraphics[width=0.48\textwidth]{figs/effPlots/fig_eff_HLTEle27_May10ReReco_template4body.pdf}
   }
   \caption{Luminosity weighted average trigger efficiency in the 
   %first 200 pb${}^{-1}$ of 2011 electron data (single electron $HLT_Ele27$) 
     electron data 
   as a function 
   of $m_{jj}$ (a). 
   The effect of this efficiency correction on W+jets shape is shown for 
   $m_{jj}$ (b) and $m_{\ell\nu jj}$ (c) templates.}
\label{fig:singleElehlteff}}
\end{figure}
%%%%%%%%%%%%%%%%%%%%

\par
To further crosscheck the impact of the trigger correction on the
$m_{jj}$ shape, we applied a single lepton trigger requirement in the MC 
simulation (chosen by an "OR" of different single muon or single 
electron paths present in the Summer11 Monte Carlo production)
and re-derived the shapes.
The rationale is that, although the trigger implemented 
for the MC does not perfectly mimic the actual trigger, it captures
the major kinematic effects on distributions. 
However, the specific trigger paths used to select events are 
distinctly different in data and Monte Carlo. 
The online isolation and ID definitions have evolved over the 
data taking epochs and the transition is not modeled in the simulation. 
Figure~\ref{fig:triggerEffect} shows the difference in shape for 
the $m_{jj}$ distribution in the W+Jets Monte Carlo, for muons (right) 
and electrons (left) with and without applying the closest trigger paths 
available (chosen by an "OR" of different single muon or single electron 
paths). The lower frame shows the difference in shape, after correcting 
for the different area of both distributions.
The two shapes are consistent with each other within the statistical errors
of a few percent.
%%%%%%%
\begin{figure}[h!] {\centering
\unitlength=0.33\linewidth
\includegraphics[width=0.48\textwidth]{figs/crosschecks/Trigger_Muons.png}
\includegraphics[width=0.48\textwidth]{figs/crosschecks/Trigger_Electrons.png}
\caption{Difference in shape for the $m_{jj}$ distribution in the W+jets 
Monte Carlo, for muons (right) and electrons (left) with and without applying 
the closest trigger paths available (chosen by an "OR" of different single 
muon or single electron paths). The lower frame shows the difference in shape, 
after correcting for the different area of both distributions.
The ratio of the two distributions is consistent with unity.} 
\label{fig:triggerEffect}}
\end{figure}
%%%%%%%
%%%%%%%%%%%%%%%%%%%%%%%%%%%%%%%%%%%%%%%
\subsection{Effect of trigger efficiency on yields}
The signal templates are corrected event-by-event for trigger 
efficiency. Since the efficiency is high (90--100\%) with weak  
dependence on lepton kinematics, the overall effect is small. 
The background yields remain unaffected by the absolute value 
of the trigger efficiency because the background yields are 
allowed to float in the fit.
%%%%%%%%%%%%%%%%%%%%%%%%%%%%%%%%%%%%%%%%%%%%%%%%%%%%%%%%%%%%%%%%%%%%
%%%%%%%%%%%%%%%%%%%%%%%%%%%%%%%%%%%%%%%%%%%%%%%%%%%%%%%%%%%%%%%%%%%%

\clearpage
\input{TriggerJetPtStudy}

\clearpage


%\clearpage
%\section{Conclusions}

We have used 2.875 pb$^{-1}$ of CMS data to measure the dijet mass spectrum
with the following eta cuts on the two leading jets: $\mid \Delta\eta \mid < 1.3$ and 
$\mid \eta \mid < 2.5$. The event with the largest observed dijet mass is at
 $m=2.05$ TeV. The measured dijet mass spectrum 
is in good agreement with a QCD prediction from PYTHIA and the full simulation of 
the CMS detector.


We have performed direct searches for high-mass dijet resonances in the
dijet mass distribution. The dijet mass data is well fit by a simple parameterization. There is no significant evidence for new particle production
in the data.  We set 95\% confidence level upper limits on the cross section for
a dijet resonance, applicable to any narrow resonance producing the following
specific pairs of partons:  $qq$, $qg$, and $gg$.  We have compared our cross section
limits with the expected cross sections from several existing models. 
We exclude at the 95\% confidence level new particles predicted 
in the following specific models: string resonances with mass less than 2.50~TeV, excited quarks with mass less than 1.58~TeV, 
and axigluons, colorons and $E_6$ diquarks in specific mass intervals, extending previously 
published limits on all models.


\section{Acknowledgments}

We would like to thank Can Kilic for his work on the string resonance
cross section, both at LHC and the Tevatron, exotica conveners Greg Landsberg and Chris Hill for their
careful reading of the note and suggestions for improvement, and the analysis review committee 
Bob Cousins, Valerie Halyo, and Jesus Marco for their effort and valuable suggestions.

%%%%%%%%%%%%%%%%%%%%%%%%%%%%%%%%%%%%%%%%%%%%%%%%%%%%%%%%%%%%%%%%%%%%%%%%%%

% >> acknowledgements (for journal papers)
% Please include the latest version from https://twiki.cern.ch/twiki/bin/viewauth/CMS/Internal/PubAcknow.
%\section*{Acknowledgements}
% ack-text

%% **DO NOT REMOVE BIBLIOGRAPHY**
\bibliography{auto_generated}   % will be created by the tdr script.

%%%% DO NOT ADD \end{document}!

