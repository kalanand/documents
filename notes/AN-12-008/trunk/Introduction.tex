
In the present iteration of the analysis our goal is to 
reconstruct either a linear discriminant or likelihood of 
minimum number of uncorrelated variables necessary to 
describe the event topology. 
This is because the individual cuts are highly 
inefficient. 
To keep the analysis as simple as possible we have intentionally 
not focused on exploiting the full potential of the 
multivariate analysis because this would involve 
inclusion of additional correlated variables.



\subsection{Training and validation method}
The Toolkit for Multivariate Data Analysis with ROOT 
(TMVA)~\cite{tmva}
is a toolkit to be implemented using ROOT that allows the user 
to carry out a significant number of multivariate analysis 
techniques such as boosted decision trees, neural networks, 
projected likelihood estimators and linear discriminants.

Using TMVA to implement the majority of these techniques 
consists of two phases: the Training Phase and the Testing Phase. 
To begin the Training Phase, the user must have samples where 
it is known which events are to be classified as signal and which 
are to be classified as background. 
Using this information, TMVA is trained to separate these classes 
as efficiently as possible. 
Next, the Testing Phase simply stands to implement the trained 
method of separation on a dataset where it is unknown which 
events are to be classified as signal or background. 

In the present analysis we try three classifiers to separate 
Higgs signal from the background: linear discriminant (LD),
likelihood, and boosted decision tree (BDT).
We have tried to use a complete set of minimum number of 
input variables necessary to describe the whole event topology, 
as will be described in detail in the next section.


The TMVA has some knobs to optimize the
performance of various classifiers, but we didn't tune these knobs. 
In the region of interest to us (90--95\% background
rejection points) the likelihood and BDT classifiers have almost 
identical performance. 
This is expected because the input variables are mostly uncorrelated, 
and there are no large correlations to be exploited.

We have also seen in the previous iterations of training that
with inclusion of additional (correlated) variables the BDT
performs better than likelihood. However, some of these
variables were correlated with the 4-body mass ($m_{WW} = m_{\ell\nu jj}$), 
so we decided to trim down to the minimum possible complete
set comprising of 10 variables. 


\subsection{Training and validation samples}
We perform a two-component multivariate analysis (MVA) to 
separate the Higgs signal from the dominant W+jets background.
We used the Fall11 MC samples for both Higgs signal and the 
W+jets background, as listed in Table~\ref{tab:MCsamples}.
%%%%%%%%%%%%%%%%%%%%%%%%%%
\begin{table}[htb]
  \begin{center}
    \begin{tabular}{l|l} 
      \hline
       Category & Sample name\\
      \hline
      Background & {\footnotesize /WJetsToLNu\_TuneZ2\_7TeV-madgraph-tauola/Fall11-PU\_S6\_START42\_V14B-v1/AODSIM}  \\
      \hline
      Signal &  {\footnotesize /GluGluToHToWWToLNuQQ\_M-*\_7TeV-powheg-pythia6/Fall11-PU\_S6\_START42\_}\\
       &  {\footnotesize V14B-v1/AODSIM} (with Higgs mass from 170 to 600~GeV) \\
      \hline
    \end{tabular}
  \end{center}
  \caption{Summary of Monte Carlo training and testing samples used in the analysis. 
    We use 50\% of the events in each category for training and the other 50\% for testing/validation.}
  \label{tab:MCsamples}
\end{table}
%%%%%%%%%%%%%%%%%%%%%%%%%%

\subsection{Training method}
We train the MVA separately for the following 12 Higgs mass 
points: 170, 180, 190, 200, 250, 300, 350, 400, 450, 500, 550, and 600~GeV.
Exactly 50\% of the events in each category are used for training 
the classifier and the other 50\% are used for testing/validation.
We train separately the events with two jets 
and three jets in the final state, because the background composition 
and kinematics are different for the two categories.
We also train separately for the two lepton species, \textit{i.e.}, 
muons and electrons.
In the following sections we will show the plots for muon channel 
only because the corresponding plots for the electron channel are 
very similar.


\subsection{Selection cuts}
Selection cuts are identical to the pre-selection described in 
CMS AN-2011/110. These include requirements on lepton ID, loose 
jet ID, second lepton veto in the event, and minimum missing transverse 
energy (25 GeV for muon and 35 GeV for electron events). 

