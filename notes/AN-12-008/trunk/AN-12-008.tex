% Customizable fields and text areas start with % >> below.
% Lines starting with the comment character (%) are normally removed before release outside the collaboration, but not those comments ending lines

% svn info. These are modified by svn at checkout time.
% The last version of these macros found before the maketitle will be the one on the front page,
% so only the main file is tracked.
% Do not edit by hand!
\RCS$Revision: 110744 $
\RCS$HeadURL: svn+ssh://svn.cern.ch/reps/tdr2/notes/AN-12-008/trunk/AN-12-008.tex $
\RCS$Id: AN-12-008.tex 110744 2012-03-15 00:17:51Z dudero $
%%%%%%%%%%%%% local definitions %%%%%%%%%%%%%%%%%%%%%
% This allows for switching between one column and two column (cms@external) layouts
% The widths should  be modified for your particular figures. You'll need additional copies if you have more than one standard figure size.
\newlength\cmsFigWidth
\ifthenelse{\boolean{cms@external}}{\setlength\cmsFigWidth{0.85\columnwidth}}{\setlength\cmsFigWidth{0.4\textwidth}}
\ifthenelse{\boolean{cms@external}}{\providecommand{\cmsLeft}{top}}{\providecommand{\cmsLeft}{left}}
\ifthenelse{\boolean{cms@external}}{\providecommand{\cmsRight}{bottom}}{\providecommand{\cmsRight}{right}}
%%%%%%%%%%%%%%%  Title page %%%%%%%%%%%%%%%%%%%%%%%%
\cmsNoteHeader{AN-12-008} % This is over-written in the CMS environment: useful as preprint no. for export versions
% >> Title: please make sure that the non-TeX equivalent is in PDFTitle below
\title{Multivariate optimization and background estimation for the Standard Model Higgs boson search in H$\to$WW$\to\ell\nu jj$ decay}

% >> Authors
%Author is always "The CMS Collaboration" for PAS and papers, so author, etc, below will be ignored in those cases
%For multiple affiliations, create an address entry for the combination


\address[ttu]{Texas Tech University, Lubbock, Texas, USA}
\address[fnal]{Fermi National Accelerator Laboratory, Batavia, Illinois, USA}
\address[tamu]{Texas A\&M University, College Station, Texas, USA}
\address[nebr]{University of Nebraska at Lincoln, Nebraska, USA}
\address[uerj]{Universidade do Estado do Rio de Janeiro (UERJ), Brazil}

\author[ttu]{Nural~Akchurin}
\author[fnal]{Jake~Anderson}
\author[fnal]{Andrew Beretvas}
\author[fnal]{Jeffrey~Berryhill}
\author[fnal]{Pushpa~Bhat}
\author[ttu]{Phil~Dudero}
\author[tamu]{Ricardo~Eusebi}
\author[fnal]{Dan~Green}
\author[nebr]{Pratima~Jindal}
\author[ttu]{Sung-Won Lee}
\author[fnal]{Kalanand~Mishra}
\author[tamu]{Ilya~Osipenkov}
\author[tamu]{Alexx~Perloff}
\author[uerj]{Andre~Sznajder}
\author[fnal]{Nhan~V.~Tran}
\author[fnal]{Fan~Yang}
\author[fnal]{Francisco~Yumiceva}


% >> Date
% The date is in yyyy/mm/dd format. Today has been
% redefined to match, but if the date needs to be fixed, please write it in this fashion.
% For papers and PAS, \today is taken as the date the head file (this one) was last modified according to svn: see the RCS Id string above.
% For the final version it is best to "touch" the head file to make sure it has the latest date.
\date{\today}

% >> Abstract
% Abstract processing:
% 1. **DO NOT use \include or \input** to include the abstract: our abstract extractor will not search through other files than this one.
% 2. **DO NOT use %**                  to comment out sections of the abstract: the extractor will still grab those lines (and they won't be comments any longer!).
% 3. **DO NOT use tex macros**         in the abstract: External TeX parsers used on the abstract don't understand them.
\abstract{
We present a study of the multi-variate optimization
for the Higgs boson search in the H$\to$WW$\to\ell\nu jj$
final state in the gluon fusion production mechanism. 
Using a complete set of mostly uncorrelated variables 
we optimize each Higgs mass points separately to distinguish 
between the Higgs signal and the dominant W+jets background.
We also develop a data-driven technique to derive 4-body 
$m_{\ell\nu jj}$ invariant mass shape for W+jets background. 
For this we use events in the upper and lower sidebands 
of the $m_{jj}$ distribution to extract the 4-body mass 
shape for events in the signal region  
65 GeV ~$< m_{jj} <$~95 GeV. 
Using this data-driven technique we are able to keep 
the systematic uncertainties (dominated by background 
shape systematics) to within a few percent. 
This analysis note complements the documentation of the 
main analysis in AN-11/110.
}

% >> PDF Metadata
% Do not comment out the following hypersetup lines (metadata). They will disappear in NODRAFT mode and are needed by CDS.
% Also: make sure that the values of the metadata items are sensible and are in plain text (no TeX! -- for \sqrt{s} use sqrt(s) -- this will show with extra quote marks in the draft version but is okay).

\hypersetup{%
pdfauthor={Jake Anderson, Phil Dudero, 
Pratima Jindal, Kalanand Mishra, Ilya Osipenkov, Fan Yang},%
pdftitle={Multivariate optimization and background estimation for the Standard Model Higgs boson search in HtoWWtoellnujj decay},%
pdfsubject={CMS},%
pdfkeywords={CMS, physics, software, computing}}

\maketitle %maketitle comes after all the front information has been supplied

% >> Text
%%%%%%%%%%%%%%%%%%%%%%%%%%%%%%%%  Begin text %%%%%%%%%%%%%%%%%%%%%%%%%%%%%
%% **DO NOT REMOVE THE BIBLIOGRAPHY** which is located before the appendix.
%% You can take the text between here and the bibiliography as an example which you should replace with the actual text of your document.
%% If you include other TeX files, be sure to use "\input{filename}" rather than "\input filename".
%% The latter works for you, but our parser looks for the braces and will break when uploading the document.
%%%%%%%%%%%%%%%
\tableofcontents
\newpage
\section{Introduction}
\section{Introduction}

Jets are the experimental signatures of quarks and gluons produced in high-energy processes such as hard scattering of partons in proton-proton collisions. The detailed understanding of both the jet energy scale and of the transverse momentum resolution is of crucial importance for many physics analyses, and it is an important component of the systematic uncertainty. This paper presents studies for the determination of the energy scale and resolution of jets, performed with the Compact Muon Solenoid (CMS) at the CERN Large Hadron Collider (LHC), on proton-proton collisions at $\sqrt{s}=7\TeV$, using a data sample corresponding to an integrated luminosity of $36\pbinv$.

The paper is organized as follows: Section~\ref{sec:detector} describes briefly the CMS detector, while Section~\ref{sec:jets} describes the jet reconstruction methods considered here. Sections~\ref{sec:data} and~\ref{sec:methods} present the data samples and the experimental techniques used for the various measurements. The jet energy calibration scheme is discussed in Section~\ref{sec:jec} and the jet transverse momentum resolution is presented in Section~\ref{sec:res}. 

\newpage
\section{Input variables used for MVA training}
\input{MVAInputs}
\newpage
\section{Correlation among the input variables}
\input{MVAcorrelations}
\newpage
\section{Output of the MVA training}
\input{MVAoutputs}
\newpage
\section{Performance: signal efficiency vs. background rejection rate}

The performance of the three classifiers -- linear discriminant, 
likelihood, and boosted decision tree -- 
are shown in Figs~\ref{fig:perf170mu}--\ref{fig:perf600mu}
for various Higgs mass points in terms of signal efficiency versus background rejection rate 
as the figure of merit.
The likelihood gives the best discrimination, therefore we use it 
for our analysis as will be described in detail in the next section.
%%%%%%%%%%%%%%%%%%%
\subsection{MVA performance: \texorpdfstring{$M_H$}{M(H)} = 170~GeV}
%%%%%%%%%%%%%%%%%%%
\begin{figure}[ht]
  \centering
  \includegraphics[width=0.48\textwidth]{figs/TMVA_170_nJ2_mu_rejBvsS}
  \includegraphics[width=0.48\textwidth]{figs/TMVA_170_nJ3_mu_rejBvsS}	
  \caption{\label{fig:perf170mu}Inputs to the multivariate discriminant for SM Higgs mass of 170~GeV for (a) 2-jet and (b) 3-jet $W\to\mu\nu$ category}
\end{figure}
%%%%%%%%%%%%%%%%%%%
\subsection{MVA performance: \texorpdfstring{$M_H$}{M(H)} = 180~GeV}
%%%%%%%%%%%%%%%%%%%
\begin{figure}[ht]
  \centering
  \includegraphics[width=0.48\textwidth]{figs/TMVA_180_nJ2_mu_rejBvsS}
  \includegraphics[width=0.48\textwidth]{figs/TMVA_180_nJ3_mu_rejBvsS}	
  \caption{\label{fig:perf180mu}Inputs to the multivariate discriminant for SM Higgs mass of 180~GeV for (a) 2-jet and (b) 3-jet $W\to\mu\nu$ category}
\end{figure}
%%%%%%%%%%%%%%%%%%%
\newpage
\subsection{MVA performance: \texorpdfstring{$M_H$}{M(H)} = 190~GeV}
%%%%%%%%%%%%%%%%%%%
\begin{figure}[ht]
  \centering
  \includegraphics[width=0.48\textwidth]{figs/TMVA_190_nJ2_mu_rejBvsS}
  \includegraphics[width=0.48\textwidth]{figs/TMVA_190_nJ3_mu_rejBvsS}	
  \caption{\label{fig:perf190mu}Inputs to the multivariate discriminant for SM Higgs mass of 190~GeV for (a) 2-jet and (b) 3-jet $W\to\mu\nu$ category}
\end{figure}
%%%%%%%%%%%%%%%%%%%
\subsection{MVA performance: \texorpdfstring{$M_H$}{M(H)} = 200~GeV}
%%%%%%%%%%%%%%%%%%%
\begin{figure}[ht]
  \centering
  \includegraphics[width=0.48\textwidth]{figs/TMVA_200_nJ2_mu_rejBvsS}
  \includegraphics[width=0.48\textwidth]{figs/TMVA_200_nJ3_mu_rejBvsS}	
  \caption{\label{fig:perf200mu}Inputs to the multivariate discriminant for SM Higgs mass of 200~GeV for (a) 2-jet and (b) 3-jet $W\to\mu\nu$ category}
\end{figure}
%%%%%%%%%%%%%%%%%%%
\newpage
\subsection{MVA performance: \texorpdfstring{$M_H$}{M(H)} = 250~GeV}
%%%%%%%%%%%%%%%%%%%
\begin{figure}[ht]
  \centering
  \includegraphics[width=0.48\textwidth]{figs/TMVA_250_nJ2_mu_rejBvsS}
  \includegraphics[width=0.48\textwidth]{figs/TMVA_250_nJ3_mu_rejBvsS}	
  \caption{\label{fig:perf250mu}Inputs to the multivariate discriminant for SM Higgs mass of 250~GeV for (a) 2-jet and (b) 3-jet $W\to\mu\nu$ category}
\end{figure}
%%%%%%%%%%%%%%%%%%%
\subsection{MVA performance: \texorpdfstring{$M_H$}{M(H)} = 300~GeV}
%%%%%%%%%%%%%%%%%%%
\begin{figure}[ht]
  \centering
  \includegraphics[width=0.48\textwidth]{figs/TMVA_300_nJ2_mu_rejBvsS}
  \includegraphics[width=0.48\textwidth]{figs/TMVA_300_nJ3_mu_rejBvsS}	
  \caption{\label{fig:perf300mu}Inputs to the multivariate discriminant for SM Higgs mass of 300~GeV for (a) 2-jet and (b) 3-jet $W\to\mu\nu$ category}
\end{figure}
%%%%%%%%%%%%%%%%%%%
\newpage
\subsection{MVA performance: \texorpdfstring{$M_H$}{M(H)} = 350~GeV}
%%%%%%%%%%%%%%%%%%%
\begin{figure}[ht]
  \centering
  \includegraphics[width=0.48\textwidth]{figs/TMVA_350_nJ2_mu_rejBvsS}
  \includegraphics[width=0.48\textwidth]{figs/TMVA_350_nJ3_mu_rejBvsS}	
  \caption{\label{fig:perf350mu}Inputs to the multivariate discriminant for SM Higgs mass of 350~GeV for (a) 2-jet and (b) 3-jet $W\to\mu\nu$ category}
\end{figure}
%%%%%%%%%%%%%%%%%%%
\subsection{MVA performance: \texorpdfstring{$M_H$}{M(H)} = 400~GeV}
%%%%%%%%%%%%%%%%%%%
\begin{figure}[ht]
  \centering
  \includegraphics[width=0.48\textwidth]{figs/TMVA_400_nJ2_mu_rejBvsS}
  \includegraphics[width=0.48\textwidth]{figs/TMVA_400_nJ3_mu_rejBvsS}	
  \caption{\label{fig:perf400mu}Inputs to the multivariate discriminant for SM Higgs mass of 400~GeV for (a) 2-jet and (b) 3-jet $W\to\mu\nu$ category}
\end{figure}
%%%%%%%%%%%%%%%%%%%
\newpage
\subsection{MVA performance: \texorpdfstring{$M_H$}{M(H)} = 450~GeV}
%%%%%%%%%%%%%%%%%%%
\begin{figure}[ht]
  \centering
  \includegraphics[width=0.48\textwidth]{figs/TMVA_450_nJ2_mu_rejBvsS}
  \includegraphics[width=0.48\textwidth]{figs/TMVA_450_nJ3_mu_rejBvsS}	
  \caption{\label{fig:perf450mu}Inputs to the multivariate discriminant for SM Higgs mass of 450~GeV for (a) 2-jet and (b) 3-jet $W\to\mu\nu$ category}
\end{figure}
%%%%%%%%%%%%%%%%%%%
\subsection{MVA performance: \texorpdfstring{$M_H$}{M(H)} = 500~GeV}
%%%%%%%%%%%%%%%%%%%
\begin{figure}[ht]
  \centering
  \includegraphics[width=0.48\textwidth]{figs/TMVA_500_nJ2_mu_rejBvsS}
  \includegraphics[width=0.48\textwidth]{figs/TMVA_500_nJ3_mu_rejBvsS}	
  \caption{\label{fig:perf500mu}Inputs to the multivariate discriminant for SM Higgs mass of 500~GeV for (a) 2-jet and (b) 3-jet $W\to\mu\nu$ category}
\end{figure}
%%%%%%%%%%%%%%%%%%%
\newpage
\subsection{MVA performance: \texorpdfstring{$M_H$}{M(H)} = 550~GeV}
%%%%%%%%%%%%%%%%%%%
\begin{figure}[ht]
  \centering
  \includegraphics[width=0.48\textwidth]{figs/TMVA_550_nJ2_mu_rejBvsS}
  \includegraphics[width=0.48\textwidth]{figs/TMVA_550_nJ3_mu_rejBvsS}	
  \caption{\label{fig:perf550mu}Inputs to the multivariate discriminant for SM Higgs mass of 550~GeV for (a) 2-jet and (b) 3-jet $W\to\mu\nu$ category}
\end{figure}
%%%%%%%%%%%%%%%%%%%
\subsection{MVA performance: \texorpdfstring{$M_H$}{M(H)} = 600~GeV}
%%%%%%%%%%%%%%%%%%%
\begin{figure}[ht]
  \centering
  \includegraphics[width=0.48\textwidth]{figs/TMVA_600_nJ2_mu_rejBvsS}
  \includegraphics[width=0.48\textwidth]{figs/TMVA_600_nJ3_mu_rejBvsS}	
  \caption{\label{fig:perf600mu}Inputs to the multivariate discriminant for SM Higgs mass of 600~GeV for (a) 2-jet and (b) 3-jet $W\to\mu\nu$ category}
\end{figure}
%%%%%%%%%%%%%%%%%%%

\newpage
\section{Ranking of the input variables}
\input{VariableRanking}

\newpage
\section{Optimal MVA cuts to maximize signal significance}

As we already introduced in previous section, we now train the MVA in
a Higgs mass window, which is a roughly fitted Gaussian mean
$\pm$ one sigma. The ten variables used in MVA are almost uncorrelated
and do not sculpt the four body mass distribution. We apply the MVA
output requirement just as a single cut as other cuts we are applying
for this analysis.

We optimized the likelihood on the basis of a scan over a range of MVA
cut values at discrete intervals. The figure of merit was chosen to be
the expected median exclusion limit achieved for the cut value based
on the same limit setup described in AN-12-024, and a reasonable
assumption on the systematics. The results in all cases revealed an
optimum cut value that minimized the expected limit with a resolution
better than a tenth of the signal strength excluded.

The data-MC comparison plots are shown in
Fig.~\ref{fig:mva:plots-mva2j170mu} to~\ref{fig:mva:plots-mva3j600mu}.
Agreement on MVA ouput between data and MC is good in the region of
phase space where the cuts are made. The MVA output for top pair
events in data and MC indicates that the agreement is good at high MVA
values where the Higgs sits.


\clearpage
%%%%%%%%%%%%%%%%%%%%%%%%%%%%%%%%%%%%%%%%%%%%%%%%%%%%%%%%%%%%%%%%%%%%%%%%%%%%
%%%%%%%%%%%%%%%%%%% mva2j170mu
\begin{figure}[!t]
  \centering
  \includegraphics[width=0.49\textwidth]{figs/cl-mva2j170mu-normal.pdf}
  \includegraphics[width=0.49\textwidth]{figs/cl-mva2j170mu-inTTbar.pdf}
  \caption{\label{fig:mva:plots-mva2j170mu} The data-MC comparisons
    after standard event selection (left) and top pair
    selection (right) for the working point: mva2j170mu.}
\end{figure}

%%%%%%%%%%%%%%%%%%% mva2j180mu
\begin{figure}[!t]
  \centering
  \includegraphics[width=0.49\textwidth]{figs/cl-mva2j180mu-normal.pdf}
  \includegraphics[width=0.49\textwidth]{figs/cl-mva2j180mu-inTTbar.pdf}
  \caption{\label{fig:mva:plots-mva2j180mu} The data-MC comparisons
    after standard event selection (left) and top pair
    selection (right) for the working point: mva2j180mu.}
\end{figure}

%%%%%%%%%%%%%%%%%%% mva2j190mu
\begin{figure}[!t]
  \centering
  \includegraphics[width=0.49\textwidth]{figs/cl-mva2j190mu-normal.pdf}
  \includegraphics[width=0.49\textwidth]{figs/cl-mva2j190mu-inTTbar.pdf}
  \caption{\label{fig:mva:plots-mva2j190mu} The data-MC comparisons
    after standard event selection (left) and top pair
    selection (right) for the working point: mva2j190mu.}
\end{figure}

%%%%%%%%%%%%%%%%%%% mva2j200mu
\begin{figure}[!t]
  \centering
  \includegraphics[width=0.49\textwidth]{figs/cl-mva2j200mu-normal.pdf}
  \includegraphics[width=0.49\textwidth]{figs/cl-mva2j200mu-inTTbar.pdf}
  \caption{\label{fig:mva:plots-mva2j200mu} The data-MC comparisons
    after standard event selection (left) and top pair
    selection (right) for the working point: mva2j200mu.}
\end{figure}

%%%%%%%%%%%%%%%%%%% mva2j250mu
\begin{figure}[!t]
  \centering
  \includegraphics[width=0.49\textwidth]{figs/cl-mva2j250mu-normal.pdf}
  \includegraphics[width=0.49\textwidth]{figs/cl-mva2j250mu-inTTbar.pdf}
  \caption{\label{fig:mva:plots-mva2j250mu} The data-MC comparisons
    after standard event selection (left) and top pair
    selection (right) for the working point: mva2j250mu.}
\end{figure}

%%%%%%%%%%%%%%%%%%% mva2j300mu
\begin{figure}[!t]
  \centering
  \includegraphics[width=0.49\textwidth]{figs/cl-mva2j300mu-normal.pdf}
  \includegraphics[width=0.49\textwidth]{figs/cl-mva2j300mu-inTTbar.pdf}
  \caption{\label{fig:mva:plots-mva2j300mu} The data-MC comparisons
    after standard event selection (left) and top pair
    selection (right) for the working point: mva2j300mu.}
\end{figure}

%%%%%%%%%%%%%%%%%%% mva2j350mu
\begin{figure}[!t]
  \centering
  \includegraphics[width=0.49\textwidth]{figs/cl-mva2j350mu-normal.pdf}
  \includegraphics[width=0.49\textwidth]{figs/cl-mva2j350mu-inTTbar.pdf}
  \caption{\label{fig:mva:plots-mva2j350mu} The data-MC comparisons
    after standard event selection (left) and top pair
    selection (right) for the working point: mva2j350mu.}
\end{figure}

%%%%%%%%%%%%%%%%%%% mva2j400mu
\begin{figure}[!t]
  \centering
  \includegraphics[width=0.49\textwidth]{figs/cl-mva2j400mu-normal.pdf}
  \includegraphics[width=0.49\textwidth]{figs/cl-mva2j400mu-inTTbar.pdf}
  \caption{\label{fig:mva:plots-mva2j400mu} The data-MC comparisons
    after standard event selection (left) and top pair
    selection (right) for the working point: mva2j400mu.}
\end{figure}

%%%%%%%%%%%%%%%%%%% mva2j450mu
\begin{figure}[!t]
  \centering
  \includegraphics[width=0.49\textwidth]{figs/cl-mva2j450mu-normal.pdf}
  \includegraphics[width=0.49\textwidth]{figs/cl-mva2j450mu-inTTbar.pdf}
  \caption{\label{fig:mva:plots-mva2j450mu} The data-MC comparisons
    after standard event selection (left) and top pair
    selection (right) for the working point: mva2j450mu.}
\end{figure}

%%%%%%%%%%%%%%%%%%% mva2j500mu
\begin{figure}[!t]
  \centering
  \includegraphics[width=0.49\textwidth]{figs/cl-mva2j500mu-normal.pdf}
  \includegraphics[width=0.49\textwidth]{figs/cl-mva2j500mu-inTTbar.pdf}
  \caption{\label{fig:mva:plots-mva2j500mu} The data-MC comparisons
    after standard event selection (left) and top pair
    selection (right) for the working point: mva2j500mu.}
\end{figure}

%%%%%%%%%%%%%%%%%%% mva2j550mu
\begin{figure}[!t]
  \centering
  \includegraphics[width=0.49\textwidth]{figs/cl-mva2j550mu-normal.pdf}
  \includegraphics[width=0.49\textwidth]{figs/cl-mva2j550mu-inTTbar.pdf}
  \caption{\label{fig:mva:plots-mva2j550mu} The data-MC comparisons
    after standard event selection (left) and top pair
    selection (right) for the working point: mva2j550mu.}
\end{figure}

%%%%%%%%%%%%%%%%%%% mva2j600mu
\begin{figure}[!t]
  \centering
  \includegraphics[width=0.49\textwidth]{figs/cl-mva2j600mu-normal.pdf}
  \includegraphics[width=0.49\textwidth]{figs/cl-mva2j600mu-inTTbar.pdf}
  \caption{\label{fig:mva:plots-mva2j600mu} The data-MC comparisons
    after standard event selection (left) and top pair
    selection (right) for the working point: mva2j600mu.}
\end{figure}

\clearpage
%%%%%%%%%%%%%%%%%%%%%%%%%%%%%%%%%%%%%%%%%%%%%%%%%%%%%%%%%%%%%%%%%%%%%%%%%%%%
%%%%%%%%%%%%%%%%%%% mva3j170mu
\begin{figure}[!t]
  \centering
  \includegraphics[width=0.49\textwidth]{figs/cl-mva3j170mu-normal.pdf}
  \includegraphics[width=0.49\textwidth]{figs/cl-mva3j170mu-inTTbar.pdf}
  \caption{\label{fig:mva:plots-mva3j170mu} The data-MC comparisons
    after standard event selection (left) and top pair
    selection (right) for the working point: mva3j170mu.}
\end{figure}

%%%%%%%%%%%%%%%%%%% mva3j180mu
\begin{figure}[!t]
  \centering
  \includegraphics[width=0.49\textwidth]{figs/cl-mva3j180mu-normal.pdf}
  \includegraphics[width=0.49\textwidth]{figs/cl-mva3j180mu-inTTbar.pdf}
  \caption{\label{fig:mva:plots-mva3j180mu} The data-MC comparisons
    after standard event selection (left) and top pair
    selection (right) for the working point: mva3j180mu.}
\end{figure}

%%%%%%%%%%%%%%%%%%% mva3j190mu
\begin{figure}[!t]
  \centering
  \includegraphics[width=0.49\textwidth]{figs/cl-mva3j190mu-normal.pdf}
  \includegraphics[width=0.49\textwidth]{figs/cl-mva3j190mu-inTTbar.pdf}
  \caption{\label{fig:mva:plots-mva3j190mu} The data-MC comparisons
    after standard event selection (left) and top pair
    selection (right) for the working point: mva3j190mu.}
\end{figure}

%%%%%%%%%%%%%%%%%%% mva3j200mu
\begin{figure}[!t]
  \centering
  \includegraphics[width=0.49\textwidth]{figs/cl-mva3j200mu-normal.pdf}
  \includegraphics[width=0.49\textwidth]{figs/cl-mva3j200mu-inTTbar.pdf}
  \caption{\label{fig:mva:plots-mva3j200mu} The data-MC comparisons
    after standard event selection (left) and top pair
    selection (right) for the working point: mva3j200mu.}
\end{figure}

%%%%%%%%%%%%%%%%%%% mva3j250mu
\begin{figure}[!t]
  \centering
  \includegraphics[width=0.49\textwidth]{figs/cl-mva3j250mu-normal.pdf}
  \includegraphics[width=0.49\textwidth]{figs/cl-mva3j250mu-inTTbar.pdf}
  \caption{\label{fig:mva:plots-mva3j250mu} The data-MC comparisons
    after standard event selection (left) and top pair
    selection (right) for the working point: mva3j250mu.}
\end{figure}

%%%%%%%%%%%%%%%%%%% mva3j300mu
\begin{figure}[!t]
  \centering
  \includegraphics[width=0.49\textwidth]{figs/cl-mva3j300mu-normal.pdf}
  \includegraphics[width=0.49\textwidth]{figs/cl-mva3j300mu-inTTbar.pdf}
  \caption{\label{fig:mva:plots-mva3j300mu} The data-MC comparisons
    after standard event selection (left) and top pair
    selection (right) for the working point: mva3j300mu.}
\end{figure}

%%%%%%%%%%%%%%%%%%% mva3j350mu
\begin{figure}[!t]
  \centering
  \includegraphics[width=0.49\textwidth]{figs/cl-mva3j350mu-normal.pdf}
  \includegraphics[width=0.49\textwidth]{figs/cl-mva3j350mu-inTTbar.pdf}
  \caption{\label{fig:mva:plots-mva3j350mu} The data-MC comparisons
    after standard event selection (left) and top pair
    selection (right) for the working point: mva3j350mu.}
\end{figure}

%%%%%%%%%%%%%%%%%%% mva3j400mu
\begin{figure}[!t]
  \centering
  \includegraphics[width=0.49\textwidth]{figs/cl-mva3j400mu-normal.pdf}
  \includegraphics[width=0.49\textwidth]{figs/cl-mva3j400mu-inTTbar.pdf}
  \caption{\label{fig:mva:plots-mva3j400mu} The data-MC comparisons
    after standard event selection (left) and top pair
    selection (right) for the working point: mva3j400mu.}
\end{figure}

%%%%%%%%%%%%%%%%%%% mva3j450mu
\begin{figure}[!t]
  \centering
  \includegraphics[width=0.49\textwidth]{figs/cl-mva3j450mu-normal.pdf}
  \includegraphics[width=0.49\textwidth]{figs/cl-mva3j450mu-inTTbar.pdf}
  \caption{\label{fig:mva:plots-mva3j450mu} The data-MC comparisons
    after standard event selection (left) and top pair
    selection (right) for the working point: mva3j450mu.}
\end{figure}

%%%%%%%%%%%%%%%%%%% mva3j500mu
\begin{figure}[!t]
  \centering
  \includegraphics[width=0.49\textwidth]{figs/cl-mva3j500mu-normal.pdf}
  \includegraphics[width=0.49\textwidth]{figs/cl-mva3j500mu-inTTbar.pdf}
  \caption{\label{fig:mva:plots-mva3j500mu} The data-MC comparisons
    after standard event selection (left) and top pair
    selection (right) for the working point: mva3j500mu.}
\end{figure}

%%%%%%%%%%%%%%%%%%% mva3j550mu
\begin{figure}[!t]
  \centering
  \includegraphics[width=0.49\textwidth]{figs/cl-mva3j550mu-normal.pdf}
  \includegraphics[width=0.49\textwidth]{figs/cl-mva3j550mu-inTTbar.pdf}
  \caption{\label{fig:mva:plots-mva3j550mu} The data-MC comparisons
    after standard event selection (left) and top pair
    selection (right) for the working point: mva3j550mu.}
\end{figure}

%%%%%%%%%%%%%%%%%%% mva3j600mu
\begin{figure}[!t]
  \centering
  \includegraphics[width=0.49\textwidth]{figs/cl-mva3j600mu-normal.pdf}
  \includegraphics[width=0.49\textwidth]{figs/cl-mva3j600mu-inTTbar.pdf}
  \caption{\label{fig:mva:plots-mva3j600mu} The data-MC comparisons
    after standard event selection (left) and top pair
    selection (right) for the working point: mva3j600mu.}
\end{figure}

\clearpage
%%%%%%%%%%%%%%%%%%%%%%%%%%%%%%%%%%%%%%%%%%%%%%%%%%%%%%%%%%%%%%%%%%%%%%%%%%%%
%%%%%%%%%%%%%%%%%%% mva2j170el
\begin{figure}[!t]
  \centering
  \includegraphics[width=0.49\textwidth]{figs/cl-mva2j170el-normal.pdf}
  \includegraphics[width=0.49\textwidth]{figs/cl-mva2j170el-inTTbar.pdf}
  \caption{\label{fig:mva:plots-mva2j170el} The data-MC comparisons
    after standard event selection (left) and top pair
    selection (right) for the working point: mva2j170el.}
\end{figure}

%%%%%%%%%%%%%%%%%%% mva2j180el
\begin{figure}[!t]
  \centering
  \includegraphics[width=0.49\textwidth]{figs/cl-mva2j180el-normal.pdf}
  \includegraphics[width=0.49\textwidth]{figs/cl-mva2j180el-inTTbar.pdf}
  \caption{\label{fig:mva:plots-mva2j180el} The data-MC comparisons
    after standard event selection (left) and top pair
    selection (right) for the working point: mva2j180el.}
\end{figure}

%%%%%%%%%%%%%%%%%%% mva2j190el
\begin{figure}[!t]
  \centering
  \includegraphics[width=0.49\textwidth]{figs/cl-mva2j190el-normal.pdf}
  \includegraphics[width=0.49\textwidth]{figs/cl-mva2j190el-inTTbar.pdf}
  \caption{\label{fig:mva:plots-mva2j190el} The data-MC comparisons
    after standard event selection (left) and top pair
    selection (right) for the working point: mva2j190el.}
\end{figure}

%%%%%%%%%%%%%%%%%%% mva2j200el
\begin{figure}[!t]
  \centering
  \includegraphics[width=0.49\textwidth]{figs/cl-mva2j200el-normal.pdf}
  \includegraphics[width=0.49\textwidth]{figs/cl-mva2j200el-inTTbar.pdf}
  \caption{\label{fig:mva:plots-mva2j200el} The data-MC comparisons
    after standard event selection (left) and top pair
    selection (right) for the working point: mva2j200el.}
\end{figure}

%%%%%%%%%%%%%%%%%%% mva2j250el
\begin{figure}[!t]
  \centering
  \includegraphics[width=0.49\textwidth]{figs/cl-mva2j250el-normal.pdf}
  \includegraphics[width=0.49\textwidth]{figs/cl-mva2j250el-inTTbar.pdf}
  \caption{\label{fig:mva:plots-mva2j250el} The data-MC comparisons
    after standard event selection (left) and top pair
    selection (right) for the working point: mva2j250el.}
\end{figure}

%%%%%%%%%%%%%%%%%%% mva2j300el
\begin{figure}[!t]
  \centering
  \includegraphics[width=0.49\textwidth]{figs/cl-mva2j300el-normal.pdf}
  \includegraphics[width=0.49\textwidth]{figs/cl-mva2j300el-inTTbar.pdf}
  \caption{\label{fig:mva:plots-mva2j300el} The data-MC comparisons
    after standard event selection (left) and top pair
    selection (right) for the working point: mva2j300el.}
\end{figure}

%%%%%%%%%%%%%%%%%%% mva2j350el
\begin{figure}[!t]
  \centering
  \includegraphics[width=0.49\textwidth]{figs/cl-mva2j350el-normal.pdf}
  \includegraphics[width=0.49\textwidth]{figs/cl-mva2j350el-inTTbar.pdf}
  \caption{\label{fig:mva:plots-mva2j350el} The data-MC comparisons
    after standard event selection (left) and top pair
    selection (right) for the working point: mva2j350el.}
\end{figure}

%%%%%%%%%%%%%%%%%%% mva2j400el
\begin{figure}[!t]
  \centering
  \includegraphics[width=0.49\textwidth]{figs/cl-mva2j400el-normal.pdf}
  \includegraphics[width=0.49\textwidth]{figs/cl-mva2j400el-inTTbar.pdf}
  \caption{\label{fig:mva:plots-mva2j400el} The data-MC comparisons
    after standard event selection (left) and top pair
    selection (right) for the working point: mva2j400el.}
\end{figure}

%%%%%%%%%%%%%%%%%%% mva2j450el
\begin{figure}[!t]
  \centering
  \includegraphics[width=0.49\textwidth]{figs/cl-mva2j450el-normal.pdf}
  \includegraphics[width=0.49\textwidth]{figs/cl-mva2j450el-inTTbar.pdf}
  \caption{\label{fig:mva:plots-mva2j450el} The data-MC comparisons
    after standard event selection (left) and top pair
    selection (right) for the working point: mva2j450el.}
\end{figure}

%%%%%%%%%%%%%%%%%%% mva2j500el
\begin{figure}[!t]
  \centering
  \includegraphics[width=0.49\textwidth]{figs/cl-mva2j500el-normal.pdf}
  \includegraphics[width=0.49\textwidth]{figs/cl-mva2j500el-inTTbar.pdf}
  \caption{\label{fig:mva:plots-mva2j500el} The data-MC comparisons
    after standard event selection (left) and top pair
    selection (right) for the working point: mva2j500el.}
\end{figure}

%%%%%%%%%%%%%%%%%%% mva2j550el
\begin{figure}[!t]
  \centering
  \includegraphics[width=0.49\textwidth]{figs/cl-mva2j550el-normal.pdf}
  \includegraphics[width=0.49\textwidth]{figs/cl-mva2j550el-inTTbar.pdf}
  \caption{\label{fig:mva:plots-mva2j550el} The data-MC comparisons
    after standard event selection (left) and top pair
    selection (right) for the working point: mva2j550el.}
\end{figure}

%%%%%%%%%%%%%%%%%%% mva2j600el
\begin{figure}[!t]
  \centering
  \includegraphics[width=0.49\textwidth]{figs/cl-mva2j600el-normal.pdf}
  \includegraphics[width=0.49\textwidth]{figs/cl-mva2j600el-inTTbar.pdf}
  \caption{\label{fig:mva:plots-mva2j600el} The data-MC comparisons
    after standard event selection (left) and top pair
    selection (right) for the working point: mva2j600el.}
\end{figure}

\clearpage
%%%%%%%%%%%%%%%%%%%%%%%%%%%%%%%%%%%%%%%%%%%%%%%%%%%%%%%%%%%%%%%%%%%%%%%%%%%%
%%%%%%%%%%%%%%%%%%% mva3j170el
\begin{figure}[!t]
  \centering
  \includegraphics[width=0.49\textwidth]{figs/cl-mva3j170el-normal.pdf}
  \includegraphics[width=0.49\textwidth]{figs/cl-mva3j170el-inTTbar.pdf}
  \caption{\label{fig:mva:plots-mva3j170el} The data-MC comparisons
    after standard event selection (left) and top pair
    selection (right) for the working point: mva3j170el.}
\end{figure}

%%%%%%%%%%%%%%%%%%% mva3j180el
\begin{figure}[!t]
  \centering
  \includegraphics[width=0.49\textwidth]{figs/cl-mva3j180el-normal.pdf}
  \includegraphics[width=0.49\textwidth]{figs/cl-mva3j180el-inTTbar.pdf}
  \caption{\label{fig:mva:plots-mva3j180el} The data-MC comparisons
    after standard event selection (left) and top pair
    selection (right) for the working point: mva3j180el.}
\end{figure}

%%%%%%%%%%%%%%%%%%% mva3j190el
\begin{figure}[!t]
  \centering
  \includegraphics[width=0.49\textwidth]{figs/cl-mva3j190el-normal.pdf}
  \includegraphics[width=0.49\textwidth]{figs/cl-mva3j190el-inTTbar.pdf}
  \caption{\label{fig:mva:plots-mva3j190el} The data-MC comparisons
    after standard event selection (left) and top pair
    selection (right) for the working point: mva3j190el.}
\end{figure}

%%%%%%%%%%%%%%%%%%% mva3j200el
\begin{figure}[!t]
  \centering
  \includegraphics[width=0.49\textwidth]{figs/cl-mva3j200el-normal.pdf}
  \includegraphics[width=0.49\textwidth]{figs/cl-mva3j200el-inTTbar.pdf}
  \caption{\label{fig:mva:plots-mva3j200el} The data-MC comparisons
    after standard event selection (left) and top pair
    selection (right) for the working point: mva3j200el.}
\end{figure}

%%%%%%%%%%%%%%%%%%% mva3j250el
\begin{figure}[!t]
  \centering
  \includegraphics[width=0.49\textwidth]{figs/cl-mva3j250el-normal.pdf}
  \includegraphics[width=0.49\textwidth]{figs/cl-mva3j250el-inTTbar.pdf}
  \caption{\label{fig:mva:plots-mva3j250el} The data-MC comparisons
    after standard event selection (left) and top pair
    selection (right) for the working point: mva3j250el.}
\end{figure}

%%%%%%%%%%%%%%%%%%% mva3j300el
\begin{figure}[!t]
  \centering
  \includegraphics[width=0.49\textwidth]{figs/cl-mva3j300el-normal.pdf}
  \includegraphics[width=0.49\textwidth]{figs/cl-mva3j300el-inTTbar.pdf}
  \caption{\label{fig:mva:plots-mva3j300el} The data-MC comparisons
    after standard event selection (left) and top pair
    selection (right) for the working point: mva3j300el.}
\end{figure}

%%%%%%%%%%%%%%%%%%% mva3j350el
\begin{figure}[!t]
  \centering
  \includegraphics[width=0.49\textwidth]{figs/cl-mva3j350el-normal.pdf}
  \includegraphics[width=0.49\textwidth]{figs/cl-mva3j350el-inTTbar.pdf}
  \caption{\label{fig:mva:plots-mva3j350el} The data-MC comparisons
    after standard event selection (left) and top pair
    selection (right) for the working point: mva3j350el.}
\end{figure}

%%%%%%%%%%%%%%%%%%% mva3j400el
\begin{figure}[!t]
  \centering
  \includegraphics[width=0.49\textwidth]{figs/cl-mva3j400el-normal.pdf}
  \includegraphics[width=0.49\textwidth]{figs/cl-mva3j400el-inTTbar.pdf}
  \caption{\label{fig:mva:plots-mva3j400el} The data-MC comparisons
    after standard event selection (left) and top pair
    selection (right) for the working point: mva3j400el.}
\end{figure}

%%%%%%%%%%%%%%%%%%% mva3j450el
\begin{figure}[!t]
  \centering
  \includegraphics[width=0.49\textwidth]{figs/cl-mva3j450el-normal.pdf}
  \includegraphics[width=0.49\textwidth]{figs/cl-mva3j450el-inTTbar.pdf}
  \caption{\label{fig:mva:plots-mva3j450el} The data-MC comparisons
    after standard event selection (left) and top pair
    selection (right) for the working point: mva3j450el.}
\end{figure}

%%%%%%%%%%%%%%%%%%% mva3j500el
\begin{figure}[!t]
  \centering
  \includegraphics[width=0.49\textwidth]{figs/cl-mva3j500el-normal.pdf}
  \includegraphics[width=0.49\textwidth]{figs/cl-mva3j500el-inTTbar.pdf}
  \caption{\label{fig:mva:plots-mva3j500el} The data-MC comparisons
    after standard event selection (left) and top pair
    selection (right) for the working point: mva3j500el.}
\end{figure}

%%%%%%%%%%%%%%%%%%% mva3j550el
\begin{figure}[!t]
  \centering
  \includegraphics[width=0.49\textwidth]{figs/cl-mva3j550el-normal.pdf}
  \includegraphics[width=0.49\textwidth]{figs/cl-mva3j550el-inTTbar.pdf}
  \caption{\label{fig:mva:plots-mva3j550el} The data-MC comparisons
    after standard event selection (left) and top pair
    selection (right) for the working point: mva3j550el.}
\end{figure}

%%%%%%%%%%%%%%%%%%% mva3j600el
\begin{figure}[!t]
  \centering
  \includegraphics[width=0.49\textwidth]{figs/cl-mva3j600el-normal.pdf}
  \includegraphics[width=0.49\textwidth]{figs/cl-mva3j600el-inTTbar.pdf}
  \caption{\label{fig:mva:plots-mva3j600el} The data-MC comparisons
    after standard event selection (left) and top pair
    selection (right) for the working point: mva3j600el.}
\end{figure}


\clearpage
%%%%%%%%%%%%%%%%%%%%%%%%%%%%%%%%%%%%%%%%%%%%%%%%%%%%%%%%%%%%%%%%%%%%%%%%%%%%%%%%%%%%%%%%%%%%%%%
\section{MVA Output Distributions}
We use an almost pure top data control sample to study the systematics
on the Higgs signal efficiency due to the MVA output cut. These
semileptonic top events are selected by requiring exactly four jets in
the event, out of which two are b-tagged and the other two are
anti-btagged. To better understand the MVA ouput distributions, we
compare those distributions in top MC samples with Higgs signal MC
samples as shown in Fig.~\ref{fig:mva:sigvsttbar-mva2j170} to
Fig.~\ref{fig:mva:sigvsttbar-mva2j600}. 

%%%%%%%%%%%%%%%%%%% 
\begin{figure}[!t]
  \centering
  \includegraphics[width=0.49\textwidth]{figs/cl-mva2j170mu-mvaTopvsHiggs.pdf}
  \includegraphics[width=0.49\textwidth]{figs/cl-mva3j170mu-mvaTopvsHiggs.pdf}
  \includegraphics[width=0.49\textwidth]{figs/cl-mva2j170el-mvaTopvsHiggs.pdf}
  \includegraphics[width=0.49\textwidth]{figs/cl-mva3j170el-mvaTopvsHiggs.pdf}
  \caption{\label{fig:mva:sigvsttbar-mva2j170}The MVA output
    distributions in top MC sample compared with in Higgs MC
    samples for Higgs mass 170~GeV. Top-left plot is muon 2-jet category,
    top-right is muon 3-jet category, bottom-left is electron 2-jet
    category, and bottom-right is electron 3-jet category. }
\end{figure}

%%%%%%%%%%%%%%%%%%% 
\begin{figure}[!t]
  \centering
  \includegraphics[width=0.49\textwidth]{figs/cl-mva2j180mu-mvaTopvsHiggs.pdf}
  \includegraphics[width=0.49\textwidth]{figs/cl-mva3j180mu-mvaTopvsHiggs.pdf}
  \includegraphics[width=0.49\textwidth]{figs/cl-mva2j180el-mvaTopvsHiggs.pdf}
  \includegraphics[width=0.49\textwidth]{figs/cl-mva3j180el-mvaTopvsHiggs.pdf}
  \caption{\label{fig:mva:sigvsttbar-mva2j180}The MVA output
    distributions in top MC sample compared with in Higgs MC
    samples for Higgs mass 180~GeV. Top-left plot is muon 2-jet category,
    top-right is muon 3-jet category, bottom-left is electron 2-jet
    category, and bottom-right is electron 3-jet category. }
\end{figure}

%%%%%%%%%%%%%%%%%%% 
\begin{figure}[!t]
  \centering
  \includegraphics[width=0.49\textwidth]{figs/cl-mva2j190mu-mvaTopvsHiggs.pdf}
  \includegraphics[width=0.49\textwidth]{figs/cl-mva3j190mu-mvaTopvsHiggs.pdf}
  \includegraphics[width=0.49\textwidth]{figs/cl-mva2j190el-mvaTopvsHiggs.pdf}
  \includegraphics[width=0.49\textwidth]{figs/cl-mva3j190el-mvaTopvsHiggs.pdf}
  \caption{\label{fig:mva:sigvsttbar-mva2j190}The MVA output
    distributions in top MC sample compared with in Higgs MC
    samples for Higgs mass 190~GeV. Top-left plot is muon 2-jet category,
    top-right is muon 3-jet category, bottom-left is electron 2-jet
    category, and bottom-right is electron 3-jet category. }
\end{figure}

%%%%%%%%%%%%%%%%%%% 
\begin{figure}[!t]
  \centering
  \includegraphics[width=0.49\textwidth]{figs/cl-mva2j200mu-mvaTopvsHiggs.pdf}
  \includegraphics[width=0.49\textwidth]{figs/cl-mva3j200mu-mvaTopvsHiggs.pdf}
  \includegraphics[width=0.49\textwidth]{figs/cl-mva2j200el-mvaTopvsHiggs.pdf}
  \includegraphics[width=0.49\textwidth]{figs/cl-mva3j200el-mvaTopvsHiggs.pdf}
  \caption{\label{fig:mva:sigvsttbar-mva2j200}The MVA output
    distributions in top MC sample compared with in Higgs MC
    samples for Higgs mass 200~GeV. Top-left plot is muon 2-jet category,
    top-right is muon 3-jet category, bottom-left is electron 2-jet
    category, and bottom-right is electron 3-jet category. }
\end{figure}

%%%%%%%%%%%%%%%%%%% 
\begin{figure}[!t]
  \centering
  \includegraphics[width=0.49\textwidth]{figs/cl-mva2j250mu-mvaTopvsHiggs.pdf}
  \includegraphics[width=0.49\textwidth]{figs/cl-mva3j250mu-mvaTopvsHiggs.pdf}
  \includegraphics[width=0.49\textwidth]{figs/cl-mva2j250el-mvaTopvsHiggs.pdf}
  \includegraphics[width=0.49\textwidth]{figs/cl-mva3j250el-mvaTopvsHiggs.pdf}
  \caption{\label{fig:mva:sigvsttbar-mva2j250}The MVA output
    distributions in top MC sample compared with in Higgs MC
    samples for Higgs mass 250~GeV. Top-left plot is muon 2-jet category,
    top-right is muon 3-jet category, bottom-left is electron 2-jet
    category, and bottom-right is electron 3-jet category. }
\end{figure}

%%%%%%%%%%%%%%%%%%% 
\begin{figure}[!t]
  \centering
  \includegraphics[width=0.49\textwidth]{figs/cl-mva2j300mu-mvaTopvsHiggs.pdf}
  \includegraphics[width=0.49\textwidth]{figs/cl-mva3j300mu-mvaTopvsHiggs.pdf}
  \includegraphics[width=0.49\textwidth]{figs/cl-mva2j300el-mvaTopvsHiggs.pdf}
  \includegraphics[width=0.49\textwidth]{figs/cl-mva3j300el-mvaTopvsHiggs.pdf}
  \caption{\label{fig:mva:sigvsttbar-mva2j300}The MVA output
    distributions in top MC sample compared with in Higgs MC
    samples for Higgs mass 300~GeV. Top-left plot is muon 2-jet category,
    top-right is muon 3-jet category, bottom-left is electron 2-jet
    category, and bottom-right is electron 3-jet category. }
\end{figure}

%%%%%%%%%%%%%%%%%%% 
\begin{figure}[!t]
  \centering
  \includegraphics[width=0.49\textwidth]{figs/cl-mva2j350mu-mvaTopvsHiggs.pdf}
  \includegraphics[width=0.49\textwidth]{figs/cl-mva3j350mu-mvaTopvsHiggs.pdf}
  \includegraphics[width=0.49\textwidth]{figs/cl-mva2j350el-mvaTopvsHiggs.pdf}
  \includegraphics[width=0.49\textwidth]{figs/cl-mva3j350el-mvaTopvsHiggs.pdf}
  \caption{\label{fig:mva:sigvsttbar-mva2j350}The MVA output
    distributions in top MC sample compared with in Higgs MC
    samples for Higgs mass 350~GeV. Top-left plot is muon 2-jet category,
    top-right is muon 3-jet category, bottom-left is electron 2-jet
    category, and bottom-right is electron 3-jet category. }
\end{figure}

%%%%%%%%%%%%%%%%%%% 
\begin{figure}[!t]
  \centering
  \includegraphics[width=0.49\textwidth]{figs/cl-mva2j400mu-mvaTopvsHiggs.pdf}
  \includegraphics[width=0.49\textwidth]{figs/cl-mva3j400mu-mvaTopvsHiggs.pdf}
  \includegraphics[width=0.49\textwidth]{figs/cl-mva2j400el-mvaTopvsHiggs.pdf}
  \includegraphics[width=0.49\textwidth]{figs/cl-mva3j400el-mvaTopvsHiggs.pdf}
  \caption{\label{fig:mva:sigvsttbar-mva2j400}The MVA output
    distributions in top MC sample compared with in Higgs MC
    samples for Higgs mass 400~GeV. Top-left plot is muon 2-jet category,
    top-right is muon 3-jet category, bottom-left is electron 2-jet
    category, and bottom-right is electron 3-jet category. }
\end{figure}

%%%%%%%%%%%%%%%%%%% 
\begin{figure}[!t]
  \centering
  \includegraphics[width=0.49\textwidth]{figs/cl-mva2j450mu-mvaTopvsHiggs.pdf}
  \includegraphics[width=0.49\textwidth]{figs/cl-mva3j450mu-mvaTopvsHiggs.pdf}
  \includegraphics[width=0.49\textwidth]{figs/cl-mva2j450el-mvaTopvsHiggs.pdf}
  \includegraphics[width=0.49\textwidth]{figs/cl-mva3j450el-mvaTopvsHiggs.pdf}
  \caption{\label{fig:mva:sigvsttbar-mva2j450}The MVA output
    distributions in top MC sample compared with in Higgs MC
    samples for Higgs mass 450~GeV. Top-left plot is muon 2-jet category,
    top-right is muon 3-jet category, bottom-left is electron 2-jet
    category, and bottom-right is electron 3-jet category. }
\end{figure}

%%%%%%%%%%%%%%%%%%% 
\begin{figure}[!t]
  \centering
  \includegraphics[width=0.49\textwidth]{figs/cl-mva2j500mu-mvaTopvsHiggs.pdf}
  \includegraphics[width=0.49\textwidth]{figs/cl-mva3j500mu-mvaTopvsHiggs.pdf}
  \includegraphics[width=0.49\textwidth]{figs/cl-mva2j500el-mvaTopvsHiggs.pdf}
  \includegraphics[width=0.49\textwidth]{figs/cl-mva3j500el-mvaTopvsHiggs.pdf}
  \caption{\label{fig:mva:sigvsttbar-mva2j500}The MVA output
    distributions in top MC sample compared with in Higgs MC
    samples for Higgs mass 500~GeV. Top-left plot is muon 2-jet category,
    top-right is muon 3-jet category, bottom-left is electron 2-jet
    category, and bottom-right is electron 3-jet category. }
\end{figure}

%%%%%%%%%%%%%%%%%%% 
\begin{figure}[!t]
  \centering
  \includegraphics[width=0.49\textwidth]{figs/cl-mva2j550mu-mvaTopvsHiggs.pdf}
  \includegraphics[width=0.49\textwidth]{figs/cl-mva3j550mu-mvaTopvsHiggs.pdf}
  \includegraphics[width=0.49\textwidth]{figs/cl-mva2j550el-mvaTopvsHiggs.pdf}
  \includegraphics[width=0.49\textwidth]{figs/cl-mva3j550el-mvaTopvsHiggs.pdf}
  \caption{\label{fig:mva:sigvsttbar-mva2j550}The MVA output
    distributions in top MC sample compared with in Higgs MC
    samples for Higgs mass 550~GeV. Top-left plot is muon 2-jet category,
    top-right is muon 3-jet category, bottom-left is electron 2-jet
    category, and bottom-right is electron 3-jet category. }
\end{figure}

%%%%%%%%%%%%%%%%%%% 
\begin{figure}[!t]
  \centering
  \includegraphics[width=0.49\textwidth]{figs/cl-mva2j600mu-mvaTopvsHiggs.pdf}
  \includegraphics[width=0.49\textwidth]{figs/cl-mva3j600mu-mvaTopvsHiggs.pdf}
  \includegraphics[width=0.49\textwidth]{figs/cl-mva2j600el-mvaTopvsHiggs.pdf}
  \includegraphics[width=0.49\textwidth]{figs/cl-mva3j600el-mvaTopvsHiggs.pdf}
  \caption{\label{fig:mva:sigvsttbar-mva2j600}The MVA output
    distributions in top MC sample compared with in Higgs MC
    samples for Higgs mass 600~GeV. Top-left plot is muon 2-jet category,
    top-right is muon 3-jet category, bottom-left is electron 2-jet
    category, and bottom-right is electron 3-jet category. }
\end{figure}

\clearpage
%\clearpage
%\section{Conclusions}

We have used 2.875 pb$^{-1}$ of CMS data to measure the dijet mass spectrum
with the following eta cuts on the two leading jets: $\mid \Delta\eta \mid < 1.3$ and 
$\mid \eta \mid < 2.5$. The event with the largest observed dijet mass is at
 $m=2.05$ TeV. The measured dijet mass spectrum 
is in good agreement with a QCD prediction from PYTHIA and the full simulation of 
the CMS detector.


We have performed direct searches for high-mass dijet resonances in the
dijet mass distribution. The dijet mass data is well fit by a simple parameterization. There is no significant evidence for new particle production
in the data.  We set 95\% confidence level upper limits on the cross section for
a dijet resonance, applicable to any narrow resonance producing the following
specific pairs of partons:  $qq$, $qg$, and $gg$.  We have compared our cross section
limits with the expected cross sections from several existing models. 
We exclude at the 95\% confidence level new particles predicted 
in the following specific models: string resonances with mass less than 2.50~TeV, excited quarks with mass less than 1.58~TeV, 
and axigluons, colorons and $E_6$ diquarks in specific mass intervals, extending previously 
published limits on all models.


\section{Acknowledgments}

We would like to thank Can Kilic for his work on the string resonance
cross section, both at LHC and the Tevatron, exotica conveners Greg Landsberg and Chris Hill for their
careful reading of the note and suggestions for improvement, and the analysis review committee 
Bob Cousins, Valerie Halyo, and Jesus Marco for their effort and valuable suggestions.

%%%%%%%%%%%%%%%%%%%%%%%%%%%%%%%%%%%%%%%%%%%%%%%%%%%%%%%%%%%%%%%%%%%%%%%%%%

% >> acknowledgements (for journal papers)
% Please include the latest version from https://twiki.cern.ch/twiki/bin/viewauth/CMS/Internal/PubAcknow.
%\section*{Acknowledgements}
% ack-text

%% **DO NOT REMOVE BIBLIOGRAPHY**
\bibliography{auto_generated}   % will be created by the tdr script.

%%%% DO NOT ADD \end{document}!

