
Several triggers are used to collect events consistent with 
the vector boson topollogy. For the $W$ + jet channels the trigger 
paths consist of several single-lepton triggers with tight lepton 
identification. Leptons are also required to be isolated from other 
tracks and calorimeter energy depositions to maintain an acceptable 
trigger rate. For the $W(\mu\nu)$ channel, the trigger thresholds for the 
muon $p_T$, are in the range of 17 to 40 GeV. 
The higher thresholds are used for the periods of higher instantaneous 
 luminosity. The combined trigger efficiency is 
 around 90\% for signal events that pass all offline requirements, 
 described in the following. 
 For the $W(e\nu)$ channel the electron $p_T$ threshold ranges 
 from 25 to 32 GeV. To preferentially keep only 
 real W events, the single electron triggers typically also require 
 minimum thresholds on the missing transverse energy (MET) and the transverse mass 
 $m_T$ of the electron plus MET system. 
 The combined efficiency for these triggers for signal events 
 that pass the final offline selection criteria is $>95\%$.
 The $Z(\mu\mu)$ channel uses the same single-muon triggers as the 
 $W(\mu\nu)$ channel. For the $Z(ee)$ channel, di-electron triggers 
 with lower pT thresholds (17 and 8 GeV) and tight isolation 
 requirements are used. These triggers are 99\% efficient for all 
 $Z(ee)$ plus jet events that pass the final offline selection criteria.

