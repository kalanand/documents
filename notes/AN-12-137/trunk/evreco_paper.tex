The reconstructed interaction vertex with the largest value 
of $\sum_i p_{T_i}^2$, where $p_{T_i}$ is the transverse momentum of 
the $i$-th track associated to the vertex, is selected as the primary event 
vertex. This vertex is used as the reference vertex for all 
relevant objects in the event, which are reconstructed with 
the particle-flow algorithm. The PU interactions affect jet momentum 
reconstruction, missing transverse energy reconstruction, and lepton isolation.
 To mitigate these effects, a track-based algorithm that filters all 
 charged hadrons that do not originate from the primary interaction is used. 
 In addition, a calorimeter-based algorithm evaluates the energy density in 
 the calorimeter from interactions not related to the primary vertex and 
 subtracts it from reconstructed jets in the event~\cite{}.

Electron reconstruction requires the matching of an energy cluster in the 
ECAL with a track in the silicon tracker~\cite{}.  
Identification criteria based on the ECAL shower shape, track-ECAL cluster 
matching, and consistency with the primary vertex are imposed. Additional 
requirements are imposed to remove electrons produced by photon conversions. 
In this analysis, electrons are considered in the pseudorapidity range 
$|\eta|<2.5$, excluding the $1.44<|\eta|<1.57$ transition region between the 
ECAL barrel and endcap.
Muons are reconstructed using two algorithms~\cite{}: 
one in which tracks in the silicon tracker are matched to signals in 
the muon chambers, and another in which a global track fit is performed 
seeded by signals in the muon system. The muon candidates used in the 
analysis are required to be reconstructed successfully by both algorithms. 
Further identification criteria are imposed on the muon candidates to reduce 
the fraction of tracks misidentified as muons. These include the number of 
measurements in the tracker and the muon system, the fit quality of the 
 muon track, and its consistency with the primary vertex.

Charged leptons from $W$ and $Z$ boson decays are expected to be isolated from 
other activity in the event. For each lepton candidate, a cone 
is constructed around the track direction at the
 event vertex. The scalar sum of the transverse energy of each 
 reconstructed particle compatible with the primary vertex and contained 
 within the cone is calculated excluding the contribution from the 
lepton candidate itself. If this sum exceeds approximately 10% of the 
candidate $p_T$ the lepton is rejected; the exact requirement depends 
on the lepton $\eta$, $p_T$ and flavor.
Muons (electrons) are required to have a $p_T$, greater than 30 GeV (80 GeV). The very high offline threshold on the eletron momentum is motivated by the criteria to avoid turn-off effects in the trigger efficiency for the single electron trigger. 

An accurate MET measurement is essential for distinguishing the $W$ signal from QCD backgrounds. We use the MET measured in the event using the full particle-flow reconstruction. The MET resolution, measured as a function of the sum $E_T$ ($\sum E_T$) of the particle-flow~\cite{} objects in the event, varies from 4\% at $\Sjm E_T$ =60 GeV to10\% at $\sum E_T$ =350 GeV~\cite{}.We require MET >30(50) GeV in the event in case of muon~(electron) data. 

Jets are reconstructed from particle-flow objects~\cite{} using
different clustering algorithmd as described in the following. 



