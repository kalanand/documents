\label{sec:summary}

In summary, we have presented the particle-level differential jet mass distribution for 
anti-$k_{\mathrm{T}}$ jets with a $D$-parameter of 0.7 for standard jets, as well as
for jets with several grooming algorithms applied,
filtering, trimming, and pruning. In addition, we have presented similar distributions for the CA-0.8 pruned and CA-12 filtered jets. The behavior of the jet mass with pileup has also been
investigated, as well as comparisons at the detector level
between anti-$k_{\mathrm{T}}$-0.5, anti-$k_{\mathrm{T}}$-0.7,
anti-$k_{\mathrm{T}}$-0.8, CA-0.8 and CA-1.2 jets.  

Leading-order parton-shower Monte Carlo predictions for the jet mass are found to be in good agreement with the data. Some recurrent patterns can already be observed: in general, both pythia and herwig reproduce the data reasonably well, with the data in better agreement with herwig predictions for more aggressive grooming algorithms. In comparing the results for the $V$+jet analysis and the di-jet one, it can be noticed that the MC predictions are in slightly better agreement with the data for the $V$+jet analysis, that could be an indication of a better simulation response for quark-originated jet w.r.t. gluon ones. The biggest discrepancies between data and simulation are evident at low mass values, that is also the region more afftected by PU.    

These studies represent the first investigation of jet substructure techniques using data collected by the CMS experiment at a center-of-mass energy of 7 TeV. For a few of these techniques, these studies have been completed for the first time using data from the LHC, and are an important benchmark toward their use in new physics or Higgs searches that rely on substructure algorithms to improve their sensitivity. In addition, the intrinsic robustness of these algorithms to PU effects will most likely contribute to a more widespread use of these techniques in the future high luminosity runs at the LHC.

