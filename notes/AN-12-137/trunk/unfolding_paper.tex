The measured spectrum of a physical observable, like the jet mass distribution, is usually distorted by	detector effects, su
ch as finite resolution and limited acceptance. Moreover, in this analysis the chosen bin size is comparable to the resoluti
on, so there is a significant migration of events generated in one jet bin mass and ending up in a different bin of reconstr
ucted jet mass. A comparison of the measured mass spectrum with that predicted at generator level requires that we remove th
ese effects to obtain the true underlying mass spectrum. There are several possible ways to achieve the unfolding of detecto
r effects on measured spectra. We use an unfolding procedure described by G.~D.~Agostini in~\cite{agostini}. Repeated application of Bayes theorem is used 
to invert the response matrix. Regularization is achieved by stopping iterations before reaching the ``true'' (but wildly fl
uctuating) inverse. The regularization parameter is just the number of iterations.
 In principle, this has to be tuned to prevent the statistical fluctuations being interpreted as structure in the true distr
ibution, according to the sample statistics and binning. In practice, the results are fairly insensitive to the precise sett
ing used and four iterations are usually sufficient.

