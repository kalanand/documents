%%%%%%%%%%%%%%%%%%%%%%%%%%%%%%%%%%%%%%%%%%%%%%%%%%%%%%%%%%%%%%%%%%%%%%%%%%
%%%%%%%%%%%%%%%%%%%%%%%%%%%%%%%%%%%%%%%%%%%%%%%%%%%%%%%%%%%%%%%%%%%%%%%%%%
\clearpage{}
\section{Physics objects reconstruction}
\label{sec:reco}
\label{sec:firstStep}
% ---- ---- ---- ---- ---- ---- ---- ---- ---- ---- ---- ---- ---- ---- ----
The analysis relies on the standard reconstruction algorithms 
produced by the CMS community.
Event data is reconstructed using the particle-flow (PF) reconstruction 
technique~\cite{pflow}. 
Particle flow attempts to reconstruct all stable particles in an event by 
combining information from all subdetectors. The algorithm categorizes all 
particles into the following five types: muons, electrons, photons, charged 
and neutral hadrons. The list of reconstructed particles is used as the set of
inputs for a jet clustering algorithm to create particle-flow jets.
%%%%%%%%%%%%%%%%%%%%%%%%%%%%
\subsection{Electron selection}
\label{sec:electron_cuts}

Electrons are reconstructed using a gaussian-sum filter (GSF)
algorithm \cite{CMS-PAS-EGM-10-004}, and are required to pass electron
ID cuts according to a multi-variate identification
technique~\cite{cite:elemva}.  We also require that selected
electron candidates are isolated. Particle flow-based relative
isolation is defined as
%%%
\begin{equation*}
\mathrm{RelIso_{\mathrm{PF}}} = \frac{I_{\mathrm{CH}}+max(0,I_{\mathrm{NH}}+I_{\mathrm{PHOTON}}-(\mathrm{EA}\cdot\rho))}{E_\mathrm{T}},
\end{equation*}
%%%
where $I_{\mathrm{CH}}$, $I_{\mathrm{NH}}$ and $I_{\mathrm{PHOTON}}$
are the charged hadron, neutral hadron and photon isolation variables
(using an isolation cone of 0.3). The isolation is corrected for
contributions from pile-up using charged hadron subtraction in the
isolation cone using fastjet algorithm \cite{FastJetPUSubtraction} and
for neutral particles using the effective area correction,
$(\mathrm{EA}\cdot\rho)$ where $\mathrm{EA}$ is the cone effective area
and $\rho$ is the average neutral particle density of the event.

The ID and isolation cuts used are shown in Table~\ref{tab:EleID} and
have been tuned to give the same efficiency bin-by-bin with respect to
the working point (WP) used for 2011 analysis.

Additionally, we require
%%%%%%%%%%%%%%%%%%%
\begin{itemize}
\item Electron $E_\mathrm{T} > 30GeV$.%,\mathrm{GeV}$ ($> 35\,\mathrm{GeV}$) for the resolved (merged) topology .
\item Pseudorapidity $|\eta| < 2.5$. There is an exclusion range due
        to the ECAL barrel-endcap transition region, defined by
        $1.4442 < |\eta_{\mathrm{sc}}| < 1.566$, where
        $\eta_{\mathrm{sc}}$ is the pseudorapidity of the ECAL
        supercluster.
%\item Impact parameter: We cut on the absolute value of the impact
%       parameter calculated with respect to the average primary vertex (PV). We
%       require: $d_0(\mathrm{PV}) < 0.02\,\mathrm{cm}.$

%\item In order to make sure that the selected electron and the selected
%jets come from the same hard interaction and not from pile up events,
%we require that the $z$ coordinate of the PV of the event and the $z$
%coordinate of the electron's vertex lie within a distance of
%less than $0.1~\mathrm{cm}$.

\item
In order to reject events in which the electron candidate actually
originates from a conversion of a photon into an $e^{+}e^{-}$ pair, we
use an approach using the vertex fit probability of fully
reconstructed conversions combined with the requirement that the
number of missed inner tracker layers of the electron track must be
exactly zero (i.e. there are no missed layers before the first hit of
the electron track from the beam line).
\end{itemize}

\begin{table}[bthp]
\begin{center}
{\footnotesize
\begin{tabular}{|c|c|c|c|}
\hline
Lepton $\eta$ & $|\eta| < 0.8$ & $0.8 < |\eta| < 1.479$ & $1.479 < |\eta| < 2.5$  \\
\hline
ID MVA cut value (tight lepton) & 0.913 & 0.964 & 0.899 \\
Isolation cut value (tight lepton) & 0.105 & 0.178 & 0.150 \\
ID MVA cut value (loose lepton) & 0.877 & 0.811 & 0.707 \\
Isolation cut value (loose lepton) & 0.426 & 0.481 & 0.390 \\
\hline
\end{tabular}
\caption[.]{\label{tab:EleID} Cut values for electron identification
MVA output and for isolation which are tuned to give the same
efficiency as VBTF Working Point (WP) 80, as used for the tight
electron selection, and VBTF Working Point (WP) 90, as used in the
loose electron selection.}}
\end{center}
\end{table}


%%%%%%%%%%%%%%%%%%%
%%%%%%%%%%%%%%%%%%%%%%%%%%%%%%%%%%%%%%%%%%%%%%%%%%%%%%%%%%%%%%%%%%%%%%%%%%%%
%%%%%%%%%%%%%%%%%%%%%%%%%%%%%%%%%%%%%%%%%%%%%%%%%%%%%%%%%%%%%%%%%%%%%%%%%%%%
\subsection{Muon selection}
\label{sec:muon_cuts}

Muon candidates are identified by two different
algorithms~\cite{MUONPAS}: one proceeds from the inner tracker outwards,
the other one starts from tracks measured in the muon chambers and matches
and combines them with tracks reconstructed in the inner tracker.
These selection criteria are summarized below:
%%%%%%%%%%%%%%%%%%%
\begin{itemize}
\item The muon candidate is reconstructed both as a global muon and
as a tracker muon.
\item Number of pixel hits of the Tracker track $\ge 1$;
\item Number of muon system hits of the Global track $\ge 1$;
\item Normalized $\chi^{2}$ of the Global track $< 10.0$.
\item Muon $p_{\mathrm{T}} > 25GeV$.%,\mathrm{GeV}$ ($> 30\,\mathrm{GeV}$) for the resolved (merged) topology.
\item Pseudorapidity $|\eta| < 2.1$.
\item Impact parameter: We cut on the absolute value of the impact
parameter calculated with respect to the primary vertex. We require:
$d_0(\mathrm{PV}) < 0.02\,\mathrm{cm}.$
\item In order to make sure that the selected muon and the selected
jets come from the same hard interaction and not from pile up events,
we require that the $z$ coordinate of the PV of the event and the $z$
coordinate of the muon's inner track vertex lie within a distance of
less than 0.5~cm.
\item The number of tracker layers with hits from the muon track has to be
$N_{\mathrm{layers}} > 5$.
\end{itemize}

The selected muon candidates also have to be isolated. Particle
flow-based relative isolation for muons is defined as
\begin{equation*}
\mathrm{RelIso_{\mathrm{PF}}} = \frac{I_{\mathrm{CH}}+max(0,I_{\mathrm{NH}}+I_{\mathrm{PHOTON}}-(0.5~{p}_{T}^\mathrm{sumPU}))}{p_\mathrm{T}},
\end{equation*}

where $I_{\mathrm{CH}}$, $I_{\mathrm{NH}}$ and $I_{\mathrm{PHOTON}}$
are the charged hadron, neutral hadron and photon isolation variables
(using an isolation cone of 0.4). The isolation is corrected for
contributions from pile-up using charged hadron subtraction in the
isolation cone using fastjet algorithm \cite{FastJetPUSubtraction} and
for neutral particles using the DeltaBeta correction, $(0.5 p_T^\mathrm{sumPU})$. We require the muon to have $\mathrm{RelIso_{\mathrm{PF}}} < 0.12$ in order to be considered isolated.



%%%%%%%%%%%%%%%%%%%
%%%%%%%%%%%%%%%%%%%%%%%%%%%%%%%%%%%%%%%%%%%%%%%%%%%%%%%%%%%%%%%%%%%%%%%%%%%%
%%%%%%%%%%%%%%%%%%%%%%%%%%%%%%%%%%%%%%%%%%%%%%%%%%%%%%%%%%%%%%%%%%%%%%%%%%%%
%\subsection{Loose leptons for ``lepton veto'' and ``jet cleaning''}
% For the purposes of rejecting events with more than one lepton, we
% define ``loose leptons'', which pass less restrictive sets of selection
% criteria. For electrons, we consider those
% that have $p_{\mathrm{T}} > 15$~GeV, $|\eta| < 2.5$, and
% $\mathrm{RelIso_{\mathrm{Trk}}} < 0.2$ and that satisfy electron ID
% cuts according to ``VBTF Working Point 95'' to be ``loose''. The cut
% values for the electron ID variables used in the analysis can be found
% in Table~\ref{tab:EleID}. Similarly, we define a loose muon as a global muon that passes
% $p_{\mathrm{T}} > 10\,\mathrm{GeV}$, $|\eta| < 2.5$, and
% $\mathrm{RelIso_{\mathrm{Trk}}} < 0.2$.

\subsubsection{Loose Electron}
For the purposes of rejecting events with more than one lepton we
define a loose electron, which has looser cuts. We consider electrons
which have $p_{\mathrm{T}} > 20\,\mathrm{GeV}/c$, $|\eta| < 2.5$,
and which satisfy electron $\mathrm{RelIso_{\mathrm{PF}}}$ and MVA ID
cuts. The cut values for the electron ID and isolation used in the
analysis can be found in Table~\ref{tab:EleID}.
%As in the case of the
%tight electrons, we also require $d_0(\mathrm{PV}) <
%0.02\,\mathrm{cm}$ and that the $z$ coordinate of the PV of the event
%and the $z$ coordinate of the electron's vertex lie within a distance
%of less than $0.1~\mathrm{cm}$.

\subsubsection{Loose Muon}
Additionally, to reject events with more than one lepton, we define a
loose muon, which has looser cuts. We consider all global muons which
have $p_{\mathrm{T}} > 10\,\mathrm{GeV}/c$, $|\eta| < 2.5$, and
$\mathrm{RelIso_{\mathrm{PF}}} < 0.2$ to be loose muons.

%%%%%%%%%%%%%%%%%%%%%%%%%%%%%%%%%%%%%%%%%%%%%%%%%%%%%%%%%%%%%%%%%%%%%%%%%%%%
%%%%%%%%%%%%%%%%%%%%%%%%%%%%%%%%%%%%%%%%%%%%%%%%%%%%%%%%%%%%%%%%%%%%%%%%%%%%
\subsection{Jet Selection}
\label{sec:firstStep_jets}

Jets are reconstructed starting from the set of objects reconstructed by the particle flow algorithm\cite{pflow,CMS-PAS-JME-10-003,CMS-PAS-PFT-10-002}.
The CMS standard R=0.5 jets reconstructed with the anti-kT algorithm~\cite{cacciari} (AK5) is used in this analysis. 

\subsubsection{Jets}
Jets are corrected such that the measured energy of the jet 
correctly reproduces the energy of the initial particle. 
The CMS standard L2 (relative) correction makes the jet response flat in $\eta$.
The standard L3 (absolute) correction brings the jet closer to the $\PT$ of 
a matched generated jet created using generator level input and a similar 
jet clustering algorithm.
The L2 and L3 corrections are calculated using Monte Carlo, and thus a 
L2L3 residual correction is applied that fixes the discrepancies between 
Monte Carlo and data~\cite{newjes-cms}.
For this analysis jets with measured (corrected) $\PT$  
greater than 30~$\gev$ and jet $|\eta| < 4.7$, which is widely used in higgs analyis of VBF channel, are required in the event pre-selection.  Jets from pile-up are identified and removed with PileupJetID tool ~\cite{cite:PileupJetID}.

Jets are required to pass a set of loose identification
criteria; this requirement eliminates jets originating from or being seeded by
noisy channels in the calorimeter~\cite{Chatrchyan:2009hy}: 

%%%%%%%%%%%%%%
\begin{itemize}
\item Fraction of energy due to neutral hadrons $<$ 0.99.
\item Fraction of energy due to neutral EM deposits $<$ 0.99.
\item Number of constituents $>$ 1.
\item Number of charged hadrons candidates $>$ 0.
\item Fraction of energy due to charged hadrons candidates $>$ 0.
\item Fraction of energy due to charged EM deposits $<$ 0.99.
\end{itemize}
%%%%%%%%%%
All energy fractions are calculated from uncorrected jets.

\par
In order to account for electron and muon objects that
have been reconstructed as jets, we remove from the jet
collection any jet that falls within a
cone of radius $R= 0.3$ of a loose electron or a loose muon. 
This ``cleaning'' procedure is applied in order to ensure that the same
particle is not double counted as two different physics objects.


%\subsection{CA8 Jets}
%CA8 jets are used to identify high $\pt$ merged W bosons. Dedicated corrections for CA8 jets do not exist so AK7 jet
%corrections are applied to CA8 uncorrected jets. Their overall performance in high pile-up envronment is discussed in SMP-12-019, while systematic differences due to using AK7 corrections 
%are computed in Sec.~\ref{sec:systematicsCA8}. AK5 jets are used for some supporting cuts, such as number of
%additional jets originating from a $b$ quark, and have their own dedicated corrections.
%
%Jets are corrected such that their measured energy
%correctly reproduces the energy of the initial particle. 
%The CMS standard L2 (relative) correction makes the jet response flat in $\eta$.
%The standard L3 (absolute) correction brings the jet closer to the $\PT$ of 
%a matched generated jet created using generator level input and a similar 
%jet clustering algorithm.
%The L2 and L3 corrections are calculated using Monte Carlo, and thus a 
%L2L3 residual correction is applied that fixes the discrepancies between 
%Monte Carlo and data~\cite{newjes-cms}.
%In this analysis we use jets with measured (corrected) $\PT$  
%greater than 30~$\gev$. 
%We require $|\eta| < 2.4$ so that the jets fall within the
%tracker acceptance. Jets from pile-up are identified and removed with PileupJetID tool ~\cite{cite:PileupJetID}.  
%
%\subsubsection{Grooming techniques}
%
%Jet grooming techniques are typically used to reduce the impact of underlying event (UE), pileup (PU), and soft QCD contributions to the jet.  
%Three proposed algorithms for accomplishing this are filtering~\cite{Butterworth:2008iy}, trimming~\cite{Krohn:2009th}, and pruning~\cite{Ellis:2009me}.
%All three are meant to clean up the jet but do so in different ways.  
%
%Filtering reclusters the jet with a smaller radius jet, $r_{\rm filt}$ and keeps the $n_{\rm filt}$ subjets.
%The default value of these parameters are $r_{\rm filt} = 0.3$ and $n_{\rm filt} = 3$ and the jet is reclustered with the CA algorithm.
%
%Trimming is similar to filtering in that it reclusters the jet with a smaller radius jetl; however, instead of keeping 
%a certain number of subjets, it keeps all subjects which contain at least a given fraction of the original jet $pT$, $pT_{\rm frac}$. 
%The default values of these parameters are $r_{\rm filt} = 0.2$ and $pT_{\rm frac} = 0.03$ and the jet is reclustered with the $k_T$ algorithm.
%
%In pruning, the jet is reclustered and all soft or large angle recombinations are removed.  Soft contributions are removed via a cut on $z_{ij}$
%and large angle contributions are removed via a cut on $\Delta R_{ij}$.
%where:
%\begin{equation}
%z_{ij} = \frac{{\rm min}(p_{T,i},p_{T,j})}{p_{T,(i+j)}} < z_{\rm cut}
%\end{equation}
%and 
%\begin{equation}
%\Delta R_{ij} > D_{\rm cut} = \alpha \times \frac{m}{p_T}.
%\end{equation}
%Here, $p_{T,i}$ and $p_{T,j}$ are the transverse momenta of the $i$ and $j$ subjets and $m$ and $p_T$ are with respect to the original jet.  
%The default values for $\alpha$ and $z_{\rm cut}$ are $0.5$ and $0.1$, respectively.
%
%Grooming techniques typically serve to push the jet mass distribution of background jets towards zero while preserving the mass of $W$ jet.  
%The more aggressive the technique, better discrimination can be achieved however one must be wary of grooming away the hard part of the jet.  
%From previous studies, we find that, given the default values of the grooming algorithms, pruning is the most aggressive grooming technique and 
%best suited for our analysis.
%
%
%\subsubsection{N-subjettiness}
%
%N-subjettiness was introduced in~\cite{Thaler:2010tr} and is generalized jet shape observable. 
%For N subjets of a given jet, here defined via the exclusive $k_T$ algorithm, we can define the 
%N-subjettiness observables as:
%\begin{equation}
%\tau_N = \frac{1}{d_0} \sum_k p_{T,k} {\rm min} \{ \Delta R_{1,k},\Delta R_{2,k},\cdots, \Delta R_{N,k} \}
%\end{equation} 
%where $k$ runs over all constituent particles.
%The normalization factor is $d_0 = \sum_k p_{T,k}R_0$ and $R_0$ is the original jet radius.  
%The $\tau_N$ observable is in effect, a way to quantify to a certain degree how much a jet is likely to be compose of N subjets.  
%For distinguishing QCD jets (typically 1 subjet) from a $W$-jet (typically 2 subjets), we place the cut of $\tau_2/\tau_1<0.55$.
%

%
%%%%%%%%%%%%%%%%%%%%%%%%%%%%%%%%%%%%%%%%%%%%%%%%%%%%%%
%%%%%%%%%%%%%%%%%%%%%%%%%%%%%%%%%%%%%%%%%%%%%%%%%%%%%%
\subsection{Missing Transverse Energy}
\label{sec:MET}
An accurate MET measurement is essential for distinguishing
the $\Wo$ signal from QCD backgrounds. 
We use the MET estimate provided by the Particle Flow algorithm.
PF MET showed the best performance
among several MET algorithms~\cite{PFMET}.
The MET is computed as the vector sum of all PF objects.
A good agreement is found between the MET
distributions of $\Wln$ events in data and simulation~\cite{metPAS}.
The resolution for inclusive multi-jet samples and for
$\Wln$ events is also well reproduced by the simulation.  
A relative broadening of a few percent is observed in the data compared to MC,  
and has a negligible impact on the
extraction of the W yields~\cite{WZCMS:2010}.
%%%%%%%%%%%%%%%%%%%%%%%%%%%%%%%%%%%%%%%%%%%%%%%%%%%%%%
%%%%%%%%%%%%%%%%%%%%%%%%%%%%%%%%%%%%%%%%%%%%%%%%%%%%%%
%%%%%%%%%%%%%%%%%%%%%%%%%%%
%\subsection{Trigger\label{sec:trigger}}
%For 2011 data rely on single muon triggers HLT\_IsoMu17 and HLT\_IsoMu24.
%For electrons use logical OR of several triggers: 
%HLT\_Ele27, HLT\_Ele25\_CentralJet30\_CentralJet25\_MHT20, and inclusive 
%W trigger.
%%% need to get all the names correct.
%%%%%%%%%%%%%%%%%%%%%%%
%\begin{table}[h!]
%\caption{HLT paths used by run range in 2010 data}
%\label{tab:HLT}
%\begin{center}
%\begin{tabular}{ | c | c |}
%\hline
%Run Range & Trigger Name \\
%\hline
%136033 - 137028 & HLT\_Ele10\_LW\_L1R \\
%138564 - 140401 & HLT\_Ele15\_SW\_L1R \\
%141956 - 144114 & HLT\_Ele15\_SW\_CaloEleId\_L1R \\
%146428 - 147116 & HLT\_Ele17\_SW\_CaloEleId\_L1R \\
%147196 - 148058 & HLT\_Ele17\_SW\_TightEleId\_L1R\_v1 \\
%148822 - 149063 & HLT\_Ele17\_SW\_TighterEleIdIsol\_L1R\_v2 \\
%149181 - 149442 & HLT\_Ele17\_SW\_TighterEleIdIsol\_L1R\_v3 \\
%\hline
%\end{tabular}
%\end{center}
%\end{table}
%%%%%%%%%%%%%%%%%%%%%%%
\section{Event Selection And Categories}
\label{sec:evtSelCat}
The event should have a good primary vertex (PV). This means selecting
the primary vertex with the highest sum of $p_{T}^2$ of the tracks
associated with it and requiring it to have a number of degrees of
freedom (ndof) $\ge 4$, where ndof corresponds to the weighted sum of
the number of tracks used for the construction of the PV. In addition,
the PV must lie in the central detector region of $|z| \le 24$~cm
and $\rho \le 2$~cm around the nominal interaction point.

%\par
%Since the signal to background ratio, the background 
%composition, shape, and the resolution of the signal shape 
%depend strongly on whether the jets are b-jets or not and whether we are dealing with merged jets or not,
%we divide our data sample into three disjoint categories:
%\begin{itemize}
%\item 
%Dijet anti-btagged sample:
%Neither of the two jets is b-tagged, leptonic $W_{\pt}<200~\mathrm{GeV}$.
%\item 
%Dijet btagged sample:
%At least one jet is b-tagged, leptonic $W_{\pt}<200~\mathrm{GeV}$.
%\item 
%Boosted sample:
%Leptonic $W_{\pt}>200~\mathrm{GeV}$.
%\end{itemize}
%We use the Combined Secondary Vertex High Efficiency Medium (CSVHEM) operationg point 
%recommended by the BTAG POG. As suggested by the label, for the dijet samples we require exactly two jets passing the cuts
%described in Section~\ref{sec:firstStep_jets}.

\par
In the electron channel, we select events that contain exactly one
tight electron candidate fulfilling the criteria described in
Section~\ref{sec:electron_cuts} and reject events that contain a
loose electron or a loose muon in addition to the tight electron. 
In the muon channel, we select events that contain exactly one
tight muon candidate whose criteria are described in
Section~\ref{sec:muon_cuts} and reject events that contain an
additional loose lepton.
We require an event to have missing transverse energy
MET in excess of 25~GeV(30~GeV) for in muons(electrons) channel.
%in the dijet scenario and greater than
%50~GeV (70~GeV) for muons (electrons) in the boosted case. 
In addition we require the W boson transverse mass to be greater than 30~GeV.  These cuts are designed to reduce the background
from QCD multijet production.


%%%%%%%%%%%%%%%%%%%%%%%%%%
%%\subsection{Event categories}
%%\label{sec:evtCat}

%%%%%%%%%%%%%%%%%%%%%%%%
\subsection{Additional quality criteria}
\label{sec:evtSelAdditionalCuts}
We apply the following additional cuts to improve 
the signal over background ratio and reduce the systematic
uncertainty in the signal regime:
%%%%%%
\begin{itemize}
%\item 
%W transverse mass $ > 50\,\mathrm{GeV}$, which further reduces QCD multijet 
%background.
\item 
Leading jet $\pt > 60\,\mathrm{GeV}$, second leading jet $\pt > 50\,\mathrm{GeV}$. The leading and second leading jets are considered as the tagjet pair.%and following jet $\pt < 30\,\mathrm{GeV}$.
\item 
%Dijet $\Delta \eta$ cut: $|\Delta\eta (\mathrm{jet1, jet2})| < 1.5$ 
Zeppenfeld variable: $|y_{W} -(y_{jet1} + y_{jet2})/2.0| < 1.2$
\item 
%The $\pt$ of the dijet system formed by the two highest $p_{T}$
%jets in the event $ > 70\,\mathrm{GeV}$.
%The invariant mass of tag jet in the event $ > 600\,\mathrm{GeV}$(control region) and $ > 1000\,\mathrm{GeV}$ (signal region). We perform the tagjet ppair invariant mass fit in the signal region.
The tagjet pair invariant mass $m_{jj} > 1000\,\mathrm{GeV}$ is used in the fit to extract the signal.
\end{itemize}
%%%%%
%Likewise, we improve the signal to background ratio for merged jets by placing the following cuts:
%%%%%%%
%\begin{itemize}
%\item Leading CA8 jet $\pt > 200\,\mathrm{GeV}$
%\item "anti-btag" where the number of AK5 jets in the event that have a b-tag is zero (the b-tag criteria in this case is the CSV medium working point).
%\item $\Delta R_{l,j} > \pi/2$; the distance between the lepton and the jet should be large
%\item $\Delta \Phi_{MET,j} > 2.0$; the azimuthal distance between the missing energy and the jet should be large
%\item $\Delta \Phi_{V,j} > 2.0$; the azimuthal distance between the $W$ boson and the jet should be large
%\end{itemize}
%%%%%%
Table~\ref{tab:SELECTION} provides a summary of the selection requirements. 
%%%%%%%%%%%%%%%%%
\\
\begin{table}
\begin{center}
\caption{Summary of selection criteria.}
\label{tab:SELECTION}
\begin{tabular}{l l}
\hline \hline
         %\Wln\ Selection  & AK5 Jet selection & CA8 Jet Selection\\ \hline 
         \Wln\ Selection  & Tagjet selection\\ \hline 
          Single lepton trigger & $p_{T}^{\textrm{jet 1,2}} >60,>50~\GeV$\\
          High-quality lepton ID and isolation  & $ |y_{W} -(y_{jet1} + y_{jet2})/2.0| < 1.2$\\
          Muon (electron) $\pt> 25 (30)~\GeV$   & tagjet pair invariant mass $m_{jj} > 1000~\GeV$\\ %is used in the fit\\
          $\MET> 25, 30$~GeV for muons(electrons) samples & \\
          %$\MET> 25, 30$~GeV for muons(electrons) samples & tagjet pair invariant mass $> 1000~\GeV$(signal region)\\
          W transverse mass $> 30$~GeV & \\
          Second lepton veto & \\
\hline \hline
\end{tabular}
\end{center}
\end{table}
%%%%%%%%%%%%%%%%%
