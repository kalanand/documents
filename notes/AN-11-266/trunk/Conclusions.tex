\section{Conclusions}
\label{sec:conclusions}
We have presented a search for a resonant enhancement 
in the dijet mass spectrum near $150~\gev$ in events containing a 
leptonically decaying W boson and two or three jets using 5.0 fb${}^{-1}$ 
integrated luminosity.
We have gone through 
the whole analysis chain using 
similar event selection criteria to those used 
by the CDF and D\O\ collaborations,  
with some adjustments 
needed to deal with a much larger $gg$- and $qg$-initiated 
backgrounds at the LHC.  
We have studied detailed Monte Carlo simulations of a wide range 
of signal scenarios.
We find no evidence for anomalous resonant dijet production 
and derive upper limits on the production cross section of 
a dijet resonance recently reported by the CDF 
Collaboration.
At 95\% confidence level, we exclude a dijet bump 
as large as the one observed by CDF at $M_{jj} = 150$ GeV.
We also exclude several important benchmark New Physics models such as
technicolor and a leptophobic Z' that were proposed to explain the 
CDF bump.
We also investigate the range of dijet invariant mass from 
123 to 186 GeV, and find no evidence for any resonant enhancement.

%The goal for this preliminary analysis is to go through 
%the whole analysis chain (``from soup to nuts'') using the 
%same event selection criteria as used by the CDF and D\O\ collaborations. 
%We understand that these selection criteria are not necessarily 
%the most optimal for LHC conditions since the S/B is almost three times worse. 
%In the next iteration of the analysis we will focus on a more
%optimized event selection.  
%We have presented detailed Monte Carlo simulation of a wide range 
%of signal scenarios.
%We generically set limit on the production rate of the $150~\gev$ 
%dijet resonance predicted by these models.
