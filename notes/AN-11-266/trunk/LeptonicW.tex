\subsection{Comparison of Data and Monte Carlo simulation}
To assess the quality of the modeling provided by the MC simulation we 
make comparisons between the MC shape normalized to the total
prediction for the background compared to the overall data
yield after applying the event selection criteria. 
We show the distributions of various kinematic
variables after applying all the selection cuts in 
Figures ~\ref{fig:mu_jet_pt}-\ref{fig:mu_njet}
for the muon+jets sample and in 
Figures ~\ref{fig:elec_jet_pt}-\ref{fig:elec_njet} for
the electron+jets sample. The MC has been been corrected for
lepton reconstruction efficiency and the trigger efficiency.
Seeing reasonable agreement gives us confidence in
the qualitative aspects of the MC modeling. 
%%%%%%%%%%%%%%%%%%%%%%%%%%%%
\begin{figure}[h!t]
  {\centering
    \includegraphics[width=0.49\textwidth]{figs/n-1_plots_mu/cl_jetld_pt.pdf}
    \includegraphics[width=0.49\textwidth]{figs/n-1_plots_mu/cl_jetnt_pt.pdf}
    \caption{Comparison of the leading jet (left) and 
      the second jet (right) $p_{T}$ distributions from data and MC for the muon+jets
      selection. }
    \label{fig:mu_jet_pt}}
\end{figure}
%%%%%%%%%%%%%%%%%%%%%%%%%%%%
\begin{figure}[h!t]
  {\centering
    \includegraphics[width=0.49\textwidth]{figs/n-1_plots_mu/cl_mjj.pdf}
    \includegraphics[width=0.49\textwidth]{figs/n-1_plots_mu/cl_jet2pt_ov_mjj.pdf}
    \caption{Comparison of the distributions from data and MC of the
    dijet mass (left) and the ratio of the $2^{nd} $ jet $p_T $ and dijet mass (right)
    for the muon+jets selection. }
\label{fig:mu_dijetmass}}
\end{figure}
\begin{figure}[h!t]
  {\centering
    \includegraphics[width=0.49\textwidth]{figs/n-1_plots_mu/cl_jetld_eta.pdf}
    \includegraphics[width=0.49\textwidth]{figs/n-1_plots_mu/cl_jetnt_eta.pdf}
    \caption{Comparison of the leading jet $\eta $ (left) and the
    second leading jet $\eta $ (right) distributions from data and MC
    for the muon+jets selection.  }
\label{fig:mu_jet_eta}}
\end{figure}
%%%%%%%%%%%%%%%%%%%%%%%%%%%%
\begin{figure}[h!t]
  {\centering
    \includegraphics[width=0.49\textwidth]{figs/n-1_plots_mu/cl_W_muon_pt.pdf}
    \includegraphics[width=0.49\textwidth]{figs/n-1_plots_mu/cl_W_muon_eta.pdf}
    \caption{Comparison of the muon $p_{T} $ (left) and the muon
      $\eta $ (right) distributions from data and MC for the muon+jets selection.
      }
    \label{fig:mu_muon}}
\end{figure}
%%%%%%%%%%%%%%%%%%%%%%%%%%%%
\begin{figure}[h!t]
  {\centering
    \includegraphics[width=0.49\textwidth]{figs/n-1_plots_mu/cl_W_mt.pdf}
    \includegraphics[width=0.49\textwidth]{figs/n-1_plots_mu/cl_event_met_pfmet.pdf}
    \caption{Comparison of the distributions from data and MC of the
     transverse mass of the muon / MET system (left) and the MET (right)
    for the muon+jets selection. 
    }
    \label{fig:mu_W_Mt}}
\end{figure}
%%%%%%%%%%%%%%%%%%%%%%%%%%%%
%%%%%%%%%%%%%%%%%%%%%%%%%%%%
\begin{figure}[h!t]
  {\centering
    \includegraphics[width=0.49\textwidth]{figs/n-1_plots_mu/cl_deltaRjj.pdf}
    \includegraphics[width=0.49\textwidth]{figs/n-1_plots_mu/cl_deltaeta_jj.pdf}
    \caption{Comparison of the distributions from
      data and MC of the  $\delta R $ (left)
      and the $\delta \eta $ (right) separation between the two jets for the muon+jets selection. 
      }
    \label{fig:mu_dijet}}
\end{figure}
%%%%%%%%%%%%%%%%%%%%%%%%%%%%
%%%%%%%%%%%%%%%%%%%%%%%%%%%%
\begin{figure}[h!t]
  {\centering
    \includegraphics[width=0.49\textwidth]{figs/n-1_plots_mu/cl_dijet_pt.pdf}
    \includegraphics[width=0.49\textwidth]{figs/n-1_plots_mu/cl_deltaphi_jetldmet.pdf}
%    \includegraphics[width=0.49\textwidth]{figs/n-1_plots_mu/cl_njets.pdf}
    \caption{Comparison of the distributions from data and MC of the
    dijet system $p_{T}$ (left) and $\delta \phi $ between the leading jet and MET for the muon+jets
    selection. 
    }
    \label{fig:mu_njet}}
\end{figure}
%%%%%%%%%%%%%%%%%%%%%%%%%%%%
%%%%%%%%%%%%%%%%%%%%%%%%%%%%
\begin{figure}[h!t]
  {\centering
    \includegraphics[width=0.49\textwidth]{figs/n-1_plots_el/elec_jetld_pt.pdf}
    \includegraphics[width=0.49\textwidth]{figs/n-1_plots_el/elec_jetnt_pt.pdf}
    \caption{Comparison of the leading jet $p_{T} $ (left) and the
      second leading jet $p_{T} $ (right) distributions from data and MC
      for the electron+jets selection.       }
    \label{fig:elec_jet_pt}}
\end{figure}
%%%%%%%%%%%%%%%%%%%%%%%%%%%%
\begin{figure}[h!t]
  {\centering
    \includegraphics[width=0.49\textwidth]{figs/n-1_plots_el/elec_mjj.pdf}
    \includegraphics[width=0.49\textwidth]{figs/n-1_plots_el/elec_jet2pt_ov_mjj.pdf}
    \caption{Comparisons of the distributions from data and MC of the dijet mass
     (left) and the ratio of the $2^{nd}$ jet $p_T $ and dijet mass (right)
     for the electron+jets selection. }
    \label{fig:elec_dijetmass}}
\end{figure}
\begin{figure}[h!t]
  {\centering
    \includegraphics[width=0.49\textwidth]{figs/n-1_plots_el/elec_jetld_eta.pdf}
    \includegraphics[width=0.49\textwidth]{figs/n-1_plots_el/elec_jetnt_eta.pdf}
    \caption{Comparison of the leading jet $\eta $ (left) and the
      second leading jet $\eta $ (right) distributions from data and MC for the electron+jets
      selection. 
      }
    \label{fig:elec_jet_eta}}
\end{figure}
%%%%%%%%%%%%%%%%%%%%%%%%%%%%
\begin{figure}[h!t]
  {\centering
    \includegraphics[width=0.49\textwidth]{figs/n-1_plots_el/elec_W_electron_et.pdf}
    \includegraphics[width=0.49\textwidth]{figs/n-1_plots_el/elec_W_electron_eta.pdf}
    \caption{Comparison of the electron $E_{T} $ (left) and the
    electron $\eta $ (right) distributions from data and MC for the
    electron+jets selection.   }
   \label{fig:elec_electron}}
\end{figure}
%%%%%%%%%%%%%%%%%%%%%%%%%%%%
\begin{figure}[h!t]
  {\centering
    \includegraphics[width=0.49\textwidth]{figs/n-1_plots_el/elec_W_mt.pdf}
    \includegraphics[width=0.49\textwidth]{figs/n-1_plots_el/elec_event_met_pfmet.pdf}
    \caption{Comparison of the distributions from data and MC of the transverse mass
     of electron / MET system (left) and the MET (right) for the
      electron+jets selection. 
      }
    \label{fig:elec_W_Mt}}
\end{figure}
%%%%%%%%%%%%%%%%%%%%%%%%%%%%
\begin{figure}[h!t]
  {\centering
     \includegraphics[width=0.49\textwidth]{figs/n-1_plots_el/elec_deltaRjj.pdf}
     \includegraphics[width=0.49\textwidth]{figs/n-1_plots_el/elec_deltaeta_jj.pdf}
    \caption{Comparison of the distributions from data and MC of the $\delta R $
    (left) and the $\delta \eta $ (right) separation between the two jets for the
      electron+jets selection. 
      }
    \label{fig:elec_dijet}}
\end{figure}
%%%%%%%%%%%%%%%%%%%%%%%%%%%%
\begin{figure}[h!t]
  {\centering
    \includegraphics[width=0.49\textwidth]{figs/n-1_plots_el/elec_dijet_pt.pdf}
    \includegraphics[width=0.49\textwidth]{figs/n-1_plots_el/elec_deltaphi_jetldmet.pdf}
%    \includegraphics[width=0.49\textwidth]{figs/n-1_plots_el/elec_njets.pdf}
    \caption{Comparison of the distributions from data and MC of the dijet system
     $p_{T}$ (left) and  $\delta \phi$ between the leading jet and MET for the electron+jets selection. 
      }
    \label{fig:elec_njet}}
\end{figure}
%%%%%%%%%%%%%%%%%%%%%%%%%%%%
%%%%%%%%%%%%%%%%%%%%%%%%%%%%
%%%%%%%%%%%%%%%%%%%%%%%%%%%%
%%%%%%%%%%%%%%%%%%%%%%%%%%%%

\clearpage
%%%%%%%%%%%%%%%%%%%%%%%%%%%%%%%%%%%%%%%%%%%%%%%%%%%%%%%%%%%%%%%%%%%%
%%%%%%%%%%%%%%%%%%%%%%%%%%%%%%%%%%%%%%%%%%%%%%%%%%%%%%%%%%%%%%%%%%%%
%%%%%%%%%%%%%%%%%%%%%%%%%%%%%%%%%%%%%%%%%%%%%%%%%%%%%%%%%%%%%%%%%%%%
\section{Lepton reconstruction/selection efficiencies}
\label{sec:lepeff}
Lepton reconstruction and selection efficiencies are computed using
a ``Tag \& Probe'' technique on the Drell-Yan Z$\to\ell\ell$ events in
both data and MC~\cite{tagnprobe}.  The ratios of data versus MC efficiencies for
various steps of lepton reconstruction/selection are given below. If such
a ratio is found to be statistically inconsistent with unity, it is
applied as a scale factor correction to the MC samples.
%%%%%%%%%%%%%%%%%%
\subsection{Muon reconstruction/isolation corrections}
%%%%%%%%%%%
The efficiency scale factor for muon isolation is given in
Table~\ref{tab:muonEffsRecoToIso_ScaleFactors}.  The scale factor is
statistically consistent with unity throughout.  Results consistent
with these were also obtained by the Drell-Yan group using
Z$\rightarrow\mu\mu$ events.
Figure~\ref{fig:muonisoeffsf} shows this scale factor variation in $m_{jj}$.
%\verbatiminput{muonEffsRecoToIso_ScaleFactors.txt}
%%%%%%%%%%%%%%%%%%%%%%%%%%%%%
\begin{table}[bthp]
\begin{center}
  \begin{tabular}{l l c | l c}
    \hline  \hline
    $p_T$ range (GeV) & $\eta$ range  &
    $\frac{\epsilon_{\rm{Data}}}{\epsilon_{\rm{MC}}}$ & 
    $\eta$ range  & $\frac{\epsilon_{\rm{Data}}}{\epsilon_{\rm{MC}}}$\\
    \hline  
    25--30 &	-2.1-- -1.5 &	1.00  & 1.5--2.1   & 1.00 \\
           &	-1.5-- -1.0 &	0.99  & 1.0--1.5   & 1.00 \\
           &	-1.0-- -0.5 &	1.00  & 0.5--1.0   & 1.00 \\
           &	-0.5--	0.0 &	1.00  & 0.0--0.5   & 1.00 \\
    \hline  
    30--35 &	-2.1-- -1.5 &	1.00  & 1.5--2.1   & 1.00 \\
           &	-1.5-- -1.0 &	0.99  & 1.0--1.5   & 1.00 \\
           &	-1.0-- -0.5 &	1.00  & 0.5--1.0   & 1.00 \\
           &	-0.5--	0.0 &	1.00  & 0.0--0.5   & 1.00 \\
    \hline  
    35--40 &	-2.1-- -1.5 &	0.99  & 1.5--2.1   & 1.00 \\
           &	-1.5-- -1.0 &	0.99  & 1.0--1.5   & 1.00 \\
           &	-1.0-- -0.5 &	1.00  & 0.5--1.0   & 1.00 \\
           &	-0.5--	0.0 &	1.00  & 0.0--0.5   & 1.00 \\
    \hline  
    40--45 &	-2.1-- -1.5 &	1.00  & 1.5--2.1   & 1.00 \\
           &	-1.5-- -1.0 &	0.99  & 1.0--1.5   & 1.00 \\
           &	-1.0-- -0.5 &	1.00  & 0.5--1.0   & 1.00 \\
           &	-0.5--	0.0 &	1.00  & 0.0--0.5   & 1.00 \\
    \hline  
    45--50 &	-2.1-- -1.5 &	1.00  & 1.5--2.1   & 1.00 \\
           &	-1.5-- -1.0 &	0.99  & 1.0--1.5   & 1.00 \\
           &	-1.0-- -0.5 &	1.00  & 0.5--1.0   & 1.00 \\
           &	-0.5--	0.0 &	1.00  & 0.0--0.5   & 1.00 \\
    \hline  
    50--200&	-2.1-- -1.5 &	1.00  & 1.5--2.1   & 1.00 \\
           &	-1.5-- -1.0 &	0.99  & 1.0--1.5   & 1.00 \\
           &	-1.0-- -0.5 &	1.00  & 0.5--1.0   & 1.00 \\
           &	-0.5--	0.0 &	1.00  & 0.0--0.5   & 1.00 \\
    \hline  \hline
  \end{tabular}
\end{center}
\caption{\label{tab:muonEffsRecoToIso_ScaleFactors}
Muon isolation efficiency data/MC scale factors. The statistical uncertainties were found
to be negligible, while the systematic uncertainty is $\sim$1\%.}
\end{table}
%%%%%%%%%%%%
\subsection{Electron reconstruction scale factors}
The efficiency scale factor for electron reconstruction is given in 
Table~\ref{tab:eleEffsSCToReco_ScaleFactors}.
The scale factor is statistically consistent with unity throughout.
Figure~\ref{fig:electronRecoIDeffsf} shows this scale factor variation in $m_{jj}$.
%%%%%%%%%%%
%\verbatiminput{eleEffsSCToReco_ScaleFactors.txt}
%%%%%%%%%%%%%%%%%%%%%%%%%%%%%
\begin{table}[bthp]
\begin{center}
  \begin{tabular}{l l c | l c}
    \hline  \hline
    $p_T$ range (GeV) & $\eta$ range  &
    $\frac{\epsilon_{\rm{Data}}}{\epsilon_{\rm{MC}}}$ & 
    $\eta$ range  & $\frac{\epsilon_{\rm{Data}}}{\epsilon_{\rm{MC}}}$\\
    \hline  
    30--35 &	-2.5-- -1.5 & 1.0096 $\pm$ 0.0062 & 1.5--2.5 & 1.0094 $\pm$ 0.0015 \\
           &	-1.5-- 0.0  & 1.0060 $\pm$ 0.0029 & 0.0--1.5 & 1.0021 $\pm$ 0.0029 \\
    \hline  
    35--40 &	-2.5-- -1.5 & 1.0038 $\pm$ 0.0043 & 1.5--2.5 & 1.0135 $\pm$ 0.0040 \\
           &	-1.5-- 0.0  & 0.9987 $\pm$ 0.0016 & 0.0--1.5 & 0.9935 $\pm$ 0.0016 \\
    \hline  
    40--45 &	-2.5-- -1.5 & 1.0002 $\pm$ 0.0070 & 1.5--2.5 & 1.0111 $\pm$ 0.0034 \\
           &	-1.5-- 0.0  & 0.9951 $\pm$ 0.0012 & 0.0--1.5 & 0.9941 $\pm$ 0.0012 \\
    \hline  
    45--50 &	-2.5-- -1.5 & 1.0202 $\pm$ 0.0021 & 1.5--2.5 & 1.0170 $\pm$ 0.0080 \\
           &	-1.5-- 0.0  & 0.9941 $\pm$ 0.0014 & 0.0--1.5 & 0.9967 $\pm$ 0.0013 \\
    \hline  
    50--200&	-2.5-- -1.5 & 1.0287 $\pm$ 0.0049 & 1.5--2.5 & 1.0421 $\pm$ 0.0092 \\
           &	-1.5-- 0.0  & 0.9805 $\pm$ 0.0130 & 0.0--1.5 & 0.9989 $\pm$ 0.0018 \\
    \hline  \hline
  \end{tabular}
\end{center}
\caption{\label{tab:eleEffsSCToReco_ScaleFactors}
Electron reconstruction efficiency data/MC scale factors. The uncertainties are statistical only.}
\end{table}
%%%%%%%%%%%%%%%%%%%%%%%%%%%%%
%%%%%%%%%%%%%%%%%%%%
\begin{figure}[h!t]
  {\centering
  \includegraphics[width=0.48\textwidth]{figs/effPlots/fig_eff_mu_RecoToIso_ScaleFactors.pdf}
   \caption{Luminosity weighted average efficiency scale factors (data/MC) for muon isolation.}
\label{fig:muonisoeffsf}}
\end{figure}
%%%%%%%%%%%%%%%%%%%%
%%%%%%%%%%%%%%%%%%%%
\begin{figure}[h!t]
  {\centering
  \subfigure[]{
  \includegraphics[width=0.48\textwidth]{figs/effPlots/fig_eff_ele_SCToReco_ScaleFactors.pdf}
   }
   \subfigure[]{
  \includegraphics[width=0.48\textwidth]{figs/effPlots/fig_eff_ele_RecoToID_ScaleFactors.pdf}
   }
   \caption{Luminosity weighted average efficiency scale factors (data/MC) for electron 
   reconstruction, \textit{i.e.}, super cluster $\to$ GSF electron (a) and electron ID (b).}
\label{fig:electronRecoIDeffsf}}
\end{figure}
%%%%%%%%%%%%%%%%%%%%
%%%%%%%%%%%%%%%%%%%%%%%%%%%%%
\subsection{Electron selection (isolation and ID) scale factors}
The efficiency scale factor for electron selection is given in 
Table~\ref{tab:eleEffsRecoToWP80_ScaleFactors}.
The scale factor is statistically consistent with unity in the ECAL barrel and in the endcaps 
within systematic uncertainties.
Figure~\ref{fig:electronRecoIDeffsf} shows this scale factor variation in $m_{jj}$.
%%%%%%%%%%%
%\verbatiminput{eleEffsRecoToWP80_ScaleFactors.txt}
%%%%%%%%%%%%%%%%%%%%%%%%%%%%%
\begin{table}[bthp]
\begin{center}
  \begin{tabular}{l l c | l c}
    \hline  \hline
    $p_T$ range (GeV) & $\eta$ range  &
    $\frac{\epsilon_{\rm{Data}}}{\epsilon_{\rm{MC}}}$ & 
    $\eta$ range  & $\frac{\epsilon_{\rm{Data}}}{\epsilon_{\rm{MC}}}$\\
    \hline  
    30--35 &	-2.5-- -1.5 & 0.9937 $\pm$ 0.0073 & 1.5--2.5 & 0.9372 $\pm$ 0.0074 \\
           &	-1.5-- 0.0  & 1.0018 $\pm$ 0.0009 & 0.0--1.5 & 0.9958 $\pm$ 0.0039 \\
    \hline  
    35--40 &	-2.5-- -1.5 & 0.9545 $\pm$ 0.0055 & 1.5--2.5 & 0.9607 $\pm$ 0.0053 \\
           &	-1.5-- 0.0  & 0.9910 $\pm$ 0.0024 & 0.0--1.5 & 0.9960 $\pm$ 0.0025 \\
    \hline  
    40--45 &	-2.5-- -1.5 & 0.9661 $\pm$ 0.1567 & 1.5--2.5 & 0.9648 $\pm$ 0.0024 \\
           &	-1.5-- 0.0  & 0.9946 $\pm$ 0.0019 & 0.0--1.5 & 0.9892 $\pm$ 0.0877 \\
    \hline  
    45--50 &	-2.5-- -1.5 & 0.9672 $\pm$ 0.0050 & 1.5--2.5 & 0.9729 $\pm$ 0.0051 \\
           &	-1.5-- 0.0  & 0.9938 $\pm$ 0.0773 & 0.0--1.5 & 0.9917 $\pm$ 0.0022 \\
    \hline  
    50--200&	-2.5-- -1.5 & 0.9836 $\pm$ 0.0066 & 1.5--2.5 & 0.9813 $\pm$ 0.0068 \\
           &	-1.5-- 0.0  & 0.9915 $\pm$ 0.0030 & 0.0--1.5 & 0.9857 $\pm$ 0.0030 \\
    \hline  \hline
  \end{tabular}
\end{center}
\caption{\label{tab:eleEffsRecoToWP80_ScaleFactors}
Electron selection efficiency data/MC scale factors. The uncertainties are statistical only.}
\end{table}
%%%%%%%%%%%%%%%%%%%%%%%%%%%%%%%%%%%%%%%%%%%%%%%%%%%%%%%%%%%%%%%%%%%%
%%%%%%%%%%%%%%%%%%%%%%%%%%%%%%%%%%%%%%%%%%%%%%%%%%%%%%%%%%%%%%%%%%%%
%%%%%%%%%%%%%%%%%%%%%%%%%%%%%%%%%%%%%%%%%%%%%%%%%%%%%%%%%%%%%%%%%%%%
\section{Trigger selection}
\label{sec:trigger}
%%%%%%%%%%%%%%%%%%%%%%%%%%%%%%%
%%%%%%%%%%%%%%%%%%%%%%%%%%%%%%%
\subsection{Run2010: Runs 136033--149442}
\begin{itemize}
\item
Muon data:\\
     Mu9 OR Mu11 OR Mu13 OR Mu15\_v* OR Mu17\_v* OR Mu24\_v*  
\item
Electron data:\\   
     Ele10\_* OR Ele15\_* OR Ele17\_* 
\end{itemize}
%%%%%%%%%%%%%%%%%%%%%%%%%%%%%%%
%%%%%%%%%%%%%%%%%%%%%%%%%%%%%%%
\subsection{Run2011A: Menus 5E32 (Runs: 160404--163869), 
1E33 (Runs:165088--166967), and 1.4E33 (Runs:167039--167913)}
\begin{itemize}
\item
Muon data:\\
     IsoMu17\_v* OR Mu30\_v* \\
Note: We really needed to OR in the nonisolated muon 
trigger as it recovers about half of the offline-isolated 
muons rejected by IsoMu, increasing the trigger efficiency 
by ~5\%. 
\item
Electron data:\\   
Ele27\_CaloIdVT\_CaloIsoT\_TrkIdT\_TrkIsoT\_v* \, \, \, \textcolor{red}{5E32 epoch}\\
Ele25\_WP80\_PFMT40\_v1 \, \, \, \textcolor{red}{1E33 epoch}\\
Ele27\_WP80\_PFMT50\_v* \, \, \, \textcolor{red}{1.4E33 epoch}
\end{itemize}
%%%%%%%%%%%%%%%%%%%%%%%%%%%%%%%
%%%%%%%%%%%%%%%%%%%%%%%%%%%%%%%
\subsection{Run2011A:Menu 2E33, Runs 170249--173198}
\begin{itemize}
\item
Muon data:\\
     (IsoMu17\_v13 OR IsoMu20\_v8 OR IsoMu24\_v8) \, \, OR \, \, (Mu30\_v7 OR Mu40\_v5)\\

Note: This epoch was complicated because Mu30, IsoMu17, 
and IsoMu20 were all prescaled for brief periods, so we 
could either break it down into sub-epochs or lump them 
together. We chose the latter because it is predominantly 
IsoMu17 and the sub-epoch lumi accounting is painful.   
\item
Electron data:\\   
Ele27\_WP80\_PFMT50\_v*
\end{itemize}
%%%%%%%%%%%%%%%%%%%%%%%%%%%%%%%
%%%%%%%%%%%%%%%%%%%%%%%%%%%%%%%
\subsection{Run2011A:Menu 3E33, Runs: 173236--173692}
\begin{itemize}
\item
Muon data:\\
HLT\_IsoMu20\_v9 OR HLT\_Mu40\_eta2p1\_v1
\item
Electron data:\\
 Ele27\_CaloIdVT\_CaloIsoT\_TrkIdT\_TrkIsoT\_CentralJet30\_CentralJet25\_PFMHT20\_v*  
\end{itemize}
%%%%%%%%%%%%%%%%%%%%%%%%%%%%%%%
%%%%%%%%%%%%%%%%%%%%%%%%%%%%%%%
\subsection{Run2011B: Menu 3E33, Runs: 175832--178380}
\begin{itemize}
\item
Muon data:\\
    (IsoMu30\_eta2p1\_v3  OR IsoMu24\_eta2p1\_v3  OR IsoMu24\_v9 OR IsoMu20\_v9) \\
     OR \\  
    (Mu40\_eta2p1\_v1  OR  HLT\_Mu40\_v6)
\item
Electron data:\\  
Ele27\_WP80\_PFMT50\_v* OR Ele27\_WP70\_PFMT50\_v*
\end{itemize}
%%%%%%%%%%%%%%%%%%%%%%%%%%%%%%%
%%%%%%%%%%%%%%%%%%%%%%%%%%%%%%%
\subsection{Run2011B: Menu 5E33, Runs: 178420--180252}
\begin{itemize}
\item
Muon data:\\
       (IsoMu30\_eta2p1\_v6 OR IsoMu24\_eta2p1\_v6 OR IsoMu24\_v12 OR \\
       IsoMu30\_eta2p1\_v7 OR IsoMu24\_eta2p1\_v7 OR IsoMu24\_v13) \\
       OR \\
      (Mu40\_eta2p1\_v4 OR  Mu40\_v9) \, \, \,
      \textcolor{red}{(v1.4, 178420-179889)} \\
       OR (Mu40\_eta2p1\_v5 OR  Mu40\_v10) \, \, \,
      \textcolor{red}{(v2.2, 179959--180252)}
\item
Electron data:\\
 Ele32\_WP70\_PFMT50\_v*
\end{itemize}
%%%%%%%%%%%%%%%%%%%%%%%%%%%%%%%%%%%%%%%%%%%%%%%%%%%%%%%%%%%
\section{Trigger efficiency computation}
\label{sec:trigeff}
%%%%%%%%%%%%%%%%%%%%%%%%%%%%%%%%%%%%%%%%%%%%%%%%%%%%%%%%%%%
The efficiency of the single lepton triggers are computed 
using tag \& probe technique from Z$\to\ell^+\ell^-$ events.
The procedure is straightforward and is described in detail 
in \cite{tagnprobe} and \cite{eleceff}.
%%%%%%%%%%%%%%%%%%%%%%%%%%%%%%%%%%%%%%%%%%%%%%%%%%%%%%%%%%
\subsection{Muon trigger efficiency table}
\label{sec:trigeff_mu}
The luminosity weighted average (LWA) trigger efficiency 
for single muon triggers in data is given 
in Table~\ref{tab:muonEffsIsoToHLT_data_LP_LWA}. 
The efficiency is slowly varying
with changes in lepton transverse momentum and rapidity. 
The efficiency is typically
about 90\%.
%%%%%%%%%%%
%\verbatiminput{muonEffsIsoToHLT_data_LP_LWA.txt}
%%%%%%%%%%%%%%%%%%%%%%%%%%%%%
\begin{table}[bthp]
\begin{center}
  \begin{tabular}{l l c | l c}
    \hline  \hline
    $p_T$ range (GeV) & $\eta$ range  & $\epsilon_{\rm{Data}}$ & 
    $\eta$ range  & $\epsilon_{\rm{Data}}$\\
    \hline  
    25--30 &	-2.1-- -1.5 &	0.8490 $\pm$ 0.0032  & 1.5--2.1   & 0.8457 $\pm$ 0.0033 \\
           &	-1.5-- -1.0 &	0.8725 $\pm$ 0.0032  & 1.0--1.5   & 0.8628 $\pm$ 0.0032 \\
           &	-1.0-- -0.5 &	0.9057 $\pm$ 0.0026  & 0.5--1.0   & 0.8999 $\pm$ 0.0027 \\
           &	-0.5--	0.0 &	0.9211 $\pm$ 0.0022  & 0.0--0.5   & 0.9251 $\pm$ 0.0022 \\
    \hline  
    30--35 &	-2.1-- -1.5 &	0.8797 $\pm$ 0.0031  & 1.5--2.1   & 0.8768 $\pm$ 0.0031 \\
           &	-1.5-- -1.0 &	0.9136 $\pm$ 0.0030  & 1.0--1.5   & 0.9016 $\pm$ 0.0031 \\
           &	-1.0-- -0.5 &	0.9397 $\pm$ 0.0025  & 0.5--1.0   & 0.9387 $\pm$ 0.0025 \\
           &	-0.5--	0.0 &	0.9579 $\pm$ 0.0022  & 0.0--0.5   & 0.9556 $\pm$ 0.0021 \\
    \hline  
    35--40 &	-2.1-- -1.5 &	0.8816 $\pm$ 0.0027  & 1.5--2.1   & 0.8894 $\pm$ 0.0026 \\
           &	-1.5-- -1.0 &	0.9142 $\pm$ 0.0025  & 1.0--1.5   & 0.9008 $\pm$ 0.0026 \\
           &	-1.0-- -0.5 &	0.9385 $\pm$ 0.0022  & 0.5--1.0   & 0.9385 $\pm$ 0.0021 \\
           &	-0.5--	0.0 &	0.9571 $\pm$ 0.0019  & 0.0--0.5   & 0.9546 $\pm$ 0.0019 \\
    \hline  
    40--45 &	-2.1-- -1.5 &	0.8878 $\pm$ 0.0024  & 1.5--2.1   & 0.8902 $\pm$ 0.0024 \\
           &	-1.5-- -1.0 &	0.9221 $\pm$ 0.0021  & 1.0--1.5   & 0.9076 $\pm$ 0.0022 \\
           &	-1.0-- -0.5 &	0.9443 $\pm$ 0.0020  & 0.5--1.0   & 0.9457 $\pm$ 0.0019 \\
           &	-0.5--	0.0 &	0.9622 $\pm$ 0.0018  & 0.0--0.5   & 0.9617 $\pm$ 0.0018 \\
    \hline  
    45--50 &	-2.1-- -1.5 &	0.8922 $\pm$ 0.0029  & 1.5--2.1   & 0.8934 $\pm$ 0.0028 \\
           &	-1.5-- -1.0 &	0.9202 $\pm$ 0.0027  & 1.0--1.5   & 0.9069 $\pm$ 0.0027 \\
           &	-1.0-- -0.5 &	0.9458 $\pm$ 0.0024  & 0.5--1.0   & 0.9437 $\pm$ 0.0025 \\
           &	-0.5--	0.0 &	0.9625 $\pm$ 0.0023  & 0.0--0.5   & 0.9615 $\pm$ 0.0023 \\
    \hline  
    50--200&	-2.1-- -1.5 &	0.8920 $\pm$ 0.0031  & 1.5--2.1   & 0.8903 $\pm$ 0.0032 \\
           &	-1.5-- -1.0 &	0.9178 $\pm$ 0.0030  & 1.0--1.5   & 0.9041 $\pm$ 0.0030 \\
           &	-1.0-- -0.5 &	0.9419 $\pm$ 0.0027  & 0.5--1.0   & 0.9424 $\pm$ 0.0028 \\
           &	-0.5--	0.0 &	0.9606 $\pm$ 0.0026  & 0.0--0.5   & 0.9604 $\pm$ 0.0025 \\
    \hline  \hline
  \end{tabular}
\end{center}
\caption{\label{tab:muonEffsIsoToHLT_data_LP_LWA}
Muon trigger efficiency in data (luminosity 
weighted average). The uncertainties are statistical only.}
\end{table}
%%%%%%%%%%%%
%%%%%%%%%%%
\subsection{Electron trigger efficiency table}
\label{sec:trigeff_eleEle27}
The luminosity weighted average trigger efficiency for the 
electron leg of the single electron triggers is shown in
Table~\ref{tab:eleEffsHLTEle}.  
The value is typically about 99\% in the barrel and 92--97\% in 
the endcaps, and is weakly dependent on the electron
$p_T$ and pseudorapidity.
The efficiency for the W transverse mass leg is shown in 
Table~\ref{tab:eleEffsHLTEleMT}.  
The average value is 91.74 $\pm$ 0.07\% with a large variation depending on 
whether the electron is in the barrel or endcaps.
Fortunately for us this is just an overall scaling effect in the 
$m_{jj}$ and $m_{\ell\nu jj}$ distributions.
The impact of this turnon on the dijet invariant mass or WW invariant mass templates
(used in Higgs analysis) is negligible as shown in 
Fig~\ref{}.

%%%%%%%%%%%
%\verbatiminput{eleEffsWP80ToHLTEle27_May10ReReco.txt}
%%%%%%%%%%%%%%%%%%%%%%%%%%%%%%
%\begin{table}[bthp]
%\begin{center}
%  \begin{tabular}{l l c | l c}
%    \hline  \hline
%    $p_T$ range (GeV) & $\eta$ range  & $\epsilon_{\rm{Data}}$ & 
%    $\eta$ range  & $\epsilon_{\rm{Data}}$\\
%    \hline  
%    30--35 &	-2.5-- -1.5 & 0.96 $\pm$ 0.01 & 1.5--2.5 & 0.93 $\pm$ 0.01 \\
%           &	-1.5-- 0.0  & 0.97 $\pm$ 0.00 & 0.0--1.5 & 0.97 $\pm$ 0.00 \\
%    \hline  
%    35--40 &	-2.5-- -1.5 & 0.97 $\pm$ 0.00 & 1.5--2.5 & 0.97 $\pm$ 0.00 \\
%           &	-1.5-- 0.0  & 0.97 $\pm$ 0.00 & 0.0--1.5 & 0.97 $\pm$ 0.00 \\
%    \hline  
%    40--45 &	-2.5-- -1.5 & 0.97 $\pm$ 0.00 & 1.5--2.5 & 0.97 $\pm$ 0.00 \\
%           &	-1.5-- 0.0  & 0.98 $\pm$ 0.00 & 0.0--1.5 & 0.98 $\pm$ 0.00 \\
%    \hline  
%    45--50 &	-2.5-- -1.5 & 0.97 $\pm$ 0.00 & 1.5--2.5 & 0.97 $\pm$ 0.00 \\
%           &	-1.5-- 0.0  & 0.98 $\pm$ 0.00 & 0.0--1.5 & 0.98 $\pm$ 0.00 \\
%    \hline  
%    50--200&	-2.5-- -1.5 & 0.97 $\pm$ 0.01 & 1.5--2.5 & 0.98 $\pm$ 0.00 \\
%           &	-1.5-- 0.0  & 0.98 $\pm$ 0.00 & 0.0--1.5 & 0.98 $\pm$ 0.00 \\
%    \hline  \hline
%  \end{tabular}
%\end{center}
%\caption{\label{tab:eleEffsWP80ToHLTEle27_May10ReReco}
%Electron trigger efficiency in data for the Ele27 trigger in the May10ReReco dataset. 
%The uncertainties are statistical only.}
%\end{table}
%%%%%%%%%%%%%%%%%%%%%%%%%%%%%%%%%%%%%%%%%%%%%%%%%%%%%%%%%%%%%%%%%%%%%
%%%%%%%%%%%%%%%%%%%%%%%%%%%%%
\begin{table}[bthp]
\begin{center}
  \begin{tabular}{l l c | l c}
    \hline  \hline
    $p_T$ range (GeV) & $\eta$ range  & $\epsilon_{\rm{Data}}$ & 
    $\eta$ range  & $\epsilon_{\rm{Data}}$\\
    \hline  
    35--40 &	-2.5-- -1.5 & 0.92 $\pm$ 0.01 & 1.5--2.5 & 0.91 $\pm$ 0.01 \\
           &	-1.5-- 0.0  & 0.99 $\pm$ 0.01 & 0.0--1.5 & 0.99 $\pm$ 0.01 \\
    \hline  
    40--45 &	-2.5-- -1.5 & 0.96 $\pm$ 0.01 & 1.5--2.5 & 0.96 $\pm$ 0.01 \\
           &	-1.5-- 0.0  & 0.99 $\pm$ 0.01 & 0.0--1.5 & 0.99 $\pm$ 0.01 \\
    \hline  
    45--50 &	-2.5-- -1.5 & 0.97 $\pm$ 0.01 & 1.5--2.5 & 0.97 $\pm$ 0.01 \\
           &	-1.5-- 0.0  & 0.99 $\pm$ 0.01 & 0.0--1.5 & 0.99 $\pm$ 0.01 \\
    \hline  
    50--200&	-2.5-- -1.5 & 0.93 $\pm$ 0.01 & 1.5--2.5 & 0.92 $\pm$ 0.01 \\
           &	-1.5-- 0.0  & 0.98 $\pm$ 0.01 & 0.0--1.5 & 0.98 $\pm$ 0.01 \\
    \hline  \hline
  \end{tabular}
\end{center}
\caption{\label{tab:eleEffsHLTEle}
Electron trigger efficiency in data (luminosity weighted average). 
The uncertainties are statistical only.}
\end{table}
%%%%%%%%%%%%%%%%%%%%%%%%%%%%%
%%%%%%%%%%%%%%%%%%%%%%%%%%%%%
\begin{table}[bthp]
\begin{center}
  \begin{tabular}{l l c | l c}
    \hline  \hline
    Offline $m_T$ range (GeV) & electron $\eta$ range  & $\epsilon_{\rm{Data}}$ & 
    electron $\eta$ range  & $\epsilon_{\rm{Data}}$\\
    \hline  
     50--55 &	-2.5-- -1.5 & 0.3580 $\pm$ 0.0167 &	+1.5--2.5 & 0.3580 $\pm$ 0.0167 \\ 
            &	-1.5--0.0 & 0.7315 $\pm$ 0.0129  &	+0.0--1.5 & 0.7315 $\pm$ 0.0129 \\
    \hline
     55--60 &	-2.5-- -1.5 & 0.4796 $\pm$ 0.0165  &	+1.5--2.5 & 0.4796 $\pm$ 0.0165 \\ 
            &	-1.5--0.0 & 0.8151 $\pm$ 0.0112  &	+0.0--1.5 & 0.8151 $\pm$ 0.0112 \\ 
    \hline
     60--65 &	-2.5-- -1.5 & 0.6073 $\pm$ 0.0144  &	+1.5--2.5 & 0.6073 $\pm$ 0.0144 \\ 
            &	-1.5--0.0 & 0.9035 $\pm$ 0.0085  &	+0.0--1.5 & 0.9035 $\pm$ 0.0085 \\ 
    \hline
     65--70 &	-2.5-- -1.5 & 0.7473 $\pm$ 0.0100 &	+1.5--2.5 & 0.7473 $\pm$ 0.0100 \\  
            &	-1.5--0.0 & 0.9548 $\pm$ 0.0047  &	+0.0--1.5 & 0.9548 $\pm$ 0.0047  \\
    \hline
     70--75 &	-2.5-- -1.5 & 0.8256 $\pm$ 0.0069  &	+1.5--2.5 & 0.8256 $\pm$ 0.0069 \\ 
            &	-1.5--0.0 & 0.9756 $\pm$ 0.0036  &	+0.0--1.5 & 0.9756 $\pm$ 0.0036 \\ 
    \hline
     75--80 &	-2.5-- -1.5 & 0.8711 $\pm$ 0.0060  &	+1.5--2.5 & 0.8711 $\pm$ 0.0060 \\ 
            &	-1.5--0.0 & 0.9866 $\pm$ 0.0034  &	+0.0--1.5 & 0.9866 $\pm$ 0.0034 \\
    \hline
     80--85 &	-2.5-- -1.5 & 0.9047 $\pm$ 0.0059  &	+1.5--2.5 & 0.9047 $\pm$ 0.0059  \\
            &	-1.5--0.0 & 0.9934 $\pm$ 0.0034  &	+0.0--1.5 & 0.9934 $\pm$ 0.0034 \\ 
    \hline
     85--90 &	-2.5-- -1.5 & 0.9308 $\pm$ 0.0061  &	+1.5--2.5 & 0.9308 $\pm$ 0.0061 \\ 
            &	-1.5--0.0 & 0.9958 $\pm$ 0.0038  &	+0.0--1.5 & 0.9958 $\pm$ 0.0038  \\
    \hline
     90--95 &	-2.5-- -1.5 & 0.9415 $\pm$ 0.0068  &	+1.5--2.5 & 0.9415 $\pm$ 0.0068  \\
            &	-1.5--0.0 & 0.9975 $\pm$ 0.0046  &	+0.0--1.5 & 0.9975 $\pm$ 0.0046  \\
    \hline
     95--100 &	-2.5-- -1.5 & 0.9441 $\pm$ 0.0080 &	+1.5--2.5 & 0.9441 $\pm$ 0.0080  \\ 
             &	-1.5--0.0 & 0.9973 $\pm$ 0.0057  &	+0.0--1.5 & 0.9973 $\pm$ 0.0057  \\
    \hline
     100--110 &	-2.5-- -1.5 & 0.9358 $\pm$ 0.0074  &	+1.5--2.5 & 0.9358 $\pm$ 0.0074 \\ 
              &	-1.5--0.0 & 0.9980 $\pm$ 0.0059  &	+0.0--1.5 & 0.9980 $\pm$ 0.0059 \\ 
    \hline
     110--120 &	-2.5-- -1.5 & 0.9120 $\pm$ 0.0109  &	+1.5--2.5 & 0.9120 $\pm$ 0.0109 \\
              &	-1.5--0.0 & 0.9963 $\pm$ 0.0101  &	+0.0--1.5 & 0.9963 $\pm$ 0.0101  \\
    \hline
     120--140 &	-2.5-- -1.5 & 0.8721 $\pm$ 0.0117  &	+1.5--2.5 & 0.8721 $\pm$ 0.0117  \\
              &	-1.5--0.0 & 0.9950 $\pm$ 0.0123  &	+0.0--1.5 & 0.9950 $\pm$ 0.0123  \\
    \hline
     140--180 &	-2.5-- -1.5 & 0.8311 $\pm$ 0.0153  &	+1.5--2.5 & 0.8311 $\pm$ 0.0153  \\
              &	-1.5--0.0 & 0.9899 $\pm$ 0.0171  &	+0.0--1.5 & 0.9899 $\pm$ 0.0171  \\
    \hline
     180--240 &	-2.5-- -1.5 & 0.8011 $\pm$ 0.0266  &	+1.5--2.5 & 0.8011 $\pm$ 0.0266  \\
              &	-1.5--0.0 & 0.9915 $\pm$ 0.0290  &	+0.0--1.5 & 0.9915 $\pm$ 0.0290  \\
    \hline
     240--300 &	-2.5-- -1.5 & 0.8110 $\pm$ 0.0710  &	+1.5-+2.5 & 0.8110 $\pm$ 0.0710 \\ 
              &	-1.5--0.0 & 1.0000 $\pm$ 0.0829  &	+0.0--1.5 & 1.0000 $\pm$ 0.0829  \\
    \hline  \hline
  \end{tabular}
\end{center}
\caption{\label{tab:eleEffsHLTEleMT}
Efficiency for W transverse mass cut ($> 50$~GeV for most epochs) in HLT 
for single electron trigger in data (luminosity weighted average). 
The uncertainties are all inclusive.}
\end{table}
%%%%%%%%%%%%%%%%%%%%%%%%%%%%%
%%%%%%%%%%%%%%%%%%%%%%%%%%%%%%%%%%%%%%%%%%%%%%%%%%%%%%%%%%%%%%%%%%%%
%%%%%%%%%%%%%%%%%%%%%%%%%%%%%%%%%%%%%%%
\subsection{Effect of trigger efficiency on shapes}
As shown in Figs.~\ref{fig:muonhlteff}-\ref{fig:EleHadhlteffMT}, 
the overall trigger efficiency with respect to the analysis selection criteria
is uniform~\footnote{Except for the Electron+2Jet+MHT cross-object trigger, for 
which the efficiency has significant variations in the kinematic variables of 
interest. We are currently working to take these variations into account.} 
across various trigger epochs within the systematic uncertainty. 
Since the trigger efficiency is flat, it does not alter the 
di-jet mass shape, other than introducing a 
global factor that will be absorbed in the normalization of the fit.
Therefore, we have followed the strategy not to apply any trigger 
correction in Monte Carlo. 
We compute the systematic error due to the efficiency 
uncertainty by recomputing the envelope for the MC shape templates 
and propagating these templates to the $m_{jj}$ fit as described in 
a later section. 
%%%%%%%%%%%%%%%%%%%%
\begin{figure}[h!t]
  {\centering
  \subfigure[]{
  \includegraphics[width=0.48\textwidth]{figs/effPlots/fig_eff_HLTMu.pdf}
  }   
\vspace*{1mm} \\
  \subfigure[]{
  \includegraphics[width=0.48\textwidth]{figs/effPlots/fig_eff_HLTMu_template.pdf}
   }
   \subfigure[]{
   \includegraphics[width=0.48\textwidth]{figs/effPlots/fig_eff_HLTMu_template4body.pdf}
   }
   \caption{Luminosity weighted average trigger efficiency in the muon data as a function 
   of $m_{jj}$ (a). 
   The effect of this efficiency correction on W+jets shape is shown for 
   $m_{jj}$ (b) and $m_{\ell\nu jj}$ (c) templates.}
\label{fig:muonhlteff}}
\end{figure}
%%%%%%%%%%%%%%%%%%%%
%%%%%%%%%%%%%%%%%%%%
\begin{figure}[h!t]
  {\centering
  \subfigure[]{
  \includegraphics[width=0.48\textwidth]{figs/effPlots/fig_eff_HLTEle27_May10ReReco.pdf}
  }   
\vspace*{1mm} \\
  \subfigure[]{
  \includegraphics[width=0.48\textwidth]{figs/effPlots/fig_eff_HLTEle27_May10ReReco_template.pdf}
   }
   \subfigure[]{
   \includegraphics[width=0.48\textwidth]{figs/effPlots/fig_eff_HLTEle27_May10ReReco_template4body.pdf}
   }
   \caption{Luminosity weighted average trigger efficiency in the 
   %first 200 pb${}^{-1}$ of 2011 electron data (single electron $HLT_Ele27$) 
     electron data for single electron leg as a function 
   of $m_{jj}$ (a). 
   The effect of this efficiency correction on W+jets shape is shown for 
   $m_{jj}$ (b) and $m_{\ell\nu jj}$ (c) templates.}
\label{fig:singleElehlteff}}
\end{figure}
%%%%%%%%%%%%%%%%%%%%
%%%%%%%%%%%%%%%%%%%%
\begin{figure}[h!t]
  {\centering
  \subfigure[]{
  \includegraphics[width=0.48\textwidth]{figs/effPlots/WMt50TriggerEfficiency.png}
  }   
\vspace*{1mm} \\
  \subfigure[]{
  \includegraphics[width=0.48\textwidth]{figs/effPlots/fig_eff_HLTWMT50_template.png}
   }
   \subfigure[]{
   \includegraphics[width=0.48\textwidth]{figs/effPlots/fig_eff_HLTWMT50_template4body.png}
   }
   \caption{Luminosity weighted average trigger efficiency in the 
   %first 200 pb${}^{-1}$ of 2011 electron data (single electron $HLT_Ele27$) 
     electron data for W transverse mass leg as a function 
   of $m_{jj}$ (a). 
   The effect of this efficiency correction on W+jets shape is shown for 
   $m_{jj}$ (b) and $m_{\ell\nu jj}$ (c) templates.}
\label{fig:singleElehlteffMT}}
\end{figure}
%%%%%%%%%%%%%%%%%%%%

\par
To further crosscheck the impact of the trigger correction on the
$m_{jj}$ shape, we applied a single lepton trigger requirement in the MC 
simulation (chosen by an "OR" of different single muon or single 
electron paths present in the Summer11 Monte Carlo production)
and re-derived the shapes.
The rationale is that, although the trigger implemented 
for the MC does not perfectly mimic the actual trigger, it captures
the major kinematic effects on distributions. 
However, the specific trigger paths used to select events are 
distinctly different in data and Monte Carlo. 
The online isolation and ID definitions have evolved over the 
data taking epochs and the transition is not modeled in the simulation. 
Figure~\ref{fig:triggerEffect} shows the difference in shape for 
the $m_{jj}$ distribution in the W+Jets Monte Carlo, for muons (right) 
and electrons (left) with and without applying the closest trigger paths 
available (chosen by an "OR" of different single muon or single electron 
paths). The lower frame shows the difference in shape, after correcting 
for the different area of both distributions.
The two shapes are consistent with each other within the statistical errors
of a few percent.
%%%%%%%
\begin{figure}[h!] {\centering
\unitlength=0.33\linewidth
\includegraphics[width=0.48\textwidth]{figs/crosschecks/Trigger_Muons.png}
\includegraphics[width=0.48\textwidth]{figs/crosschecks/Trigger_Electrons.png}
\caption{Difference in shape for the $m_{jj}$ distribution in the W+jets 
Monte Carlo, for muons (right) and electrons (left) with and without applying 
the closest trigger paths available (chosen by an "OR" of different single 
muon or single electron paths). The lower frame shows the difference in shape, 
after correcting for the different area of both distributions.
The ratio of the two distributions is consistent with unity.} 
\label{fig:triggerEffect}}
\end{figure}
%%%%%%%
%%%%%%%%%%%%%%%%%%%%%%%%%%%%%%%%%%%%%%%
\subsection{Effect of trigger efficiency on yields}
The signal templates are corrected event-by-event for trigger 
efficiency. Since the efficiency is high (90--100\%) with weak  
dependence on lepton kinematics, the overall effect is small. 
The background yields remain unaffected by the absolute value 
of the trigger efficiency because the background yields are 
allowed to float in the fit.
%%%%%%%%%%%%%%%%%%%%%%%%%%%%%%%%%%%%%%%%%%%%%%%%%%%%%%%%%%%%%%%%%%%%
%%%%%%%%%%%%%%%%%%%%%%%%%%%%%%%%%%%%%%%%%%%%%%%%%%%%%%%%%%%%%%%%%%%%
\clearpage
\section{Effect of pileup}
\label{sec:pileup}
The presence of additional interactions, known as Pile-up (PU), is expected to affect
this analysis in the following ways:
%%%%%
\begin{itemize}
\item additional energy deposits from PU will be added to the jets from the main 
interaction
\item additional low $p_{T}$ jets composed of PU energy will be
added to the event
\item tracks and calorimetric towers from PU energy deposits will be
added to the isolation energy sum of the lepton, thus making isolation 
cuts less efficient.
\end{itemize}
%%%%%
Particle flow (PF) algorithms can be used to decrease the effects of pileup. 
Charged PF particles with tracks pointing to non-primary 
vertices are removed from the list of particles used to reconstruct jets. 
Neutral particles do not leave tracks, and therefore cannot be associated with
a vertex and removed. 

\par
Various techniques have been developed and centrally validated in CMS to 
alleviate the degradation in object reconstruction due to PU effects.
The present analysis makes full use of these improvements. 
The so-called Fastjet and L1-offset corrections remove the additional 
energy released in the event from PU interactions.
The charged particles coming from PU are removed prior to 
jet clustering by requiring that all the tracks come from the primary vertex. 
Similarly, for leptons we subtract from the isolation 
energy sum the pileup contribution from charged particles.
As a result of these corrections, 
the effect of pile-up reweighing in the analysis should be small.
Therefore, no additional corrections have been applied in Monte Carlo 
derived $m_{jj}$ templates to account for differences in the number 
of pile-up events compared to the data.
In order to verify this hypothesis, the Summer11 Monte Carlo samples have 
been corrected, with a reweighing factor obtained from the comparison of 
the number of vertices in data and in Monte Carlo .
The effect of PU on the $m_{jj}$ shape from the W+jets Monte Carlo is 
shown in Figure~\ref{fig:pileup}, in which the di-jet mass distribution 
is shown before (red)  and after (blue) the PU correction. 
The lower frame shows the ratio between the two.
We conclude from these plots that the effect of pileup on the dijet mass
distributions is statistically insignificant.
%%%%%%%
\begin{figure}[h!] {\centering
\unitlength=0.33\linewidth
\includegraphics[width=0.48\textwidth]{figs/crosschecks/PU_Muons.png}
\includegraphics[width=0.48\textwidth]{figs/crosschecks/PU_Electrons.png}
\caption{Difference in shape for the $m_{jj}$ distribution in the W+jets 
Monte Carlo, for muons (right) and electrons (left) with and without applying 
PU re-weighting according to the number of primary vertices in the event. 
The lower frame shows the difference in shape, 
after correcting for the different area of both distributions.
The ratio of the two distributions is consistent with unity.} 
\label{fig:pileup}}
\end{figure}
%%%%%%%
%%%%%%%%%%%%%%%%%%%%%%%%%%%%%%%%%%%%%%%%%%%%%%%%%%%%%%%%%%%%%%%%%%%%
%%%%%%%%%%%%%%%%%%%%%%%%%%%%%%%%%%%%%%%%%%%%%%%%%%%%%%%%%%%%%%%%%%%%
%%%%%%%%%%%%%%%%%%%%%%%%%%%%%%%%%%%%%%%%%%%%%%%%%%%%%%%%%%%%%%%%%%%%
%%%%%%%%%%%%%%%%%%%%%%%%%%%%%%%%%%%%%%%%%%%%%%%%%%%%%%%%%%%%%%%%%%%%
%%%%%%%%%%%%%%%%%%%%%%%%%%%%%%%%%%%%%%%%%%%%%%%%%%%%%%%%%%%%%%%%%%%%
%%%%%%%%%%%%%%%%%%%%%%%%%%%%%%%%%%%%%%%%%%%%%%%%%%%%%%%%%%%%%%%%%%%%
%%%%%%%%%%%%%%%%%%%%%%%%%%%%%%%%%%%%%%%%%%%%%%%%%%%%%%%%%%%%%%%%%%%%
\section{Data driven QCD estimation}
\label{sec:qcd}
A background from QCD multijet events comes from 3- or 4-jet events
with one jet passing the lepton criteria as a 'fake'. However, it is
not practical to generate sufficient MC to create a statistically
significant sample that passes the selection criteria. Therefore we
rely on a data-driven approach in which the the sideband samples from
data, which mirror the QCD background, are used instead. Specifically,
we invert the lepton isolation to be $>0.1$ (default selection uses
Iso$_{mu}<0.1$ and Iso$_{el}<0.05$).  We also relax the MET cut to
$>20GeV$ and fit the MET to obtain the expected event counts.

Note that the W transverse mass distributions from the data and MC are
statistically consistent, as shown in
Figure~\ref{fig:QCDCutLoosening_MET} for muons; for electrons there's
an insufficient number of MC events to make the comparison.  The MET
for QCD processes is also 'fake'; i.e., it originates from badly
measured jets, and therefore has an exponentially falling spectrum.
By contrast, all other backgrounds exhibit a wide peak at $\sim
35$~GeV from a real neutrino (with the exception of Z+Jets, where the
MET is the result of a poorly measured lepton). We perform a fit
to the data using the W$jj$ and QCD templates as shown in
Figure~\ref{fig:QCDTemplateFit_MET}. The fraction of QCD relative to 
the data (after accounting for acceptances between MET$>20GeV$ and 
MET$>30GeV$) is given in Table~\ref{tab:qcdfrac}. To account for 
discrepancies in template modeling (e.g. using W+jets MC as a proxy 
for all non-QCD processes) a sufficiently large uncertainty on these 
fractions is assumed. 

%%  $el_{2J}$ $frac_{QCD}=0.0617\pm 0.00384$,
%%  $el_{3J}$ $frac_{QCD}=0.0213\pm 0.00678$.
%%
%% $\mu_{2J}$ $frac_{QCD}=0.001625\pm 0.004214$,
%% $\mu_{3J}$ $frac_{QCD}=0.0\pm 0.0040797$,


\begin{table}[bthp]
\begin{center}
  \begin{tabular}{l c c}
    \hline  \hline
     & 2 jets & 3 jets \\
    \hline  
    electron  &	6.2 $\pm$ 0.4\% & 2.1 $\pm$ 0.7\% \\
    muon      &	0.2 $\pm$ 0.4\% & 0.0 $\pm$ 0.4\% \\
    \hline  \hline
  \end{tabular}
\end{center}
\caption{\label{tab:qcdfrac} Estimates of the percentage of QCD in data
for the muon and electron datasets after selection, separated into 2- and
3-jet bins.}
\end{table}

\subsection{QCD Uncertainties}
\label{sec:qcd_Uncertainty}

When performing the fit (Section~\ref{sec:mjj_fit}) the QCD yield 
is Gaussian-constrained with a mean given by the value shown in
Table~\ref{tab:qcdfrac}.
In the case of electrons, the error on the QCD fraction
is small and we (conservatively) estimate
the uncertainty to be one half of the expected value. For muons the 
uncertainty is the error on the relative fraction (i.e.,
0.4\% for both 2- and 3-jet bins).
When fitting the sum of electron and muon data, the uncertainties
are combined using the standard error propagation machinery.

\subsection{Cross-Checks}
In order to ensure that our sidebands provide a consistent representation of
QCD events, we perform the following cross-checks:
\begin{itemize}
\item Fit the QCD with a Raileigh Function: $xe^{-x^2/2(\sigma_0+\sigma_1x)^2}$,
used during the inclusive cross section measurements~\cite{WZCMS:2010}. 
As can be seen from Fig.~\ref{fig:QCDMETRaileighFit},
the function accurately fits the overall shape as well as the parameter
corresponding to the intrinsic MET resolution ($\sigma_0\simeq 10$~GeV).
\item Compare the W transverse mass shapes for the data sidebands with MET$>20$~GeV vs 
MET$>30$~GeV (Fig.~\ref{fig:QCDMETCutsWmTShape}). Naturally, events with MET$>30$~GeV do not have the
same exponential falloff, since they contain a higher percentage of W's.
\item Examine the impact of setting Iso$>0.1$, rather than Iso$>0.2$.
We compare the MET (Fig.~\ref{fig:QCDISOCutsMETShape}) and W transverse mass
(Fig.~\ref{fig:QCDISOCutsWmTShape}) distributions, and conclude that there is no statistically
significant discrepancy introduced by the looser isolation requirement.
\end{itemize}


%%%%%%%%%%%%%%%%%%%%%%%%%%%%
%%%%%%%
\begin{figure}[h!] {\centering
\unitlength=0.33\linewidth
\includegraphics[width=0.48\textwidth]{figs/qcd/QCDDataVSMC_Muons2J_MET.pdf}
\caption{ Comparison of the MET shapes for MC vs data-driven muon QCD events in the 2-jet bin. The two are statistically consistent.} 
\label{fig:QCDCutLoosening_MET}
}
\end{figure}
%%%%%%%
%%%%%%%
\begin{figure}[h!] {\centering
\unitlength=0.33\linewidth
\includegraphics[width=0.48\textwidth]{figs/qcd/TemplateFit_MET_mu2j.pdf}
\put(-0.80,0.0){(a)} 
\unitlength=0.33\linewidth
\includegraphics[width=0.48\textwidth]{figs/qcd/TemplateFit_MET_mu3j.pdf}
\put(-0.80,0.0){(b)} \\
\unitlength=0.33\linewidth
\includegraphics[width=0.48\textwidth]{figs/qcd/TemplateFit_MET_el2j.pdf}
\put(-0.80,0.0){(c)} 
\unitlength=0.33\linewidth
\includegraphics[width=0.48\textwidth]{figs/qcd/TemplateFit_MET_el3j.pdf}
\put(-0.80,0.0){(d)} 
\caption{MET distributions fit to the QCD and W$jj$ templates for: (a) muons - 2-jet bin, (b) muons - 3-jet bin, (c) electrons - 2-jet bin, (d) electrons - 3-jet bin.} 
\label{fig:QCDTemplateFit_MET}
}
\end{figure}
%%%%%%%
%%%%%%%
\begin{figure}[h!] {\centering
\unitlength=0.33\linewidth
\includegraphics[width=0.48\textwidth]{figs/qcd/RaileighFitQCD_mu2j.pdf}
\put(-0.80,0.0){(a)} 
\unitlength=0.33\linewidth
\includegraphics[width=0.48\textwidth]{figs/qcd/RaileighFitQCD_mu3j.pdf}
\put(-0.80,0.0){(b)} \\
\unitlength=0.33\linewidth
\includegraphics[width=0.48\textwidth]{figs/qcd/RaileighFitQCD_el2j.pdf}
\put(-0.80,0.0){(c)} 
\unitlength=0.33\linewidth
\includegraphics[width=0.48\textwidth]{figs/qcd/RaileighFitQCD_el3j.pdf}
\put(-0.80,0.0){(d)} 
\caption{QCD MET distributions fitted with a Raileigh Function for: (a) muons - 2-jet bin, (b) muons - 3-jet bin, (c) electrons - 2-jet bin, (d) electrons - 3-jet bin.} 
\label{fig:QCDMETRaileighFit}
}
\end{figure}
%%%%%%%
%%%%%%%
\begin{figure}[h!] {\centering
\unitlength=0.33\linewidth
\includegraphics[width=0.48\textwidth]{figs/qcd/METShapeComp_mu2j.pdf}
\put(-0.80,0.0){(a)} 
\unitlength=0.33\linewidth
\includegraphics[width=0.48\textwidth]{figs/qcd/METShapeComp_mu3j.pdf}
\put(-0.80,0.0){(b)} \\
\unitlength=0.33\linewidth
\includegraphics[width=0.48\textwidth]{figs/qcd/METShapeComp_el2j.pdf}
\put(-0.80,0.0){(c)} 
\unitlength=0.33\linewidth
\includegraphics[width=0.48\textwidth]{figs/qcd/METShapeComp_el3j.pdf}
\put(-0.80,0.0){(d)} 
\caption{ QCD W transverse mass shapes with MET$>20$~GeV vs MET$>30$~GeV for: (a) muons - 2-jet bin, (b) muons - 3-jet bin, (c) electrons - 2-jet bin, (d) electrons - 3-jet bin.} 
\label{fig:QCDMETCutsWmTShape}
}
\end{figure}
%%%%%%%
%%%%%%%
\begin{figure}[h!] {\centering
\unitlength=0.33\linewidth
\includegraphics[width=0.48\textwidth]{figs/qcd/ISOShapeComp_MET_mu2j_g01vg02.pdf}
\put(-0.80,0.0){(a)} 
\unitlength=0.33\linewidth
\includegraphics[width=0.48\textwidth]{figs/qcd/ISOShapeComp_MET_mu3j_g01vg02.pdf}
\put(-0.80,0.0){(b)} \\
\unitlength=0.33\linewidth
\includegraphics[width=0.48\textwidth]{figs/qcd/ISOShapeComp_MET_el2j_g01vg02.pdf}
\put(-0.80,0.0){(c)} 
\unitlength=0.33\linewidth
\includegraphics[width=0.48\textwidth]{figs/qcd/ISOShapeComp_MET_el3j_g01vg02.pdf}
\put(-0.80,0.0){(d)} 
\caption{ QCD MET shapes with Iso$>0.1$ vs Iso$>0.2$ for: (a) muons - 2-jet bin, (b) muons - 3-jet bin, (c) electrons - 2-jet bin, (d) electrons - 3-jet bin.} 
\label{fig:QCDISOCutsMETShape}
}
\end{figure}
%%%%%%%
%%%%%%%
\begin{figure}[h!] {\centering
\unitlength=0.33\linewidth
\includegraphics[width=0.48\textwidth]{figs/qcd/ISOShapeComp_WmT_mu2j_g01vg02.pdf}
\put(-0.80,0.0){(a)} 
\unitlength=0.33\linewidth
\includegraphics[width=0.48\textwidth]{figs/qcd/ISOShapeComp_WmT_mu3j_g01vg02.pdf}
\put(-0.80,0.0){(b)} \\
\unitlength=0.33\linewidth
\includegraphics[width=0.48\textwidth]{figs/qcd/ISOShapeComp_WmT_el2j_g01vg02.pdf}
\put(-0.80,0.0){(c)} 
\unitlength=0.33\linewidth
\includegraphics[width=0.48\textwidth]{figs/qcd/ISOShapeComp_WmT_el3j_g01vg02.pdf}
\put(-0.80,0.0){(d)} 
\caption{ QCD W transverse mass shapes with Iso$>0.1$ vs Iso$>0.2$ for: (a) muons - 2-jet bin, (b) muons - 3-jet bin, (c) electrons - 2-jet bin, (d) electrons - 3-jet bin.} 
\label{fig:QCDISOCutsWmTShape}
}
\end{figure}
%%%%%%%
%%%%%%%%%%%%%%%%%%%%%%%%%%%%%%%%%%%%%%%%%%%%%%%%%%%%%%%%%%%%%%%%%%%%
%%%%%%%%%%%%%%%%%%%%%%%%%%%%%%%%%%%%%%%%%%%%%%%%%%%%%%%%%%%%%%%%%%%%
%%%%%%%%%%%%%%%%%%%%%%%%%%%%%%%%%%%%%%%%%%%%%%%%%%%%%%%%%%%%%%%%%%%%
\clearpage
\section{W+jets shape}
\label{sec:wjetsShape}

%\begin{figure}
%\begin{center}
%\includegraphics[width=\textwidth,trim=0 150 0 150,clip]{figs/mjjdfd2body}
%\end{center} 
%\caption{\label{fig:mjj2body}A depiction of the inputs and work flow of the W+jets template and the final fit.}
%\end{figure} 

In order to get a good description of the W+jets shape in data, the
simulation needs to describe well both the matrix elements for the
hard processes, and the subsequent development of the hard partons
into jets of hadrons.  However, no factorization theorem exists to
rigorously separate these two components.  A given (n + 1)-jet event
can be obtained in two ways: from the collinear/soft-radiation
evolution of an appropriate (n + 1)-parton final state, or from an
n-parton configuration where hard, large-angle emission during its
evolution leads to the extra jet.  A factorization scheme defines, on
an event-by-event basis, which of the two paths should be followed.
The two relevant parameters defining such a scheme are: the
factorization/renormalization scale $q^2$ and the matrix element -
parton shower matching threshold.  Optimized values of these
parameters should give the best possible approximation to the W+jets
kinematics for a given fixed-order calculation.  We know that the
physics has to be independent of the relative contributions of the two
components.  Therefore, it is important that any uncertainty in the
W+jets shape due to the choice of these parameters is propagated to
our final limits.

%%%%%%%%%%%%%%%
\begin{figure}
\begin{center}
\includegraphics[width=0.75\textwidth]{figs/Wjets_shapes}
\end{center}
\caption{\label{fig:wjetshapes}The $m_{jj}$ distribution in W+jets events for various
MC samples.}
\end{figure}
%%%%%%%%%%%%%%%

The CMS MadGraph W+jets production uses MLM matching
\cite{Hoche:2006ph} with $k_T$ jets.  The default matching threshold
is 10~GeV (i.e., if the parton $p_{T}$ is greater than 10~GeV, then it
is assumed to have originated from the hard scattering process and
contributes to the matrix element calculation; if the parton $p_{T}$
is less than 10~GeV then it is assumed to come from the parton
shower).  The factorization/renormalization scale $q^2$ corresponds to
the ``transverse mass'' of the W boson: $\sqrt{M_W^2 + p_{T, W}^2}$.

%%%%%%%%%%%%%%%%%%%%%%%%%%%%%%%%%%%%%
\subsection{Factorization/renormalization scale and matrix element - parton shower matching threshold}
\label{sec:wjetsShapeMatchingQ2}
%%%%%%%%%%%
To perform studies of the uncertainty due to the choice of
the $q^2$ and matching scales, alternative MadGraph W+jets samples are
produced in which the corresponding scales are changed by a factor of 2.
Thus, we have ``matching-up'', ``matching-down'', ``scale-up'', and
``scale-down'' samples, each yielding an $m_{jj}$ distribution, or
template.  %Our use of these templates is shown schematically in
%Fig.~\ref{fig:mjj2body} and described below.
Figure~\ref{fig:wjetshapes} shows the input MC $m_{jj}$ templates that
are we have availible.

We can use our samples to find an optimum MC template
$\mathcal{F}_{W+\text{jets}}$,
\[
\mathcal{F}_{W+\text{jets}} = \sum_\alpha f_\alpha \mathcal{G}_\alpha\, + (1-\sum_\alpha f_\alpha)\times\mathcal{G}_\text{nominal},
\]
where $\alpha \in
\{\text{matchingUp,matchingDown,scaleUp,scaleDown}\}$,
$\mathcal{G}_\text{nominal}$ is the template from the default MadGraph
generation and $\mathcal{G}_\alpha$ is the template from the specified
sample.  We define a 2D coordinate system in scale and matching.  The
origin corresponds to the default MadGraph sample.  To move in the
positive scaleUp direction, one sets $f_\text{scaleDown}=0$ and
increases $f_\text{scaleUp}$. To move in the negative scaleUp
direction, one sets $f_\text{scaleUp}=0$ and increases
$f_\text{scaleDown}$.  The same is true for the matching samples.  We
allow the relative contributions from these templates to float in our
fit.  This lets the data determine the best W+jets shape and the
uncertainties on these parameters is automatically propagated to the
uncertainties on the yields that go into the limit setting.  %We scan
%this 2D parameter space and find values of these parameters that
%minimize the negative log likelihood, where we have excluded the
%putative signal region of 130 to 170~GeV from the optimization.  The
%optimal values so obtained are shown in Table~\ref{tab:optimize}.

%%%%%%%%%%%%%%%%%%%
%\begin{table}
%\caption{\label{tab:optimize} Optimal mixture to describe the W+jets shape.}
%\begin{center}
%\begin{tabular}{rc|rc}
%\hline\hline
%\multicolumn{2}{c|}{2 jets} & \multicolumn{2}{c}{3 jets} \\
%\hline
%$f_\text{matchingUp}$ & $0.376 \pm 0.083$ & $f_\text{matchingUp}$ & $0.075 \pm 0.075$ \\
%$f_\text{scaleUp}$ & $0.207 \pm 0.060$ & $f_\text{scaleUp}$ & $0.078 \pm 0.060$ \\
%\hline\hline
%\end{tabular} 
%\end{center} 
%\end{table} 
%%%%%%%%%%%%%%%%%%%
%%%%%%%%%%%%%%%%%%%%%%%%%%%%%%%%%%%%%
%\subsection{Uncertainty in \texorpdfstring{$q^2$}{q-squared} scale and ME-PS matching threshold}
%\label{sec:wjetsShapeMatchingQ2error}
%%%%%%%%%%%
%The errors assigned to this optimization are evaluated using toy MC
%samples generated from the centrally produced MadGraph shapes.  We
%produce these samples and then perform an identical optimization for
%each pseudo dataset.  The distribution of the optimization of these
%pseudo datasets is shown in
%Figs.~\ref{fig:WjetsTemplateQ2MatchingScan}
%and~\ref{fig:WjetsTemplateQ2MatchingScan3}.  To compute the systematic
%uncertainty, we use the W+jets shapes that
%correspond to $\pm 1\sigma$ of the mean values.
%%%%%%%
%\begin{figure}[h!] {\centering
%\includegraphics[width=0.67\textwidth]{figs/2Jet_fSUfMU_systematic_distribution.pdf}
%\includegraphics[width=0.48\textwidth]{figs/2Jet_fSU_systematic_distribution.pdf}
%\includegraphics[width=0.48\textwidth]{figs/2Jet_fMU_systematic_distribution.pdf}
%\caption{Scan of the factorization/renormalization scale and the ME-PS matching scale for the W+jets sample in the 2 jet bin.} 
%\label{fig:WjetsTemplateQ2MatchingScan}}
%\end{figure}
%\begin{figure}[h!] {\centering
%\includegraphics[width=0.67\textwidth]{figs/3Jet_fSUfMU_systematic_distribution.pdf}
%\includegraphics[width=0.48\textwidth]{figs/3Jet_fSU_systematic_distribution.pdf}
%\includegraphics[width=0.48\textwidth]{figs/3Jet_fMU_systematic_distribution.pdf}
%\caption{Scan of the factorization/renormalization scale and the ME-PS matching scale for the W+jets sample in the 3 jet bin.} 
%\label{fig:WjetsTemplateQ2MatchingScan3}}
%\end{figure}
%%%%%%%

As a cross-check we perform the our fit using only the default MadGraph sample and ignore the systematic samples. 
This results in a considerably worse fit to the data with the $\chi^2$ probability dropping from 55\% for our default approach
with floating fractions to only 8\% when fixing those fractions to zero.  The projections of the $\mu$ 2-jet channel are shown
if Fig.~\ref{fig:mjj_nomorph}.  The problems are apparent in the modeling near the diboson peak.  In addition there is 
no particularly good method to now include the systematic error due to these deficiencies.  For this reason we adopt 
the morphing strategy described above.

\begin{figure}[h!]
\begin{center}
\includegraphics[width=0.45\textwidth]{figs/noMorph/Wjj_Mjj_Muon_2jets_Stacked}
\includegraphics[width=0.45\textwidth]{figs/noMorph/Wjj_Mjj_Muon_2jets_Stacked_log}
\includegraphics[width=0.45\textwidth]{figs/noMorph/Wjj_Mjj_Muon_2jets_Subtracted}
\includegraphics[width=0.45\textwidth]{figs/noMorph/Wjj_Mjj_Muon_2jets_Pull_Corrected}
\end{center}
\caption{\label{fig:mjj_nomorph}The $m_{jj}$ distribution fit to data for muons 2-jets where only the default MadGraph MC is used and the shape morphing using the systematic samples is removed.}
\end{figure}
