\section{Physics objects reconstruction}
\label{sec:reco}
\label{sec:firstStep}
% ---- ---- ---- ---- ---- ---- ---- ---- ---- ---- ---- ---- ---- ---- ----
The analysis relies on the standard reconstruction algorithms 
produced by the CMS community.
Event data is reconstructed using the particle-flow (PF) reconstruction 
technique~\cite{pflow}. 
Particle flow attempts to reconstruct all stable particles in an event by 
combining information from all subdetectors. The algorithm categorizes all 
particles into the following five types: muons, electrons, photons, charged 
and neutral hadrons. The list of reconstructed particles is used as the set of
inputs for a jet clustering algorithm to create particle-flow jets.
%%%%%%%%%%%%%%%%%%%%%%%%%%%%
\subsection{Electron selection}
\label{sec:electron_cuts}
Electrons are reconstructed using a gaussian sum 
filter (GSF) algorithm \cite{CMS-PAS-EGM-10-004},
and are required to pass electron ID cuts according 
to the simple cut-based electron ID~\cite{simplecutbasedelectronid}, 
with the ``VBTF Working Point 70''. 
The GSF algorithm accounts for possible energy loss due to
bremsstrahlung in the tracker layers.
The energy of an electron candidate with $\et>30~\gev$ is essentially
determined by the ECAL cluster energy, while its momentum direction
is determined by that of the associated track.
The simple cut based electron ID relies on three shower
shape variables with different cut values for the barrel and
the endcap regions. The three variables are:
%%%%%%%%%%%%%%%%%%%
\begin{itemize}
\item $\sigma_{i\eta i\eta}$, the supercluster $\eta$ width.
\item $\eta_{\mathrm{SC}} - \eta_{\mathrm{trk}}$: Difference between
      the $\eta$ of the supercluster (SC) and the $\eta$ of the track,
      extrapolated from the vertex.
\item $\phi_{\mathrm{SC}} - \phi_{\mathrm{trk}}$: Difference between
      the $\phi$ of the supercluster and the $\phi$ of the track,
      extrapolated from the vertex.
\end{itemize}
%%%%%%%%%%%%%%%%%%%
The cut values used in the analysis can be found in
Table~\ref{tab:EleID}.
%%%%%%%%%%%%%%%%%%%
\begin{table}[t]
\begin{center}
{\footnotesize
\begin{tabular}{|c|c|c|c|c|}
\hline
ID Variable & WP70 Barrel & WP70 Endcaps & WP95 Barrel & WP95 Endcaps  \\
\hline
$\sigma_{i\eta i\eta}$ & 0.01 & 0.03 & 0.01 & 0.03 \\
$\phi_{\mathrm{SC}} - \phi_{\mathrm{trk}}$ & 0.03 & 0.02 & 0.8 & 0.7 \\
$\eta_{\mathrm{SC}} - \eta_{\mathrm{trk}}$ & 0.004 & 0.005 & 0.007 & 0.01 \\
\hline
\end{tabular}
\caption[.]{\label{tab:EleID} Cut values for electron identification
variables for VBTF Working Point (WP) 70 (barrel and endcap), as used
for the tight electron selection, and VBTF Working Point (WP) 95
(barrel and endcap), as used in the loose electron selection.}}
\end{center}
\end{table}
%%%%%%%%%%%%%%%%%%%

Additionally, we require
%%%%%%%%%%%%%%%%%%%
\begin{itemize}
\item Electron $E_\mathrm{T} > 30\,\mathrm{GeV}$.
\item Pseudorapidity $|\eta| < 2.5$. There is an exclusion range due
        to the ECAL barrel-endcap transition region, defined by
        $1.4442 < |\eta_{\mathrm{sc}}| < 1.566$, where
        $\eta_{\mathrm{sc}}$ is the pseudorapidity of the ECAL
        supercluster.
\item Impact parameter: We cut on the absolute value of the impact
       parameter calculated with respect to the average beamspot. We
       require:
\begin{equation*}
 d_0(\mathrm{Bsp}) < 0.02\,\mathrm{cm}.    
\end{equation*}
\item The selected electron candidates have to be isolated simultaneously in
the tracker, and in the electromagnetic and hadronic calorimters.  Combined
relative isolation is defined as
%%%
\begin{equation*}
\mathrm{RelIso_{\mathrm{Comb}}} = \frac{I_{\mathrm{Trk}}+I_{\mathrm{EM}}+I_{\mathrm{had}}}{E_\mathrm{T}}.
\end{equation*} 
%%%
The electron candidate is required to have 
$\mathrm{RelIso_{\mathrm{Comb}}} < 0.05$ in order 
to be considered isolated. 
A pile-up offset subtraction in the isolation cone 
using fastjet algorithm \cite{FastJetPUSubtraction} is applied.
\item 
In order to reject events in which the electron candidate actually
originates from a conversion of a photon into an $e^{+}e^{-}$ pair, we
require the number of missed inner tracker layers of the electron
track to be exactly zero (i.e. there are no missed layers before the
first hit of the electron track from the beamline). In addition, we
reject any event in which the selected electron is flagged as a
conversion, \textit{i.e.}, an electron that has a 
distance of the partner track $|$\texttt{dist}$|$ $< 0.02$~mm and an
opening angle $|$\texttt{dcot}$|$ $< 0.02$~\cite{ConversionRejection}.
\end{itemize}
%%%%%%%%%%%%%%%%%%%
%%%%%%%%%%%%%%%%%%%%%%%%%%%%%%%%%%%%%%%%%%%%%%%%%%%%%%%%%%%%%%%%%%%%%%%%%%%%
%%%%%%%%%%%%%%%%%%%%%%%%%%%%%%%%%%%%%%%%%%%%%%%%%%%%%%%%%%%%%%%%%%%%%%%%%%%%
\subsection{Muon selection}
\label{sec:muon_cuts}
Muons are obtained from the CMS reconstruction \cite{MUONPAS}.
Muon candidates are identified by two different 
algorithms~\cite{MUONPAS}: one proceeds from the inner tracker outwards, 
the other one starts from tracks measured in the muon chambers and matches 
and combines them with tracks reconstructed in the inner tracker. 
Muons from decays in flight of hadrons and punch-through particles are 
reduced by applying a cut on $\chi^2/dof$
of a global fit including tracker and muon detector hits.
In order to ensure a precise estimate of the momentum and impact parameter
only tracks with more than 10 hits in the inner tracker and at least 
one hit in the pixel detector are used.
We require hits in at least two muon detection layers in the measurement,
to ensure a good quality momentum estimate at the trigger level, and
to suppress further any remaining fake muon candidates.
Cosmic muons are rejected by imposing a maximum allowed transverse impact parameter 
distance to the beam spot position.
These criteria are summarized below:
%%%%%%%%%%%%%%%%%%%
\begin{itemize}
\item The muon candidate is reconstructed both as a global muon and
as a tracker muon.
\item The number of hits of the muon track in the silicon tracker has
to be $N_{\mathrm{Hits}} > 10$.
\item Number of pixel hits of the Tracker track $\ge 1$;
\item Number of muon hits of the Global track $\ge 2$;
\item Normalized $\chi^{2}$ of the Global track $< 10.0$.
\item Muon $p_{\mathrm{T}} > 25\,\mathrm{GeV}$.
\item Pseudorapidity $|\eta| < 2.1$.
\item Impact parameter: We cut on the absolute value of the impact
parameter calculated with respect to the beamspot. We require:
\begin{equation*}
 d_0(\mathrm{Bsp}) < 0.02\,\mathrm{cm}.
\end{equation*}
\item In order to make sure that the selected muon and the selected
jets come from the same hard interaction and not from pile up events,
we require that the $z$ coordinate of the PV of the event and the $z$
coordinate of the muon's inner track vertex lie within a distance of
less than 1~cm.
\item The selected muon candidates also have to be isolated.
We require the muon to be isolated simultaneusly in the
tracker, and in the electromagnetic and hadronic calorimeters.  
This ``combined relative isolation'' is defined as
\begin{equation*}
\mathrm{RelIso_{\mathrm{Comb}}} = \frac{I_{\mathrm{Trk}}+I_{\mathrm{EM}}+I_{\mathrm{had}}}{p_\mathrm{T}}.
\end{equation*} 
The muon candidate is required to have
$\mathrm{RelIso_{\mathrm{Comb}}} < 0.1$ in order to be considered
isolated.
\end{itemize}
%%%%%%%%%%%%%%%%%%%
%%%%%%%%%%%%%%%%%%%%%%%%%%%%%%%%%%%%%%%%%%%%%%%%%%%%%%%%%%%%%%%%%%%%%%%%%%%%
%%%%%%%%%%%%%%%%%%%%%%%%%%%%%%%%%%%%%%%%%%%%%%%%%%%%%%%%%%%%%%%%%%%%%%%%%%%%
\subsection{Loose leptons for ``lepton veto'' and ``jet cleaning''}
For the purposes of rejecting events with more than one lepton, we
define ``loose leptons'', which pass less restrictive sets of selection
criteria. For electrons, we consider those
that have $p_{\mathrm{T}} > 15$~GeV, $|\eta| < 2.5$, and
$\mathrm{RelIso_{\mathrm{Trk}}} < 0.2$ and that satisfy electron ID
cuts according to ``VBTF Working Point 95'' to be ``loose''. The cut
values for the electron ID variables used in the analysis can be found
in Table~\ref{tab:EleID}. Similarly, we define a loose muon as a global muon that passes
$p_{\mathrm{T}} > 10\,\mathrm{GeV}$, $|\eta| < 2.5$, and
$\mathrm{RelIso_{\mathrm{Trk}}} < 0.2$.
%%%%%%%%%%%%%%%%%%%%%%%%%%%%%%%%%%%%%%%%%%%%%%%%%%%%%%%%%%%%%%%%%%%%%%%%%%%%
%%%%%%%%%%%%%%%%%%%%%%%%%%%%%%%%%%%%%%%%%%%%%%%%%%%%%%%%%%%%%%%%%%%%%%%%%%%%
\subsection{Jet selection}
\label{sec:firstStep_jets}
Jets are reconstructed with the anti-KT algorithm \cite{cacciari}, 
starting from the set of objects reconstructed by the particle 
flow \cite{pflow,CMS-PAS-JME-10-003,CMS-PAS-PFT-10-002}.
Jets are corrected such that the measured energy of the jet 
correctly reproduces the energy of the initial particle. 
The CMS standard L2 (relative) correction makes the jet response flat in $\eta$.
The standard L3 (absolute) correction brings the jet closer to the $\PT$ of 
a matched generated jet created using generator level input and a similar 
jet clustering algorithm.
The L2 and L3 corrections are calculated using Monte Carlo, and thus a 
L2L3 residual correction is applied that fixes the discrepancies between 
Monte Carlo and data~\cite{newjes-cms}.
In this analysis we use jets with measured (corrected) $\PT$  
greater than 30~$\gev$. 
We require $|\eta| < 2.4$ so that the jets fall within the
tracker acceptance.  
Jets are required to pass a set of loose identification
criteria; this requirement eliminates jets originating from or being seeded by
noisy channels in the calorimeter~\cite{Chatrchyan:2009hy}: 
%%%%%%%%%%%%%%
\begin{itemize}
\item Fraction of energy due to neutral hadrons $<$ 0.99.
\item Fraction of energy due to neutral EM deposits $<$ 0.99.
\item Number of constituents $>$ 1.
\item Number of charged hadrons candidates $>$ 0.
\item Fraction of energy due to charged hadrons candidates $>$ 0.
\item Fraction of energy due to charged EM deposits $<$ 0.99.
\end{itemize}
%%%%%%%%%%
All energy fractions are calculated from uncorrected jets.

\par
In order to account for electron and muon objects that
have been reconstructed as jets, we remove from the jet
collection any jet that falls within a
cone of radius $R= 0.3$ of a loose electron or a loose muon. 
This ``cleaning'' procedure is applied in order to ensure that the same
particle is not double counted as two different physics objects.
%The analysis is performed by requiring exactly one lepton 
%passing very strict identification criteria and 2 jets with 
%selection requirements listed in the following subsection.
%Additionally, we require relatively high particle flow 
%$\met$ in the event ($\met > 25~\gev$) and W transverse 
%mass greater than $40~\gev$. 
%The last two cuts ensure a highly pure W sample along with 
%two jets.  
%
%\subsection{Lepton reconstruction and identification \label{sec:leptonId}}
%\par
%Events with high-$\et$ electrons are selected online when they pass a 
%L1 trigger
%filter that requires a coarse-granularity region of the ECAL to have
%$\et > 12$~GeV. They subsequently must pass an HLT~\cite{HLT}
%filter that requires an ECAL cluster with $\et$ above a threshold, using
%the full granularity of the ECAL and $\et$ measurements
%corrected using offline calibration~\cite{CMS-PAS-EGM-10-003}. 
%The threshold required for this analysis is $\et > 30~\gev$. 
%For the analysis we have used a combination of unprescaled 
%electron triggers.
%
%%%%%%%%%%%%%%%%%%%%%%%%%%%%%%%%%%%%%%%%%%%%%%%%%%%%%%
%%%%%%%%%%%%%%%%%%%%%%%%%%%%%%%%%%%%%%%%%%%%%%%%%%%%%%
\subsection{Missing Transverse Energy}
\label{sec:MET}
An accurate MET measurement is essential for distinguishing
the $\Wo$ signal from QCD backgrounds. 
We use the MET estimate provided by the Particle Flow algorithm.
PF MET showed the best performance
among several MET algorithms~\cite{PFMET}.
The MET is computed as the vector sum of all PF objects.
A good agreement is found between the MET
distributions of $\Wln$ events in data and simulation~\cite{metPAS}.
The resolution for inclusive multi-jet samples and for
$\Wln$ events is also well reproduced by the simulation.  
A relative broadening of a few percent is observed in the data compared to MC,  
and has a negligible impact on the
extraction of the W yields~\cite{WZCMS:2010}.
%%%%%%%%%%%%%%%%%%%%%%%%%%%%%%%%%%%%%%%%%%%%%%%%%%%%%%
%%%%%%%%%%%%%%%%%%%%%%%%%%%%%%%%%%%%%%%%%%%%%%%%%%%%%%
%%%%%%%%%%%%%%%%%%%%%%%%%%%
%\subsection{Trigger\label{sec:trigger}}
%For 2011 data rely on single muon triggers HLT\_IsoMu17 and HLT\_IsoMu24.
%For electrons use logical OR of several triggers: 
%HLT\_Ele27, HLT\_Ele25\_CentralJet30\_CentralJet25\_MHT20, and inclusive 
%W trigger.
%%% need to get all the names correct.
%%%%%%%%%%%%%%%%%%%%%%%
%\begin{table}[h!]
%\caption{HLT paths used by run range in 2010 data}
%\label{tab:HLT}
%\begin{center}
%\begin{tabular}{ | c | c |}
%\hline
%Run Range & Trigger Name \\
%\hline
%136033 - 137028 & HLT\_Ele10\_LW\_L1R \\
%138564 - 140401 & HLT\_Ele15\_SW\_L1R \\
%141956 - 144114 & HLT\_Ele15\_SW\_CaloEleId\_L1R \\
%146428 - 147116 & HLT\_Ele17\_SW\_CaloEleId\_L1R \\
%147196 - 148058 & HLT\_Ele17\_SW\_TightEleId\_L1R\_v1 \\
%148822 - 149063 & HLT\_Ele17\_SW\_TighterEleIdIsol\_L1R\_v2 \\
%149181 - 149442 & HLT\_Ele17\_SW\_TighterEleIdIsol\_L1R\_v3 \\
%\hline
%\end{tabular}
%\end{center}
%\end{table}
%%%%%%%%%%%%%%%%%%%%%%%
\section{Event selection}
\label{sec:evtSel}
The event should have a good primary vertex (PV). This means selecting
the primary vertex with the highest sum of $p_{T}^2$ of the tracks
associated with it and requiring it to have a number of degrees of
freedom (ndof) $\ge 4$, where ndof corresponds to the weighted sum of
the number of tracks used for the construction of the PV. In addition,
the PV must lie in the central detector region of $|z| \le 24$~cm
and $\rho \le 2$~cm around the nominal interaction point.

\par
In the electron channel, we select events that contain exactly one
tight electron candidate fulfilling the criteria described in
Section~\ref{sec:electron_cuts} and reject events that contain a
loose electron or a loose muon in addition to the tight electron. 
In the muon channel, we select events that contain exactly one
tight muon candidate whose criteria are described in
Section~\ref{sec:muon_cuts} and reject events that contain an
additional loose lepton.
In both channels we require an event to have missing transverse energy
MET in excess of 30~GeV and to have transverse mass greater than
30~GeV.  These cuts are designed to reduce the background
from QCD multijet production.

We further require exactly two or three jets passing the cuts
described in Section~\ref{sec:firstStep_jets}.

%%%%%%%%%%%%%%%%%%%%%%%%
\subsection{Additional quality criteria}
\label{sec:evtSelAdditionalCuts}
We apply the following additional cuts to improve 
the signal over background ratio to reduce the systematic 
uncertainty:
%%%%%
\begin{itemize}
\item 
Dijet transverse mass $ > 50\,\mathrm{GeV}$, which further reduces QCD multijet 
background.
\item 
Leading jet $\pt > 40\,\mathrm{GeV}$. 
\item 
Dijet $\Delta \eta$ cut: $\Delta\eta (\mathrm{jet1, jet2}) < 1.2$ 
\item 
The $\pt$ of the dijet system formed by the two highest $p_{T}$
jets in the event $ > 45\,\mathrm{GeV}$.
\item 
Kinematic cut: $0.3 < \frac{\pt^{\mathrm{jet2}}}{m_{jj}} < 0.7$, which
reduces the W+jets background (Section~\ref{sec:jet2ptovermjj}).
\end{itemize}
%%%%%
%%%%%%%%%%%%%%%%%%%%%%%%
\subsection{``Eichten, Lane, and Martin (ELM)'' recommendations}
\label{sec:elmrecommendations}
Some of the above cuts (most notably, $p_T(j1)>40$~GeV, 
$p_T(jj)>45$~GeV, and $|\Delta\eta (jj)|<1.2$) were suggested 
by Eichten, Lane, and Martin~\cite{ELM} to suppress the dominant 
W+jets background at the LHC, which is about 10--20 times higher 
than at the Tevatron. 
However, we do not apply the $Q = M_{W_{jj}} - M_{jj} - M_W > 20$~GeV cut recommended 
in Ref.~\cite{ELM}, as it is specific to the $\rho_t\to W_{\pi_T}$ 
kinematics. 
We also do not apply the $p_T(W)>60$~GeV cut because it removes 
a large fraction of the diboson as well as new physics signal 
along with W$jj$, as shown in Appendix~\ref{sec:CutsAndFitterConfigs}.
