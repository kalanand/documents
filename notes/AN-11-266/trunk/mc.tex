
\section{Signal and background expectations from CMS simulation}
\label{sec:MCexpectations}
% ---- ---- ---- ---- ---- ---- ---- ---- ---- ---- --

\subsection {The signal}
Table~\ref{tab:signals} lists the effective cross-section (production
cross-section multiplied by the branching fraction for subsequent decay
to a lepton, neutrino and two or three jets) of the various proposed signals.
In addition, we calculate the equivalent cross-section of a dijet resonance
analogous to that observed by the CDF experiment as follows:

\begin{equation*}
\sigma_{\textrm{LHC}}^{dijet-resonance} = \sigma_{\textrm{Tevatron}}^{dijet-resonance}  \times \frac{\sigma_{\textrm{LHC}}^{WH}}{\sigma_{\textrm{Tevatron}}^{WH}},\label{eqTevToLHC2}
\end{equation*}

where
\begin{eqnarray*}
  \sigma_{\textrm{Tevatron}}^{dijet-resonance} &=& 4~\textrm{pb (taken from \cite{cdf_paper})} \\
  \sigma_{\textrm{LHC}}^{WH} &=&   46.8~\textrm{fb (taken from \cite{LHC4PDFxsec}, \cite{LHC4PDFbr})} \\
  \sigma_{\textrm{Tevatron}}^{WH} &=& 12~\textrm{fb (taken from \cite{Djouadi199856})} \\
\end{eqnarray*}

Then
\begin{equation*}
\sigma_{\textrm{LHC}}^{dijet-resonance} \times BR(W\to\ell\nu) = 3.43~\textrm{pb.}
\end{equation*}

%%%%%%%%%%%%%%%%%%%%%%%%%%%%%
\begin{table}[bthp]
\begin{center}
  \begin{tabular}{l c c c c c}
    \hline  \hline
    Signal Mode &  $\sigma \times$BR (pb) & $A\times\varepsilon$ ($ejj$) & $A\times\varepsilon$ ($ejjj$) & $A\times\varepsilon$ ($\mu jj$) & $A\times\varepsilon$ ($\mu jjj$) \\
    \hline
    $\Wo\Ho$/$\Zo\Ho$ & 0.0145 &  0.0380 & 0.0132 & 0.0599 & 0.0192 \\
    $\Zo^{\prime}$    & 1.72   &  0.0421 & 0.0138 & 0.0700 & 0.0234 \\
    Technicolor       & 1.58   &  0.0387 & 0.0111 & 0.0648 & 0.0200 \\
    \hline  \hline
  \end{tabular}
\end{center}
\caption{\label{tab:signals}
The cross section times branching ratio for 
several signal models and their overall reconstruction efficiency, which includes
the (unknown) ratio between the two-jet and three-jet final state occurrences.}
\end{table}
%%%%%%%%%%%%%%%%%%%%%%%%%%%%%
%%%%%%%%%%%
\subsection {The backgrounds}
%%%%%%%%%%%
All processes yielding one lepton, two or more jets, 
and missing transverse energy are possible sources 
of background. The most relevant ones are:
%%%%%%%%%%%
\begin{itemize}
  \item W+jets: this is the production of a single W vector boson 
        in association with quarks or gluons that mimic the final state 
        signature. 
        Because of its large cross section, it is by far the most important 
	background to the analysis.
  \item Drell-Yan Z/$\gamma^{*}$+jets: this is the production of 
        a single Z/$\gamma^{*}$ boson in association with quarks or gluons, 
	where one lepton escapes undetected because of acceptance or 
        inefficiency effects, and the hadronic activity mimics the final 
	 state signature.
  \item WW: this electroweak diboson production is an irreducible 
        background for this analysis. Because its cross-section is measured
	at the LHC, it also serves as a standard candle.
  \item WZ: in case the Z decays hadronically, or the W decays hadronically 
              and one Z lepton is not identified by the detector,
              this sample also contributes to the backgrounds.
  \item ZZ: in case one Z decays hadronically, and one lepton is not identified by the detector,
              this sample contributes to the backgrounds.
  \item $t\bar{t}$: top quarks pairs are produced at LHC via the gluon fusion process
              $gg\to{}t\bar{t}$ or via QCD quark annihilation $q\bar{q}\to{}t\bar{t}$.
              The semi-leptonic component, in which one W decays hadronically,
              is reduced by requiring exactly 2 or 3 jets,
              while the fully leptonic decay mode is reduced by requiring anly 
	      one good lepton in the event.
              However, because of acceptance and inefficiencies, 
              this background still contaminates the signal.
  \item Single top production: it proceeds through three separate channels:
       \begin{enumerate}
         \item t-channel: top is produced after a quark-gluon interaction 
               with the exchange of a virtual W.
         \item s-channel: top is produced in association with an anti-bottom, 
               after the annihilation of a pair of quarks in a weak vertex.
         \item tW-channel: top is produced in association with a charged vector boson in a weak process, 
               from a gluon-bottom pair in the initial state.
       \end{enumerate}
%       The first two can be distinguished from signal thanks to a selection on the energy of b-quarks, 
%       which are quite different from signal tag quarks, while tW-channel has got a missing jet.
  \item QCD multi-jet events generate a background 
       because of the non-negligible probability of jets to be reconstructed as leptons.
\end{itemize}
%%%%%%%%%%%%%%%%%%%%%%
The cross section for the backgrounds, multiplied by the branching ratio when meaningful, 
are reported in Table~\ref{tab:bkg_XS}. 
%%%%%%%%%%%
\begin{table}[htb]
  \begin{center}
  \begin{tabular}{c|c}
  \hline  \hline
  Channel & Cross-section (pb) \\
  \hline
  W+jets                        & $31300 \pm 1600 $ \\
  Z+jets                        & $3050 \pm 130$ \\
  WW                            & $43.0 \pm 1.5$ \\
  WZ                            & $18.2 \pm 0.7$\\
  ZZ                            & $5.90 \pm 0.15$\\
  t$\bar{\textnormal{t}}$+jets  & $157.5 \pm 10$ \\
  t+jets ($s$-channel)          & $4.63 \pm 0.19$ \\
  t+jets ($t$-channel)          & $64.57 \pm 2.58$\\
  t+jets ($t$W-channel)         & $15.74 \pm 1.17$ \\
  QCD (ele enriched)            & $6740000 \pm 100\%$\\
  QCD (mu enriched)             & $84700 \pm  100\%$\\
  \hline  \hline
  \end{tabular}
  \end{center}
  \caption{The cross section for the backgrounds, multiplied by the branching ratio when meaningful.}
  \label{tab:bkg_XS}
\end{table}%
%%%%%%%%%%%
%%%%%%%%%%%%%%%%%%%%%%%%%%%%%%%%%%%%%%%%%%%%
\section{Datasets}
\label{sec:technicalities}
% ---- ---- ---- ---- ---- ---- ---- ---- ---- ---- ---- ---- --
\subsection{Data samples}
The data samples used in this analysis were recorded by the CMS experiment in 2010 and 2011.
Only certified runs and luminosity sections are considered, which means that a good functioning
of all CMS sub-detectors is required. The total statistics analyzed correspond to an integrated
luminosity of about $2.1\fbinv$.
% The analysis relies on centrally-produced Primary Datasets (PDs), each of which consists of a collection
% of High Level Trigger (HLT) paths. As explained in Section~\ref{sec:trigger}, single-lepton triggers
% are used to select the events interesting for this analysis in both
% the $e$ and $\mu$ channels, up to an instantaneous luminosity $L \sim 1\cdot10^{33}\percms$. At higher
% luminosities, we move to cross electron+di-jet triggers for the $e$ channel to cope with the
% unsustainable single-electron HLT rate. Therefore, the analysis relies on the so-called ``SingleElectron''
% and ``SingleMu'' PDs for the first data-taking period, and on the ``ElectronHad'' and ``SingleMu'' ones 
% for the most runs. 
The datasets used for the analysis and the corresponding run ranges are listed in Table~\ref{tab:datasets}.
All data samples were reconstructed using a \texttt{CMSSW\_4\_2\_X} release version.
%%%%%%%%%%%
\begin{table}[htb]
  \begin{center}
  \begin{tabular}{r|r}
  \hline  \hline
  Dataset name & Run range \\
  \hline
  /EG/Run2010A-Apr21ReReco-v1/AOD   & 136033 - 144114  \\
  /Mu/Run2010A-Apr21ReReco-v1/AOD   &            \\ 
  \hline
  /Electron/Run2010B-Apr21ReReco-v1/AOD   &  144919 - 149442  \\
  /Mu/Run2010B-Apr21ReReco-v1/AOD         &             \\
  \hline
  /SingleElectron/Run2011A-May10ReReco-v1/AOD   & 160431 - 163869 \\
  /SingleMu/Run2011A-May10ReReco-v1/AOD         &                 \\
  \hline                                     
  /ElectronHad/Run2011A-PromptReco-v4/AOD       & 165088 - 167913   \\
  /SingleMu/Run2011A-PromptReco-v4/AOD          &                 \\
  \hline                                     
  /SingleElectron/Run2011A-05Aug2011-v1/AOD        & 170826 - 172619 \\
  /SingleMu/Run2011A-05Aug2011-v1/AOD           &                 \\
  \hline                                     
  /SingleElectron/Run2011A-PromptReco-v6/AOD       & 172620 - 173692 \\
  /SingleMu/Run2011A-PromptReco-v6/AOD          &                 \\
  \hline                                     
  /SingleElectron/Run2011B-PromptReco-v1/AOD       & 175832 - 180252 \\
 /SingleMu/Run2011B-PromptReco-v1/AOD          &                 \\
  \hline  \hline
  \end{tabular}
  \end{center}
  \caption{Summary of the data samples used and the run ranges of applicability.}
  \label{tab:datasets}
\end{table}%
%%%%%%%%%%%
%%%%%%%%%%%%%%%%%%%%%%%%%%%%%%%%%%%%%%%%%%%%
\subsection{Monte Carlo samples}
Samples for a large variety of electroweak and QCD-induced background sources, 
as well as for the technicolor, Z' (WZ'$\to\ell\nu{}jj$) and associated 
Higgs production (WH$\to\ell\nu{}jj$) 
have been generated and showered using different Monte Carlo generators.
To better reproduce the actual data-taking conditions, where there is a significant probability
that more than two protons interact in the same bunch crossing, pile-up (PU) events are
added on top of the hard scattering. Particle interactions with the detector were reproduced through
a detailed description of CMS.


Technicolor and WH events were produced with PYTHIA6 \cite{pythia}.
The Z' events were produced using MadGraph with parton showering from PYTHIA6 \cite{pythia}.
The background and signal samples used for the studies are listed in Table~\ref{tab:MCsamples},
together with the equivalent luminosity available for the study.
All background MC samples considered in this analysis come from 
the official ``Summer11'' production. Events were
reconstructed making use of a \texttt{CMSSW\_4\_2\_X} release version. 
% The pile-up scenario used for
% these samples consists of a flat distribution of PU events up to ten additional interactions on top
% of the hard scattering plus a poissonian tail with a mean of ten interactions; only pile-up occurring 
% in the same bunch crossing of the main event (in time PU) was considered.
%%%%%%%%%%%%%%%%%%%%%%%%%%%%%%%%%
\begin{sidewaystable}[htb]
  \begin{center}
    \begin{tabular}{l} 
      \hline\hline
      sample \\
      \hline
      /WJetsToLNu\_TuneZ2\_7TeV-madgraph-tauola/Fall11-PU\_S6\_START42\_V14B-v1/AODSIM    \\
      /WW\_TuneZ2\_7TeV\_pythia6\_tauola/Fall11-PU\_S6\_START42\_V14B-v1/AODSIM     \\
      /WZ\_TuneZ2\_7TeV\_pythia6\_tauola/Fall11-PU\_S6\_START42\_V14B-v1/AODSIM     \\
      /TTJets\_TuneZ2\_7TeV-madgraph-tauola/Fall11-PU\_S6\_START42\_V14B-v2/AODSIM    \\
      /QCD\_Pt-20\_MuEnrichedPt-15\_TuneZ2\_7TeV-pythia6/Fall11-PU\_S6\_START42\_V14B-v2/AODSIM  \\      
      /DYJetsToLL\_TuneZ2\_M-50\_7TeV-madgraph-tauola/Fall11-PU\_S6\_START42\_V14B-v1/AODSIM   \\
      /Tbar\_TuneZ2\_s-channel\_7TeV-powheg-tauola/Fall11-PU\_S6\_START42\_V14B-v1/AODSIM     \\
      /Tbar\_TuneZ2\_t-channel\_7TeV-powheg-tauola/Fall11-PU\_S6\_START42\_V14B-v1/AODSIM     \\
      /Tbar\_TuneZ2\_tW-channel-DS\_7TeV-powheg-tauola/Fall11-PU\_S6\_START42\_V14B-v1/AODSIM    \\
      /T\_TuneZ2\_s-channel\_7TeV-powheg-tauola/Fall11-PU\_S6\_START42\_V14B-v1/AODSIM     \\
      /T\_TuneZ2\_t-channel\_7TeV-powheg-tauola/Fall11-PU\_S6\_START42\_V14B-v1/AODSIM     \\
      /T\_TuneZ2\_tW-channel-DS\_7TeV-powheg-tauola/Fall11-PU\_S6\_START42\_V14B-v1/AODSIM     \\
      \hline
      /QCD\_Pt-20to30\_EMEnriched\_TuneZ2\_7TeV-pythia/Summer11-PU\_S4\_START42\_V11-v1/AODSIM\\
      /QCD\_Pt-30to80\_EMEnriched\_TuneZ2\_7TeV-pythia/Summer11-PU\_S4\_START42\_V11-v1/AODSIM\\
      /QCD\_Pt-80to170\_EMEnriched\_TuneZ2\_7TeV-pythia6/Summer11-PU\_S4\_START42\_V11-v1/AODSIM \\
      /QCD\_Pt-20to30\_BCtoE\_TuneZ2\_7TeV-pythia6/Summer11-PU\_S4\_START42\_V11-v1/AODSIM \\
      /QCD\_Pt-30to80\_BCtoE\_TuneZ2\_7TeV-pythia6/Summer11-PU\_S4\_START42\_V11-v1/AODSIM \\
      /QCD\_Pt-80to170\_BCtoE\_TuneZ2\_7TeV-pythia/Summer11-PU\_S4\_START42\_V11-v1/AODSIM   \\
      \hline
      /WJetsToLNu\_TuneZ2\_matchingdown\_7TeV-madgraph-tauola/Summer11-PU\_S4\_START42\_V11-v1/AODSIM  \\
      /WJetsToLNu\_TuneZ2\_matchingup\_7TeV-madgraph-tauola/Summer11-PU\_S4\_START42\_V11-v1/AODSIM  \\
      /WJetsToLNu\_TuneZ2\_scaledown\_7TeV-madgraph-tauola/Summer11-PU\_S4\_START42\_V11-v1/AODSIM          \\
      /WJetsToLNu\_TuneZ2\_scaleup\_7TeV-madgraph-tauola/Summer11-PU\_S4\_START42\_V11-v1/AODSIM            \\
      /WToLNu\_1jEnh2\_2jEnh35\_3jEnh40\_4jEnh50\_7TeV-sherpa/Summer11-PU\_S4\_START42\_V11-v1/AODSIM      \\
      \hline 
      /Zprime\_Wjj\_4\_2\_3\_SIM/andersj-Zprime\_Wjj\_4\_2\_3\_SIM-65edf0a8ef6070859fe5bacc74af2960/USER      \\
      /Zprime\_Wjj\_4\_2\_3\_SIM/andersj-Zprime\_Wjj\_4\_2\_3\_RAW-ca6deeacdf15095bc2c1568bed374b31/USER      \\
      /Zprime\_Wjj\_4\_2\_3\_SIM/andersj-Zprime\_Wjj\_4\_2\_3\_AODSIM-f71d043e41acd38c60e3392468355a0e/USER      \\
      /WHZH\_Wjj\_4\_2\_3\_SIM/andersj-WHZH\_Wjj\_4\_2\_3\_SIM-c3d0044b4728087b67b7d100b833fffe/USER      \\
      /WHZH\_Wjj\_4\_2\_3\_SIM/andersj-WHZH\_Wjj\_4\_2\_3\_RAW-ca6deeacdf15095bc2c1568bed374b31/USER      \\
      /WHZH\_Wjj\_4\_2\_3\_SIM/andersj-WHZH\_Wjj\_4\_2\_3\_AODSIM-f71d043e41acd38c60e3392468355a0e/USER      \\
      /Technirho\_Wjj\_4\_2\_3\_SIM/andersj-Technirho\_Wjj\_4\_2\_3\_SIM-48ae79793c15f90553d38fb9d6b12e45/USER      \\
      /Technirho\_Wjj\_4\_2\_3\_SIM/andersj-Technirho\_Wjj\_4\_2\_3\_RAW-ca6deeacdf15095bc2c1568bed374b31/USER      \\
      /Technirho\_Wjj\_4\_2\_3\_SIM/andersj-Technirho\_Wjj\_4\_2\_3\_AODSIM-f71d043e41acd38c60e3392468355a0e/USER      \\
      /Technicolor\_Wjj\_4\_2\_3\_SIM/andersj-Technicolor\_Wjj\_4\_2\_3\_SIM-90fffd1b741c647cc2d2f30511523c8f/USER      \\
      /Technicolor\_Wjj\_4\_2\_3\_SIM/andersj-Technicolor\_Wjj\_4\_2\_3\_RAW-ca6deeacdf15095bc2c1568bed374b31/USER      \\
      /Technicolor\_Wjj\_4\_2\_3\_SIM/andersj-Technicolor\_Wjj\_4\_2\_3\_AODSIM-f71d043e41acd38c60e3392468355a0e/USER      \\
      \hline\hline
    \end{tabular}
  \end{center}
  \caption{Summary of Monte Carlo samples used in the analysis for background and signal modeling and for systematic studies.}
  \label{tab:datasets:mcstat}
  %FIXME add the corresponding cross-sections
  \label{tab:MCsamples}
\end{sidewaystable}
%%%%%%%%%%%%%%%%%%%%%%%%%%%%%%%%%
\clearpage
%%%%%%%%%%%%%%%%%%%%%%%%%%%%%%%%%%%%%%%%%%%%%%%%%%
%\begin{table*}[hb]
%\caption{
%Monte Carlo samples with $t\bar{t}$, $t$W, ZZ, WZ, WW.
%The last column shows equivalent luminosity of the available MC sample.
%}
%\label{table-mc}
%\vspace*{\medskipamount}
%\begin{center}
%\small
%\begin{tabular}{|c|l|c|}
%\hline
%MC ID & name  & $\sigma$ LO(NLO) [pb]\\
%\hline
%1000030  &  {\footnotesize /WW\_TuneZ2\_7TeV\_pythia6\_tauola/Summer11-PU\_S4\_START42}         & 27.8 (42.9)\\
%1000031  &  {\footnotesize /WZ\_TuneZ2\_7TeV\_pythia6\_tauola/Summer11-PU\_S4\_START42}          & 10.4 (18.3) \\
%4000041  &  {\footnotesize /WJetsToLNu\_TuneZ2\_7TeV-madgraph-tauola/Summer11-PU\_S4\_START42}  & 24640 (31539)\\   
%MC:2924&  {\footnotesize /WJetsToLNu\_TuneZ2\_matchingup\_7TeV-madgraph-tauola/Summer11-PU\_S4\_START42}    & \\
%MC:2924&  {\footnotesize /WJetsToLNu\_TuneZ2\_matchingdown\_7TeV-madgraph-tauola/Summer11-PU\_S4\_START42}  & \\
%MC:2924&  {\footnotesize /WJetsToLNu\_TuneZ2\_scaleup\_7TeV-madgraph-tauola/Summer11-PU\_S4\_START42}       & \\
%MC:2924&  {\footnotesize /WJetsToLNu\_TuneZ2\_scaledown\_7TeV-madgraph-tauola/Summer11-PU\_S4\_START42}     &  \\
%MC:2924&  {\footnotesize /DYJetsToLL\_TuneZ2\_M-50\_7TeV-madgraph-tauola/Summer11-PU\_S4\_START42}  & 2469 (3048)\\     
%\hline
%4000040  & {\footnotesize /TTJets\_TuneZ2\_7TeV-madgraph-tauola/Summer11-PU\_S4\_START42}             & 121 (157.5)\\
%         & {\footnotesize /Tbar\_TuneZ2\_s-channel\_7TeV-powheg-tauola/Summer11-PU\_S4\_START42}      &  (1.44)  \\
%         & {\footnotesize /Tbar\_TuneZ2\_t-channel\_7TeV-powheg-tauola/Summer11-PU\_S4\_START42}      &  (22.65)   \\
%         & {\footnotesize /Tbar\_TuneZ2\_tW-channel-DS\_7TeV-powheg-tauola/Summer11-PU\_S4\_START42}  &   (7.87) \\
%         & {\footnotesize /T\_TuneZ2\_s-channel\_7TeV-powheg-tauola/Summer11-PU\_S4\_START42}         &  (3.19)  \\
%         & {\footnotesize /T\_TuneZ2\_t-channel\_7TeV-powheg-tauola/Summer11-PU\_S4\_START42}         & (41.92)   \\
%         & {\footnotesize /T\_TuneZ2\_tW-channel-DS\_7TeV-powheg-tauola/Summer11-PU\_S4\_START42}     &  (7.87)  \\
%\hline
%1000041  & {\footnotesize /QCD\_Pt-30to80\_EMEnriched\_TuneZ2\_7TeV-pythia/Summer11-PU\_S4\_START42}   &  3866200\\
%1000043  & {\footnotesize /QCD\_Pt-80to170\_EMEnriched\_TuneZ2\_7TeV-pythia6/Summer11-PU\_S4\_START42} & 139500 \\
%1000040  & {\footnotesize /QCD\_Pt-30to80\_BCtoE\_TuneZ2\_7TeV-pythia6/Summer11-PU\_S4\_START42}       & 136804 \\
%1000042  & {\footnotesize /QCD\_Pt-80to170\_BCtoE\_TuneZ2\_7TeV-pythia/Summer11-PU\_S4\_START42}       & 9360  \\
%1000039  & {\footnotesize /QCD\_Pt-20\_MuEnrichedPt-15\_TuneZ2\_7TeV-pythia6/Summer11-PU\_S4\_START42} &  136804 \\
%\hline
%\end{tabular}
%\end{center}
%\end{table*}
%%%%%%%%%%%%%%%%%%%%%%%%%%%%%%%%%%%%%%%%%%%%%%%%%%
